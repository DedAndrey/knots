\section{Регулируемые}

Можно сказать, это затягивающиеся петли наоборот. В них не коренной, а ходовой конец имеет возможность в определенных пределах проскальзывать внутри узла, после чего легко фиксируется.

\subsection{Хонда}

Косая петля, Косой скользящий узел. Английские названия - Farmer’s Halter Loop, Honda or Bowstring Knot.

%TODO Нарисовать быстрый способ \ref{ris:Noname_loop_3}

\begin{figure}[H]\centering
	\begin{minipage}{1\linewidth}
		\begin{center}
			\tcbox[enhanced jigsaw,colframe=black,opacityframe=0.5,opacityback=0.5]
			{\centering{\includesvg[width=0.6\linewidth]{Nonslide/Honda}}}
		\end{center}
	\end{minipage}
\caption{Хонда.}
\label{ris:Honda}
\end{figure}

\subsection{Скользящая петля}

Очень похожа на Хонду. Можно завязать не только прямым способом, но и быстрым. Фактически это Арбор (рис.~\ref{ris:Arbor}, только ходовой и коренной концы поменяны местами и на ходовом конце завязан узел для фиксации. В отличие от Хонды, если потянуть за ходовой конец - узел развяжется.

\begin{figure}[H]\centering
	\begin{minipage}{1\linewidth}
		\begin{center}
			\tcbox[enhanced jigsaw,colframe=black,opacityframe=0.5,opacityback=0.5]
			{\centering{\includesvg[width=0.6\linewidth]{Nonslide/Honda_2}}}
		\end{center}
	\end{minipage}
\caption{Скользящая петля.}
\label{ris:Honda_2}
\end{figure}

\subsection{Fisherman's Bowline}

\begin{figure}[H]\centering
	\begin{minipage}{1\linewidth}
		\begin{center}
			\tcbox[enhanced jigsaw,colframe=black,opacityframe=0.5,opacityback=0.5]
			{\centering{\includesvg[width=0.6\linewidth]{Bowline/Fishermans_Bowline}}}
		\end{center}
	\end{minipage}
\caption{Fisherman's Bowline.}
\label{ris:Fishermans_Bowline}
\end{figure}

\addtocounter{LoopNoName}{1}

\subsection{Петля без названия \arabic{LoopNoName}}

Узел без названия в книге г-на Эшли указан под номером 1025. По свойствам похож на Fishermans Bowline (рис.~\ref{ris:Fishermans_Bowline}).

\begin{figure}[H]\centering
	\begin{minipage}{1\linewidth}
		\begin{center}
			\tcbox[enhanced jigsaw,colframe=black,opacityframe=0.5,opacityback=0.5]
			{\centering{\includesvg[width=0.8\linewidth]{Nonslide/ABOK_1025}}}
		\end{center}
	\end{minipage}
\caption{Петля без названия \arabic{LoopNoName}.}
\label{ris:ABOK_1025}
\end{figure}

\subsection{Bowstring Knot}
% 
Похож на Булинь, только здесь вместо полуштыка простой узел.

\begin{figure}[H]\centering
	\begin{minipage}{1\linewidth}
		\begin{center}
			\tcbox[enhanced jigsaw,colframe=black,opacityframe=0.5,opacityback=0.5]
			{\centering{\includesvg[width=0.65\linewidth]{Nonslide/Bowstring_Knot}}}
		\end{center}
	\end{minipage}
\caption{Bowstring Knot.}
\label{ris:Bowstring_Knot}
\end{figure}

\subsection{Tucked Double Overhand}

%TODO перенести в Регулируемые

\begin{figure}[H]\centering
	\subfloat[Первый вариант]{\label{ris:Tucked_Double_Overhand_1}
	\tcbox[enhanced jigsaw,colframe=black,opacityframe=0.5,opacityback=0.5]
		{\centering
			\includesvg[width=0.55\linewidth]{Nonslide/Tucked_Double_Overhand}}
		}
\end{figure}

\begin{figure}[H]\centering
	\subfloat[Второй вариант]{\label{ris:Tucked_Double_Overhand_2}
	\tcbox[enhanced jigsaw,colframe=black,opacityframe=0.5,opacityback=0.5]
		{\centering
			\includesvg[width=0.55\linewidth]{Nonslide/Tucked_Double_Overhand_1}}
		}
	\caption{Tucked Double Overhand.}\label{ris:Tucked_Double_Overhand}
\end{figure}

\subsection{Петля из Колокольного узла}

Из Bell Ringers Knot (рис.~\ref{ris:Bell_Ringers_Knot}). Похож на Фламандскую петлю, конец выходит немного не так.

\begin{figure}[H]\centering
	\subfloat[Завязывание]{\label{ris:From_Bell_Ringer_Knot_1}
	\tcbox[enhanced jigsaw,colframe=black,opacityframe=0.5,opacityback=0.5]
		{\centering
			\includesvg[width=0.75\linewidth]{Nonslide/From_Bell_Ringer_Knot}}
		}
\end{figure}

\begin{figure}[H]\centering
	\subfloat[Результат]{\label{ris:From_Bell_Ringer_Knot_2}
	\tcbox[enhanced jigsaw,colframe=black,opacityframe=0.5,opacityback=0.5]
		{\centering
			\includesvg[width=0.85\linewidth]{Nonslide/From_Bell_Ringer_Knot_1}}
		}
	\caption{Петля из Колокольного узла.}\label{ris:From_Bell_Ringer_Knot}
\end{figure}

\subsection{Decorative loop}

%TODO Посмотреть в Булинях, вроде там есть похожее

\begin{figure}[H]\centering
	\begin{minipage}{1\linewidth}
		\begin{center}
			\tcbox[enhanced jigsaw,colframe=black,opacityframe=0.5,opacityback=0.5]
			{\centering{\includesvg[width=0.65\linewidth]{Nonslide/decorative_loop_that_starts_with_an_Overhand_Knot}}}
		\end{center}
	\end{minipage}
\caption{Decorative loop.}
\label{ris:decorative_loop_that_starts_with_an_Overhand_Knot}
\end{figure}

\addtocounter{LoopNoName}{1}

\subsection{Петля без названия \arabic{LoopNoName}}

\begin{figure}[H]\centering
	\subfloat[Завязывание]{\label{ris:Noname_loop_3_1}
	\tcbox[enhanced jigsaw,colframe=black,opacityframe=0.5,opacityback=0.5]
		{\centering
			\includesvg[width=0.6\linewidth]{Nonslide/Noname_loop_3_1}}
		}
\hfil
	\subfloat[Завязывание]{\label{ris:Noname_loop_3_2}
	\tcbox[enhanced jigsaw,colframe=black,opacityframe=0.5,opacityback=0.5]
		{\centering
			\includesvg[width=0.6\linewidth]{Nonslide/Noname_loop_3_2}}
		}
\end{figure}

\begin{figure}[H]\centering
	\subfloat[Результат]{\label{ris:Noname_loop_3_3}
	\tcbox[enhanced jigsaw,colframe=black,opacityframe=0.5,opacityback=0.5]
		{\centering
			\includesvg[width=0.6\linewidth]{Nonslide/Noname_loop_3_3}}
		}
	\caption{Петля без названия \arabic{LoopNoName}.}\label{ris:Noname_loop_3}
\end{figure}

\addtocounter{LoopNoName}{1}

\subsection{Петля без названия \arabic{LoopNoName}}

\begin{figure}[H]\centering
	\subfloat[Завязывание]{\label{ris:Noname_loop_4_1}
	\tcbox[enhanced jigsaw,colframe=black,opacityframe=0.5,opacityback=0.5]
		{\centering
			\includesvg[width=0.6\linewidth]{Nonslide/Noname_loop_4}}
		}
\end{figure}

\begin{figure}[H]\centering
	\subfloat[Результат]{\label{ris:Noname_loop_4_2}
	\tcbox[enhanced jigsaw,colframe=black,opacityframe=0.5,opacityback=0.5]
		{\centering
			\includesvg[width=0.6\linewidth]{Nonslide/Noname_loop_4_1}}
		}
	\caption{Петля без названия \arabic{LoopNoName}.}\label{ris:Noname_loop_4}
\end{figure}

\addtocounter{LoopNoName}{1}

\subsection{Петля без названия \arabic{LoopNoName}}

\begin{figure}[H]\centering
	\begin{minipage}{1\linewidth}
		\begin{center}
			\tcbox[enhanced jigsaw,colframe=black,opacityframe=0.5,opacityback=0.5]
			{\centering{\includesvg[width=0.45\linewidth]{Nonslide/Noname_loop_6}}}
		\end{center}
	\end{minipage}
\caption{Петля без названия \arabic{LoopNoName}.}
\label{ris:Noname_loop_6}
\end{figure}

\addtocounter{LoopNoName}{1}

\subsection{Петля без названия \arabic{LoopNoName}}

\begin{figure}[H]\centering
	\begin{minipage}{1\linewidth}
		\begin{center}
			\tcbox[enhanced jigsaw,colframe=black,opacityframe=0.5,opacityback=0.5]
			{\centering{\includesvg[width=0.45\linewidth]{Nonslide/Noname_loop_5}}}
		\end{center}
	\end{minipage}
\caption{Петля без названия \arabic{LoopNoName}.}
\label{ris:Noname_loop_5}
\end{figure}

\addtocounter{LoopNoName}{1}

\subsection{Петля без названия \arabic{LoopNoName}.}

Петля вяжется из Clove Hitch (рис.~\ref{ris:Clove_Hitch}). Одновременно имеет признаки регулируемой, зажимаемой и трансформируемой петли. На стадии промежуточного результата является затягивающейся петлей.

\begin{figure}[H]\centering
	\subfloat[Завязывание]{\label{ris:loop_From_Clove_Hitch_1_1}
	\tcbox[enhanced jigsaw,colframe=black,opacityframe=0.5,opacityback=0.5]
		{\centering
			\includesvg[width=0.3\linewidth]{Nonslide/From_Clove_Hitch}}
		}
\end{figure}

\begin{figure}[H]\centering
	\subfloat[Промежуточный результат]{\label{ris:loop_From_Clove_Hitch_1_2}
	\tcbox[enhanced jigsaw,colframe=black,opacityframe=0.5,opacityback=0.5]
		{\centering
			\includesvg[width=0.6\linewidth]{Nonslide/From_Clove_Hitch_0_1}}
		}
\end{figure}

\begin{figure}[H]\centering
	\subfloat[Результат]{\label{ris:loop_From_Clove_Hitch_1_3}
	\tcbox[enhanced jigsaw,colframe=black,opacityframe=0.5,opacityback=0.5]
		{\centering
			\includesvg[width=0.6\linewidth]{Nonslide/From_Clove_Hitch_0}}
		}
\end{figure}

\begin{figure}[H]\centering
	\subfloat[Быстроразвязывающийся вариант]{\label{ris:loop_From_Clove_Hitch_1_4}
	\tcbox[enhanced jigsaw,colframe=black,opacityframe=0.5,opacityback=0.5]
		{\centering
			\includesvg[width=0.6\linewidth]{Nonslide/From_Clove_Hitch_fast_1}}
		}
	\caption{Петля без названия \arabic{LoopNoName}.}\label{ris:loop_From_Clove_Hitch_1}
\end{figure}

\addtocounter{LoopNoName}{1}

\subsection{Петля без названия \arabic{LoopNoName}}

Также имеет признаки регулируемой, зажимаемой и трансформируемой петли. На стадии промежуточного результата является затягивающейся петлей.

\begin{figure}[H]\centering
	\subfloat[Завязывание]{\label{ris:loop_From_Clove_Hitch_2_1}
	\tcbox[enhanced jigsaw,colframe=black,opacityframe=0.5,opacityback=0.5]
		{\centering
			\includesvg[width=0.35\linewidth]{Nonslide/From_Clove_Hitch_1}}
		}
\end{figure}

\begin{figure}[H]\centering
	\subfloat[Промежуточный результат]{\label{ris:loop_From_Clove_Hitch_2_2}
	\tcbox[enhanced jigsaw,colframe=black,opacityframe=0.5,opacityback=0.5]
		{\centering
			\includesvg[width=0.6\linewidth]{Nonslide/From_Clove_Hitch_0_2}}
		}
\end{figure}

\begin{figure}[H]\centering
	\subfloat[Результат]{\label{ris:loop_From_Clove_Hitch_2_3}
	\tcbox[enhanced jigsaw,colframe=black,opacityframe=0.5,opacityback=0.5]
		{\centering
			\includesvg[width=0.6\linewidth]{Nonslide/From_Clove_Hitch_2}}
		}
\end{figure}

\begin{figure}[H]\centering
	\subfloat[Быстроразвязывающийся вариант]{\label{ris:loop_From_Clove_Hitch_2_4}
	\tcbox[enhanced jigsaw,colframe=black,opacityframe=0.5,opacityback=0.5]
		{\centering
			\includesvg[width=0.6\linewidth]{Nonslide/From_Clove_Hitch_fast}}
		}
	\caption{Петля без названия \arabic{LoopNoName}.}\label{ris:loop_From_Clove_Hitch_2}
\end{figure}

\subsection{Улучшенные петли из Clove Hitch.}

\begin{figure}[H]\centering
	\subfloat[Завязывание]{\label{ris:loop_From_Clove_Hitch_3_1}
	\tcbox[enhanced jigsaw,colframe=black,opacityframe=0.5,opacityback=0.5]
		{\centering
			\includesvg[width=0.6\linewidth]{Nonslide/From_Clove_Hitch_3}}
		}
\end{figure}

\begin{figure}[H]\centering
	\subfloat[Быстроразвязывающийся вариант]{\label{ris:loop_From_Clove_Hitch_3_2}
	\tcbox[enhanced jigsaw,colframe=black,opacityframe=0.5,opacityback=0.5]
		{\centering
			\includesvg[width=0.65\linewidth]{Nonslide/From_Clove_Hitch_4}}
		}
	\caption{Улучшенные петли из Clove Hitch.}\label{ris:loop_From_Clove_Hitch_3}
\end{figure}

\addtocounter{LoopNoName}{1}

\subsection{Петля без названия \arabic{LoopNoName}.}

\begin{figure}[H]\centering
	\subfloat[Завязывание]{\label{ris:loop_From_Clove_Hitch_4_1}
	\tcbox[enhanced jigsaw,colframe=black,opacityframe=0.5,opacityback=0.5]
		{\centering
			\includesvg[width=0.33\linewidth]{Nonslide/From_Clove_Hitch_5}}
		}
\end{figure}

\begin{figure}[H]\centering
	\subfloat[Промежуточный результат]{\label{ris:loop_From_Clove_Hitch_4_2}
	\tcbox[enhanced jigsaw,colframe=black,opacityframe=0.5,opacityback=0.5]
		{\centering
			\includesvg[width=0.65\linewidth]{Nonslide/From_Clove_Hitch_5_1}}
		}
\end{figure}

\begin{figure}[H]\centering
	\subfloat[Результат]{\label{ris:loop_From_Clove_Hitch_4_3}
	\tcbox[enhanced jigsaw,colframe=black,opacityframe=0.5,opacityback=0.5]
		{\centering
			\includesvg[width=0.65\linewidth]{Nonslide/From_Clove_Hitch_5_2}}
		}
\end{figure}

\begin{figure}[H]\centering
	\subfloat[Улучшенная петля]{\label{ris:loop_From_Clove_Hitch_4_4}
	\tcbox[enhanced jigsaw,colframe=black,opacityframe=0.5,opacityback=0.5]
		{\centering
			\includesvg[width=0.65\linewidth]{Nonslide/From_Clove_Hitch_5_3}}
		}
	\caption{Петля без названия \arabic{LoopNoName}.}\label{ris:loop_From_Clove_Hitch_4}
\end{figure}

\subsection{Эскимосская петля}

Самая простая незатягивающаяся петля с возможностью регулирования диаметра. Английские названия - Eskimo Bowline Loop Knot, Overhand knot and Half Hitch.

\begin{figure}[H]\centering
	\begin{minipage}{1\linewidth}
		\begin{center}
			\tcbox[enhanced jigsaw,colframe=black,opacityframe=0.5,opacityback=0.5]
			{\centering{\includesvg[width=0.6\linewidth]{Nonslide/Eskimo}}}
		\end{center}
	\end{minipage}
\caption{Эскимосская петля.}
\label{ris:Eskimo}
\end{figure}

\subsection{Эскимосская петля со штыком}

\begin{figure}[H]\centering
	\begin{minipage}{1\linewidth}
		\begin{center}
			\tcbox[enhanced jigsaw,colframe=black,opacityframe=0.5,opacityback=0.5]
			{\centering{\includesvg[width=0.65\linewidth]{Nonslide/Eskimo_2}}}
		\end{center}
	\end{minipage}
\caption{Эскимосская петля со штыком.}
\label{ris:Eskimo_shtyk}
\end{figure}

\subsection{Быстроразвязывающаяся Эскимосская петля}

\begin{figure}[H]\centering
	\begin{minipage}{1\linewidth}
		\begin{center}
			\tcbox[enhanced jigsaw,colframe=black,opacityframe=0.5,opacityback=0.5]
			{\centering{\includesvg[width=0.65\linewidth]{Nonslide/Eskimo_fast}}}
		\end{center}
	\end{minipage}
\caption{Быстроразвязывающаяся Эскимосская петля.}
\label{ris:Eskimo_fast}
\end{figure}

\subsection{Эскимосская петля с двумя шлагами}

\begin{figure}[H]\centering
	\begin{minipage}{1\linewidth}
		\begin{center}
			\tcbox[enhanced jigsaw,colframe=black,opacityframe=0.5,opacityback=0.5]
			{\centering{\includesvg[width=0.65\linewidth]{Nonslide/Eskimo_Double}}}
		\end{center}
	\end{minipage}
\caption{Эскимосская петля с двумя шлагами.}
\label{ris:Eskimo_Double}
\end{figure}

\subsection{Lock Bowline}

\begin{figure}[H]\centering
	\begin{minipage}{1\linewidth}
		\begin{center}
			\tcbox[enhanced jigsaw,colframe=black,opacityframe=0.5,opacityback=0.5]
			{\centering{\includesvg[width=0.65\linewidth]{Bowline/Lock_bowline}}}
		\end{center}
	\end{minipage}
\caption{Lock bowline.}
\label{ris:Lock_bowline}
\end{figure}

\subsection{Петля с обвивом}

Вяжется немного проще, чем рапаловская (рис.~\ref{ris:Rapala}), с меньшим числом изгибов лески. Начинают с прямого скользящего узла, а не косого, как в рапаловской петле. Затем ходовой конец несколько раз обносят вокруг коренного, загибают назад и пропускают в просвет скользящего узла с той же стороны, на какую он из него вышел в вытягивающейся петле. Сколько раз обносить ходовой конец вокруг коренного при вязке этого и подобных узлов? Считается, что толстая леска требует меньшего числа обносов, чем тонкая. Например, для лесок с разрывной нагрузкой более 27 кг (60 фунтов) достаточно 2 обносов, от 22 до 27 кг (50 - 60 фунтов) - 3 обносов, а вот на лесках с разрывной нагрузкой менее 4,5 кг (10 фунтов) возможно придется сделать и 7 обносов, чтобы узел получился максимально прочным. В любом случае, лучший советчик в этом - опыт. Разбираемый узел предварительно затягивается одновременно в трех направлениях - за петлю, коренной и свободный концы, окончательно - за петлю и коренной конец. Затягивать - как и любой другой сложный узел на синтетической леске - лучше предварительно увлажнив.

% TODO Ругается на какой-то символ

\begin{figure}[H]\centering
	\begin{minipage}{1\linewidth}
		\begin{center}
			\tcbox[enhanced jigsaw,colframe=black,opacityframe=0.5,opacityback=0.5]
			{\centering{\includesvg[width=0.7\linewidth]{Nonslide/Petlya_s_obvivom}}}
		\end{center}
	\end{minipage}
\caption{Петля с обвивом.}
\label{ris:Petlya_s_obvivom}
\end{figure}

\subsection{Rapala}

Рапаловская петля. Этот узел, а точнее петля, рекомендован широко известной фирмой Rapala для привязывания воблеров, откуда и происходит такое название. Вязка узла начинается с Хонды (рис.~\ref{ris:Honda}).

% TODO Куда делась Рапала?

\begin{figure}[H]\centering
	\begin{minipage}{1\linewidth}
		\begin{center}
			\tcbox[enhanced jigsaw,colframe=black,opacityframe=0.5,opacityback=0.5]
			{\centering{\includesvg[width=0.7\linewidth]{Nonslide/Rapala}}}
		\end{center}
	\end{minipage}
\caption{Rapala.}
\label{ris:Rapala}
\end{figure}

\subsection{Похож на Rapala}

Если начать вязку узла не с Хонды, а со Скользящего узла (рис.~\ref{ris:Honda_2}), то получится другой узел.

\begin{figure}[H]\centering
	\begin{minipage}{1\linewidth}
		\begin{center}
			\tcbox[enhanced jigsaw,colframe=black,opacityframe=0.5,opacityback=0.5]
			{\centering{\includesvg[width=0.7\linewidth]{Nonslide/Rapala_1}}}
		\end{center}
	\end{minipage}
\caption{Похож на Rapala.}
\label{ris:Rapala_2}
\end{figure}

\subsection{Рыболовная петля}

Совершенная петля (Петля удильщика, Узел с глухой петлей, Compound knot, Angler’s Loop). Одна из классических петель, в основе которой простой узел. Эта петля широко применялась во времена волосяных и шелковых лесок, но может быть использована и на многих современных материалах.

Существует 3 способа вязки рыболовной петли - 1 прямой и 2 непрямых (быстрых). Первый из них более пригоден для подвязывания небольших частей оснастки, например, застежек и вертлюжков, второй - более крупных, таких, как блесны и воблеры. Прямой способ вязки рыболовной петли несколько сложнее, чем непрямой. Ходовым концом делается колышка (закрытая петля, отличающаяся от открытой наличием перехлеста), потом шлаг (полный оборот) вокруг коренного конца, и ходовой конец заводится между колышкой и шлагом. Теперь надо пропустить самую верхнюю часть узла через колышку и затянуть петлю, используемую дальше для крепления элементов оснастки.

\begin{figure}[H]\centering
	\subfloat[Прямой способ вязки]{\label{ris:Fish_Loop_1}
	\tcbox[enhanced jigsaw,colframe=black,opacityframe=0.5,opacityback=0.5]
		{\centering
			\includesvg[width=0.65\linewidth]{Nonslide/Angler_Loop_metod_1}}
		}
\end{figure}

\begin{figure}[H]\centering
	\subfloat[Быстрый способ вязки 1]{\label{ris:Fish_Loop_2}
	\tcbox[enhanced jigsaw,colframe=black,opacityframe=0.5,opacityback=0.5]
		{\centering
			\includesvg[width=0.65\linewidth]{Nonslide/Angler_Loop_metod_2}}
		}
\end{figure}

\begin{figure}[H]\centering
	\subfloat[Быстрый способ вязки 2]{\label{ris:Fish_Loop_3}
	\tcbox[enhanced jigsaw,colframe=black,opacityframe=0.5,opacityback=0.5]
		{\centering
			\includesvg[width=0.4\linewidth]{Nonslide/Angler_Loop_metod_3}}
		}
\end{figure}

\begin{figure}[H]\centering
	\subfloat[Результат]{\label{ris:Fish_Loop_4}
	\tcbox[enhanced jigsaw,colframe=black,opacityframe=0.5,opacityback=0.5]
		{\centering
			\includesvg[width=0.65\linewidth]{Nonslide/Angler_Loop}}
		}
\end{figure}

\begin{figure}[H]\centering
	\subfloat[Быстроразвязывающаяся Рыболовная петля]{\label{ris:Fish_Loop_5}
	\tcbox[enhanced jigsaw,colframe=black,opacityframe=0.5,opacityback=0.5]
		{\centering
			\includesvg[width=0.65\linewidth]{Nonslide/Fast_Angler_Loop}}
		}
	\caption{Рыболовная петля.}\label{ris:Fish_Loop}
\end{figure}

\subsection{Усиленная Рыболовная петля}

Для повышения надежности можно обнести ходовой конец 2-3 раза вокруг коренного конца. Однако, в этом случае возможность быстрого регулирования петли практически исчезает.

\begin{figure}[H]\centering
	\subfloat[Первый вариант]{\label{ris:Fish_Loop_2_1}
	\tcbox[enhanced jigsaw,colframe=black,opacityframe=0.5,opacityback=0.5]
		{\centering
			\includesvg[width=0.8\linewidth]{Nonslide/Angler_Loop_2}}
		}
\end{figure}

\begin{figure}[H]\centering
	\subfloat[Второй вариант]{\label{ris:Fish_Loop_2_2}
	\tcbox[enhanced jigsaw,colframe=black,opacityframe=0.5,opacityback=0.5]
		{\centering
			\includesvg[width=0.8\linewidth]{Nonslide/Angler_Loop_3}}
		}
	\caption{Усиленная Рыболовная петля.}\label{ris:Fish_Loop_2}
\end{figure}

\subsection{Узел, похожий на Рыболовную петлю 1}

По принципу вязки похожа на восьмерку, а по устройству на лиановый узел.

\begin{figure}[H]\centering
	\subfloat[Завязывание]{\label{ris:Fish_Loop_seem_1_1}
	\tcbox[enhanced jigsaw,colframe=black,opacityframe=0.5,opacityback=0.5]
		{\centering
			\includesvg[width=0.55\linewidth]{Nonslide/Angler_Loop_4}}
		}
\end{figure}

\begin{figure}[H]\centering
	\subfloat[Результат]{\label{ris:Fish_Loop_seem_1_2}
	\tcbox[enhanced jigsaw,colframe=black,opacityframe=0.5,opacityback=0.5]
		{\centering
			\includesvg[width=0.65\linewidth]{Nonslide/Angler_Loop_5}}
		}
\end{figure}

\begin{figure}[H]\centering
	\subfloat[Быстроразвязывающийся вариант]{\label{ris:Fish_Loop_seem_1_3}
	\tcbox[enhanced jigsaw,colframe=black,opacityframe=0.5,opacityback=0.5]
		{\centering
			\includesvg[width=0.65\linewidth]{Nonslide/Fast_Angler_Loop_2}}
		}
	\caption{Узел, похожий на Рыболовную петлю 1.}\label{ris:Fish_Loop_seem_1}
\end{figure}

\subsection{Узел, похожий на Рыболовную петлю 2}

\begin{figure}[H]\centering
	\subfloat[Завязывание]{\label{ris:Fish_Loop_seem_2_1}
	\tcbox[enhanced jigsaw,colframe=black,opacityframe=0.5,opacityback=0.5]
		{\centering
			\includesvg[width=0.55\linewidth]{Nonslide/Angler_Loop_6}}
		}
\end{figure}

\begin{figure}[H]\centering
	\subfloat[Результат]{\label{ris:Fish_Loop_seem_2_2}
	\tcbox[enhanced jigsaw,colframe=black,opacityframe=0.5,opacityback=0.5]
		{\centering
			\includesvg[width=0.65\linewidth]{Nonslide/Angler_Loop_7}}
		}
\end{figure}

\begin{figure}[H]\centering
	\subfloat[Быстроразвязывающийся вариант]{\label{ris:Fish_Loop_seem_2_3}
	\tcbox[enhanced jigsaw,colframe=black,opacityframe=0.5,opacityback=0.5]
		{\centering
			\includesvg[width=0.7\linewidth]{Nonslide/Fast_Angler_Loop_3}}
		}
	\caption{Узел, похожий на Рыболовную петлю 2.}\label{ris:Fish_Loop_seem_2}
\end{figure}

%TODO А вот это уже заново изобретенный на базе описанного мной узла, похожего на лиановый. По устройству сильно похож на скользящий баттерфляй, но петля не скользит. На тонких и скользких веревках держится прилично. Не затягивается намертво и не зажимает свободную петлю. Вяжется только на ненатянутой веревке, запоминается плохо. При переменных нагрузках не ползет.

\subsection{Бергвахт}

Карл Пруссик в ноябрьском номере 1940г. австрийского альпинистского журнала \enquote{Bergsteiger} дает описание нового узла \enquote{бергвахт} для завязывания идущего в середине (при охранении веревкой). В противоположность прежнему узлу (\enquote{Lackstich}???), очень трудно развязываемому на намокшей или обледенелой веревке, узел \enquote{бергвахт} развязывается легко, быстро и требует меньшей на 5 \% длины веревки. Прежний узел завязывался из двойной веревки. В узле \enquote{бергвахт} петля делается на каждой веревке отдельно, затем одна петля накладывается на другую. Через образовавшуюся таким образом двойную петлю пропускается широкая петля. Образуется петля, которая может быть легко ослаблена. Существует и альтернативный способ завязывания узла \enquote{бергвахт}. Он также очень прост. В веревке закладывается большая петля. Перед ее узлом делается вторая небольшая петля, через которую пропускается большая петля. Узел \enquote{бергвахт} не следует завязывать на конце веревки, так как его свойство легко развязываться может стать опасным. Карл Пруссик назвал свой узел в честь работников спасательной службы.

\begin{figure}[H]\centering
	\subfloat[Завязывание. Первый вариант]{\label{ris:Bergvaht_1}
	\tcbox[enhanced jigsaw,colframe=black,opacityframe=0.5,opacityback=0.5]
		{\centering
			\includesvg[width=0.6\linewidth]{Nonslide/Bergvaht_1}}
		}
\end{figure}

\begin{figure}[H]\centering
	\subfloat[Завязывание. Второй вариант]{\label{ris:Bergvaht_2}
	\tcbox[enhanced jigsaw,colframe=black,opacityframe=0.5,opacityback=0.5]
		{\centering
			\includesvg[width=0.55\linewidth]{Nonslide/Bergvaht_2}}
		}
\end{figure}

\begin{figure}[H]\centering
	\subfloat[Результат]{\label{ris:Bergvaht_3}
	\tcbox[enhanced jigsaw,colframe=black,opacityframe=0.5,opacityback=0.5]
		{\centering
			\includesvg[width=0.65\linewidth]{Nonslide/Bergvaht}}
		}
	\caption{Бергвахт.}\label{ris:Bergvaht}
\end{figure}

\subsection{Регулируемая петля}

Adjustable Loop, Прочный огон. Похож на Бергвахт, только фиксирующий шлаг вяжется в обратную сторону.

\begin{figure}[H]\centering
	\begin{minipage}{1\linewidth}
		\begin{center}
			\tcbox[enhanced jigsaw,colframe=black,opacityframe=0.5,opacityback=0.5]
			{\centering{\includesvg[width=0.65\linewidth]{Nonslide/Reguliruemaya_1}}}
		\end{center}
	\end{minipage}
\caption{Регулируемая петля.}
\label{ris:Reguliruemaya}
\end{figure}

\subsection{Калмыцкий узел}

Очень похож на Эскимосский булинь в быстроразвязывающемся варианте (рис.~\ref{ris:Kazak_3}). Однако, петля коренного конца повернута не против часовой стрелки, а в обратную сторону, в связи с чем ходовой конец ею зажимается несколько по-другому.

\begin{figure}[H]\centering
	\begin{minipage}{1\linewidth}
		\begin{center}
			\tcbox[enhanced jigsaw,colframe=black,opacityframe=0.5,opacityback=0.5]
			{\centering{\includesvg[width=0.55\linewidth]{Nonslide/Kalmyk}}}
		\end{center}
	\end{minipage}
\caption{Калмыцкий узел.}
\label{ris:Kalmyk}
\end{figure}

\subsection{Чабанский узел}

Развитие быстроразвязывающегося Эскимосского булиня и Калмыцкого узла. Ходовой конец страхует узел, временно блокируя от возможности случайного саморазвязывания. Отличие только в направлении прохождения ходового конца через фиксируемую петлю. Разница не принципиальна.

\begin{figure}[H]\centering
	\subfloat[Эскимосский.\\Первый вариант]{\label{ris:Chaban_1}
	\tcbox[enhanced jigsaw,colframe=black,opacityframe=0.5,opacityback=0.5]
		{\centering
			\includesvg[angle=90,width=0.55\linewidth]{Nonslide/Chaban}}
		}
\hfil
	\subfloat[Калмыцкий.\\Первый вариант]{\label{ris:Chaban_2}
	\tcbox[enhanced jigsaw,colframe=black,opacityframe=0.5,opacityback=0.5]
		{\centering
			\includesvg[angle=90,width=0.55\linewidth]{Nonslide/Chaban_2}}
		}
\end{figure}

\begin{figure}[H]\centering
	\subfloat[Эскимосский.\\Второй вариант]{\label{ris:Chaban_3}
	\tcbox[enhanced jigsaw,colframe=black,opacityframe=0.5,opacityback=0.5]
		{\centering
			\includesvg[angle=90,width=0.55\linewidth]{Nonslide/Chaban_4}}
		}
\hfil
	\subfloat[Калмыцкий.\\Второй вариант]{\label{ris:Chaban_4}
	\tcbox[enhanced jigsaw,colframe=black,opacityframe=0.5,opacityback=0.5]
		{\centering
			\includesvg[angle=90,width=0.55\linewidth]{Nonslide/Chaban_5}}
		}
\end{figure}

\begin{figure}[H]\centering
	\subfloat[Эскимосский.\\Третий вариант]{\label{ris:Chaban_5}
	\tcbox[enhanced jigsaw,colframe=black,opacityframe=0.5,opacityback=0.5]
		{\centering
			\includesvg[angle=90,width=0.55\linewidth]{Nonslide/Chaban_1}}
		}
\hfil
	\subfloat[Калмыцкий.\\Третий вариант]{\label{ris:Chaban_6}
	\tcbox[enhanced jigsaw,colframe=black,opacityframe=0.5,opacityback=0.5]
		{\centering
			\includesvg[angle=90,width=0.55\linewidth]{Nonslide/Chaban_3}}
		}
	\caption{Чабанский узел.}\label{ris:Chaban}
\end{figure}

\subsection{Якутский узел}

По якутски - \enquote{Туомтуу баайыы}. Другое название - Распускной. Интереснейшее исследование есть \href{http://ilin-yakutsk.narod.ru/2002-4/savinov.htm}{в журнале \enquote{Илин} за 2002г}. Там, в частности, описаны способы вязки и отличия Калмыцкого и Якутского узлов.

Вновь \enquote{открыт} \href{http://www.muzel.ru/article/za/skifovsky.htm}{Юрием Елизаровым}, где он называет его Скифовский узел.

\begin{figure}[H]\centering
	\subfloat[Результат]{\label{ris:Skif_1}
	\tcbox[enhanced jigsaw,colframe=black,opacityframe=0.5,opacityback=0.5]
		{\centering
			\includesvg[width=0.55\linewidth]{Nonslide/Skif}}
		}
\end{figure}

На рисунках ниже красными кружками отмечены единственные пересечения, которые отличают Якутский (Скифовский) от Калмыцкого и Беседочного узлов.

\begin{figure}[H]\centering
	\subfloat[Точка отличия от\\Калмыцкого узла]{\label{ris:Skif_2}
	\tcbox[enhanced jigsaw,colframe=black,opacityframe=0.5,opacityback=0.5]
		{\centering
			\includesvg[width=0.55\linewidth]{Nonslide/Skif_0}}
		}
\end{figure}

\begin{figure}[H]\centering
	\subfloat[Точка отличия от\\Беседочного узла]{\label{ris:Skif_3}
	\tcbox[enhanced jigsaw,colframe=black,opacityframe=0.5,opacityback=0.5]
		{\centering
			\includesvg[width=0.55\linewidth]{Nonslide/Skif_00}}
		}
	\caption{Якутский узел.}\label{ris:Skif}
\end{figure}

\subsection{Простой Скифовский узел}

Тоже самое, но без возврата ходового конца обратно для быстрого развязывания.

\begin{figure}[H]\centering
	\setcounter{subfigure}{0}
	\addtocounter{figure}{1}
	\begin{minipage}{1\linewidth}
		\begin{center}
			\tcbox[enhanced jigsaw,colframe=black,opacityframe=0.5,opacityback=0.5]
			{\centering{\includesvg[width=0.55\linewidth]{Nonslide/Skif_2}}}
		\end{center}
	\end{minipage}
	\addtocounter{figure}{-1}
	\caption{Простой Скифовский узел.}
	\label{ris:Skif_simpl}
\end{figure}

\subsection{Петля из плоского узла}

Carrick Loop. Плоский узел, но в измененном по способу приложения нагрузки виде. Не скользит, не ползет, не затягивается, не зажимает свободную петлю, не позволяет затянуться основной петле. Работает на скользких, мокрых и тонких веревках. На толстых веревках затягивается слабо.

\begin{figure}[H]\centering
	\subfloat[Первый вариант]{\label{ris:Carrick_Loop_1}
	\tcbox[enhanced jigsaw,colframe=black,opacityframe=0.5,opacityback=0.5]
		{\centering
			\includesvg[width=0.55\linewidth]{Nonslide/Carrick_Loop}}
		}
\end{figure}

\begin{figure}[H]\centering
	\subfloat[Второй вариант]{\label{ris:Carrick_Loop_2}
	\tcbox[enhanced jigsaw,colframe=black,opacityframe=0.5,opacityback=0.5]
		{\centering
			\includesvg[angle=180,width=0.55\linewidth]{Nonslide/From_plosky_knot}}
		}
	\caption{Петля из плоского узла.}\label{ris:Carrick_Loop}
\end{figure}
