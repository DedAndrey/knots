\section{Бухты}

Для связывания веревки в бухты можно использовать Fool's Bowline (рис.~\ref{ris:Fools_Bowline}), Португальский беседочный (рис.~\ref{ris:Portuguese_bowline}) или Португальский беседочный 2 (рис.~\ref{ris:Portuguese_bowline_2}).

\subsection{Doughnut}

Пончик, Бублик.

\begin{figure}[H]\centering
	\begin{minipage}{1\linewidth}
		\begin{center}
			\tcbox[enhanced jigsaw,colframe=black,opacityframe=0.5,opacityback=0.5]
			{\centering{\includesvg[width=0.6\linewidth]{Utolsh/Doughnut}}}
		\end{center}
	\end{minipage}
\caption{Doughnut.}
\label{ris:Doughnut}
\end{figure}

\subsection{Heaving Line}

\begin{figure}[H]\centering
	\subfloat[Завязывание)]{\label{ris:Heaving_Line_1}
	\tcbox[enhanced jigsaw,colframe=black,opacityframe=0.5,opacityback=0.5]
		{\centering
			\includesvg[width=0.5\linewidth]{Buhty/Heaving_Line}}
		}
\end{figure}

\begin{figure}[H]\centering
	\subfloat[Результат]{\label{ris:Heaving_Line_2}
	\tcbox[enhanced jigsaw,colframe=black,opacityframe=0.5,opacityback=0.5]
		{\centering
			\includesvg[width=0.65\linewidth]{Buhty/Heaving_Line_1}}
		}
	\caption{Heaving Line.}\label{ris:Heaving_Line}
\end{figure}

\subsection{Heaving Line Knot with coil}

Легость на основе бухты Heaving Line (рис.~\ref{ris:Heaving_Line}).

\begin{figure}[H]\centering
	\begin{minipage}{1\linewidth}
		\begin{center}
			\tcbox[enhanced jigsaw,colframe=black,opacityframe=0.5,opacityback=0.5]
			{\centering{\includesvg[width=0.65\linewidth]{Utolsh/Heaving_Line_2}}}
		\end{center}
	\end{minipage}
\caption{Heaving Line Knot with coil.}
\label{ris:Heaving_Line_with_coil}
\end{figure}

\subsection{Doughnut 2}

\begin{figure}[H]\centering
	\subfloat[Завязывание)]{\label{ris:Doughnut_2_1}
	\tcbox[enhanced jigsaw,colframe=black,opacityframe=0.5,opacityback=0.5]
		{\centering
			\includesvg[width=0.55\linewidth]{Utolsh/Doughnut_1}}
		}
\end{figure}

\begin{figure}[H]\centering
	\subfloat[Результат]{\label{ris:Doughnut_2_2}
	\tcbox[enhanced jigsaw,colframe=black,opacityframe=0.5,opacityback=0.5]
		{\centering
			\includesvg[width=0.55\linewidth]{Utolsh/Doughnut_2}}
		}
	\caption{Doughnut 2.}\label{ris:Doughnut_2}
\end{figure}

\subsection{Martha’s Vineyard Boat Heaving Line Knot}

\begin{figure}[H]\centering
	\subfloat[Завязывание]{\label{ris:Martha_Vineyard_1}
	\tcbox[enhanced jigsaw,colframe=black,opacityframe=0.5,opacityback=0.5]
		{\centering
			\includesvg[width=1\linewidth]{Buhty/Martha_Vineyard}}
		}
\end{figure}

\begin{figure}[H]\centering
	\subfloat[Результат]{\label{ris:Martha_Vineyard_2}
	\tcbox[enhanced jigsaw,colframe=black,opacityframe=0.5,opacityback=0.5]
		{\centering
			\includesvg[width=1\linewidth]{Buhty/Martha_Vineyard_1}}
		}
	\caption{Martha’s Vineyard Boat Heaving Line Knot.}\label{ris:Martha_Vineyard}
\end{figure}

\subsection{Вязание бухты на основе калмыцкого узла}

% TODO “По устройству узел похож на двухпетельный беседочный узел с подвижными петлями”. Что это?

\begin{figure}[H]\centering
	\subfloat[Завязывание]{\label{ris:From_Kalmyk_1}
	\tcbox[enhanced jigsaw,colframe=black,opacityframe=0.5,opacityback=0.5]
		{\centering
			\includesvg[width=0.55\linewidth]{Buhty/From_Kalmyk}}
		}
\end{figure}

\begin{figure}[H]\centering
	\subfloat[Промежуточный результат]{\label{ris:From_Kalmyk_2}
	\tcbox[enhanced jigsaw,colframe=black,opacityframe=0.5,opacityback=0.5]
		{\centering
			\includesvg[width=0.55\linewidth]{Buhty/From_Kalmyk_1}}
		}
\end{figure}

Бухта на основе Калмыцкого узла (рис.~\ref{ris:Kalmyk}). Промежуточный результат (рис.~\ref{ris:From_Kalmyk_2}) вполне самодостаточен, но что бы иметь возможность подвесить бухту за петлю, нужно сделать дополнительный шлаг.

\begin{figure}[H]\centering
	\subfloat[Окончательный результат]{\label{ris:From_Kalmyk_3}
	\tcbox[enhanced jigsaw,colframe=black,opacityframe=0.5,opacityback=0.5]
		{\centering
			\includesvg[width=0.6\linewidth]{Buhty/From_Kalmyk_2}}
		}
	\caption{Бухта на основе калмыцкого узла.}\label{ris:From_Kalmyk}
\end{figure}
