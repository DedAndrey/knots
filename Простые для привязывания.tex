\section{Простые}

Трос можно привязать к чему-нибудь или связать несколько предметов между собой.

\subsection{Single or Simple Hitch}

\begin{figure}[H]\centering
	\begin{minipage}{1\linewidth}
		\begin{center}
			\tcbox[enhanced jigsaw,colframe=black,opacityframe=0.5,opacityback=0.5]
			{\centering{\includesvg[width=0.3\linewidth]{Hitch/Single_or_Simple_Hitch}}}
		\end{center}
	\end{minipage}
\caption{Single or Simple Hitch.}
\label{ris:Single_or_Simple_Hitch}
\end{figure}

\subsection{Slippery Hitch}

\begin{figure}[H]\centering
	\subfloat[Первый вариант]{\label{ris:Slippery_Hitch_1}
	\tcbox[enhanced jigsaw,colframe=black,opacityframe=0.5,opacityback=0.5]
		{\centering
			\includesvg[width=0.3\linewidth]{Hitch/Slippery_Hitch}}
		}
\hfil
	\subfloat[Второй вариант]{\label{ris:Slippery_Hitch_2}
	\tcbox[enhanced jigsaw,colframe=black,opacityframe=0.5,opacityback=0.5]
		{\centering
			\includesvg[width=0.3\linewidth]{Hitch/Slippery_Hitch_1}}
		}
\hfil
	\subfloat[Третий вариант]{\label{ris:Slippery_Hitch_3}
	\tcbox[enhanced jigsaw,colframe=black,opacityframe=0.5,opacityback=0.5]
		{\centering
			\includesvg[width=0.3\linewidth]{Hitch/Slippery_Hitch_2}}
		}
	\caption{Slippery Hitch. Три варианта.}\label{ris:Slippery_Hitch}
\end{figure}

Slippery Pin Hitch. Вариант \ref{ris:Slippery_Hitch_1} самый надежный, \ref{ris:Slippery_Hitch_3} - легче всего развязывается.

\subsection{Slippery turn}

\begin{figure}[H]\centering
	\begin{minipage}{1\linewidth}
		\begin{center}
			\tcbox[enhanced jigsaw,colframe=black,opacityframe=0.5,opacityback=0.5]
			{\centering{\includesvg[width=0.3\linewidth]{Hitch/Slippery_turn}}}
		\end{center}
	\end{minipage}
\caption{Slippery turn.}
\label{ris:Slippery_turn}
\end{figure}

\subsection{Single turn and Single Hitch}

\begin{figure}[H]\centering
	\begin{minipage}{1\linewidth}
		\begin{center}
			\tcbox[enhanced jigsaw,colframe=black,opacityframe=0.5,opacityback=0.5]
			{\centering{\includesvg[width=0.3\linewidth]{Hitch/Single_turn_and_Single_Hitch}}}
		\end{center}
	\end{minipage}
\caption{Single turn and Single Hitch.}
\label{ris:Single_turn_and_Single_Hitch}
\end{figure}

\subsection{Double Single Hitch}

\begin{figure}[H]\centering
	\begin{minipage}{1\linewidth}
		\begin{center}
			\tcbox[enhanced jigsaw,colframe=black,opacityframe=0.5,opacityback=0.5]
			{\centering{\includesvg[width=0.3\linewidth]{Hitch/Double_Single_Hitch}}}
		\end{center}
	\end{minipage}
\caption{Double Single Hitch.}
\label{ris:Double_Single_Hitch}
\end{figure}

\subsection{Right-hand turn}

\begin{figure}[H]\centering
	\begin{minipage}{1\linewidth}
		\begin{center}
			\tcbox[enhanced jigsaw,colframe=black,opacityframe=0.5,opacityback=0.5]
			{\centering{\includesvg[width=0.3\linewidth]{Hitch/right-hand_turn}}}
		\end{center}
	\end{minipage}
\caption{Right-hand turn.}
\label{ris:right-hand_turn}
\end{figure}

\subsection{Left-hand turn}

\begin{figure}[H]\centering
	\begin{minipage}{1\linewidth}
		\begin{center}
			\tcbox[enhanced jigsaw,colframe=black,opacityframe=0.5,opacityback=0.5]
			{\centering{\includesvg[width=0.3\linewidth]{Hitch/left-hand_turn}}}
		\end{center}
	\end{minipage}
\caption{Left-hand turn.}
\label{ris:left-hand_turn}
\end{figure}

\subsection{Make fast}

%FIXME Посмотреть!!! (contra, “Cast off”) - команды

\begin{figure}[H]\centering
	\begin{minipage}{1\linewidth}
		\begin{center}
			\tcbox[enhanced jigsaw,colframe=black,opacityframe=0.5,opacityback=0.5]
			{\centering{\includesvg[width=0.3\linewidth]{Hitch/Make_fast}}}
		\end{center}
	\end{minipage}
\caption{Make fast.}
\label{ris:Make_fast}
\end{figure}

\subsection{Double make fast}

\begin{figure}[H]\centering
	\begin{minipage}{1\linewidth}
		\begin{center}
			\tcbox[enhanced jigsaw,colframe=black,opacityframe=0.5,opacityback=0.5]
			{\centering{\includesvg[width=0.3\linewidth]{Hitch/Double_make_fast}}}
		\end{center}
	\end{minipage}
\caption{Double make fast.}
\label{ris:Double_make_fast}
\end{figure}

\subsection{Anti-Galligan Hitch}

\begin{figure}[H]\centering
	\begin{minipage}{1\linewidth}
		\begin{center}
			\tcbox[enhanced jigsaw,colframe=black,opacityframe=0.5,opacityback=0.5]
			{\centering{\includesvg[width=0.3\linewidth]{Hitch/Anti-Galligan_Hitch}}}
		\end{center}
	\end{minipage}
\caption{Anti-Galligan Hitch.}
\label{ris:Anti-Galligan_Hitch}
\end{figure}

\subsection{Фаловый узел}

\begin{figure}[H]\centering
	\begin{minipage}{1\linewidth}
		\begin{center}
			\tcbox[enhanced jigsaw,colframe=black,opacityframe=0.5,opacityback=0.5]
			{\centering{\includesvg[width=0.3\linewidth]{Hitch/Cleat_hitch}}}
		\end{center}
	\end{minipage}
\caption{Фаловый узел.}
\label{ris:Cleat_hitch}
\end{figure}

Другое название - Кнехтовый узел. По-английски - Cleat hitch.

\subsection{Кнехтовый с петелькой}

\begin{figure}[H]\centering
	\begin{minipage}{1\linewidth}
		\begin{center}
			\tcbox[enhanced jigsaw,colframe=black,opacityframe=0.5,opacityback=0.5]
			{\centering{\includesvg[width=0.3\linewidth]{Hitch/Cleat_hitch_with_bight}}}
		\end{center}
	\end{minipage}
\caption{Кнехтовый с петелькой.}
\label{ris:Cleat_hitch_with_bight}
\end{figure}

\subsection{Швартовочный узел со стопорным узлом}

\begin{figure}[H]\centering
	\begin{minipage}{1\linewidth}
		\begin{center}
			\tcbox[enhanced jigsaw,colframe=black,opacityframe=0.5,opacityback=0.5]
			{\centering{\includesvg[width=0.5\linewidth]{Hitch/Cleat_hitch_with_stop}}}
		\end{center}
	\end{minipage}
\caption{Швартовочный узел со стопорным узлом.}
\label{ris:Cleat_hitch_with_stop}
\end{figure}

\subsection{Lighterman’s Hitch}

\begin{figure}[H]\centering
	\subfloat[Завязывание]{\label{ris:Lightermans_Hitch_metod_1}
	\tcbox[enhanced jigsaw,colframe=black,opacityframe=0.5,opacityback=0.5]
		{\centering
			\includesvg[width=0.3\linewidth]{Hitch/Lightermans_Hitch}}
		}
\hfil
	\subfloat[Завязывание]{\label{ris:Lightermans_Hitch_metod_2}
	\tcbox[enhanced jigsaw,colframe=black,opacityframe=0.5,opacityback=0.5]
		{\centering
			\includesvg[width=0.3\linewidth]{Hitch/Lightermans_Hitch_1}}
		}
\hfil
	\subfloat[Первый вариант]{\label{ris:Lightermans_Hitch_1}
	\tcbox[enhanced jigsaw,colframe=black,opacityframe=0.5,opacityback=0.5]
		{\centering
			\includesvg[width=0.32\linewidth]{Hitch/Lightermans_Hitch_2}}
		}
\hfil
	\subfloat[Второй вариант]{\label{ris:Lightermans_Hitch_2}
	\tcbox[enhanced jigsaw,colframe=black,opacityframe=0.5,opacityback=0.5]
		{\centering
			\includesvg[width=0.32\linewidth]{Hitch/Lightermans_Hitch_3}}
		}
	\caption{Lighterman’s Hitch.}\label{ris:Lightermans_Hitch}
\end{figure}

\subsection{Half Hitch}

\begin{figure}[H]\centering
	\begin{minipage}{1\linewidth}
		\begin{center}
			\tcbox[enhanced jigsaw,colframe=black,opacityframe=0.5,opacityback=0.5]
			{\centering{\includesvg[width=0.35\linewidth]{Hitch/Half_Hitch}}}
		\end{center}
	\end{minipage}
\caption{Half Hitch.}
\label{ris:Half_Hitch}
\end{figure}

Простой полуштык, Самозатягивающийся узел. В данном варианте безопасен, только если маркой прихватить ходовой конец к коренному как на рис.~\ref{ris:Half_Hitch_seized}. Полуузел, является основой для вязки многих других узлов.

\subsection{Seized Half Hitch}

\begin{figure}[H]\centering
	\begin{minipage}{1\linewidth}
		\begin{center}
			\tcbox[enhanced jigsaw,colframe=black,opacityframe=0.5,opacityback=0.5]
			{\centering{\includesvg[width=0.35\linewidth]{Hitch/Half_Hitch_seized}}}
		\end{center}
	\end{minipage}
\caption{Seized Half Hitch.}
\label{ris:Half_Hitch_seized}
\end{figure}

\subsection{Half Hitched Half Hitch}

\begin{figure}[H]\centering
	\begin{minipage}{1\linewidth}
		\begin{center}
			\tcbox[enhanced jigsaw,colframe=black,opacityframe=0.5,opacityback=0.5]
			{\centering{\includesvg[width=0.35\linewidth]{Hitch/Half_Hitched_Half_Hitch}}}
		\end{center}
	\end{minipage}
\caption{Half Hitched Half Hitch.}
\label{ris:Half_Hitched_Half_Hitch}
\end{figure}

Самозатягивающийся узел с полуштыком.

\subsection{Two Half Hitches}

\begin{figure}[H]\centering
	\subfloat[Правильный]{\label{ris:Two_Half_Hitches_1}
	\tcbox[enhanced jigsaw,colframe=black,opacityframe=0.5,opacityback=0.5]
		{\centering
			\includesvg[width=0.33\linewidth]{Hitch/Two_Half_Hitches}}
		}
\hfil
	\subfloat[Перевернутый]{\label{ris:Two_Half_Hitches_2}
	\tcbox[enhanced jigsaw,colframe=black,opacityframe=0.5,opacityback=0.5]
		{\centering
			\includesvg[width=0.33\linewidth]{Hitch/Two_Half_Hitches_1}}
		}
	\caption{Two Half Hitches.}\label{ris:Two_Half_Hitches}
\end{figure}

Слева – правильно завязанный; справа – перевернутый, неправильный (Reverse Hitches). Разница в том, какой узел образуют полуузлы. Или Clove Hitch (рис.~\ref{ris:Clove_Hitch}) или Cow Hitch (рис.~\ref{ris:Cow_Hitch}). Другие названия - Простой штык, Two Half или Lighterman's Hitch (странно, Lighterman's Hitch другой).

\subsection{Slipped Half Hitch}

\begin{figure}[H]\centering
	\begin{minipage}{1\linewidth}
		\begin{center}
			\tcbox[enhanced jigsaw,colframe=black,opacityframe=0.5,opacityback=0.5]
			{\centering{\includesvg[width=0.35\linewidth]{Hitch/Slipped_Half_Hitch}}}
		\end{center}
	\end{minipage}
\caption{Slipped Half Hitch.}
\label{ris:Slipped_Half_Hitch}
\end{figure}

Он же Шлюпочный, Шлаг с петлей.

\subsection{Round Turn and Half Hitch}

\begin{figure}[H]\centering
	\begin{minipage}{1\linewidth}
		\begin{center}
			\tcbox[enhanced jigsaw,colframe=black,opacityframe=0.5,opacityback=0.5]
			{\centering{\includesvg[width=0.35\linewidth]{Hitch/Round_Turn_and_Half_Hitch}}}
		\end{center}
	\end{minipage}
\caption{Round Turn and Half Hitch.}
\label{ris:Round_Turn_and_Half_Hitch}
\end{figure}

Полуштык со шлагом.

\subsection{Round Turn and Two Half Hitches}

\begin{figure}[H]\centering
	\begin{minipage}{1\linewidth}
		\begin{center}
			\tcbox[enhanced jigsaw,colframe=black,opacityframe=0.5,opacityback=0.5]
			{\centering{\includesvg[width=0.35\linewidth]{Hitch/Round_Turn_and_Two_Half_Hitches}}}
		\end{center}
	\end{minipage}
\caption{Round Turn and Two Half Hitches.}
\label{ris:Round_Turn_and_Two_Half_Hitches}
\end{figure}

Простой штык со шлагом.

\subsection{Two Round Turns and Two Half Hitches}

\begin{figure}[H]\centering
	\begin{minipage}{1\linewidth}
		\begin{center}
			\tcbox[enhanced jigsaw,colframe=black,opacityframe=0.5,opacityback=0.5]
			{\centering{\includesvg[width=0.35\linewidth]{Hitch/Two_Round_Turns_and_Two_Half_Hitches}}}
		\end{center}
	\end{minipage}
\caption{Two Round Turns and Two Half Hitches.}
\label{ris:Two_Round_Turns_and_Two_Half_Hitches}
\end{figure}

Простой штык с двумя шлагами.

\subsection{Two round turns}

\begin{figure}[H]\centering
	\begin{minipage}{1\linewidth}
		\begin{center}
			\tcbox[enhanced jigsaw,colframe=black,opacityframe=0.5,opacityback=0.5]
			{\centering{\includesvg[width=0.35\linewidth]{Hitch/Two_round_turns}}}
		\end{center}
	\end{minipage}
\caption{Two round turns.}
\label{ris:Two_round_turns}
\end{figure}

Пенберти, Penberthy. Один из основных узлов! Наряду с Cow Hitch (рис.~\ref{ris:Cow_Hitch}), Стременем (рис.~\ref{ris:Clove_Hitch}) и Простым узлом (рис.~\ref{ris:Single_Knot}).

% TODO Если этот узел вязать вокруг более толстой веревки и сделать больше шлагов (4-5), потом связать концы, например, Шкотовым узлом, то получится Penberthy-Pierson (Valdôtain Tress) (рис.~\ref{ris:VT}). Penberthy Knot. Нарисовать! Это уже к “схватывающим”... До этого была привязка к чему-то жесткому. Теперь к основной веревке, “перилам”, то есть к относительно мягкой основе.

\subsection{Two round turns (seized)}

\begin{figure}[H]\centering
	\begin{minipage}{1\linewidth}
		\begin{center}
			\tcbox[enhanced jigsaw,colframe=black,opacityframe=0.5,opacityback=0.5]
			{\centering{\includesvg[width=0.35\linewidth]{Hitch/Two_round_turns_seized}}}
		\end{center}
	\end{minipage}
\caption{Two round turns (seized).}
\label{ris:Two_round_turns_seized}
\end{figure}

\subsection{Single Pass Hitch}

\begin{figure}[H]\centering
	\begin{minipage}{1\linewidth}
		\begin{center}
			\tcbox[enhanced jigsaw,colframe=black,opacityframe=0.5,opacityback=0.5]
			{\centering{\includesvg[width=0.35\linewidth]{Hitch/Single_Pass_Hitch}}}
		\end{center}
	\end{minipage}
\caption{Single Pass Hitch.}
\label{ris:Single_Pass_Hitch}
\end{figure}

\subsection{Fisherman’s Bend}

\begin{figure}[H]\centering
	\begin{minipage}{1\linewidth}
		\begin{center}
			\tcbox[enhanced jigsaw,colframe=black,opacityframe=0.5,opacityback=0.5]
			{\centering{\includesvg[width=0.35\linewidth]{Hitch/Fishermans_Bend}}}
		\end{center}
	\end{minipage}
\caption{Fisherman’s Bend.}
\label{ris:Fishermans_Bend}
\end{figure}

or Bucket Hitch, Gaff Topsail Halyard bend. Полуштык рыбацкий.

\subsection{Fisherman’s Bend with two hitches}

\begin{figure}[H]\centering
	\begin{minipage}{1\linewidth}
		\begin{center}
			\tcbox[enhanced jigsaw,colframe=black,opacityframe=0.5,opacityback=0.5]
			{\centering{\includesvg[width=0.35\linewidth]{Hitch/Fishermans_Bend_with_two_hitches}}}
		\end{center}
	\end{minipage}
\caption{Fisherman’s Bend with two hitches.}
\label{ris:Fishermans_Bend_with_two_hitches}
\end{figure}

\subsection{Backhanded Hitch 2}

%TODO Выяснить почему такое название!

\begin{figure}[H]\centering
	\begin{minipage}{1\linewidth}
		\begin{center}
			\tcbox[enhanced jigsaw,colframe=black,opacityframe=0.5,opacityback=0.5]
			{\centering{\includesvg[width=0.35\linewidth]{Hitch/Backhanded_Hitch_2}}}
		\end{center}
	\end{minipage}
\caption{Backhanded Hitch 2.}
\label{ris:Backhanded_Hitch_2}
\end{figure}

Обратный штык.

\subsection{Мачтовый штык}

\begin{figure}[H]\centering
	\begin{minipage}{1\linewidth}
		\begin{center}
			\tcbox[enhanced jigsaw,colframe=black,opacityframe=0.5,opacityback=0.5]
			{\centering{\includesvg[width=0.35\linewidth]{Hitch/Machtovy}}}
		\end{center}
	\end{minipage}
\caption{Мачтовый штык.}
\label{ris:Machtovy}
\end{figure}

Комбинация двух узлов – Выбленочный (рис.~\ref{ris:Clove_Hitch}) и Простой штык (рис.~\ref{ris:Two_Half_Hitches}).

\subsection{Half Hitch восьмеркой}

\begin{figure}[H]\centering
	\begin{minipage}{1\linewidth}
		\begin{center}
			\tcbox[enhanced jigsaw,colframe=black,opacityframe=0.5,opacityback=0.5]
			{\centering{\includesvg[width=0.35\linewidth]{Hitch/Half_Hitch_8}}}
		\end{center}
	\end{minipage}
\caption{Half Hitch восьмеркой.}
\label{ris:Half_Hitch_8}
\end{figure}

Почти тоже самое, что и Half Hitch на рис.~\ref{ris:Half_Hitch}, но ходовой конец обходит коренной в обратном направлении, восьмеркой.

\subsection{Figure-Eight Hitch}

\begin{figure}[H]\centering
	\begin{minipage}{1\linewidth}
		\begin{center}
			\tcbox[enhanced jigsaw,colframe=black,opacityframe=0.5,opacityback=0.5]
			{\centering{\includesvg[width=0.35\linewidth]{Hitch/Figure-Eight_Hitch}}}
		\end{center}
	\end{minipage}
\caption{Figure-Eight Hitch.}
\label{ris:Figure-Eight_Hitch}
\end{figure}

\subsection{Slipped Figure-Eight Hitch}

\begin{figure}[H]\centering
	\begin{minipage}{1\linewidth}
		\begin{center}
			\tcbox[enhanced jigsaw,colframe=black,opacityframe=0.5,opacityback=0.5]
			{\centering{\includesvg[width=0.35\linewidth]{Hitch/Slipped_Figure-Eight_Hitch}}}
		\end{center}
	\end{minipage}
\caption{Slipped Figure-Eight Hitch.}
\label{ris:Slipped_Figure-Eight_Hitch}
\end{figure}

\subsection{Еще восьмерка}

\begin{figure}[H]\centering
	\begin{minipage}{1\linewidth}
		\begin{center}
			\tcbox[enhanced jigsaw,colframe=black,opacityframe=0.5,opacityback=0.5]
			{\centering{\includesvg[width=0.35\linewidth]{Hitch/Figure-Eight_2}}}
		\end{center}
	\end{minipage}
\caption{Еще восьмерка.}
\label{ris:Figure-Eight_2}
\end{figure}

Как и предыдущая на рис.~\ref{ris:Slipped_Figure-Eight_Hitch}, Быстроразвязывающаяся вяжется так же.

\subsection{Timber Hitch}

\begin{figure}[H]\centering
	\begin{minipage}{1\linewidth}
		\begin{center}
			\tcbox[enhanced jigsaw,colframe=black,opacityframe=0.5,opacityback=0.5]
			{\centering{\includesvg[width=0.35\linewidth]{Hitch/Timber_Hitch}}}
		\end{center}
	\end{minipage}
\caption{Timber Hitch.}
\label{ris:Timber_Hitch}
\end{figure}

Lumberman’s Knot, Countryman’s Knot. Удавка (плотницкий штык), Узел Циммермана.

\subsection{Клинч из Timber Hitch}

\begin{figure}[H]\centering
	\begin{minipage}{1\linewidth}
		\begin{center}
			\tcbox[enhanced jigsaw,colframe=black,opacityframe=0.5,opacityback=0.5]
			{\centering{\includesvg[width=0.35\linewidth]{Hitch/Clinch_2}}}
		\end{center}
	\end{minipage}
\caption{Клинч из Timber Hitch.}
\label{ris:Clinch_2}
\end{figure}

\subsection{Figure-Eight Timber Hitch}

\begin{figure}[H]\centering
	\begin{minipage}{1\linewidth}
		\begin{center}
			\tcbox[enhanced jigsaw,colframe=black,opacityframe=0.5,opacityback=0.5]
			{\centering{\includesvg[width=0.35\linewidth]{Hitch/Figure-Eight_Timber_Hitch}}}
		\end{center}
	\end{minipage}
\caption{Figure-Eight Timber Hitch.}
\label{ris:Figure-Eight_Timber_Hitch}
\end{figure}

Удавка восьмеркой. В принципе, разница только в направлении обматывания коренного конца ходовым.

\subsection{Клинч}

%FIXME Шпульковый узел?

\begin{figure}[H]\centering
	\begin{minipage}{1\linewidth}
		\begin{center}
			\tcbox[enhanced jigsaw,colframe=black,opacityframe=0.5,opacityback=0.5]
			{\centering{\includesvg[width=0.35\linewidth]{Hitch/Clinch}}}
		\end{center}
	\end{minipage}
\caption{Клинч.}
\label{ris:Clinch}
\end{figure}

\subsection{Killick hitch}

\begin{figure}[H]\centering
	\begin{minipage}{1\linewidth}
		\begin{center}
			\tcbox[enhanced jigsaw,colframe=black,opacityframe=0.5,opacityback=0.5]
			{\centering{\includesvg[width=0.35\linewidth]{Hitch/Killick_hitch}}}
		\end{center}
	\end{minipage}
\caption{Killick Hitch.}
\label{ris:Killick_hitch}
\end{figure}

Удавка с полуштыками, лесной, бревенчатый (Timber Hitch and Half Hitches).

\subsection{Series of Half Hitches and stopped end}

\begin{figure}[H]\centering
	\begin{minipage}{1\linewidth}
		\begin{center}
			\tcbox[enhanced jigsaw,colframe=black,opacityframe=0.5,opacityback=0.5]
			{\centering{\includesvg[width=0.35\linewidth]{Hitch/series_of_Half_Hitches_and_stopped_end}}}
		\end{center}
	\end{minipage}
\caption{Series of Half Hitches and stopped end.}
\label{ris:series_of_Half_Hitches_and_stopped_end}
\end{figure}

\subsection{Figure-Eight Hitch and round turn}

\begin{figure}[H]\centering
	\begin{minipage}{1\linewidth}
		\begin{center}
			\tcbox[enhanced jigsaw,colframe=black,opacityframe=0.5,opacityback=0.5]
			{\centering{\includesvg[width=0.35\linewidth]{Hitch/Figure-Eight_Hitch_and_round_turn}}}
		\end{center}
	\end{minipage}
\caption{Figure-Eight Hitch and round turn.}
\label{ris:Figure-Eight_Hitch_and_round_turn}
\end{figure}

\subsection{Clove Hitch}

\begin{figure}[H]\centering
	\begin{minipage}{1\linewidth}
		\begin{center}
			\tcbox[enhanced jigsaw,colframe=black,opacityframe=0.5,opacityback=0.5]
			{\centering{\includesvg[width=0.35\linewidth]{Hitch/Clove_Hitch}}}
		\end{center}
	\end{minipage}
\caption{Clove Hitch.}
\label{ris:Clove_Hitch}
\end{figure}

Выбленочный узел, Стремя.

\subsection{Cow Hitch}

\begin{figure}[H]\centering
	\begin{minipage}{1\linewidth}
		\begin{center}
			\tcbox[enhanced jigsaw,colframe=black,opacityframe=0.5,opacityback=0.5]
			{\centering{\includesvg[width=0.35\linewidth]{Hitch/Cow_Hitch}}}
		\end{center}
	\end{minipage}
\caption{Cow Hitch.}
\label{ris:Cow_Hitch}
\end{figure}

Lanyard Hitch. Running Eye (без использования Eye). Коровий узел, Привязочный узел, Ларк, Талрепный, Тросовый талреп, Фриволите, Рымный, Коушный, Глухая петля, Глухой узел, Бирочный узел.

\subsection{Cow Hitch Variant}

\begin{figure}[H]\centering
	\begin{minipage}{1\linewidth}
		\begin{center}
			\tcbox[enhanced jigsaw,colframe=black,opacityframe=0.5,opacityback=0.5]
			{\centering{\includesvg[width=0.35\linewidth]{Hitch/Cow_Hitch_Variant}}}
		\end{center}
	\end{minipage}
\caption{Cow Hitch Variant.}
\label{ris:Cow_Hitch_Variant}
\end{figure}

\subsection{Pedigree Cow Hitch}

\begin{figure}[H]\centering
	\begin{minipage}{1\linewidth}
		\begin{center}
			\tcbox[enhanced jigsaw,colframe=black,opacityframe=0.5,opacityback=0.5]
			{\centering{\includesvg[width=0.35\linewidth]{Hitch/Pedigree_Cow_Hitch}}}
		\end{center}
	\end{minipage}
\caption{Pedigree Cow Hitch.}
\label{ris:Pedigree_Cow_Hitch}
\end{figure}

В переводе - Племенной, породистый узел.

\subsection{Pedigree Cow Hitch 2}

\begin{figure}[H]\centering
	\begin{minipage}{1\linewidth}
		\begin{center}
			\tcbox[enhanced jigsaw,colframe=black,opacityframe=0.5,opacityback=0.5]
			{\centering{\includesvg[width=0.35\linewidth]{Hitch/Pedigree_Cow_Hitch_2}}}
		\end{center}
	\end{minipage}
\caption{Pedigree Cow Hitch 2.}
\label{ris:Pedigree_Cow_Hitch_2}
\end{figure}

\subsection{Горизонт}

\begin{figure}[H]\centering
	\begin{minipage}{1\linewidth}
		\begin{center}
			\tcbox[enhanced jigsaw,colframe=black,opacityframe=0.5,opacityback=0.5]
			{\centering{\includesvg[width=0.35\linewidth]{Hitch/Horizon}}}
		\end{center}
	\end{minipage}
\caption{Горизонт.}
\label{ris:Horizon}
\end{figure}

Он же Петельный узел.

\subsection{Running Eye Hitch}

\begin{figure}[H]\centering
	\begin{minipage}{1\linewidth}
		\begin{center}
			\tcbox[enhanced jigsaw,colframe=black,opacityframe=0.5,opacityback=0.5]
			{\centering{\includesvg[width=0.45\linewidth]{Hitch/Running_Eye_Hitch}}}
		\end{center}
	\end{minipage}
\caption{Running Eye Hitch.}
\label{ris:Running_Eye_Hitch}
\end{figure}

Running Eye (рис.~\ref{ris:Running_Eye}), который использован для привязывания к опоре.

\subsection{Long Running Eye}

\begin{figure}[H]\centering
	\begin{minipage}{1\linewidth}
		\begin{center}
			\tcbox[enhanced jigsaw,colframe=black,opacityframe=0.5,opacityback=0.5]
			{\centering{\includesvg[width=0.35\linewidth]{Hitch/Long_Running_Eye}}}
		\end{center}
	\end{minipage}
\caption{Long Running Eye.}
\label{ris:Long_Running_Eye}
\end{figure}

Cow Hitch (рис.~\ref{ris:Cow_Hitch}), концы которого зафиксированы маркой.

\subsection{Strap or Bale Sling Hitch}

\begin{figure}[H]\centering
	\begin{minipage}{1\linewidth}
		\begin{center}
			\tcbox[enhanced jigsaw,colframe=black,opacityframe=0.5,opacityback=0.5]
			{\centering{\includesvg[width=0.35\linewidth]{Hitch/Bale_Sling_Hitch}}}
		\end{center}
	\end{minipage}
\caption{Strap or Bale Sling Hitch.}
\label{ris:Bale_Sling_Hitch}
\end{figure}

Тот же Cow Hitch (рис.~\ref{ris:Cow_Hitch}), но вяжется в середине, а не на конце. Самый безопасный из всех, так как не имеет концов, которые могут развязаться.

\subsection{Strap and Becket Hitch}

\begin{figure}[H]\centering
	\begin{minipage}{1\linewidth}
		\begin{center}
			\tcbox[enhanced jigsaw,colframe=black,opacityframe=0.5,opacityback=0.5]
			{\centering{\includesvg[width=0.5\linewidth]{Hitch/Strap_and_Becket_Hitch}}}
		\end{center}
	\end{minipage}
\caption{Strap and Becket Hitch.}
\label{ris:Strap_and_Becket_Hitch}
\end{figure}

\subsection{Kellig Hitch}

\begin{figure}[H]\centering
	\subfloat[Завязывание]{\label{ris:Kellig_Hitch_1}
	\tcbox[enhanced jigsaw,colframe=black,opacityframe=0.5,opacityback=0.5,height=5cm]
		{\centering
			\includesvg[width=0.3\linewidth]{Hitch/Kellig_Hitch_1}}
		}
\hfil
	\subfloat[Результат]{\label{ris:Kellig_Hitch_2}
	\tcbox[enhanced jigsaw,colframe=black,opacityframe=0.5,opacityback=0.5,height=5cm]
		{\centering
			\includesvg[width=0.3\linewidth]{Hitch/Kellig_Hitch}}
		}
	\caption{Kellig Hitch.}\label{ris:Kellig_Hitch}
\end{figure}

Другое название Slingstone Hitch.

\subsection{Double Strap Hitch}

\begin{figure}[H]\centering
	\subfloat[Первый вариант]{\label{ris:Double_Strap_Hitch_1}
	\tcbox[enhanced jigsaw,colframe=black,opacityframe=0.5,opacityback=0.5,height=5cm]
		{\centering
			\includesvg[width=0.33\linewidth]{Hitch/Double_Ring_Hitch}}
		}
\hfil
	\subfloat[Второй вариант]{\label{ris:Double_Strap_Hitch_2}
	\tcbox[enhanced jigsaw,colframe=black,opacityframe=0.5,opacityback=0.5,height=5cm]
		{\centering
			\includesvg[width=0.33\linewidth]{Hitch/Double_Strap_Hitch}}
		}
	\caption{Double Strap Hitch.}\label{ris:Double_Strap_Hitch}
\end{figure}

Он же Double Ring Hitch, Двойной Ларк.

% TODO Может, разделить на разные?

\subsection{Knotted Strap Hitch}

\begin{figure}[H]\centering
	\begin{minipage}{1\linewidth}
		\begin{center}
			\tcbox[enhanced jigsaw,colframe=black,opacityframe=0.5,opacityback=0.5]
			{\centering{\includesvg[width=0.35\linewidth]{Hitch/Knotted_Strap_Hitch}}}
		\end{center}
	\end{minipage}
\caption{Knotted Strap Hitch.}
\label{ris:Knotted_Strap_Hitch}
\end{figure}

Узел обычно не развязывают, им закрепляют клешни крабов и лобстеров.

\subsection{Гафельный узел}

\begin{figure}[H]\centering
	\begin{minipage}{1\linewidth}
		\begin{center}
			\tcbox[enhanced jigsaw,colframe=black,opacityframe=0.5,opacityback=0.5]
			{\centering{\includesvg[width=0.35\linewidth]{Hitch/Gaff_Topsail_Halyard_Bend}}}
		\end{center}
	\end{minipage}
\caption{Гафельный узел.}
\label{ris:Gaff_Topsail_Halyard_Bend}
\end{figure}

Gaff Topsail Halyard Bend.

\subsection{Studding-Sail Bend}

\begin{figure}[H]\centering
	\begin{minipage}{1\linewidth}
		\begin{center}
			\tcbox[enhanced jigsaw,colframe=black,opacityframe=0.5,opacityback=0.5]
			{\centering{\includesvg[width=0.35\linewidth]{Hitch/Studding-Sail_Bend}}}
		\end{center}
	\end{minipage}
\caption{Studding-Sail Bend.}
\label{ris:Studding-Sail_Bend}
\end{figure}

Лисельный (Лисельный фаловый) узел.

\subsection{Topsail Halyard Bend}

\begin{figure}[H]\centering
	\subfloat[Первый вариант]{\label{ris:Topsail_Halyard_Bend_1}
	\tcbox[enhanced jigsaw,colframe=black,opacityframe=0.5,opacityback=0.5]
		{\centering
			\includesvg[width=0.33\linewidth]{Hitch/Topsail_Halyard_Bend}}
		}
\hfil
	\subfloat[Второй вариант]{\label{ris:Topsail_Halyard_Bend_2}
	\tcbox[enhanced jigsaw,colframe=black,opacityframe=0.5,opacityback=0.5]
		{\centering
			\includesvg[width=0.33\linewidth]{Hitch/Topsail_Halyard_Bend_1}}
		}
	\caption{Topsail Halyard Bend.}\label{ris:Topsail_Halyard_Bend}
\end{figure}

Фаловый (Топовый фаловый) узел.

\addtocounter{HitchNoName}{1}

\subsection{Hitch без названия \arabic{HitchNoName}}

\begin{figure}[H]\centering
	\subfloat[Первый вариант]{\label{ris:Hitch_noname_3_1}
	\tcbox[enhanced jigsaw,colframe=black,opacityframe=0.5,opacityback=0.5]
		{\centering
			\includesvg[width=0.33\linewidth]{Hitch/Hitch_noname_3}}
		}
\hfil
	\subfloat[Второй вариант]{\label{ris:Hitch_noname_3_2}
	\tcbox[enhanced jigsaw,colframe=black,opacityframe=0.5,opacityback=0.5]
		{\centering
			\includesvg[width=0.33\linewidth]{Hitch/Hitch_noname_3_1}}
		}
	\caption{Hitch без названия \arabic{HitchNoName}.}\label{ris:Hitch_noname_3}
\end{figure}

\subsection{Clinging Clara}

\begin{figure}[H]\centering
	\begin{minipage}{1\linewidth}
		\begin{center}
			\tcbox[enhanced jigsaw,colframe=black,opacityframe=0.5,opacityback=0.5]
			{\centering{\includesvg[width=0.35\linewidth]{Hitch/Clinging_Clara}}}
		\end{center}
	\end{minipage}
\caption{Clinging Clara.}
\label{ris:Clinging_Clara}
\end{figure}

\addtocounter{HitchNoName}{1}

\subsection{Hitch без названия \arabic{HitchNoName}}

\begin{figure}[H]\centering
	\begin{minipage}{1\linewidth}
		\begin{center}
			\tcbox[enhanced jigsaw,colframe=black,opacityframe=0.5,opacityback=0.5]
			{\centering{\includesvg[width=0.35\linewidth]{Hitch/Hitch_noname_2}}}
		\end{center}
	\end{minipage}
\caption{Hitch без названия \arabic{HitchNoName}.}
\label{ris:Hitch_noname_2}
\end{figure}

\addtocounter{HitchNoName}{1}

\subsection{Hitch без названия \arabic{HitchNoName}}

\begin{figure}[H]\centering
	\begin{minipage}{1\linewidth}
		\begin{center}
			\tcbox[enhanced jigsaw,colframe=black,opacityframe=0.5,opacityback=0.5]
			{\centering{\includesvg[width=0.35\linewidth]{Hitch/Hitch_noname_4}}}
		\end{center}
	\end{minipage}
\caption{Hitch без названия \arabic{HitchNoName}.}
\label{ris:Hitch_noname_4}
\end{figure}

\subsection{Snug Hitch}

\begin{figure}[H]\centering
	\begin{minipage}{1\linewidth}
		\begin{center}
			\tcbox[enhanced jigsaw,colframe=black,opacityframe=0.5,opacityback=0.5]
			{\centering{\includesvg[width=0.35\linewidth]{Hitch/Snug_Hitch}}}
		\end{center}
	\end{minipage}
\caption{Snug Hitch.}
\label{ris:Snug_Hitch}
\end{figure}

\subsection{Picket-Line Hitch}

\begin{figure}[H]\centering
	\begin{minipage}{1\linewidth}
		\begin{center}
			\tcbox[enhanced jigsaw,colframe=black,opacityframe=0.5,opacityback=0.5]
			{\centering{\includesvg[width=0.35\linewidth]{Hitch/Picket-Line_Hitch}}}
		\end{center}
	\end{minipage}
\caption{Picket-Line Hitch.}
\label{ris:Picket-Line_Hitch}
\end{figure}

Пикетный узел. Начало Four-Bight Three-Lead Turk’s-Head.% TODO (рис.~\ref{ris:Turk-Head).

\subsection{Ground line hitch}

\begin{figure}[H]\centering
	\subfloat[Обычный]{\label{ris:Ground_line_hitch_1}
	\tcbox[enhanced jigsaw,colframe=black,opacityframe=0.5,opacityback=0.5,height=4.5cm]
		{\centering
			\includesvg[width=0.32\linewidth]{Hitch/Ground_line_hitch}}
		}
\hfil
	\subfloat[Скользящий]{\label{ris:Ground_line_hitch_2}
	\tcbox[enhanced jigsaw,colframe=black,opacityframe=0.5,opacityback=0.5,height=4.5cm]
		{\centering
			\includesvg[width=0.46\linewidth]{Hitch/Ground_line_hitch_0}}
		}
	\caption{Ground line hitch.}\label{ris:Ground_line_hitch}
\end{figure}

Зеркальный вариант Picket-Line Hitch (рис.~\ref{ris:Picket-Line_Hitch}).

\subsection{Щучий узел}

\begin{figure}[H]\centering
	\subfloat[Обычный]{\label{ris:Schuchy_1_1}
	\tcbox[enhanced jigsaw,colframe=black,opacityframe=0.5,opacityback=0.5]
		{\centering
			\includesvg[width=0.33\linewidth]{Hitch/Schuchy}}
		}
\end{figure}
% \vfill
\begin{figure}[H]\centering
	\subfloat[Скользящий вариант 1]{\label{ris:Schuchy_1_2}
	\tcbox[enhanced jigsaw,colframe=black,opacityframe=0.5,opacityback=0.5]
		{\centering
			\includesvg[width=0.39\linewidth]{Hitch/Schuchy_0}}
		}
\hfil
	\subfloat[Скользящий вариант 2]{\label{ris:Schuchy_1_3}
	\tcbox[enhanced jigsaw,colframe=black,opacityframe=0.5,opacityback=0.5]
		{\centering
			\includesvg[width=0.39\linewidth]{Hitch/Schuchy_0_1}}
		}
	\caption{Щучий узел.}\label{ris:Schuchy}
\end{figure}

\subsection{СССР}

\begin{figure}[H]\centering
	\begin{minipage}{1\linewidth}
		\begin{center}
			\tcbox[enhanced jigsaw,colframe=black,opacityframe=0.5,opacityback=0.5]
			{\centering{\includesvg[width=0.4\linewidth]{Hitch/SSSR}}}
		\end{center}
	\end{minipage}
\caption{СССР.}
\label{ris:SSSR}
\end{figure}

Если Щучий узел вязать с дополнительным шлагом на опоре, получится узел, по внешнему виду похожий на аббревиатуру СССР.

\subsection{По мотивам Щучьего узла}

\begin{figure}[H]\centering
	\subfloat[Обычный]{\label{ris:Schuchy_2_1}
	\tcbox[enhanced jigsaw,colframe=black,opacityframe=0.5,opacityback=0.5]
		{\centering
			\includesvg[width=0.33\linewidth]{Hitch/Schuchy_2}}
		}
\end{figure}
% \vfill
\begin{figure}[H]\centering
	\subfloat[Скользящий вариант 1]{\label{ris:Schuchy_2_2}
	\tcbox[enhanced jigsaw,colframe=black,opacityframe=0.5,opacityback=0.5,height=3cm]
		{\centering
			\includesvg[width=0.39\linewidth]{Hitch/Schuchy_2_0}}
		}
\hfil
	\subfloat[Скользящий вариант 2]{\label{ris:Schuchy_2_3}
	\tcbox[enhanced jigsaw,colframe=black,opacityframe=0.5,opacityback=0.5,height=3cm]
		{\centering
			\includesvg[width=0.39\linewidth]{Hitch/Schuchy_2_0_1}}
		}
	\caption{Узел по мотивам Щучьего узла.}\label{ris:Schuchy_2}
\end{figure}

Ходовой конец идет не снаружи узла, а внутри шлагов.

\subsection{Jam Hitch}

\begin{figure}[H]\centering
	\subfloat[Обычный]{\label{ris:Jam_Hitch_1}
	\tcbox[enhanced jigsaw,colframe=black,opacityframe=0.5,opacityback=0.5,height=3cm]
		{\centering
			\includesvg[width=0.3\linewidth]{Hitch/Jam_Hitch_1}}
		}
\hfil
	\subfloat[Скользящий вариант]{\label{ris:Jam_Hitch_2}
	\tcbox[enhanced jigsaw,colframe=black,opacityframe=0.5,opacityback=0.5,height=3cm]
		{\centering
			\includesvg[width=0.42\linewidth]{Hitch/Jam_Hitch}}
		}
	\caption{Jam Hitch.}\label{ris:Jam_Hitch}
\end{figure}

%(структурно похож на Butcher’s Knots and Binder Knots).
Реверс Rolling Hitch 1 (рис.~\ref{ris:Rolling_Hitch}).

\subsection{Magnus Hitch}

\begin{figure}[H]\centering
	\subfloat[Обычный]{\label{ris:Magnus_Hitch_1}
	\tcbox[enhanced jigsaw,colframe=black,opacityframe=0.5,opacityback=0.5,height=3cm]
		{\centering
			\includesvg[width=0.3\linewidth]{Hitch/Magnus_Hitch_0}}
		}
\hfil
	\subfloat[Скользящий вариант]{\label{ris:Magnus_Hitch_2}
	\tcbox[enhanced jigsaw,colframe=black,opacityframe=0.5,opacityback=0.5,height=3cm]
		{\centering
			\includesvg[width=0.42\linewidth]{Hitch/Magnus_Hitch}}
		}
	\caption{Magnus Hitch.}\label{ris:Magnus_Hitch}
\end{figure}

% № 1736
Схватывающий косой австрийский. Линьковая намотка - разновидность узла Прусика.

\subsection{Magnus Hitch Revers}

\begin{figure}[H]\centering
	\subfloat[Обычный]{\label{ris:Magnus_Hitch_Revers_1}
	\tcbox[enhanced jigsaw,colframe=black,opacityframe=0.5,opacityback=0.5,height=3cm]
		{\centering
			\includesvg[width=0.3\linewidth]{Hitch/Magnus_Hitch_Revers_1}}
		}
\hfil
	\subfloat[Скользящий вариант]{\label{ris:Magnus_Hitch_Revers_2}
	\tcbox[enhanced jigsaw,colframe=black,opacityframe=0.5,opacityback=0.5,height=3cm]
		{\centering
			\includesvg[width=0.42\linewidth]{Hitch/Magnus_Hitch_Revers}}
		}
	\caption{Magnus Hitch Revers.}\label{ris:Magnus_Hitch_Revers}
\end{figure}

\subsection{Magnus Hitch Revers 2}

\begin{figure}[H]\centering
	\subfloat[Обычный]{\label{ris:Magnus_Hitch_Revers_2_1}
	\tcbox[enhanced jigsaw,colframe=black,opacityframe=0.5,opacityback=0.5,height=3cm]
		{\centering
			\includesvg[width=0.3\linewidth]{Hitch/Magnus_Hitch_Revers_2_1}}
		}
\hfil
	\subfloat[Скользящий вариант]{\label{ris:Magnus_Hitch_Revers_2_2}
	\tcbox[enhanced jigsaw,colframe=black,opacityframe=0.5,opacityback=0.5,height=3cm]
		{\centering
			\includesvg[width=0.42\linewidth]{Hitch/Magnus_Hitch_Revers_2}}
		}
	\caption{Magnus Hitch Revers 2.}\label{ris:Magnus_Hitch_Revers_2}
\end{figure}

По сравнению с Magnus Hitch и Magnus Hitch Revers коренной и ходовой концы поменялись местами.

\subsection{Rolling Hitch}

\begin{figure}[H]\centering
	\subfloat[Вид спереди]{\label{ris:Rolling_Hitch_0_1}
	\tcbox[enhanced jigsaw,colframe=black,opacityframe=0.5,opacityback=0.5,height=4cm]
		{\centering
			\includesvg[width=0.32\linewidth]{Hitch/Rolling_Hitch_1}}
		}
\hfil
	\subfloat[Вид сзади]{\label{ris:Rolling_Hitch_0_2}
	\tcbox[enhanced jigsaw,colframe=black,opacityframe=0.5,opacityback=0.5,height=4cm]
		{\centering
			\includesvg[width=0.36\linewidth]{Hitch/Rolling_Hitch_1_1}}
		}
\hfil
	\subfloat[Скользящий вариант]{\label{ris:Rolling_Hitch_0_3}
	\tcbox[enhanced jigsaw,colframe=black,opacityframe=0.5,opacityback=0.5]
		{\centering
			\includesvg[width=0.42\linewidth]{Hitch/Rolling_Hitch_1_0}}
		}
	\caption{Rolling Hitch 1.}\label{ris:Rolling_Hitch}
\end{figure}

% № 1735? № 1734?
Ранее - Magnus Hitch (Magner’s or Magness) Hitch, Mooring Hitch. Задвижной штык, мачтовый штык, Выбленочный со шлагом. Не соскальзывает вниз.

\subsection{Линьковая намотка 2}

\begin{figure}[H]\centering
	\begin{minipage}{1\linewidth}
		\begin{center}
			\tcbox[enhanced jigsaw,colframe=black,opacityframe=0.5,opacityback=0.5]
			{\centering{\includesvg[width=0.45\linewidth]{Hitch/Linkovaya_namotka_2}}}
		\end{center}
	\end{minipage}
\caption{Линьковая намотка 2.}
\label{ris:Linkovaya_namotka_2}
\end{figure}

Зеркальное отражение Rolling Hitch (рис.~\ref{ris:Rolling_Hitch}).

\subsection{Rolling Hitch со шлагом}

\begin{figure}[H]\centering
	\subfloat[Завязывание]{\label{ris:Rolling_Hitch_shlag_1}
	\tcbox[enhanced jigsaw,colframe=black,opacityframe=0.5,opacityback=0.5]
		{\centering
			\includesvg[width=0.33\linewidth]{Hitch/Rolling_Hitch_1_2}}
		}
\end{figure}
% \vfill
\begin{figure}[H]\centering
	\subfloat[Шлаг влево]{\label{ris:Rolling_Hitch_shlag_2}
	\tcbox[enhanced jigsaw,colframe=black,opacityframe=0.5,opacityback=0.5,height=4.5cm]
		{\centering
			\includesvg[width=0.33\linewidth]{Hitch/Rolling_Hitch_1_2_1}}
		}
\hfil
	\subfloat[Шлаг вправо]{\label{ris:Rolling_Hitch_shlag_3}
	\tcbox[enhanced jigsaw,colframe=black,opacityframe=0.5,opacityback=0.5,height=4.5cm]
		{\centering
			\includesvg[width=0.33\linewidth]{Hitch/Rolling_Hitch_1_2_2}}
		}
	\caption{Rolling Hitch со шлагом.}\label{ris:Rolling_Hitch_shlag}
\end{figure}

\subsection{Rolling Hitch 2}

\begin{figure}[H]\centering
	\subfloat[Обычный]{\label{ris:Rolling_Hitch_2_1}
	\tcbox[enhanced jigsaw,colframe=black,opacityframe=0.5,opacityback=0.5,height=4cm]
		{\centering
			\includesvg[width=0.33\linewidth]{Hitch/Rolling_Hitch_2}}
		}
\hfil
	\subfloat[Скользящий]{\label{ris:Rolling_Hitch_2_2}
	\tcbox[enhanced jigsaw,colframe=black,opacityframe=0.5,opacityback=0.5,height=4cm]
		{\centering
			\includesvg[width=0.42\linewidth]{Hitch/Rolling_Hitch_2_0}}
		}
	\caption{Rolling Hitch 2.}\label{ris:Rolling_Hitch_2}
\end{figure}

Плоский штык.

\subsection{Видоизмененный плоский штык}

\begin{figure}[H]\centering
	\begin{minipage}{1\linewidth}
		\begin{center}
			\tcbox[enhanced jigsaw,colframe=black,opacityframe=0.5,opacityback=0.5]
			{\centering{\includesvg[width=0.45\linewidth]{Hitch/Plosky_hitch}}}
		\end{center}
	\end{minipage}
\caption{Видоизмененный плоский штык.}
\label{ris:Plosky_hitch}
\end{figure}

\subsection{Буйрепный узел}

\begin{figure}[H]\centering
	\begin{minipage}{1\linewidth}
		\begin{center}
			\tcbox[enhanced jigsaw,colframe=black,opacityframe=0.5,opacityback=0.5]
			{\centering{\includesvg[width=0.35\linewidth]{Hitch/Buoyrope}}}
		\end{center}
	\end{minipage}
\caption{Буйрепный узел.}
\label{ris:Buoyrope}
\end{figure}

Выбленочный узел (рис.~\ref{ris:Clove_Hitch}), коренной конец которого зафиксирован маркой на опоре.

\subsection{Buoy Rope Hitch}

\begin{figure}[H]\centering
	\begin{minipage}{1\linewidth}
		\begin{center}
			\tcbox[enhanced jigsaw,colframe=black,opacityframe=0.5,opacityback=0.5]
			{\centering{\includesvg[width=0.45\linewidth]{Hitch/Buoy_Rope_Hitch}}}
		\end{center}
	\end{minipage}
\caption{Buoy Rope Hitch.}
\label{ris:Buoy_Rope_Hitch}
\end{figure}

\addtocounter{HitchNoName}{1}

\subsection{Hitch без названия \arabic{HitchNoName}}

\begin{figure}[H]\centering
	\begin{minipage}{1\linewidth}
		\begin{center}
			\tcbox[enhanced jigsaw,colframe=black,opacityframe=0.5,opacityback=0.5]
			{\centering{\includesvg[width=0.4\linewidth]{Hitch/Hitch_noname_12}}}
		\end{center}
	\end{minipage}
\caption{Hitch без названия \arabic{HitchNoName}.}
\label{ris:Hitch_noname_12}
\end{figure}

\addtocounter{HitchNoName}{1}

\subsection{Hitch без названия \arabic{HitchNoName}}

\begin{figure}[H]\centering
	\subfloat[Завязывание]{\label{ris:Hitch_noname_5_1}
	\tcbox[enhanced jigsaw,colframe=black,opacityframe=0.5,opacityback=0.5,height=4.5cm]
		{\centering
			\includesvg[width=0.32\linewidth]{Hitch/Hitch_noname_5_1}}
		}
\hfil
	\subfloat[Результат]{\label{ris:Hitch_noname_5_2}
	\tcbox[enhanced jigsaw,colframe=black,opacityframe=0.5,opacityback=0.5,height=4.5cm]
		{\centering
			\includesvg[width=0.34\linewidth]{Hitch/Hitch_noname_5}}
		}
	\caption{Hitch без названия \arabic{HitchNoName}.}\label{ris:Hitch_noname_5}
\end{figure}

\addtocounter{HitchNoName}{1}

\subsection{Hitch без названия \arabic{HitchNoName}}

\begin{figure}[H]\centering
	\subfloat[Завязывание]{\label{ris:Hitch_noname_9_1}
	\tcbox[enhanced jigsaw,colframe=black,opacityframe=0.5,opacityback=0.5,height=4.5cm]
		{\centering
			\includesvg[width=0.32\linewidth]{Hitch/Hitch_noname_9_1}}
		}
\hfil
	\subfloat[Результат]{\label{ris:Hitch_noname_9_2}
	\tcbox[enhanced jigsaw,colframe=black,opacityframe=0.5,opacityback=0.5,height=4.5cm]
		{\centering
			\includesvg[width=0.34\linewidth]{Hitch/Hitch_noname_9}}
		}
	\caption{Hitch без названия \arabic{HitchNoName}.}\label{ris:Hitch_noname_9}
\end{figure}

\addtocounter{HitchNoName}{1}

\subsection{Hitch без названия \arabic{HitchNoName}}

\begin{figure}[H]\centering
	\subfloat[Завязывание]{\label{ris:Hitch_noname_6_1}
	\tcbox[enhanced jigsaw,colframe=black,opacityframe=0.5,opacityback=0.5,height=4.5cm]
		{\centering
			\includesvg[width=0.32\linewidth]{Hitch/Hitch_noname_6_1}}
		}
\hfil
	\subfloat[Результат]{\label{ris:Hitch_noname_6_2}
	\tcbox[enhanced jigsaw,colframe=black,opacityframe=0.5,opacityback=0.5,height=4.5cm]
		{\centering
			\includesvg[width=0.34\linewidth]{Hitch/Hitch_noname_6}}
		}
	\caption{Hitch без названия \arabic{HitchNoName}.}\label{ris:Hitch_noname_6}
\end{figure}

\addtocounter{HitchNoName}{1}

\subsection{Hitch без названия \arabic{HitchNoName}}

\begin{figure}[H]\centering
	\subfloat[Завязывание]{\label{ris:Hitch_noname_7_1}
	\tcbox[enhanced jigsaw,colframe=black,opacityframe=0.5,opacityback=0.5,height=4.5cm]
		{\centering
			\includesvg[width=0.32\linewidth]{Hitch/Hitch_noname_7_1}}
		}
\hfil
	\subfloat[Результат]{\label{ris:Hitch_noname_7_2}
	\tcbox[enhanced jigsaw,colframe=black,opacityframe=0.5,opacityback=0.5,height=4.5cm]
		{\centering
			\includesvg[width=0.34\linewidth]{Hitch/Hitch_noname_7}}
		}
	\caption{Hitch без названия \arabic{HitchNoName}.}\label{ris:Hitch_noname_7}
\end{figure}

\addtocounter{HitchNoName}{1}

\subsection{Hitch без названия \arabic{HitchNoName}}

\begin{figure}[H]\centering
	\begin{minipage}{1\linewidth}
		\begin{center}
			\tcbox[enhanced jigsaw,colframe=black,opacityframe=0.5,opacityback=0.5]
			{\centering{\includesvg[width=0.45\linewidth]{Hitch/Hitch_noname_8}}}
		\end{center}
	\end{minipage}
\caption{Hitch без названия \arabic{HitchNoName}.}
\label{ris:Hitch_noname_8}
\end{figure}

\addtocounter{HitchNoName}{1}

\subsection{Hitch без названия \arabic{HitchNoName}}

\begin{figure}[H]\centering
	\begin{minipage}{1\linewidth}
		\begin{center}
			\tcbox[enhanced jigsaw,colframe=black,opacityframe=0.5,opacityback=0.5]
			{\centering{\includesvg[width=0.35\linewidth]{Hitch/Hitch_noname_10}}}
		\end{center}
	\end{minipage}
\caption{Hitch без названия \arabic{HitchNoName}.}
\label{ris:Hitch_noname_10}
\end{figure}

\subsection{Ossel hitch}

\begin{figure}[H]\centering
	\begin{minipage}{1\linewidth}
		\begin{center}
			\tcbox[enhanced jigsaw,colframe=black,opacityframe=0.5,opacityback=0.5]
			{\centering{\includesvg[width=0.35\linewidth]{Hitch/Ossel_hitch}}}
		\end{center}
	\end{minipage}
\caption{Ossel hitch.}
\label{ris:Ossel_hitch}
\end{figure}

\subsection{Backhanded Hitch}

\begin{figure}[H]\centering
	\subfloat[Вид сзади]{\label{ris:Backhanded_Hitch_0_1}
	\tcbox[enhanced jigsaw,colframe=black,opacityframe=0.5,opacityback=0.5]
		{\centering
			\includesvg[width=0.33\linewidth]{Hitch/Backhanded_Hitch}}
		}
\hfil
	\subfloat[Вид спереди]{\label{ris:Backhanded_Hitch_0_2}
	\tcbox[enhanced jigsaw,colframe=black,opacityframe=0.5,opacityback=0.5]
		{\centering
			\includesvg[width=0.33\linewidth]{Hitch/Backhanded_Hitch_0}}
		}
	\caption{Backhanded Hitch.}\label{ris:Backhanded_Hitch}
\end{figure}

\addtocounter{HitchNoName}{1}

\subsection{Hitch без названия \arabic{HitchNoName}}

\begin{figure}[H]\centering
	\begin{minipage}{1\linewidth}
		\begin{center}
			\tcbox[enhanced jigsaw,colframe=black,opacityframe=0.5,opacityback=0.5]
			{\centering{\includesvg[width=0.35\linewidth]{Hitch/Backhanded_Hitch_1}}}
		\end{center}
	\end{minipage}
\caption{Hitch без названия \arabic{HitchNoName}.}
\label{ris:Backhanded_Hitch_1}
\end{figure}

Штык без названия, по мотивам Backhanded Hitch.

\addtocounter{HitchNoName}{1}

\subsection{Hitch без названия \arabic{HitchNoName}}

\begin{figure}[H]\centering
	\begin{minipage}{1\linewidth}
		\begin{center}
			\tcbox[enhanced jigsaw,colframe=black,opacityframe=0.5,opacityback=0.5]
			{\centering{\includesvg[width=0.35\linewidth]{Hitch/Backhanded_Hitch_1_1}}}
		\end{center}
	\end{minipage}
\caption{Hitch без названия \arabic{HitchNoName}.}
\label{ris:Backhanded_Hitch_motiv}
\end{figure}

Штык без названия, по мотивам Backhanded Hitch.

\subsection{Vibration-proof Hitch}

\begin{figure}[H]\centering
	\subfloat[Первый вариант]{\label{ris:Vibration-proof_Hitch_1}
	\tcbox[enhanced jigsaw,colframe=black,opacityframe=0.5,opacityback=0.5]
		{\centering
			\includesvg[width=0.33\linewidth]{Hitch/Vibration-proof_Hitch_1}}
		}
\hfil
	\subfloat[Второй вариант]{\label{ris:Vibration-proof_Hitch_2}
	\tcbox[enhanced jigsaw,colframe=black,opacityframe=0.5,opacityback=0.5]
		{\centering
			\includesvg[width=0.33\linewidth]{Hitch/Vibration-proof_Hitch}}
		}
	\caption{Vibration-proof Hitch.}\label{ris:Vibration-proof_Hitch}
\end{figure}

Узел, противостоящий вибрации.

\addtocounter{HitchNoName}{1}

\subsection{Hitch без названия \arabic{HitchNoName}}

\begin{figure}[H]\centering
	\begin{minipage}{1\linewidth}
		\begin{center}
			\tcbox[enhanced jigsaw,colframe=black,opacityframe=0.5,opacityback=0.5]
			{\centering{\includesvg[width=0.35\linewidth]{Hitch/Hitch_noname_11}}}
		\end{center}
	\end{minipage}
\caption{Hitch без названия \arabic{HitchNoName}.}
\label{ris:Hitch_noname_11}
\end{figure}

\subsection{Шахтерский узел}

\begin{figure}[H]\centering
	\begin{minipage}{1\linewidth}
		\begin{center}
			\tcbox[enhanced jigsaw,colframe=black,opacityframe=0.5,opacityback=0.5]
			{\centering{\includesvg[width=0.35\linewidth]{Hitch/Slack_Line_Hitch}}}
		\end{center}
	\end{minipage}
\caption{Шахтерский узел.}
\label{ris:Shahtersky_Hitch}
\end{figure}

По английски - Slack Line Hitch.

\addtocounter{HitchNoName}{1}

\subsection{Hitch без названия \arabic{HitchNoName}}

\begin{figure}[H]\centering
	\begin{minipage}{1\linewidth}
		\begin{center}
			\tcbox[enhanced jigsaw,colframe=black,opacityframe=0.5,opacityback=0.5]
			{\centering{\includesvg[width=0.35\linewidth]{Hitch/Hitch_noname_15}}}
		\end{center}
	\end{minipage}
\caption{Hitch без названия \arabic{HitchNoName}.}
\label{ris:Hitch_noname_15}
\end{figure}

\subsection{Качельный узел}

\begin{figure}[H]\centering
	\subfloat[Завязывание]{\label{ris:Kachelny_1}
	\tcbox[enhanced jigsaw,colframe=black,opacityframe=0.5,opacityback=0.5]
		{\centering
			\includesvg[width=0.34\linewidth]{Hitch/Kachelny_1}}
		}
\hfil
	\subfloat[Результат]{\label{ris:Kachelny_2}
	\tcbox[enhanced jigsaw,colframe=black,opacityframe=0.5,opacityback=0.5]
		{\centering
			\includesvg[width=0.34\linewidth]{Hitch/Kachelny}}
		}
	\caption{Качельный узел.}\label{ris:Kachelny}
\end{figure}

Hitch из Clove Hitch (рис.~\ref{ris:Clove_Hitch}).

\subsection{Studding-Sail Halyard Strap}

\begin{figure}[H]\centering
	\begin{minipage}{1\linewidth}
		\begin{center}
			\tcbox[enhanced jigsaw,colframe=black,opacityframe=0.5,opacityback=0.5]
			{\centering{\includesvg[width=0.35\linewidth]{Hitch/Studding-Sail_Halyard_Strap}}}
		\end{center}
	\end{minipage}
\caption{Studding-Sail Halyard Strap.}
\label{ris:Studding-Sail_Halyard_Strap}
\end{figure}

На веревке вяжут какой-нибудь узел для утолщения и просовывают в веревочную петлю. Петля и веревка обходят обвязываемую опору с разных сторон. Петля дополнительно посредине скрепляется маркой.

\subsection{Buntline Hitch}

\begin{figure}[H]\centering
	\subfloat[Первый вариант]{\label{ris:Buntline_Hitch_1}
	\tcbox[enhanced jigsaw,colframe=black,opacityframe=0.5,opacityback=0.5]
		{\centering
			\includesvg[width=0.34\linewidth]{Hitch/Buntline_Hitch}}
		}
\hfil
	\subfloat[Второй вариант]{\label{ris:Buntline_Hitch_2}
	\tcbox[enhanced jigsaw,colframe=black,opacityframe=0.5,opacityback=0.5]
		{\centering
			\includesvg[width=0.34\linewidth]{Hitch/Buntline_Hitch_1}}
		}
	\caption{Buntline Hitch.}\label{ris:Buntline_Hitch}
\end{figure}

%TODO Лисельный галсовый узел?

\enquote{Gordingstek} (нем.). Узел Бык-горденя, Бычий стопор, Штык перевернутый. По сути, это Clove Hitch (рис.~\ref{ris:Clove_Hitch}) завязанный вокруг коренного конца и образующий скользящую петлю.

На обоих рисунках один и тот же узел.

\subsection{Slipped Buntline Hitch}

\begin{figure}[H]\centering
	\begin{minipage}{1\linewidth}
		\begin{center}
			\tcbox[enhanced jigsaw,colframe=black,opacityframe=0.5,opacityback=0.5]
			{\centering{\includesvg[width=0.35\linewidth]{Hitch/Slipped_Buntline_Hitch}}}
		\end{center}
	\end{minipage}
\caption{Slipped Buntline Hitch.}
\label{ris:Slipped_Buntline_Hitch}
\end{figure}

\subsection{Lobster Buoy Hitch}

\begin{figure}[H]\centering
	\begin{minipage}{1\linewidth}
		\begin{center}
			\tcbox[enhanced jigsaw,colframe=black,opacityframe=0.5,opacityback=0.5]
			{\centering{\includesvg[width=0.35\linewidth]{Hitch/Lobster_Buoy_Hitch}}}
		\end{center}
	\end{minipage}
\caption{Lobster Buoy Hitch.}
\label{ris:Lobster_Buoy_Hitch}
\end{figure}

А это Cow Hitch (рис.~\ref{ris:Cow_Hitch}) завязанный вокруг коренного конца и образующий скользящую петлю. Заключительный шлаг ходовым концом делается со стороны опоры, в противном случае получится перевернутый Two Half Hitches (рис.~\ref{ris:Two_Half_Hitches_2}).

\subsection{Топовый шкотовый узел}

\begin{figure}[H]\centering
	\begin{minipage}{1\linewidth}
		\begin{center}
			\tcbox[enhanced jigsaw,colframe=black,opacityframe=0.5,opacityback=0.5]
			{\centering{\includesvg[width=0.35\linewidth]{Hitch/Top_shkot}}}
		\end{center}
	\end{minipage}
\caption{Топовый шкотовый узел.}
\label{ris:Top_shkot}
\end{figure}

Зеркальное отображение Buntline Hitch (рис.~\ref{ris:Buntline_Hitch_2}).

\subsection{Бегущий простой узел}

\begin{figure}[H]\centering
	\begin{minipage}{1\linewidth}
		\begin{center}
			\tcbox[enhanced jigsaw,colframe=black,opacityframe=0.5,opacityback=0.5]
			{\centering{\includesvg[width=0.6\linewidth]{Hitch/Simple_sliding}}}
		\end{center}
	\end{minipage}
\caption{Бегущий простой узел.}
\label{ris:Simple_sliding}
\end{figure}

\subsection{Slip Noose Hitch}

\begin{figure}[H]\centering
	\begin{minipage}{1\linewidth}
		\begin{center}
			\tcbox[enhanced jigsaw,colframe=black,opacityframe=0.5,opacityback=0.5]
			{\centering{\includesvg[width=0.6\linewidth]{Hitch/Slip_Noose_Hitch}}}
		\end{center}
	\end{minipage}
\caption{Slip Noose Hitch.}
\label{ris:Slip_Noose_Hitch}
\end{figure}

Быстроразвязывающийся Бегущий простой узел (рис.~\ref{ris:Simple_sliding}).

\subsection{Причальный узел}

\begin{figure}[H]\centering
	\begin{minipage}{1\linewidth}
		\begin{center}
			\tcbox[enhanced jigsaw,colframe=black,opacityframe=0.5,opacityback=0.5]
			{\centering{\includesvg[width=0.6\linewidth]{Slide/Prichalny}}}
		\end{center}
	\end{minipage}
\caption{Причальный узел.}
\label{ris:Prichalny}
\end{figure}

По английски - Mooring hitch. В качестве просто затягивающейся петли приведен как Швартовочная петля на рисунке \ref{ris:Shvartovochnaya}.

\subsection{Decorative loop Hitch}

\begin{figure}[H]\centering
	\begin{minipage}{1\linewidth}
		\begin{center}
			\tcbox[enhanced jigsaw,colframe=black,opacityframe=0.5,opacityback=0.5]
			{\centering{\includesvg[width=0.65\linewidth]{Hitch/Decorative_loop_hitch}}}
		\end{center}
	\end{minipage}
\caption{Decorative loop Hitch.}
\label{ris:Decorative_loop_hitch}
\end{figure}

Штык из Декоративной петли 2 (рис.~\ref{ris:Decorative_loop_2}, Одностороння). %TODO Добавить ссылку

\subsection{Шлюпочный узел}

\begin{figure}[H]\centering
	\begin{minipage}{1\linewidth}
		\begin{center}
			\tcbox[enhanced jigsaw,colframe=black,opacityframe=0.5,opacityback=0.5]
			{\centering{\includesvg[width=0.4\linewidth]{Hitch/Shlupochny}}}
		\end{center}
	\end{minipage}
\caption{Шлюпочный узел.}
\label{ris:Shlupochny}
\end{figure}

Другое название - Лавковый узел.

\addtocounter{HitchNoName}{1}

\subsection{Hitch без названия \arabic{HitchNoName}}

\begin{figure}[H]\centering
	\begin{minipage}{1\linewidth}
		\begin{center}
			\tcbox[enhanced jigsaw,colframe=black,opacityframe=0.5,opacityback=0.5]
			{\centering{\includesvg[width=0.4\linewidth]{Hitch/Hitch_noname}}}
		\end{center}
	\end{minipage}
\caption{Hitch Без названия \arabic{HitchNoName}.}
\label{ris:Hitch_noname}
\end{figure}

\subsection{Hitch без названия \arabic{HitchNoName}, быстроразвязывающийся вариант}

\begin{figure}[H]\centering
	\begin{minipage}{1\linewidth}
		\begin{center}
			\tcbox[enhanced jigsaw,colframe=black,opacityframe=0.5,opacityback=0.5]
			{\centering{\includesvg[width=0.4\linewidth]{Hitch/Hitch_noname_1}}}
		\end{center}
	\end{minipage}
\caption{Hitch Без названия \arabic{HitchNoName}, быстроразвязывающийся вариант.}
\label{ris:Hitch_noname_slipped}
\end{figure}

\subsection{Hitch to a double line}

\begin{figure}[H]\centering
	\begin{minipage}{1\linewidth}
		\begin{center}
			\tcbox[enhanced jigsaw,colframe=black,opacityframe=0.5,opacityback=0.5]
			{\centering{\includesvg[width=0.3\linewidth]{Hitch/Hitch_to_a_double_line}}}
		\end{center}
	\end{minipage}
\caption{Hitch to a double line.}
\label{ris:Hitch_to_a_double_line}
\end{figure}

\subsection{Single-hitched Clove Hitch}

\begin{figure}[H]\centering
	\subfloat[Первый вариант]{\label{ris:single-hitched_Clove_Hitch_1}
	\tcbox[enhanced jigsaw,colframe=black,opacityframe=0.5,opacityback=0.5,height=4cm]
		{\centering
			\includesvg[width=0.36\linewidth]{Hitch/single-hitched_Clove_Hitch}}
		}
\hfil
	\subfloat[Второй вариант]{\label{ris:single-hitched_Clove_Hitch_2}
	\tcbox[enhanced jigsaw,colframe=black,opacityframe=0.5,opacityback=0.5,height=4cm]
		{\centering
			\includesvg[width=0.32\linewidth]{Hitch/single-hitched_Clove_Hitch_1}}
		}
	\caption{Single-hitched Clove Hitch.}\label{ris:single-hitched_Clove_Hitch}
\end{figure}

Clove Hitch с одним дополнительным шлагом.

\subsection{Hitch Series}

\begin{figure}[H]\centering
	\begin{minipage}{1\linewidth}
		\begin{center}
			\tcbox[enhanced jigsaw,colframe=black,opacityframe=0.5,opacityback=0.5]
			{\centering{\includesvg[width=0.4\linewidth]{Hitch/Hitch_Series}}}
		\end{center}
	\end{minipage}
\caption{Hitch Series.}
\label{ris:Hitch_Series}
\end{figure}

Cow Hitch с двумя дополнительными шлагами.

\addtocounter{HitchNoName}{1}

\subsection{Hitch без названия \arabic{HitchNoName}}

\begin{figure}[H]\centering
	\begin{minipage}{1\linewidth}
		\begin{center}
			\tcbox[enhanced jigsaw,colframe=black,opacityframe=0.5,opacityback=0.5]
			{\centering{\includesvg[width=0.35\linewidth]{Hitch/Hitch_noname_13}}}
		\end{center}
	\end{minipage}
\caption{Hitch без названия \arabic{HitchNoName}.}
\label{ris:Hitch_noname_13}
\end{figure}

Не скользит при смене направления нагрузки.

\subsection{Camel Hitch}

\begin{figure}[H]\centering
	\begin{minipage}{1\linewidth}
		\begin{center}
			\tcbox[enhanced jigsaw,colframe=black,opacityframe=0.5,opacityback=0.5]
			{\centering{\includesvg[width=0.4\linewidth]{Hitch/Camel_Hitch}}}
		\end{center}
	\end{minipage}
\caption{Camel Hitch.}
\label{ris:Camel_Hitch}
\end{figure}

Верблюжий узел (Кэмел). Не только не скользит при смене направления нагрузки, но и можно легко развязать под нагрузкой.

\subsection{Reef  Hitch}

\begin{figure}[H]\centering
	\begin{minipage}{1\linewidth}
		\begin{center}
			\tcbox[enhanced jigsaw,colframe=black,opacityframe=0.5,opacityback=0.5]
			{\centering{\includesvg[width=0.4\linewidth]{Hitch/Reef_Pendant_Hitch}}}
		\end{center}
	\end{minipage}
\caption{Reef Pendant Hitch.}
\label{ris:Reef_Pendant_Hitch}
\end{figure}

\subsection{Arboreal Hitch}

\begin{figure}[H]\centering
	\begin{minipage}{1\linewidth}
		\begin{center}
			\tcbox[enhanced jigsaw,colframe=black,opacityframe=0.5,opacityback=0.5]
			{\centering{\includesvg[width=0.33\linewidth]{Hitch/Arboreal_Hitch}}}
		\end{center}
	\end{minipage}
\caption{Arboreal Hitch.}
\label{ris:Arboreal_Hitch}
\end{figure}

Tree Surgeon’s Knot.

\subsection{Steeplejack’s Hitch}

\begin{figure}[H]\centering
	\subfloat[Первый вариант]{\label{ris:Steeplejacks_Hitch_1}
	\tcbox[enhanced jigsaw,colframe=black,opacityframe=0.5,opacityback=0.5,height=3.5cm]
		{\centering
			\includesvg[width=0.36\linewidth]{Hitch/Steeplejacks_Hitch}}
		}
\hfil
	\subfloat[Второй вариант]{\label{ris:Steeplejacks_Hitch_2}
	\tcbox[enhanced jigsaw,colframe=black,opacityframe=0.5,opacityback=0.5,height=3.5cm]
		{\centering
			\includesvg[width=0.36\linewidth]{Hitch/Steeplejacks_Hitch_1}}
		}
	\caption{Steeplejack’s Hitch.}\label{ris:Steeplejacks_Hitch}
\end{figure}

Верхолазный узел. На один шлаг больше, чем в Rolling Hitch (соответственно, слева узел, основанный на Rolling Hitch 1, справа - на Rolling Hitch 2). Петля безопасности. см. Schwabich - узлы отличаются как Cow Hitch и Clove Hitch.

\subsection{A tail block stopped in the rigging}

\begin{figure}[H]\centering
	\subfloat[Первый вариант]{\label{ris:A_tail_block_stopped_in_the_rigging_1}
	\tcbox[enhanced jigsaw,colframe=black,opacityframe=0.5,opacityback=0.5]
		{\centering
			\includesvg[width=0.38\linewidth]{Hitch/A_tail_block_stopped_in_the_rigging}}
		}
\hfil
	\subfloat[Второй вариант]{\label{ris:A_tail_block_stopped_in_the_rigging_2}
	\tcbox[enhanced jigsaw,colframe=black,opacityframe=0.5,opacityback=0.5]
		{\centering
			\includesvg[width=0.38\linewidth]{Hitch/A_tail_block_stopped_in_the_rigging_1}}
		}
	\caption{A tail block stopped in the rigging.}\label{ris:A_tail_block_stopped_in_the_rigging}
\end{figure}

\subsection{A tail block hitched, dogged and hitched}

\begin{figure}[H]\centering
	\subfloat[Первый вариант]{\label{ris:A_tail_block_hitched_dogged_and_hitched_1}
	\tcbox[enhanced jigsaw,colframe=black,opacityframe=0.5,opacityback=0.5]
		{\centering
			\includesvg[width=0.5\linewidth]{Hitch/A_tail_block_hitched_dogged_and_hitched}}
		}
\end{figure}
% \vfill
\begin{figure}[H]\centering
	\subfloat[Второй вариант]{\label{ris:A_tail_block_hitched_dogged_and_hitched_2}
	\tcbox[enhanced jigsaw,colframe=black,opacityframe=0.5,opacityback=0.5]
		{\centering
			\includesvg[width=0.5\linewidth]{Hitch/A_tail_block_hitched_dogged_and_hitched_1}}
		}
	\caption{A tail block hitched, dogged and hitched.}\label{ris:A_tail_block_hitched_dogged_and_hitched}
\end{figure}

Стопорный узел (Stopper hitch).

\subsection{Midshipman’s Hitch}

\begin{figure}[H]\centering
	\subfloat[Обычный]{\label{ris:Midshipmans_Hitch_1}
	\tcbox[enhanced jigsaw,colframe=black,opacityframe=0.5,opacityback=0.5,height=3.5cm]
		{\centering
			\includesvg[width=0.3\linewidth]{Hitch/Midshipmans_Hitch}}
		}
\hfil
	\subfloat[Скользящий]{\label{ris:Midshipmans_Hitch_2}
	\tcbox[enhanced jigsaw,colframe=black,opacityframe=0.5,opacityback=0.5,height=3.5cm]
		{\centering
			\includesvg[width=0.42\linewidth]{Hitch/Midshipmans_Hitch_1}}
		}
	\caption{Midshipman’s Hitch.}\label{ris:Midshipmans_Hitch}
\end{figure}

Средиземноморский.

%TODO Внимательно почитать в интернете!!!

Регулируемая петля. ABOK \#  1855, p 310. Taut-line hitch (Toss, Brion (1998), The Complete Rigger's Apprentice, Camden: International Marine, pp. 54–55), Adjustable hitch, Rigger's hitch, Tent-line hitch, Tent hitch.

\subsection{Hitched Midshipman’s Hitch}

\begin{figure}[H]\centering
	\subfloat[С маркой]{\label{ris:Midshipmans_Hitch_hitched_1}
	\tcbox[enhanced jigsaw,colframe=black,opacityframe=0.5,opacityback=0.5,height=2.5cm]
		{\centering
			\includesvg[width=0.35\linewidth]{Hitch/Midshipmans_Hitch_hitched}}
		}
\hfil
	\subfloat[С маркой и петлей]{\label{ris:Midshipmans_Hitch_hitched_2}
	\tcbox[enhanced jigsaw,colframe=black,opacityframe=0.5,opacityback=0.5,height=2.5cm]
		{\centering
			\includesvg[width=0.35\linewidth]{Hitch/Midshipmans_Hitch_hitched_1}}
		}
\end{figure}
% \vfill
\begin{figure}[H]\centering
	\subfloat[Обычный]{\label{ris:Midshipmans_Hitch_hitched_3}
	\tcbox[enhanced jigsaw,colframe=black,opacityframe=0.5,opacityback=0.5,height=2.5cm]
		{\centering
			\includesvg[width=0.35\linewidth]{Hitch/Midshipmans_Hitch_hitched_2_1}}
		}
\hfil
	\subfloat[Обычный с петлей]{\label{ris:Midshipmans_Hitch_hitched_4}
	\tcbox[enhanced jigsaw,colframe=black,opacityframe=0.5,opacityback=0.5,height=2.5cm]
		{\centering
			\includesvg[width=0.35\linewidth]{Hitch/Midshipmans_Hitch_hitched_2}}
		}
	\caption{Hitched Midshipman’s Hitch.}\label{ris:Midshipmans_Hitch_hitched}
\end{figure}

\subsection{Spong Knot}

\begin{figure}[H]\centering
	\begin{minipage}{1\linewidth}
		\begin{center}
			\tcbox[enhanced jigsaw,colframe=black,opacityframe=0.5,opacityback=0.5]
			{\centering{\includesvg[width=0.4\linewidth]{Hitch/Spong_Knot}}}
		\end{center}
	\end{minipage}
\caption{Spong Knot.}
\label{ris:Spong_Knot}
\end{figure}

\subsection{Double tail block}

\begin{figure}[H]\centering
	\begin{minipage}{1\linewidth}
		\begin{center}
			\tcbox[enhanced jigsaw,colframe=black,opacityframe=0.5,opacityback=0.5]
			{\centering{\includesvg[width=0.35\linewidth]{Hitch/Double_tail_block}}}
		\end{center}
	\end{minipage}
\caption{Double tail block.}
\label{ris:Double_tail_block}
\end{figure}

На обоих концах вяжут Халф Хитчи, потом основной канат оплетается ими (в одну сторону, но с противоположными твистами). Может быть гораздо длиннее, чем на рисунке. Концы можно связать Рифовым узлом. У г-на Эшли в петлю одет гак.

\subsection{Single strap for a well pipe}

\begin{figure}[H]\centering
	\begin{minipage}{1\linewidth}
		\begin{center}
			\tcbox[enhanced jigsaw,colframe=black,opacityframe=0.5,opacityback=0.5]
			{\centering{\includesvg[width=0.45\linewidth]{Hitch/Single_strap_for_a_well_pipe}}}
		\end{center}
	\end{minipage}
\caption{Single strap for a well pipe.}
\label{ris:Single_strap_for_a_well_pipe}
\end{figure}

\subsection{Single strap to a telephone pole}

\begin{figure}[H]\centering
	\begin{minipage}{1\linewidth}
		\begin{center}
			\tcbox[enhanced jigsaw,colframe=black,opacityframe=0.5,opacityback=0.5]
			{\centering{\includesvg[width=0.5\linewidth]{Hitch/Single_strap_to_a_telephone_pole}}}
		\end{center}
	\end{minipage}
\caption{Single strap to a telephone pole.}
\label{ris:Single_strap_to_a_telephone_pole}
\end{figure}

\subsection{Prusik Knot}

\begin{figure}[H]\centering
	\begin{minipage}{1\linewidth}
		\begin{center}
			\tcbox[enhanced jigsaw,colframe=black,opacityframe=0.5,opacityback=0.5]
			{\centering{\includesvg[width=0.35\linewidth]{Hitch/Prusik_Knot}}}
		\end{center}
	\end{minipage}
\caption{Prusik Knot.}
\label{ris:Prusik_Knot}
\end{figure}

Пруссик (схватывающий узел), Prusik Knot (Prusik Knot four-coil, “четырехвитковый”), изобрел Карл Пруссик. Основывается на Cow Hitch (рис.~\ref{ris:Cow_Hitch}).

\subsection{Обратный Prusik}

\begin{figure}[H]\centering
	\begin{minipage}{1\linewidth}
		\begin{center}
			\tcbox[enhanced jigsaw,colframe=black,opacityframe=0.5,opacityback=0.5]
			{\centering{\includesvg[width=0.35\linewidth]{Hitch/Prusik_Revers}}}
		\end{center}
	\end{minipage}
\caption{Обратный Prusik.}
\label{ris:Prusik_Revers}
\end{figure}

\subsection{Taut-line Hitch}

\begin{figure}[H]\centering
	\begin{minipage}{1\linewidth}
		\begin{center}
			\tcbox[enhanced jigsaw,colframe=black,opacityframe=0.5,opacityback=0.5]
			{\centering{\includesvg[width=0.35\linewidth]{Hitch/Taut-line_Hitch}}}
		\end{center}
	\end{minipage}
\caption{Taut-line Hitch.}
\label{ris:Taut-line_Hitch}
\end{figure}

Узел, аналогичный Пруссику (рис.~\ref{ris:Prusik_Knot}), но основанный на Clove Hitch (рис.~\ref{ris:Clove_Hitch}).

\subsection{Double Prusik Knot}

\begin{figure}[H]\centering
	\begin{minipage}{1\linewidth}
		\begin{center}
			\tcbox[enhanced jigsaw,colframe=black,opacityframe=0.5,opacityback=0.5]
			{\centering{\includesvg[width=0.35\linewidth]{Hitch/Double_Prusik_Knot}}}
		\end{center}
	\end{minipage}
\caption{Double Prusik Knot.}
\label{ris:Double_Prusik_Knot}
\end{figure}

Double Prusik Knot (Prusik Knot six-coil, \enquote{шестивитковый}). Если не концы, а петля - Double strap for hoisting a spar at middle length, см. Bale Sling Hitch (рис.~\ref{ris:Bale_Sling_Hitch}).

\subsection{Двойной обратный Prusik}

\begin{figure}[H]\centering
	\begin{minipage}{1\linewidth}
		\begin{center}
			\tcbox[enhanced jigsaw,colframe=black,opacityframe=0.5,opacityback=0.5]
			{\centering{\includesvg[width=0.35\linewidth]{Hitch/Double_Revers_Prusik}}}
		\end{center}
	\end{minipage}
\caption{Двойной обратный Prusik.}
\label{ris:Double_Revers_Prusik}
\end{figure}

\subsection{Double Taut-line Hitch}

\begin{figure}[H]\centering
	\begin{minipage}{1\linewidth}
		\begin{center}
			\tcbox[enhanced jigsaw,colframe=black,opacityframe=0.5,opacityback=0.5]
			{\centering{\includesvg[width=0.35\linewidth]{Hitch/Double_Taut-line_Hitch}}}
		\end{center}
	\end{minipage}
\caption{Taut-line Hitch.}
\label{ris:Double_Taut-line_Hitch}
\end{figure}

Узел, аналогичный Double Prusik Knot (рис.~\ref{ris:Prusik_Knot}), но основанный на Clove Hitch (рис.~\ref{ris:Clove_Hitch}).

\subsection{Неравнобокий Prusik Knot}

\begin{figure}[H]\centering
	\subfloat[Классический, основан на Cow Hitch (рис.~\ref{ris:Cow_Hitch})]{\label{ris:Neravnoboky_Prusik_Knot_1}
	\tcbox[enhanced jigsaw,colframe=black,opacityframe=0.5,opacityback=0.5,height=3.5cm]
		{\centering
			\includesvg[width=0.3\linewidth]{Hitch/Neravnoboky_Prusik_Knot}}
		}
\hfil
	\subfloat[Основан на Clove Hitch (рис.~\ref{ris:Clove_Hitch})]{\label{ris:Neravnoboky_Prusik_Knot_2}
	\tcbox[enhanced jigsaw,colframe=black,opacityframe=0.5,opacityback=0.5,height=3.5cm]
		{\centering
			\includesvg[width=0.4\linewidth]{Hitch/Midshipmans_Hitch_hitched_1}}
		}
	\caption{Неравнобокий Prusik Knot.}\label{ris:Neravnoboky_Prusik_Knot}
\end{figure}

\subsection{Schwabich}

\begin{figure}[H]\centering
	\subfloat[Классический, основан на Cow Hitch (рис.~\ref{ris:Cow_Hitch})]{\label{ris:Schwabich_1}
	\tcbox[enhanced jigsaw,colframe=black,opacityframe=0.5,opacityback=0.5]
		{\centering
			\includesvg[width=0.3\linewidth]{Hitch/Schwabich}}
		}
\hfil
	\subfloat[Основан на Clove Hitch (рис.~\ref{ris:Clove_Hitch})]{\label{ris:Schwabich_2}
	\tcbox[enhanced jigsaw,colframe=black,opacityframe=0.5,opacityback=0.5]
		{\centering
			\includesvg[width=0.3\linewidth]{Hitch/Schwabich_1}}
		}
	\caption{Schwabich.}\label{ris:Schwabich}
\end{figure}

\subsection{Schwab}

\begin{figure}[H]\centering
	\subfloat[Классический, основан на Cow Hitch (рис.~\ref{ris:Cow_Hitch})]{\label{ris:Schwab_1}
	\tcbox[enhanced jigsaw,colframe=black,opacityframe=0.5,opacityback=0.5]
		{\centering
			\includesvg[width=0.33\linewidth]{Hitch/Schwab}}
		}
\hfil
	\subfloat[Основан на Clove Hitch (рис.~\ref{ris:Clove_Hitch})]{\label{ris:Schwab_2}
	\tcbox[enhanced jigsaw,colframe=black,opacityframe=0.5,opacityback=0.5]
		{\centering
			\includesvg[width=0.33\linewidth]{Hitch/Schwab_1}}
		}
	\caption{Schwab.}\label{ris:Schwab}
\end{figure}

\subsection{Prusik Around Carabiner}

\begin{figure}[H]\centering
	\begin{minipage}{1\linewidth}
		\begin{center}
			\tcbox[enhanced jigsaw,colframe=black,opacityframe=0.5,opacityback=0.5]
			{\centering{\includesvg[width=0.35\linewidth]{Hitch/Prusik_Around_Carabiner}}}
		\end{center}
	\end{minipage}
\caption{Prusik Around Carabiner.}
\label{ris:Prusik_Around_Carabiner}
\end{figure}

Карабинный узел, Схватывающий (Пруссик) с карабином.

\subsection{Carabiner Prusik}

\begin{figure}[H]\centering
	\begin{minipage}{1\linewidth}
		\begin{center}
			\tcbox[enhanced jigsaw,colframe=black,opacityframe=0.5,opacityback=0.5]
			{\centering{\includesvg[width=0.35\linewidth]{Hitch/Carabiner_Prusik}}}
		\end{center}
	\end{minipage}
\caption{Carabiner Prusik.}
\label{ris:Carabiner_Prusik}
\end{figure}

\subsection{Blake's Hitch}

\begin{figure}[H]\centering
	\subfloat[Первый вариант]{\label{ris:Blakes_Hitch_1}
	\tcbox[enhanced jigsaw,colframe=black,opacityframe=0.5,opacityback=0.5]
		{\centering
			\includesvg[width=0.33\linewidth]{Hitch/Blakes_Hitch}}
		}
\hfil
	\subfloat[Второй вариант]{\label{ris:Blakes_Hitch_2}
	\tcbox[enhanced jigsaw,colframe=black,opacityframe=0.5,opacityback=0.5]
		{\centering
			\includesvg[width=0.33\linewidth]{Hitch/Blakes_Hitch_1}}
		}
	\caption{Blake's Hitch.}\label{ris:Blakes_Hitch}
\end{figure}

Блэккнот или узел Блэйка. Гафельный узел (рис.~\ref{ris:Gaff_Topsail_Halyard_Bend}) с дополнительными шлагами.

\subsection{Nodo Bellunese}

\begin{figure}[H]\centering
	\begin{minipage}{1\linewidth}
		\begin{center}
			\tcbox[enhanced jigsaw,colframe=black,opacityframe=0.5,opacityback=0.5]
			{\centering{\includesvg[width=0.4\linewidth]{Hitch/Nodo_Bellunese}}}
		\end{center}
	\end{minipage}
\caption{Nodo Bellunese.}
\label{ris:Nodo_Bellunese}
\end{figure}

\subsection{Prohaska Knots Single Rope}

\begin{figure}[H]\centering
	\begin{minipage}{1\linewidth}
		\begin{center}
			\tcbox[enhanced jigsaw,colframe=black,opacityframe=0.5,opacityback=0.5]
			{\centering{\includesvg[width=0.4\linewidth]{Hitch/Prohaska_Knots_Single_Rope}}}
		\end{center}
	\end{minipage}
\caption{Prohaska Knots Single Rope.}
\label{ris:Prohaska_Knots_Single_Rope}
\end{figure}

Альтернатива Blake's Hitch - одинарный узел Прохаска. Лучше держит на мокрой и обледенелой веревке за счет нижней (подгибающей перильную веревку) части. Работает в широком диапазоне соотношения диаметров перильной и схватывающей веревки.

\subsection{Prohaska Knots Double Rope}

\begin{figure}[H]\centering
	\begin{minipage}{1\linewidth}
		\begin{center}
			\tcbox[enhanced jigsaw,colframe=black,opacityframe=0.5,opacityback=0.5]
			{\centering{\includesvg[width=0.5\linewidth]{Hitch/Prohaska_Knots_Double_Rope}}}
		\end{center}
	\end{minipage}
\caption{Prohaska Knots Double Rope.}
\label{ris:Prohaska_Knots_Double_Rope}
\end{figure}

\subsection{Косичка}

\begin{figure}[H]\centering
	\begin{minipage}{1\linewidth}
		\begin{center}
			\tcbox[enhanced jigsaw,colframe=black,opacityframe=0.5,opacityback=0.5]
			{\centering{\includesvg[width=0.7\linewidth]{Hitch/Kosichka}}}
		\end{center}
	\end{minipage}
\caption{Косичка.}
\label{ris:Kosichka}
\end{figure}

На рисунке концы узла схвачены маркой. Вместо этого можно их связать как угодно.

\subsection{Cross-gartering}

\begin{figure}[H]\centering
	\subfloat[Первый вариант]{\label{ris:cross-gartering_1}
	\tcbox[enhanced jigsaw,colframe=black,opacityframe=0.5,opacityback=0.5]
		{\centering
			\includesvg[width=0.5\linewidth]{Hitch/cross-gartering}}
		}
\end{figure}
% \vfill
\begin{figure}[H]\centering
	\subfloat[Второй вариант]{\label{ris:cross-gartering_2}
	\tcbox[enhanced jigsaw,colframe=black,opacityframe=0.5,opacityback=0.5]
		{\centering
			\includesvg[width=0.65\linewidth]{Hitch/cross-gartering_1}}
		}
	\caption{Cross-gartering.}\label{ris:cross-gartering}
\end{figure}

\subsection{French Prusik}

\begin{figure}[H]\centering
	\begin{minipage}{1\linewidth}
		\begin{center}
			\tcbox[enhanced jigsaw,colframe=black,opacityframe=0.5,opacityback=0.5]
			{\centering{\includesvg[width=0.45\linewidth]{Hitch/French_Prusik}}}
		\end{center}
	\end{minipage}
\caption{French Prusik.}
\label{ris:French_Prusik}
\end{figure}

\addtocounter{HitchNoName}{1}

\subsection{Hitch без названия \arabic{HitchNoName}}

\begin{figure}[H]\centering
	\begin{minipage}{1\linewidth}
		\begin{center}
			\tcbox[enhanced jigsaw,colframe=black,opacityframe=0.5,opacityback=0.5]
			{\centering{\includesvg[width=0.85\linewidth]{Hitch/Hitch_noname_14}}}
		\end{center}
	\end{minipage}
\caption{Hitch без названия \arabic{HitchNoName}.}
\label{ris:Hitch_noname_14}
\end{figure}

\subsection{Israeli French Prusik}

\begin{figure}[H]\centering
	\subfloat[Завязывание]{\label{ris:Israeli_French_Prusik_1}
	\tcbox[enhanced jigsaw,colframe=black,opacityframe=0.5,opacityback=0.5]
		{\centering
			\includesvg[width=0.6\linewidth]{Hitch/Israeli_French_Prusik_1}}
		}
\end{figure}
% \vfill
\begin{figure}[H]\centering
	\subfloat[Результат]{\label{ris:Israeli_French_Prusik_2}
	\tcbox[enhanced jigsaw,colframe=black,opacityframe=0.5,opacityback=0.5]
		{\centering
			\includesvg[width=0.6\linewidth]{Hitch/Israeli_French_Prusik}}
		}
	\caption{Israeli French Prusik.}\label{ris:Israeli_French_Prusik}
\end{figure}

\subsection{Hedden Knot}

\begin{figure}[H]\centering
	\begin{minipage}{1\linewidth}
		\begin{center}
			\tcbox[enhanced jigsaw,colframe=black,opacityframe=0.5,opacityback=0.5]
			{\centering{\includesvg[width=0.55\linewidth]{Hitch/Hedden_Knot}}}
		\end{center}
	\end{minipage}
\caption{Hedden Knot.}
\label{ris:Hedden_Knot}
\end{figure}

Kreutzklem (нем.).

\subsection{Heddon Knot}

\begin{figure}[H]\centering
	\begin{minipage}{1\linewidth}
		\begin{center}
			\tcbox[enhanced jigsaw,colframe=black,opacityframe=0.5,opacityback=0.5]
			{\centering{\includesvg[width=0.45\linewidth]{Hitch/Heddon_Knot}}}
		\end{center}
	\end{minipage}
\caption{Heddon Knot.}
\label{ris:Heddon_Knot}
\end{figure}

\subsection{Double Heddon Knot}

\begin{figure}[H]\centering
	\begin{minipage}{1\linewidth}
		\begin{center}
			\tcbox[enhanced jigsaw,colframe=black,opacityframe=0.5,opacityback=0.5]
			{\centering{\includesvg[width=0.5\linewidth]{Hitch/Double_Heddon_Knot}}}
		\end{center}
	\end{minipage}
\caption{Double Heddon Knot.}
\label{ris:Double_Heddon_Knot}
\end{figure}

\subsection{Обмоточный узел}

\begin{figure}[H]\centering
	\begin{minipage}{1\linewidth}
		\begin{center}
			\tcbox[enhanced jigsaw,colframe=black,opacityframe=0.5,opacityback=0.5]
			{\centering{\includesvg[width=0.7\linewidth]{Hitch/Obmotochny}}}
		\end{center}
	\end{minipage}
\caption{Обмоточный узел.}
\label{ris:Obmotochny}
\end{figure}

Double strap for hoisting a spar or hooking a tackle. Австрийский схватывающий (или обратный схватывающий?) узел (Узел Маршара, Французский прусик, Французский блокер, удавка французская, Клемхейст, Klemheist Knot).

\subsection{Арб}

\begin{figure}[H]\centering
	\begin{minipage}{1\linewidth}
		\begin{center}
			\tcbox[enhanced jigsaw,colframe=black,opacityframe=0.5,opacityback=0.5]
			{\centering{\includesvg[width=0.75\linewidth]{Hitch/Arb}}}
		\end{center}
	\end{minipage}
\caption{Арб.}
\label{ris:Arb}
\end{figure}

Карабинный узел Бахмана.

\subsection{Bachmann Ring Knot}

\begin{figure}[H]\centering
	\begin{minipage}{1\linewidth}
		\begin{center}
			\tcbox[enhanced jigsaw,colframe=black,opacityframe=0.5,opacityback=0.5]
			{\centering{\includesvg[width=0.75\linewidth]{Hitch/Bachmann_Ring_Knot}}}
		\end{center}
	\end{minipage}
\caption{Bachmann Ring Knot.}
\label{ris:Bachmann_Ring_Knot}
\end{figure}

По итальянски - Nodo FB-Anello или Sviluppo nodi FB.

\subsection{Nodo FB-asola}

\begin{figure}[H]\centering
	\begin{minipage}{1\linewidth}
		\begin{center}
			\tcbox[enhanced jigsaw,colframe=black,opacityframe=0.5,opacityback=0.5]
			{\centering{\includesvg[width=0.75\linewidth]{Hitch/Nodo_FB-asola}}}
		\end{center}
	\end{minipage}
\caption{Nodo FB-asola.}
\label{ris:Nodo_FB-asola}
\end{figure}

По итальянски - Nodo FB-Asola (Buttonhole), он же Nodo FB-asola semplice (Simple Buttonhole). Свободные концы можно связать. Получится петля. В нее, кстати, можно защелкнуть карабин.

\subsection{Klemheist Knot}

\begin{figure}[H]\centering
	\begin{minipage}{1\linewidth}
		\begin{center}
			\tcbox[enhanced jigsaw,colframe=black,opacityframe=0.5,opacityback=0.5]
			{\centering{\includesvg[width=0.9\linewidth]{Hitch/Klemheist_Knot}}}
		\end{center}
	\end{minipage}
\caption{Klemheist Knot.}
\label{ris:Klemheist_Knot}
\end{figure}

\subsection{Cheater Knot}

\begin{figure}[H]\centering
	\begin{minipage}{1\linewidth}
		\begin{center}
			\tcbox[enhanced jigsaw,colframe=black,opacityframe=0.5,opacityback=0.5]
			{\centering{\includesvg[width=0.55\linewidth]{Hitch/Cheater_Knot}}}
		\end{center}
	\end{minipage}
\caption{Cheater Knot.}
\label{ris:Cheater_Knot}
\end{figure}

\subsection{Snap Link Twist}

\begin{figure}[H]\centering
	\begin{minipage}{1\linewidth}
		\begin{center}
			\tcbox[enhanced jigsaw,colframe=black,opacityframe=0.5,opacityback=0.5]
			{\centering{\includesvg[width=0.7\linewidth]{Hitch/Snap_Link_Twist}}}
		\end{center}
	\end{minipage}
\caption{Snap Link Twist.}
\label{ris:Snap_Link_Twist}
\end{figure}

\subsection{Bachmann Knot}

\begin{figure}[H]\centering
	\begin{minipage}{1\linewidth}
		\begin{center}
			\tcbox[enhanced jigsaw,colframe=black,opacityframe=0.5,opacityback=0.5]
			{\centering{\includesvg[width=0.9\linewidth]{Hitch/Bachmann_Knot}}}
		\end{center}
	\end{minipage}
\caption{Bachmann Knot.}
\label{ris:Bachmann_Knot}
\end{figure}

Nodo Bachmann. Узел Бахмана (карабинный узел).

\subsection{Bachmann Knot with Turn}

\begin{figure}[H]\centering
	\begin{minipage}{1\linewidth}
		\begin{center}
			\tcbox[enhanced jigsaw,colframe=black,opacityframe=0.5,opacityback=0.5]
			{\centering{\includesvg[width=1\linewidth]{Hitch/Bachmann_Knot_with_Turn}}}
		\end{center}
	\end{minipage}
\caption{Bachmann Knot with Turn.}
\label{ris:Bachmann_Knot_with_Turn}
\end{figure}

Nodo Bachmann+. По немецки - Silverterknoten. Аналогично, свободные концы можно связать. Получится петля. И карабин в нее... Получится Nodo FB-asola completo con nodo e moschettone.

\subsection{Carabiner Knot}

\begin{figure}[H]\centering
	\begin{minipage}{1\linewidth}
		\begin{center}
			\tcbox[enhanced jigsaw,colframe=black,opacityframe=0.5,opacityback=0.5]
			{\centering{\includesvg[width=0.55\linewidth]{Hitch/Carabiner_Knot}}}
		\end{center}
	\end{minipage}
\caption{Carabiner Knot.}
\label{ris:Carabiner_Knot}
\end{figure}

Nodo Moschettone. Bachmann 1947.

\subsection{RBS Knot}

\begin{figure}[H]\centering
	\begin{minipage}{1\linewidth}
		\begin{center}
			\tcbox[enhanced jigsaw,colframe=black,opacityframe=0.5,opacityback=0.5]
			{\centering{\includesvg[width=0.6\linewidth]{Hitch/RBS_Knot}}}
		\end{center}
	\end{minipage}
\caption{RBS Knot.}
\label{ris:RBS_Knot}
\end{figure}

\subsection{Carabiner Hedden}

\begin{figure}[H]\centering
	\subfloat[Оригинальная версия]{\label{ris:Carabiner_Hedden_1}
	\tcbox[enhanced jigsaw,colframe=black,opacityframe=0.5,opacityback=0.5]
		{\centering
			\includesvg[width=0.6\linewidth]{Hitch/Carabiner_Hedden_Original_Version}}
		}
\end{figure}
% \vfill
\begin{figure}[H]\centering
	\subfloat[Модифицированная версия]{\label{ris:Carabiner_Hedden_2}
	\tcbox[enhanced jigsaw,colframe=black,opacityframe=0.5,opacityback=0.5]
		{\centering
			\includesvg[width=0.6\linewidth]{Hitch/Carabiner_Hedden_Modified_Version}}
		}
	\caption{Carabiner Hedden.}\label{ris:Carabiner_Hedden}
\end{figure}

\subsection{Bachmann Ring Knot Hitched}

Nodo FB-anello con mezzo ripasso. Bachmann Ring Knot вариант.

\begin{figure}[H]\centering
	\begin{minipage}{1\linewidth}
		\begin{center}
			\tcbox[enhanced jigsaw,colframe=black,opacityframe=0.5,opacityback=0.5]
			{\centering{\includesvg[width=0.85\linewidth]{Hitch/Bachmann_Ring_Knot_Hitched}}}
		\end{center}
	\end{minipage}
\caption{Bachmann Ring Knot Hitched.}
\label{ris:Bachmann_Ring_Knot_Hitched}
\end{figure}

\subsection{Bachmann Knot Hitched}

\begin{figure}[H]\centering
	\begin{minipage}{1\linewidth}
		\begin{center}
			\tcbox[enhanced jigsaw,colframe=black,opacityframe=0.5,opacityback=0.5]
			{\centering{\includesvg[width=1\linewidth]{Hitch/Bachmann_Knot_Hitched}}}
		\end{center}
	\end{minipage}
\caption{Bachmann Knot Hitched.}
\label{ris:Bachmann_Knot_Hitched}
\end{figure}

Nodo Bachmann+ con mezzo ripasso.

\subsection{Крестостан}

\begin{figure}[H]\centering
	\begin{minipage}{1\linewidth}
		\begin{center}
			\tcbox[enhanced jigsaw,colframe=black,opacityframe=0.5,opacityback=0.5]
			{\centering{\includesvg[width=0.4\linewidth]{Hitch/Krestostan}}}
		\end{center}
	\end{minipage}
\caption{Крестостан.}
\label{ris:Krestostan}
\end{figure}

Перекрестный самозатягивающийся узел. То же самое, что и Carabiner Hedden (рис.~\ref{ris:Carabiner_Hedden}), но без карабина.

\subsection{Martin}

\begin{figure}[H]\centering
	\begin{minipage}{1\linewidth}
		\begin{center}
			\tcbox[enhanced jigsaw,colframe=black,opacityframe=0.5,opacityback=0.5]
			{\centering{\includesvg[width=0.4\linewidth]{Hitch/Martin}}}
		\end{center}
	\end{minipage}
\caption{Martin.}
\label{ris:Martin}
\end{figure}

\subsection{Todd Kramer (TK)}

\begin{figure}[H]\centering
	\begin{minipage}{1\linewidth}
		\begin{center}
			\tcbox[enhanced jigsaw,colframe=black,opacityframe=0.5,opacityback=0.5]
			{\centering{\includesvg[width=0.45\linewidth]{Hitch/Todd_Kramer}}}
		\end{center}
	\end{minipage}
\caption{Todd Kramer (TK).}
\label{ris:Todd_Kramer}
\end{figure}

\subsection{Helical Knot}

\begin{figure}[H]\centering
	\subfloat[Первый вариант]{\label{ris:Helical_Knot_1}
	\tcbox[enhanced jigsaw,colframe=black,opacityframe=0.5,opacityback=0.5]
		{\centering
			\includesvg[width=0.5\linewidth]{Hitch/Helical_Knot}}
		}
\hfil
	\subfloat[Второй вариант]{\label{ris:Helical_Knot_2}
	\tcbox[enhanced jigsaw,colframe=black,opacityframe=0.5,opacityback=0.5]
		{\centering
			\includesvg[width=0.5\linewidth]{Hitch/Helical_Knot_1}}
		}
\end{figure}
% \vfill
\begin{figure}[H]\centering
	\subfloat[Третий вариант]{\label{ris:Helical_Knot_2}
	\tcbox[enhanced jigsaw,colframe=black,opacityframe=0.5,opacityback=0.5]
		{\centering
			\includesvg[width=0.5\linewidth]{Hitch/Helical_Knot_2}}
		}
\hfil
	\subfloat[Четвертый вариант]{\label{ris:Helical_Knot_2}
	\tcbox[enhanced jigsaw,colframe=black,opacityframe=0.5,opacityback=0.5]
		{\centering
			\includesvg[width=0.5\linewidth]{Hitch/Helical_Knot_3}}
		}
	\caption{Helical Knot.}\label{ris:Helical_Knot}
\end{figure}

Четыре варианта. У всех концы связаны Булинем (Шкотовым узлом).

\subsection{Knut}

\begin{figure}[H]\centering
	\begin{minipage}{1\linewidth}
		\begin{center}
			\tcbox[enhanced jigsaw,colframe=black,opacityframe=0.5,opacityback=0.5]
			{\centering{\includesvg[width=0.45\linewidth]{Hitch/Knut}}}
		\end{center}
	\end{minipage}
\caption{Knut.}
\label{ris:Knut}
\end{figure}

\subsection{Icicle}

\begin{figure}[H]\centering
	\begin{minipage}{1\linewidth}
		\begin{center}
			\tcbox[enhanced jigsaw,colframe=black,opacityframe=0.5,opacityback=0.5]
			{\centering{\includesvg[width=0.45\linewidth]{Hitch/Icicle}}}
		\end{center}
	\end{minipage}
\caption{Icicle.}
\label{ris:Icicle}
\end{figure}

\subsection{Olivier Peron Caillet (OPC)}

\begin{figure}[H]\centering
	\begin{minipage}{1\linewidth}
		\begin{center}
			\tcbox[enhanced jigsaw,colframe=black,opacityframe=0.5,opacityback=0.5]
			{\centering{\includesvg[width=0.45\linewidth]{Hitch/OPC}}}
		\end{center}
	\end{minipage}
\caption{Olivier Peron Caillet (OPC).}
\label{ris:OPC}
\end{figure}

\subsection{Machard}

\begin{figure}[H]\centering
	\begin{minipage}{1\linewidth}
		\begin{center}
			\tcbox[enhanced jigsaw,colframe=black,opacityframe=0.5,opacityback=0.5]
			{\centering{\includesvg[width=0.5\linewidth]{Hitch/Machard}}}
		\end{center}
	\end{minipage}
\caption{Machard.}
\label{ris:Machard}
\end{figure}

Autoblock, Французский схватывающий узел, Автоблок.

\subsection{Marlatt Knot}

\begin{figure}[H]\centering
	\begin{minipage}{1\linewidth}
		\begin{center}
			\tcbox[enhanced jigsaw,colframe=black,opacityframe=0.5,opacityback=0.5]
			{\centering{\includesvg[width=0.6\linewidth]{Hitch/Marlatt_Knot}}}
		\end{center}
	\end{minipage}
\caption{Marlatt Knot.}
\label{ris:Marlatt_Knot}
\end{figure}

\subsection{Distel Knot}

\begin{figure}[H]\centering
	\begin{minipage}{1\linewidth}
		\begin{center}
			\tcbox[enhanced jigsaw,colframe=black,opacityframe=0.5,opacityback=0.5]
			{\centering{\includesvg[width=0.5\linewidth]{Hitch/Distel_Knot}}}
		\end{center}
	\end{minipage}
\caption{Distel Knot.}
\label{ris:Distel_Knot}
\end{figure}

Узел Дистела. Узел, основанный на узле Schwab (рис.~\ref{ris:Schwab}). На концах завязаны петли и в них продет карабин. Петли могут быть завязаны по-разному, в данном случае - Дубовая петля (рис.~\ref{ris:Dubovaya_loop}).

\subsection{Valdôtain Tress (VT)}

\begin{figure}[H]\centering
	\begin{minipage}{1\linewidth}
		\begin{center}
			\tcbox[enhanced jigsaw,colframe=black,opacityframe=0.5,opacityback=0.5]
			{\centering{\includesvg[width=0.5\linewidth]{Hitch/VT}}}
		\end{center}
	\end{minipage}
\caption{Valdôtain Tress (VT).}
\label{ris:VT}
\end{figure}

\subsection{Simple Buttonhole}

\begin{figure}[H]\centering
	\begin{minipage}{1\linewidth}
		\begin{center}
			\tcbox[enhanced jigsaw,colframe=black,opacityframe=0.5,opacityback=0.5]
			{\centering{\includesvg[width=0.5\linewidth]{Hitch/Simple_Buttonhole}}}
		\end{center}
	\end{minipage}
\caption{Simple Buttonhole.}
\label{ris:Simple_Buttonhole}
\end{figure}

\subsection{Extended French Prusik Knot}

\begin{figure}[H]\centering
	\begin{minipage}{1\linewidth}
		\begin{center}
			\tcbox[enhanced jigsaw,colframe=black,opacityframe=0.5,opacityback=0.5]
			{\centering{\includesvg[width=0.95\linewidth]{Hitch/Extended_French_Prusik_Knot}}}
		\end{center}
	\end{minipage}
\caption{Extended French Prusik Knot.}
\label{ris:Extended_French_Prusik_Knot}
\end{figure}

\subsection{Cross-Lashed Sling}

\begin{figure}[H]\centering
	\subfloat[Завязывание]{\label{ris:Cross-Lashed_Sling_1}
	\tcbox[enhanced jigsaw,colframe=black,opacityframe=0.5,opacityback=0.5]
		{\centering
			\includesvg[width=0.8\linewidth]{Hitch/Cross-Lashed_Sling_2}}
		}
\hfil
	\subfloat[Завязывание]{\label{ris:Cross-Lashed_Sling_2}
	\tcbox[enhanced jigsaw,colframe=black,opacityframe=0.5,opacityback=0.5]
		{\centering
			\includesvg[width=0.8\linewidth]{Hitch/Cross-Lashed_Sling_1}}
		}
\end{figure}
% \vfill
\begin{figure}[H]\centering
	\subfloat[Результат]{\label{ris:Cross-Lashed_Sling_2}
	\tcbox[enhanced jigsaw,colframe=black,opacityframe=0.5,opacityback=0.5]
		{\centering
			\includesvg[width=0.8\linewidth]{Hitch/Cross-Lashed_Sling}}
		}
	\caption{Cross-Lashed Sling.}\label{ris:Cross-Lashed_Sling}
\end{figure}

Chi-Fi Knot.

\subsection{Mariner’s Knot}

\begin{figure}[H]\centering
	\begin{minipage}{1\linewidth}
		\begin{center}
			\tcbox[enhanced jigsaw,colframe=black,opacityframe=0.5,opacityback=0.5]
			{\centering{\includesvg[width=0.75\linewidth]{Hitch/Mariners_Knot}}}
		\end{center}
	\end{minipage}
\caption{Mariner’s Knot.}
\label{ris:Mariners_Knot}
\end{figure}

Маринер.

\subsection{Garda Knot}

\begin{figure}[H]\centering
	\subfloat[Первый вариант]{\label{ris:Garda_Knot_1}
	\tcbox[enhanced jigsaw,colframe=black,opacityframe=0.5,opacityback=0.5,height=9cm]
		{\centering
			\includesvg[width=0.3\linewidth]{Hitch/Garda_Knot}}
		}
\hfil
	\subfloat[Второй вариант]{\label{ris:Garda_Knot_2}
	\tcbox[enhanced jigsaw,colframe=black,opacityframe=0.5,opacityback=0.5,height=9cm]
		{\centering
			\includesvg[width=0.42\linewidth]{Hitch/Garda_Knot_1}}
		}
	\caption{Garda Knot.}\label{ris:Garda_Knot}
\end{figure}

Alpine Clutch. Петля Гарда, Реми. По сути - Single Hitch (рис.~\ref{ris:Single_or_Simple_Hitch}).

\subsection{UIAA}

\begin{figure}[H]\centering
	\subfloat[Завязывание]{\label{ris:UIAA_1}
	\tcbox[enhanced jigsaw,colframe=black,opacityframe=0.5,opacityback=0.5,height=8cm]
		{\centering
			\includesvg[width=0.3\linewidth]{Hitch/UIAA_1}}
		}
\hfil
	\subfloat[Результат]{\label{ris:UIAA_2}
	\tcbox[enhanced jigsaw,colframe=black,opacityframe=0.5,opacityback=0.5,height=8cm]
		{\centering
			\includesvg[width=0.3\linewidth]{Hitch/UIAA}}
		}
	\caption{UIAA.}\label{ris:UIAA}
\end{figure}

UIAA\footnote{Международный Союз Альпинистских Ассоциаций}, в практике советского альпинизма назывался \enquote{узел пожарника}. Другие названия - полустремя, узел Баумгартнера, Munter Friction Hitch.

\subsection{Обратный UIAA}

\begin{figure}[H]\centering
	\begin{minipage}{1\linewidth}
		\begin{center}
			\tcbox[enhanced jigsaw,colframe=black,opacityframe=0.5,opacityback=0.5]
			{\centering{\includesvg[width=0.45\linewidth]{Hitch/UIAA_revers}}}
		\end{center}
	\end{minipage}
\caption{Обратный UIAA.}
\label{ris:UIAA_revers}
\end{figure}

\subsection{Двойной UIAA}

\begin{figure}[H]\centering
	\begin{minipage}{1\linewidth}
		\begin{center}
			\tcbox[enhanced jigsaw,colframe=black,opacityframe=0.5,opacityback=0.5]
			{\centering{\includesvg[width=0.45\linewidth]{Hitch/Double_UIAA}}}
		\end{center}
	\end{minipage}
\caption{Двойной UIAA.}
\label{ris:Double_UIAA}
\end{figure}

Double UIAA, Double Munter Friction Hitch.

\subsection{UIAA для тонкой веревки}

\begin{figure}[H]\centering
	\begin{minipage}{1\linewidth}
		\begin{center}
			\tcbox[enhanced jigsaw,colframe=black,opacityframe=0.5,opacityback=0.5]
			{\centering{\includesvg[width=0.6\linewidth]{Hitch/UIAA_thin_rope}}}
		\end{center}
	\end{minipage}
\caption{UIAA для тонкой веревки.}
\label{ris:UIAA_thin_rope}
\end{figure}

\subsection{UIAA + два полуштыка}

\begin{figure}[H]\centering
	\begin{minipage}{1\linewidth}
		\begin{center}
			\tcbox[enhanced jigsaw,colframe=black,opacityframe=0.5,opacityback=0.5]
			{\centering{\includesvg[width=0.8\linewidth]{Hitch/UIAA_+_2_hitches}}}
		\end{center}
	\end{minipage}
\caption{UIAA + два полуштыка.}
\label{ris:UIAA_+_2_hitches}
\end{figure}

\subsection{UIAA + Мул}

%TODO Нарисовать МУЛ отдельно

\begin{figure}[H]\centering
	\subfloat[Завязывание]{\label{ris:UIAA_+_Mul_1}
	\tcbox[enhanced jigsaw,colframe=black,opacityframe=0.5,opacityback=0.5]
		{\centering
			\includesvg[width=0.7\linewidth]{Hitch/UIAA_+_Mul}}
		}
\hfil
	\subfloat[Завязывание]{\label{ris:UIAA_+_Mul_2}
	\tcbox[enhanced jigsaw,colframe=black,opacityframe=0.5,opacityback=0.5]
		{\centering
			\includesvg[width=0.7\linewidth]{Hitch/UIAA_+_Mul_1}}
		}
\end{figure}
% \vfill
\begin{figure}[H]\centering
	\subfloat[Результат]{\label{ris:UIAA_+_Mul_3}
	\tcbox[enhanced jigsaw,colframe=black,opacityframe=0.5,opacityback=0.5]
		{\centering
			\includesvg[width=0.8\linewidth]{Hitch/UIAA_+_Mul_2}}
		}
	\caption{UIAA + Мул.}\label{ris:UIAA_+_Mul}
\end{figure}
