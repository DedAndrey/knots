\section{Гачные}

Конечно же, можно использовать любую петлю.

\subsection{Hitch to the eye of a hook}

\begin{figure}[H]\centering
	\begin{minipage}{1\linewidth}
		\begin{center}
			\tcbox[enhanced jigsaw,colframe=black,opacityframe=0.5,opacityback=0.5]
			{\centering{\includesvg[width=0.3\linewidth]{Hook/Hitch_to_the_eye_of_a_hook}}}
		\end{center}
	\end{minipage}
\caption{Hitch to the eye of a hook.}
\label{ris:Hitch_to_the_eye_of_a_hook}
\end{figure}

Нужен жесткий трос, например, металлический. Очко крюка должно быть как можно меньше, в идеале - два диаметра троса.

\subsection{Blackwall Hitch}

\begin{figure}[H]\centering
	\begin{minipage}{1\linewidth}
		\begin{center}
			\tcbox[enhanced jigsaw,colframe=black,opacityframe=0.5,opacityback=0.5]
			{\centering{\includesvg[width=0.3\linewidth]{Hook/Blackwall_Hitch}}}
		\end{center}
	\end{minipage}
\caption{Blackwall Hitch.}
\label{ris:Blackwall_Hitch}
\end{figure}

Гачный узел. Он же Single or Simple Hitch (рис.~\ref{ris:Single_or_Simple_Hitch}) и Garda Knot (рис.~\ref{ris:Garda_Knot}).

\subsection{Double Blackwall Hitch}

\begin{figure}[H]\centering
	\begin{minipage}{1\linewidth}
		\begin{center}
			\tcbox[enhanced jigsaw,colframe=black,opacityframe=0.5,opacityback=0.5]
			{\centering{\includesvg[width=0.3\linewidth]{Hook/Double_Blackwall}}}
		\end{center}
	\end{minipage}
\caption{Double Blackwall Hitch.}
\label{ris:Double_Blackwall}
\end{figure}

Гачный узел со шлагом (Двойной).

\subsection{Stunner Hitch}

\begin{figure}[H]\centering
	\begin{minipage}{1\linewidth}
		\begin{center}
			\tcbox[enhanced jigsaw,colframe=black,opacityframe=0.5,opacityback=0.5]
			{\centering{\includesvg[width=0.3\linewidth]{Hook/Stunner_Hitch}}}
		\end{center}
	\end{minipage}
\caption{Stunner Hitch.}
\label{ris:Stunner_Hitch}
\end{figure}

\subsection{Bill Hitch}

\begin{figure}[H]\centering
	\subfloat[Завязывание]{\label{ris:Bill_Hitch_1}
	\tcbox[enhanced jigsaw,colframe=black,opacityframe=0.5,opacityback=0.5,height=6.5cm]
		{\centering
			\includesvg[width=0.35\linewidth]{Hook/Bill_Hitch_1}}
		}
\hfil
	\subfloat[Результат]{\label{ris:Bill_Hitch_2}
	\tcbox[enhanced jigsaw,colframe=black,opacityframe=0.5,opacityback=0.5,height=6.5cm]
		{\centering
			\includesvg[width=0.33\linewidth]{Hook/Bill_Hitch}}
		}
	\caption{Bill Hitch.}\label{ris:Bill_Hitch}
\end{figure}

Becket Hitch, Midshipman's Hitch (из Blackwall Hitch), Гачный.

\subsection{Marlingspike Hitch}

\begin{figure}[H]\centering
	\subfloat[Завязывание]{\label{ris:Marlingspike_Hitch_1}
	\tcbox[enhanced jigsaw,colframe=black,opacityframe=0.5,opacityback=0.5,height=6.5cm]
		{\centering
			\includesvg[width=0.36\linewidth]{Hook/Marlingspike_Hitch}}
		}
\hfil
	\subfloat[Результат]{\label{ris:Marlingspike_Hitch_2}
	\tcbox[enhanced jigsaw,colframe=black,opacityframe=0.5,opacityback=0.5,height=6.5cm]
		{\centering
			\includesvg[width=0.25\linewidth]{Hook/Marlingspike_Hitch_1}}
		}
	\caption{Marlingspike Hitch.}\label{ris:Marlingspike_Hitch}
\end{figure}

Сваечный узел. Из него получается Farmer’s Noose Hitch %TODO (рис.~\ref{ris:Far...}?

\subsection{Single Hook Hitch}

\begin{figure}[H]\centering
	\begin{minipage}{1\linewidth}
		\begin{center}
			\tcbox[enhanced jigsaw,colframe=black,opacityframe=0.5,opacityback=0.5]
			{\centering{\includesvg[width=0.3\linewidth]{Hook/Single_Hook_Hitch}}}
		\end{center}
	\end{minipage}
\caption{Single Hook Hitch.}
\label{ris:Single_Hook_Hitch}
\end{figure}

\subsection{Гачный узел}

\begin{figure}[H]\centering
	\subfloat[Завязывание]{\label{ris:Gachny_1}
	\tcbox[enhanced jigsaw,colframe=black,opacityframe=0.5,opacityback=0.5,height=4cm]
		{\centering
			\includesvg[width=0.36\linewidth]{Hook/Gachny_2}}
		}
\hfil
	\subfloat[Завязывание]{\label{ris:Gachny_2}
	\tcbox[enhanced jigsaw,colframe=black,opacityframe=0.5,opacityback=0.5,height=4cm]
		{\centering
			\includesvg[width=0.36\linewidth]{Hook/Gachny_1}}
		}
\hfil
	\subfloat[Результат]{\label{ris:Gachny_3}
	\tcbox[enhanced jigsaw,colframe=black,opacityframe=0.5,opacityback=0.5]
		{\centering
			\includesvg[width=0.25\linewidth]{Hook/Gachny}}
		}
	\caption{Гачный узел.}\label{ris:Gachny}
\end{figure}

Практически точная копия Fisherman’s Bend (рис.~\ref{ris:Fishermans_Bend}).

\subsection{Racking Hitch}

\begin{figure}[H]\centering
	\subfloat[Завязывание]{\label{ris:Racking_Hitch_1}
	\tcbox[enhanced jigsaw,colframe=black,opacityframe=0.5,opacityback=0.5,height=6cm]
		{\centering
			\includesvg[width=0.5\linewidth]{Hook/Racking_Hitch_1}}
		}
\hfil
	\subfloat[Результат]{\label{ris:Racking_Hitch_2}
	\tcbox[enhanced jigsaw,colframe=black,opacityframe=0.5,opacityback=0.5,height=6cm]
		{\centering
			\includesvg[width=0.25\linewidth]{Hook/Racking_Hitch}}
		}
	\caption{Racking Hitch.}\label{ris:Racking_Hitch}
\end{figure}

\subsection{Cat’s-Paw}

\begin{figure}[H]\centering
	\subfloat[Завязывание]{\label{ris:Cats-Paw_1}
	\tcbox[enhanced jigsaw,colframe=black,opacityframe=0.5,opacityback=0.5,height=6cm]
		{\centering
			\includesvg[width=0.5\linewidth]{Hook/Cats-Paw_1}}
		}
\hfil
	\subfloat[Результат]{\label{ris:Cats-Paw_2}
	\tcbox[enhanced jigsaw,colframe=black,opacityframe=0.5,opacityback=0.5,height=6cm]
		{\centering
			\includesvg[width=0.25\linewidth]{Hook/Cats-Paw}}
		}
	\caption{Cat’s-Paw.}\label{ris:Cats-Paw}
\end{figure}

Кошачья лапа.

\subsection{Single Cat’s-Paw}

\begin{figure}[H]\centering
	\subfloat[Завязывание]{\label{ris:Single_Cats-Paw_1_1}
	\tcbox[enhanced jigsaw,colframe=black,opacityframe=0.5,opacityback=0.5]
		{\centering
			\includesvg[width=0.5\linewidth]{Hook/Single_Cats-Paw_1}}
		}
\hfil
	\subfloat[Результат]{\label{ris:Single_Cats-Paw_1_2}
	\tcbox[enhanced jigsaw,colframe=black,opacityframe=0.5,opacityback=0.5]
		{\centering
			\includesvg[width=0.4\linewidth]{Hook/Single_Cats-Paw}}
		}
	\caption{Single Cat’s-Paw.}\label{ris:Single_Cats-Paw}
\end{figure}

\subsection{Крученый Single Cat’s-Paw}

\begin{figure}[H]\centering
	\subfloat[Завязывание]{\label{ris:Single_Cats-Paw_2_1}
	\tcbox[enhanced jigsaw,colframe=black,opacityframe=0.5,opacityback=0.5]
		{\centering
			\includesvg[width=0.5\linewidth]{Hook/Single_Cats-Paw_2_1}}
		}
\hfil
	\subfloat[Завязывание]{\label{ris:Single_Cats-Paw_2_2}
	\tcbox[enhanced jigsaw,colframe=black,opacityframe=0.5,opacityback=0.5]
		{\centering
			\includesvg[width=0.5\linewidth]{Hook/Single_Cats-Paw_2}}
		}
\hfil
	\subfloat[Результат]{\label{ris:Single_Cats-Paw_3_2}
	\tcbox[enhanced jigsaw,colframe=black,opacityframe=0.5,opacityback=0.5]
		{\centering
			\includesvg[width=0.35\linewidth]{Hook/Single_Cats-Paw_2_2}}
		}
	\caption{Крученый Single Cat’s-Paw.}\label{ris:Single_Cats-Paw_2}
\end{figure}

\subsection{A selvage strap and toggle}

\begin{figure}[H]\centering
	\begin{minipage}{1\linewidth}
		\begin{center}
			\tcbox[enhanced jigsaw,colframe=black,opacityframe=0.5,opacityback=0.5]
			{\centering{\includesvg[width=0.5\linewidth]{Hook/A_selvage_strap_and_toggle}}}
		\end{center}
	\end{minipage}
\caption{A selvage strap and toggle.}
\label{ris:A_selvage_strap_and_toggle}
\end{figure}

% TODO Сделать разные веревки разными цветами

\subsection{Мышеловка}

\begin{figure}[H]\centering
	\begin{minipage}{1\linewidth}
		\begin{center}
			\tcbox[enhanced jigsaw,colframe=black,opacityframe=0.5,opacityback=0.5]
			{\centering{\includesvg[width=0.55\linewidth]{Hook/Myshelovka}}}
		\end{center}
	\end{minipage}
\caption{Мышеловка.}
\label{ris:Myshelovka}
\end{figure}

Используется на относительно тонких канатах и широких гаках.

\subsection{To shorten slings}

\begin{figure}[H]\centering
	\subfloat[Завязывание]{\label{ris:To_shorten_slings_1}
	\tcbox[enhanced jigsaw,colframe=black,opacityframe=0.5,opacityback=0.5]
		{\centering
			\includesvg[width=0.45\linewidth]{Hook/To_shorten_slings_1}}
		}
\hfil
	\subfloat[Результат]{\label{ris:To_shorten_slings_2}
	\tcbox[enhanced jigsaw,colframe=black,opacityframe=0.5,opacityback=0.5]
		{\centering
			\includesvg[width=0.45\linewidth]{Hook/To_shorten_slings}}
		}
	\caption{To shorten slings.}\label{ris:To_shorten_slings}
\end{figure}

\subsection{Crow’s Foot}

\begin{figure}[H]\centering
	\subfloat[Завязывание]{\label{ris:Crows-Foot_1}
	\tcbox[enhanced jigsaw,colframe=black,opacityframe=0.5,opacityback=0.5,height=4cm]
		{\centering
			\includesvg[width=0.37\linewidth]{Hook/Crows-Foot_2}}
		}
\hfil
	\subfloat[Завязывание]{\label{ris:Crows-Foot_2}
	\tcbox[enhanced jigsaw,colframe=black,opacityframe=0.5,opacityback=0.5,height=4cm]
		{\centering
			\includesvg[width=0.42\linewidth]{Hook/Crows-Foot_1}}
		}
\end{figure}
% \vfill
\begin{figure}[H]\centering
	\subfloat[Результат]{\label{ris:Crows-Foot_3}
	\tcbox[enhanced jigsaw,colframe=black,opacityframe=0.5,opacityback=0.5]
		{\centering
			\includesvg[width=0.22\linewidth]{Hook/Crows-Foot}}
		}
	\caption{Crow’s Foot.}\label{ris:Crows-Foot}
\end{figure}

Воронья лапа.

\subsection{Becket Hitch}

\begin{figure}[H]\centering
	\begin{minipage}{1\linewidth}
		\begin{center}
			\tcbox[enhanced jigsaw,colframe=black,opacityframe=0.5,opacityback=0.5]
			{\centering{\includesvg[width=0.55\linewidth]{Hook/Becket_Hitch}}}
		\end{center}
	\end{minipage}
\caption{Becket Hitch.}
\label{ris:Becket_Hitch}
\end{figure}

%TODO шкотовый

\subsection{Double Becket Hitch}

\begin{figure}[H]\centering
	\begin{minipage}{1\linewidth}
		\begin{center}
			\tcbox[enhanced jigsaw,colframe=black,opacityframe=0.5,opacityback=0.5]
			{\centering{\includesvg[width=0.55\linewidth]{Hook/Becket_Double_Hitch}}}
		\end{center}
	\end{minipage}
\caption{Double Becket Hitch.}
\label{ris:Becket_Double_Hitch}
\end{figure}

%TODO брамшкотовый

\subsection{Another Double Becket Hitch}

%TODO нарисовать такие узлы двумя концами

\begin{figure}[H]\centering
	\subfloat[Первый вариант]{\label{ris:Another_Double_Becket_Hitch_1}
	\tcbox[enhanced jigsaw,colframe=black,opacityframe=0.5,opacityback=0.5]
		{\centering
			\includesvg[width=0.55\linewidth]{Hook/Another_Double_Becket_Hitch}}
		}
\end{figure}
% \vfill
\begin{figure}[H]\centering
	\subfloat[Второй вариант]{\label{ris:Another_Double_Becket_Hitch_2}
	\tcbox[enhanced jigsaw,colframe=black,opacityframe=0.5,opacityback=0.5]
		{\centering
			\includesvg[width=0.55\linewidth]{Hook/Another_Double_Becket_Hitch_2}}
		}
	\caption{Another Double Becket Hitch.}\label{ris:Another_Double_Becket_Hitch}
\end{figure}

\subsection{Becket Hitch by Ohrvall}

\begin{figure}[H]\centering
	\begin{minipage}{1\linewidth}
		\begin{center}
			\tcbox[enhanced jigsaw,colframe=black,opacityframe=0.5,opacityback=0.5]
			{\centering{\includesvg[width=0.55\linewidth]{Hook/Becket_Hitch_by_Ohrvall}}}
		\end{center}
	\end{minipage}
\caption{Becket Hitch by Ohrvall.}
\label{ris:Becket_Hitch_by_Ohrvall}
\end{figure}

\subsection{Figure-Eight Hitch}

\begin{figure}[H]\centering
	\begin{minipage}{1\linewidth}
		\begin{center}
			\tcbox[enhanced jigsaw,colframe=black,opacityframe=0.5,opacityback=0.5]
			{\centering{\includesvg[width=0.55\linewidth]{Hook/Figure-Eight_Hitch_hook}}}
		\end{center}
	\end{minipage}
\caption{Figure-Eight Hitch.}
\label{ris:Figure-Eight_Hitch_hook}
\end{figure}

%FIXME (see Leader Loop Hitches)

\subsection{Temporary hitch}

\begin{figure}[H]\centering
	\begin{minipage}{1\linewidth}
		\begin{center}
			\tcbox[enhanced jigsaw,colframe=black,opacityframe=0.5,opacityback=0.5]
			{\centering{\includesvg[width=0.65\linewidth]{Hook/Temporary_hitch}}}
		\end{center}
	\end{minipage}
\caption{Temporary hitch.}
\label{ris:Temporary_hitch}
\end{figure}

\addtocounter{GachnyNoName}{1}

\subsection{Гачный без названия \arabic{GachnyNoName}}

\begin{figure}[H]\centering
	\begin{minipage}{1\linewidth}
		\begin{center}
			\tcbox[enhanced jigsaw,colframe=black,opacityframe=0.5,opacityback=0.5]
			{\centering{\includesvg[width=0.65\linewidth]{Hook/Noname}}}
		\end{center}
	\end{minipage}
\caption{Гачный без названия \arabic{GachnyNoName}.}
\label{ris:Gachny_Noname_1}
\end{figure}

%FIXME (see Клинч)

\subsection{Гинцевый узел}

\begin{figure}[H]\centering
	\begin{minipage}{1\linewidth}
		\begin{center}
			\tcbox[enhanced jigsaw,colframe=black,opacityframe=0.5,opacityback=0.5]
			{\centering{\includesvg[width=0.65\linewidth]{Hook/Gincevy}}}
		\end{center}
	\end{minipage}
\caption{Гинцевый узел.}
\label{ris:Gincevy}
\end{figure}

\subsection{Toggled Bight}

\begin{figure}[H]\centering
	\begin{minipage}{1\linewidth}
		\begin{center}
			\tcbox[enhanced jigsaw,colframe=black,opacityframe=0.5,opacityback=0.5]
			{\centering{\includesvg[width=0.65\linewidth]{Hook/Toggled_Bight}}}
		\end{center}
	\end{minipage}
\caption{Toggled Bight.}
\label{ris:Toggled_Bight}
\end{figure}

\subsection{Toggled Bight 2}

\begin{figure}[H]\centering
	\begin{minipage}{1\linewidth}
		\begin{center}
			\tcbox[enhanced jigsaw,colframe=black,opacityframe=0.5,opacityback=0.5]
			{\centering{\includesvg[width=0.65\linewidth]{Hook/Toggled_Bight_2}}}
		\end{center}
	\end{minipage}
\caption{Toggled Bight 2.}
\label{ris:Toggled_Bight_2}
\end{figure}

\subsection{Toggled Bight with extra turns}

\begin{figure}[H]\centering
	\begin{minipage}{1\linewidth}
		\begin{center}
			\tcbox[enhanced jigsaw,colframe=black,opacityframe=0.5,opacityback=0.5]
			{\centering{\includesvg[width=0.75\linewidth]{Hook/Toggled_Bight_with_extra_turns}}}
		\end{center}
	\end{minipage}
\caption{Toggled Bight with extra turns.}
\label{ris:Toggled_Bight_with_extra_turns}
\end{figure}

\subsection{Slipped and Toggled Becket Hitch}

\begin{figure}[H]\centering
	\begin{minipage}{1\linewidth}
		\begin{center}
			\tcbox[enhanced jigsaw,colframe=black,opacityframe=0.5,opacityback=0.5]
			{\centering{\includesvg[width=0.65\linewidth]{Hook/Slipped_and_Toggled_Becket_Hitch}}}
		\end{center}
	\end{minipage}
\caption{Slipped and Toggled Becket Hitch.}
\label{ris:Slipped_and_Toggled_Becket_Hitch}
\end{figure}

\subsection{Bight and Eye, toggled}

\begin{figure}[H]\centering
	\begin{minipage}{1\linewidth}
		\begin{center}
			\tcbox[enhanced jigsaw,colframe=black,opacityframe=0.5,opacityback=0.5]
			{\centering{\includesvg[width=0.65\linewidth]{Hook/Bight_and_Eye_toggled}}}
		\end{center}
	\end{minipage}
\caption{Bight and Eye, toggled.}
\label{ris:Bight_and_Eye_toggled}
\end{figure}

В принципе, можно использовать любой штык или бенд, например, Round Turn and Two Half Hitches, Rolling Hitch, Fisherman’s Bend, Single Hook Hitch(see Pile Hitch), Clove Hitch, Cow Hitch, Ring Hitch.
