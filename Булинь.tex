\section{Булинь}

Он же Bowline, Bowling, Bolin Knot, Bowling’s Knot. По-русски - Беседочный узел, Булиневый, Пальстек. Самый известный узел в мире.

\begin{figure}[H]\centering
	\begin{minipage}{1\linewidth}
		\begin{center}
			\tcbox[enhanced jigsaw,colframe=black,opacityframe=0.5,opacityback=0.5]
			{\centering{\includesvg[width=0.6\linewidth]{Bowline/Besedochny}}}
		\end{center}
	\end{minipage}
\caption{Булинь.}
\label{ris:Besedochny}
\end{figure}

\subsection{Завязывание}\label{sec:Zavyazyvanie_bowline}

Завязывание Беседочного узла возможно несколькими способами.

\subsubsection*{Прямой способ}

Английские моряки завязывание Беседочного узла описывали в стихах:

\begin{verse}
Lay the bight to make a hole \\
Then under the back and around the pole \\
Over the top and thru the eye \\
Cinch it tight and let it lie.
\end{verse}

%TODO (Какой??? Посмотреть!)

На коренном конце делается твист примерно $270^{\circ}$ по часовой стрелке. Направление поворота важно, так как иначе получится совершенно другой узел. Далее, ходовой конец проходит снизу вверх. Это тоже важно, иначе получится Казацкий узел (рис.~\ref{ris:Kazak}) она входит в образованную твистом петлю, обносится вокруг коренного конца и возвращается обратно.

\subsubsection*{Пальстек}

Это не узел, просто разновидность прямого способа завязывания, когда нужно привязать Булинь к верхней опоре.

\begin{figure}[H]\centering
	\begin{minipage}{1\linewidth}
		\begin{center}
			\tcbox[enhanced jigsaw,colframe=black,opacityframe=0.5,opacityback=0.5]
			{\centering{\includesvg[width=0.55\linewidth]{Bowline/Palstek}}}
		\end{center}
	\end{minipage}
\caption{Пальстек.}
\label{ris:Palstek}
\end{figure}

\subsubsection*{Швартовочный способ}

По английски - Hawser Bowline (рис.~\ref{ris:Hawser_Bowline}). Булинь завязывается сам после резкого рывка за коренной конец. Обычно используется для обвязывания вокруг опоры.

\subsubsection*{Морской способ}

Можно вязать Беседочный узел одной рукой в полной темноте (на ощупь). Это очень полезный навык. Одна рука может быть повреждена или чем-то занята. Представьте самый экстремальный вариант - Вы за бортом, держитесь за веревку, судно движется. Нужно срочно обвязать вокруг себя петлю. Сделать это можно только одной рукой, другой же Вы должны держаться за веревку.

\subsection{Развязывание}

Под любой нагрузкой он никогда не затянется намертво. Для этого достаточно перегнуть коренной конец в районе охватывающей его петли ходового конца, как бы переламывая узел, после чего эта петля сдвинется вверх по ходовому концу. Тем самым петля коренного конца увеличивается в размере и освобождает ходовой конец.

\begin{figure}[H]\centering
	\begin{minipage}{1\linewidth}
		\begin{center}
			\tcbox[enhanced jigsaw,colframe=black,opacityframe=0.5,opacityback=0.5]
			{\centering{\includesvg[width=0.45\linewidth]{Bowline/Besedochny_razvyazivanie}}}
		\end{center}
	\end{minipage}
\caption{Развязывание.}
\label{ris:Besedochny_razvyazivanie}
\end{figure}

\subsection{Быстроразвязывающийся Булинь}

\begin{figure}[H]\centering
	\begin{minipage}{1\linewidth}
		\begin{center}
			\tcbox[enhanced jigsaw,colframe=black,opacityframe=0.5,opacityback=0.5]
			{\centering{\includesvg[width=0.55\linewidth]{Bowline/Besedochny_slide}}}
		\end{center}
	\end{minipage}
\caption{Быстроразвязывающийся Булинь.}
\label{ris:Besedochny_slide}
\end{figure}

\subsection{Булинь левосторонний}

Леворукий, Ковбойский булинь, Зимний булинь, Left-Handed Bowline.

\begin{figure}[H]\centering
	\begin{minipage}{1\linewidth}
		\begin{center}
			\tcbox[enhanced jigsaw,colframe=black,opacityframe=0.5,opacityback=0.5]
			{\centering{\includesvg[width=0.6\linewidth]{Bowline/Besedochny_Left-Hand}}}
		\end{center}
	\end{minipage}
\caption{Булинь левосторонний.}
\label{ris:Besedochny_Left-Hand}
\end{figure}

\subsection{Булинь обратный}

Он же Булинь Инвертированный и Bowline Inverted. Зеркальное отображение Беседочного узла.

\begin{figure}[H]\centering
	\begin{minipage}{1\linewidth}
		\begin{center}
			\tcbox[enhanced jigsaw,colframe=black,opacityframe=0.5,opacityback=0.5]
			{\centering{\includesvg[width=0.6\linewidth]{Bowline/Besedochny_Inverted}}}
		\end{center}
	\end{minipage}
\caption{Булинь обратный.}
\label{ris:Besedochny_Inverted}
\end{figure}

\subsection{Clinch Bowline}

%TODO посмотреть “Обвязочную петлю”

Он же Linfit Bowline, “Беседочный с постромкой” или Пальстек. Плохо затянутый узел может вывернуться и получится затягивающаяся петля. Специфический узел. Регулируемая петля, которая без фиксации ходового конца под малейшей нагрузкой распустится.

\begin{figure}[H]\centering
	\begin{minipage}{1\linewidth}
		\begin{center}
			\tcbox[enhanced jigsaw,colframe=black,opacityframe=0.5,opacityback=0.5]
			{\centering{\includesvg[width=0.6\linewidth]{Bowline/Clinch_Bowline}}}
		\end{center}
	\end{minipage}
\caption{Clinch Bowline.}
\label{ris:Clinch_Bowline}
\end{figure}

\subsection{Back Bowline}

\begin{figure}[H]\centering
	\begin{minipage}{1\linewidth}
		\begin{center}
			\tcbox[enhanced jigsaw,colframe=black,opacityframe=0.5,opacityback=0.5]
			{\centering{\includesvg[width=0.6\linewidth]{Bowline/Back_Bowline}}}
		\end{center}
	\end{minipage}
\caption{Back Bowline.}
\label{ris:Back_Bowline}
\end{figure}

\subsection{Китайский Беседочный}

Он же Chinese Bowline.

\begin{figure}[H]\centering
	\subfloat[Завязывание]{\label{ris:Chinese_Bowline_1}
	\tcbox[enhanced jigsaw,colframe=black,opacityframe=0.5,opacityback=0.5]
		{\centering
			\includesvg[width=0.45\linewidth]{Bowline/Chinese_Bowline}}
		}
\end{figure}

\begin{figure}[H]\centering
	\subfloat[Завязывание]{\label{ris:Chinese_Bowline_2}
	\tcbox[enhanced jigsaw,colframe=black,opacityframe=0.5,opacityback=0.5]
		{\centering
			\includesvg[width=0.6\linewidth]{Bowline/Chinese_Bowline_1}}
		}
\end{figure}

\begin{figure}[H]\centering
	\subfloat[Результат]{\label{ris:Chinese_Bowline_3}
	\tcbox[enhanced jigsaw,colframe=black,opacityframe=0.5,opacityback=0.5]
		{\centering
			\includesvg[width=0.6\linewidth]{Bowline/Chinese_Bowline_2}}
		}
	\caption{Китайский Беседочный.}\label{ris:Chinese_Bowline}
\end{figure}

\subsection{Мексиканский Беседочный}

Он же Mexican Bowline.

\begin{figure}[H]\centering
	\begin{minipage}{1\linewidth}
		\begin{center}
			\tcbox[enhanced jigsaw,colframe=black,opacityframe=0.5,opacityback=0.5]
			{\centering{\includesvg[width=0.6\linewidth]{Bowline/Mexican_Bowline}}}
		\end{center}
	\end{minipage}
\caption{Мексиканский Беседочный.}
\label{ris:Mexican_Bowline}
\end{figure}

\subsection{Compromise Bowline}

Можно сделать несколько петель.

\begin{figure}[H]\centering
	\subfloat[Завязывание]{\label{ris:Compromise_Bowline_1}
	\tcbox[enhanced jigsaw,colframe=black,opacityframe=0.5,opacityback=0.5]
		{\centering
			\includesvg[width=0.6\linewidth]{Bowline/Compromise_Bowline}}
		}
\end{figure}

\begin{figure}[H]\centering
	\subfloat[Результат]{\label{ris:Compromise_Bowline_2}
	\tcbox[enhanced jigsaw,colframe=black,opacityframe=0.5,opacityback=0.5]
		{\centering
			\includesvg[width=0.6\linewidth]{Bowline/Compromise_Bowline_1}}
		}
	\caption{Compromise Bowline.}\label{ris:Chinese_Bowline}
\end{figure}

\subsection{Bow Bowline}

%TODO Где-то я это видел...

\begin{figure}[H]\centering
	\begin{minipage}{1\linewidth}
		\begin{center}
			\tcbox[enhanced jigsaw,colframe=black,opacityframe=0.5,opacityback=0.5]
			{\centering{\includesvg[width=0.45\linewidth]{Bowline/Bow_Bowline}}}
		\end{center}
	\end{minipage}
\caption{Bow Bowline.}
\label{ris:Bow_Bowline}
\end{figure}

\subsection{Eye Bowline}

\begin{figure}[H]\centering
	\begin{minipage}{1\linewidth}
		\begin{center}
			\tcbox[enhanced jigsaw,colframe=black,opacityframe=0.5,opacityback=0.5]
			{\centering{\includesvg[width=0.55\linewidth]{Bowline/Eye_Bowline}}}
		\end{center}
	\end{minipage}
\caption{Eye Bowline.}
\label{ris:Eye_Bowline}
\end{figure}

\subsection{Китайский Беседочный с двумя петлями}

Chinese twin loop Bowline. Очень любопытный узел с редкими свойствами. Одна петля (на рисунке справа) у него скользящая, другая - глухая. На мой взгляд, ходовой конец можно провести более рационально для повышения удобства вязки. На другие качества узла это не повлияет.

\begin{figure}[H]\centering
	\subfloat[Завязывание]{\label{ris:Chinese_twin_loop_Bowline_1}
	\tcbox[enhanced jigsaw,colframe=black,opacityframe=0.5,opacityback=0.5,height=7.5cm]
		{\centering
			\includesvg[width=0.28\linewidth]{Bowline/Chinese_twin_loop_Bowline}}
		}
\hfil
	\subfloat[Завязывание]{\label{ris:Chinese_twin_loop_Bowline_2}
	\tcbox[enhanced jigsaw,colframe=black,opacityframe=0.5,opacityback=0.5,height=7.5cm]
		{\centering
			\includesvg[width=0.38\linewidth]{Bowline/Chinese_twin_loop_Bowline_1}}
		}
\end{figure}

\begin{figure}[H]\centering
	\subfloat[Результат]{\label{ris:Chinese_twin_loop_Bowline_3}
	\tcbox[enhanced jigsaw,colframe=black,opacityframe=0.5,opacityback=0.5]
		{\centering
			\includesvg[width=0.35\linewidth]{Bowline/Chinese_twin_loop_Bowline_2}}
		}
	\caption{Китайский Беседочный с двумя петлями.}\label{ris:Chinese_twin_loop_Bowline}
\end{figure}

\subsection{Улучшенный Китайский Беседочный с двумя петлями}

\begin{figure}[H]\centering
	\subfloat[Завязывание]{\label{ris:Chinese_twin_loop_Bowline_my_1_1}
	\tcbox[enhanced jigsaw,colframe=black,opacityframe=0.5,opacityback=0.5]
		{\centering
			\includesvg[width=0.38\linewidth]{Bowline/Chinese_twin_loop_Bowline_my}}
		}
\hfil
	\subfloat[Результат]{\label{ris:Chinese_twin_loop_Bowline_my_1_2}
	\tcbox[enhanced jigsaw,colframe=black,opacityframe=0.5,opacityback=0.5]
		{\centering
			\includesvg[width=0.38\linewidth]{Bowline/Chinese_twin_loop_Bowline_my_1}}
		}
	\caption{Улучшенный Китайский Беседочный с двумя петлями.}\label{ris:Chinese_twin_loop_Bowline_my_1}
\end{figure}

\subsection{Развязывающийся Китайский беседочный с двумя петлями}

В обоих вариантах правая скользящая петля, будучи полностью вытянутой, не развяжет узел полностью. Ей помешает вторая (глухая) петля, так как она проходит внутри ее. Противоположного эффекта можно добиться, если при завязывании узла провести глухую петлю поверх скользящей. Красным овалом отмечено изменение в схеме завязывания узла.

\begin{figure}[H]\centering
	\subfloat[Завязывание]{\label{ris:Chinese_twin_loop_Bowline_my_2_1}
	\tcbox[enhanced jigsaw,colframe=black,opacityframe=0.5,opacityback=0.5]
		{\centering
			\includesvg[width=0.38\linewidth]{Bowline/Chinese_twin_loop_Bowline_my_2_1}}
		}
\hfil
	\subfloat[Результат]{\label{ris:Chinese_twin_loop_Bowline_my_2_2}
	\tcbox[enhanced jigsaw,colframe=black,opacityframe=0.5,opacityback=0.5]
		{\centering
			\includesvg[width=0.38\linewidth]{Bowline/Chinese_twin_loop_Bowline_my_2}}
		}
	\caption{Развязывающийся Китайский Беседочный с двумя петлями.}\label{ris:Chinese_twin_loop_Bowline_my_2}
\end{figure}

\subsection{Bowline Shortening}

От обычного Беседочного отличается только тем, что ходовой конец сложен вдвое. Он же Укорачивающий Булинь, Булинь петлей или Double Bowline (Brimingham method). Что такое \enquote{Brimingham}, не понятно.

\begin{figure}[H]\centering
	\subfloat[Завязывание]{\label{ris:Bowline_Shortening_1}
	\tcbox[enhanced jigsaw,colframe=black,opacityframe=0.5,opacityback=0.5]
		{\centering
			\includesvg[width=0.55\linewidth]{Bowline/Bowline_Shortening}}
		}
\end{figure}

\begin{figure}[H]\centering
	\subfloat[Завязывание]{\label{ris:Bowline_Shortening_2}
	\tcbox[enhanced jigsaw,colframe=black,opacityframe=0.5,opacityback=0.5]
		{\centering
			\includesvg[width=0.55\linewidth]{Bowline/Bowline_Shortening_1}}
		}
\end{figure}

\begin{figure}[H]\centering
	\subfloat[Результат]{\label{ris:Bowline_Shortening_3}
	\tcbox[enhanced jigsaw,colframe=black,opacityframe=0.5,opacityback=0.5]
		{\centering
			\includesvg[width=0.6\linewidth]{Bowline/Double_Bowline_Brimingham_method}}
		}
	\caption{Bowline Shortening.}\label{ris:Bowline_Shortening}
\end{figure}

\addtocounter{LoopNoName}{1}

\subsection{Петля без названия \arabic{LoopNoName}}

Узел названия не имеет, но в его основе тот же Булинь, точнее, Булинь петлей (рис.~\ref{ris:Bowline_Shortening}). ABOK № 1016.

\begin{figure}[H]\centering
	\subfloat[Завязывание]{\label{ris:LoopNoName_1}
	\tcbox[enhanced jigsaw,colframe=black,opacityframe=0.5,opacityback=0.5]
		{\centering
			\includesvg[width=0.6\linewidth]{Bowline/1016}}
		}
\end{figure}

\begin{figure}[H]\centering
	\subfloat[Результат]{\label{ris:LoopNoName_2}
	\tcbox[enhanced jigsaw,colframe=black,opacityframe=0.5,opacityback=0.5]
		{\centering
			\includesvg[width=0.65\linewidth]{Bowline/1016_1}}
		}
	\caption{Петля без названия \arabic{LoopNoName}.}\label{ris:LoopNoName}
\end{figure}

\subsection{Brummycham Bowline}

Бирмингемский булинь. Этим узлом как и Португальским Беседочным можно обвязать целую бухту троса - петель может быть сколько угодно.

\begin{figure}[H]\centering
	\begin{minipage}{1\linewidth}
		\begin{center}
			\tcbox[enhanced jigsaw,colframe=black,opacityframe=0.5,opacityback=0.5]
			{\centering{\includesvg[width=0.6\linewidth]{Bowline/Brummycham_Bowline}}}
		\end{center}
	\end{minipage}
\caption{Brummycham Bowline.}
\label{ris:Brummycham_Bowline}
\end{figure}

\subsection{Булинь петлей с фиксацией}

Логическое продолжение предыдущего узла. О возможных вариантах фиксации Беседочного узла можно узнать в разделе № \ref{sec:Fiks_bowline}.

\begin{figure}[H]\centering
	\begin{minipage}{1\linewidth}
		\begin{center}
			\tcbox[enhanced jigsaw,colframe=black,opacityframe=0.5,opacityback=0.5]
			{\centering{\includesvg[width=0.6\linewidth]{Bowline/Boulin_petley}}}
		\end{center}
	\end{minipage}
\caption{Булинь петлей с фиксацией.}
\label{ris:Bowline_petley}
\end{figure}

\subsection{Двойной булинь}

Двойной беседочный узел, боцманский, воинский, петлевой. Английское название - Double Bowline. Возможен вариант с завязыванием этого узла на середине веревки, тогда он будет называться Bowline on-the-bight. Если сделать несколько витков основной петлей - получится один из вариантов Португальского Беседочного узла (рис.~\ref{ris:Portuguese_Bowline_on-the-bight}).

\begin{figure}[H]\centering
	\subfloat[Завязывание]{\label{ris:Double_Bowline_1}
	\tcbox[enhanced jigsaw,colframe=black,opacityframe=0.5,opacityback=0.5]
		{\centering
			\includesvg[width=0.55\linewidth]{Bowline/Double_Bowline_0}}
		}
\end{figure}

\begin{figure}[H]\centering
	\subfloat[Результат]{\label{ris:Double_Bowline_2}
	\tcbox[enhanced jigsaw,colframe=black,opacityframe=0.5,opacityback=0.5]
		{\centering
			\includesvg[width=0.55\linewidth]{Bowline/Double_Bowline_0_1}}
		}
	\caption{Двойной Беседочный.}\label{ris:Double_Bowline}
\end{figure}

\subsection{Тройной Беседочный}

Triple Bowline. Представляет собой обычный Булинь, только завязанный сложенной вдвое веревкой. Может быть леворуким, обратным и т.п. Тройной --потому что три петли.

\begin{figure}[H]\centering
	\subfloat[Завязывание]{\label{ris:Triple_Bowline_1}
	\tcbox[enhanced jigsaw,colframe=black,opacityframe=0.5,opacityback=0.5]
		{\centering
			\includesvg[width=0.5\linewidth]{Bowline/Triple_Bowline}}
		}
\end{figure}

\begin{figure}[H]\centering
	\subfloat[Результат]{\label{ris:Triple_Bowline_2}
	\tcbox[enhanced jigsaw,colframe=black,opacityframe=0.5,opacityback=0.5]
		{\centering
			\includesvg[width=0.5\linewidth]{Bowline/Triple_Bowline_1}}
		}
	\caption{Тройной Беседочный.}\label{ris:Triple_Bowline}
\end{figure}

\subsection{Дважды Двойной Беседочный}

Получается из Тройного Беседочного, если сдвоенные петли пропустить внутрь последней (третьей) петли и затянуть узел как обычный Булинь.

\begin{figure}[H]\centering
	\begin{minipage}{1\linewidth}
		\begin{center}
			\tcbox[enhanced jigsaw,colframe=black,opacityframe=0.5,opacityback=0.5]
			{\centering{\includesvg[width=0.5\linewidth]{Bowline/Double_Bowline_0_3}}}
		\end{center}
	\end{minipage}
\caption{Дважды Двойной Беседочный.}
\label{ris:Double_Double_Bowline}
\end{figure}

\subsection{Русский Беседочный}

Одна из петель затягивающаяся, другая фиксированная. Можно развить идею этого узла, сделав затягивающуюся петлю регулируемой или % TODO Добавить мыслей

\begin{figure}[H]\centering
	\subfloat[Завязывание]{\label{ris:Russian_bouline_1}
	\tcbox[enhanced jigsaw,colframe=black,opacityframe=0.5,opacityback=0.5]
		{\centering
			\includesvg[width=0.65\linewidth]{Bowline/Russian_bouline}}
		}
\end{figure}

\begin{figure}[H]\centering
	\subfloat[Завязывание]{\label{ris:Russian_bouline_2}
	\tcbox[enhanced jigsaw,colframe=black,opacityframe=0.5,opacityback=0.5]
		{\centering
			\includesvg[width=0.6\linewidth]{Bowline/Russian_bouline_1}}
		}
\end{figure}

\begin{figure}[H]\centering
	\subfloat[Результат]{\label{ris:Russian_bouline_3}
	\tcbox[enhanced jigsaw,colframe=black,opacityframe=0.5,opacityback=0.5]
		{\centering
			\includesvg[width=0.75\linewidth]{Bowline/Russian_bouline_2}}
		}
	\caption{Русский Беседочный.}\label{ris:Russian_bouline}
\end{figure}

\subsection{Датский булинь 1}

Dutchman's Bowline.

\begin{figure}[H]\centering
	\subfloat[Завязывание]{\label{ris:Dutchmans_bowline_1_1}
	\tcbox[enhanced jigsaw,colframe=black,opacityframe=0.5,opacityback=0.5]
		{\centering
			\includesvg[width=0.45\linewidth]{Bowline/Dutchmans_bowline_1}}
		}
\end{figure}

\begin{figure}[H]\centering
	\subfloat[Завязывание]{\label{ris:Dutchmans_bowline_1_2}
	\tcbox[enhanced jigsaw,colframe=black,opacityframe=0.5,opacityback=0.5]
		{\centering
			\includesvg[width=0.5\linewidth]{Bowline/Dutchmans_bowline_1_1}}
		}
\end{figure}

\begin{figure}[H]\centering
	\subfloat[Результат]{\label{ris:Dutchmans_bowline_1_3}
	\tcbox[enhanced jigsaw,colframe=black,opacityframe=0.5,opacityback=0.5]
		{\centering
			\includesvg[width=0.55\linewidth]{Bowline/Dutchmans_bowline_1_2}}
		}
	\caption{Датский булинь 1.}\label{ris:Dutchmans_bowline_1}
\end{figure}

\subsection{Датский булинь 2}

Вязание узла начинается с Дубовой петли (рис.~\ref{ris:Dubovaya_loop}).

\begin{figure}[H]\centering
	\subfloat[Завязывание]{\label{ris:Dutchmans_bowline_2_1}
	\tcbox[enhanced jigsaw,colframe=black,opacityframe=0.5,opacityback=0.5]
		{\centering
			\includesvg[width=0.55\linewidth]{Bowline/Dutchmans_bowline_2}}
		}
\end{figure}

\begin{figure}[H]\centering
	\subfloat[Результат]{\label{ris:Dutchmans_bowline_2_2}
	\tcbox[enhanced jigsaw,colframe=black,opacityframe=0.5,opacityback=0.5]
		{\centering
			\includesvg[width=0.6\linewidth]{Bowline/Dutchmans_bowline_2_1}}
		}
\end{figure}

Развиваем и видоизменяем узел.

\begin{figure}[H]\centering
	\subfloat[Двухпетельный]{\label{ris:Dutchmans_bowline_2_3}
	\tcbox[enhanced jigsaw,colframe=black,opacityframe=0.5,opacityback=0.5]
		{\centering
			\includesvg[width=0.6\linewidth]{Bowline/Dutchmans_bowline_3_1}}
		}
\end{figure}

\begin{figure}[H]\centering
	\subfloat[Трехпетельный]{\label{ris:Dutchmans_bowline_2_4}
	\tcbox[enhanced jigsaw,colframe=black,opacityframe=0.5,opacityback=0.5]
		{\centering
			\includesvg[width=0.65\linewidth]{Bowline/Dutchmans_bowline_3}}
		}
	\caption{Датский булинь 2.}\label{ris:Dutchmans_bowline_2}
\end{figure}

\subsection{Испанский Беседочный}

Он же Боцманский узел, Spanish Bowline. Петли перетягиваются одна в другую как сообщающиеся сосуды.

\begin{figure}[H]\centering
	\subfloat[Завязывание. \\Первый способ]{\label{ris:Spanish_bowline_1}
	\tcbox[enhanced jigsaw,colframe=black,opacityframe=0.5,opacityback=0.5,height=7cm]
		{\centering
			\includesvg[width=0.44\linewidth]{Bowline/Spanish_bowline_1_1}}
		}
\hfil
	\subfloat[Завязывание. \\Первый способ]{\label{ris:Spanish_bowline_2}
	\tcbox[enhanced jigsaw,colframe=black,opacityframe=0.5,opacityback=0.5,height=7cm]
		{\centering
			\includesvg[width=0.35\linewidth]{Bowline/Spanish_bowline_1_2}}
		}
\end{figure}

\begin{figure}[H]\centering
	\subfloat[Завязывание. \\Второй способ]{\label{ris:Spanish_bowline_3}
	\tcbox[enhanced jigsaw,colframe=black,opacityframe=0.5,opacityback=0.5]
		{\centering
			\includesvg[width=0.5\linewidth]{Bowline/Spanish_bowline_2_1}}
		}
\end{figure}

\begin{figure}[H]\centering
	\subfloat[Завязывание. \\Второй способ]{\label{ris:Spanish_bowline_4}
	\tcbox[enhanced jigsaw,colframe=black,opacityframe=0.5,opacityback=0.5]
		{\centering
			\includesvg[width=0.55\linewidth]{Bowline/Spanish_bowline_2_2}}
		}
\end{figure}

\begin{figure}[H]\centering
	\subfloat[Завязывание. \\Третий способ]{\label{ris:Spanish_bowline_5}
	\tcbox[enhanced jigsaw,colframe=black,opacityframe=0.5,opacityback=0.5]
		{\centering
			\includesvg[width=0.5\linewidth]{Bowline/Spanish_bowline_3}}
		}
\end{figure}

\begin{figure}[H]\centering
	\subfloat[Результат]{\label{ris:Spanish_bowline_6}
	\tcbox[enhanced jigsaw,colframe=black,opacityframe=0.5,opacityback=0.5]
		{\centering
			\includesvg[width=0.65\linewidth]{Bowline/Spanish_bowline}}
		}
	\caption{Испанский Беседочный.}\label{ris:Spanish_bowline}
\end{figure}

\subsection{Hitched Испанский Беседочный}

\begin{figure}[H]\centering
	\subfloat[Завязывание]{\label{ris:Hitched_Spanish_Bowline_1}
	\tcbox[enhanced jigsaw,colframe=black,opacityframe=0.5,opacityback=0.5]
		{\centering
			\includesvg[width=0.5\linewidth]{Bowline/Hitched_Spanish_Bowline}}
		}
\end{figure}

\begin{figure}[H]\centering
	\subfloat[Результат]{\label{ris:Hitched_Spanish_Bowline_2}
	\tcbox[enhanced jigsaw,colframe=black,opacityframe=0.5,opacityback=0.5]
		{\centering
			\includesvg[width=0.9\linewidth]{Bowline/Hitched_Spanish_Bowline_1}}
		}
	\caption{Hitched Испанский Беседочный.}\label{ris:Hitched_Spanish_Bowline}
\end{figure}

\subsection{Lark's head Bowline}

\begin{figure}[H]\centering
	\subfloat[Завязывание]{\label{ris:Larks_head_Bowline_1}
	\tcbox[enhanced jigsaw,colframe=black,opacityframe=0.5,opacityback=0.5,height=7.5cm]
		{\centering
			\includesvg[width=0.39\linewidth]{Bowline/Larks_head_Bowline_2}}
		}
\hfil
	\subfloat[Завязывание]{\label{ris:Larks_head_Bowline_2}
	\tcbox[enhanced jigsaw,colframe=black,opacityframe=0.5,opacityback=0.5,height=7.5cm]
		{\centering
			\includesvg[width=0.39\linewidth]{Bowline/Larks_head_Bowline_1}}
		}
\end{figure}

\begin{figure}[H]\centering
	\subfloat[Результат]{\label{ris:Larks_head_Bowline_3}
	\tcbox[enhanced jigsaw,colframe=black,opacityframe=0.5,opacityback=0.5]
		{\centering
			\includesvg[width=0.4\linewidth]{Bowline/Larks_head_Bowline}}
		}
	\caption{Lark's head Bowline.}\label{ris:Larks_head_Bowline}
\end{figure}

\subsection{Interlocking round-turn Bowline}

Если остановиться на второй фазе завязывания, получится система из одной скользящей и одной глухой петли, как в Китайском Беседочном с двумя петлями (рис.~\ref{ris:Chinese_twin_loop_Bowline}).

\begin{figure}[H]\centering
	\subfloat[Завязывание]{\label{ris:Interlocking_round-turn_bowline_1}
	\tcbox[enhanced jigsaw,colframe=black,opacityframe=0.5,opacityback=0.5]
		{\centering
			\includesvg[angle=270,width=0.6\linewidth]{Bowline/Interlocking_round-turn_bowline_5}}
		}
\hfil
	\subfloat[Результат первой фазы завязывания]{\label{ris:Interlocking_round-turn_bowline_2}
	\tcbox[enhanced jigsaw,colframe=black,opacityframe=0.5,opacityback=0.5]
		{\centering
			\includesvg[angle=270,width=0.6\linewidth]{Bowline/Interlocking_round-turn_bowline_4}}
		}
\end{figure}

\begin{figure}[H]\centering
	\subfloat[Завязывание]{\label{ris:Interlocking_round-turn_bowline_3}
	\tcbox[enhanced jigsaw,colframe=black,opacityframe=0.5,opacityback=0.5]
		{\centering
			\includesvg[angle=270,width=0.6\linewidth]{Bowline/Interlocking_round-turn_bowline_3}}
		}
\hfil
	\subfloat[Результат второй фазы завязывания]{\label{ris:Interlocking_round-turn_bowline_4}
	\tcbox[enhanced jigsaw,colframe=black,opacityframe=0.5,opacityback=0.5]
		{\centering
			\includesvg[angle=270,width=0.6\linewidth]{Bowline/Interlocking_round-turn_bowline_2}}
		}
\end{figure}

\begin{figure}[H]\centering
	\subfloat[Завязывание]{\label{ris:Interlocking_round-turn_bowline_5}
	\tcbox[enhanced jigsaw,colframe=black,opacityframe=0.5,opacityback=0.5]
		{\centering
			\includesvg[angle=270,width=0.6\linewidth]{Bowline/Interlocking_round-turn_bowline_1}}
		}
\hfil
	\subfloat[Результат]{\label{ris:Interlocking_round-turn_bowline_6}
	\tcbox[enhanced jigsaw,colframe=black,opacityframe=0.5,opacityback=0.5]
		{\centering
			\includesvg[angle=270,width=0.6\linewidth]{Bowline/Interlocking_round-turn_bowline}}
		}
	\caption{Interlocking round-turn Bowline.}\label{ris:Interlocking_round-turn_bowline}
\end{figure}

\subsection{Португальский беседочный}

Португальский булинь или Portuguese Bowline, Французский булинь или French Bowline, встречалось также название “Воинский,”. Петли в этом узле также перетягиваются одна в другую, причем, даже легче, чем в Испанском беседочном. Их можно сделать сколько угодно, целую бухту троса можно обвязать таким способом.

\begin{figure}[H]\centering
	\begin{minipage}{1\linewidth}
		\begin{center}
			\tcbox[enhanced jigsaw,colframe=black,opacityframe=0.5,opacityback=0.5]
			{\centering{\includesvg[width=0.5\linewidth]{Bowline/Portuguese_bowline}}}
		\end{center}
	\end{minipage}
\caption{Португальский беседочный.}
\label{ris:Portuguese_bowline}
\end{figure}

\subsection{Португальский беседочный 2}

\begin{figure}[H]\centering
	\begin{minipage}{1\linewidth}
		\begin{center}
			\tcbox[enhanced jigsaw,colframe=black,opacityframe=0.5,opacityback=0.5]
			{\centering{\includesvg[width=0.6\linewidth]{Bowline/Portuguese_bowline_2}}}
		\end{center}
	\end{minipage}
\caption{Португальский беседочный 2.}
\label{ris:Portuguese_bowline_2}
\end{figure}

\subsection{Portuguese Bowline on-the-bight}

\begin{figure}[H]\centering
	\begin{minipage}{1\linewidth}
		\begin{center}
			\tcbox[enhanced jigsaw,colframe=black,opacityframe=0.5,opacityback=0.5]
			{\centering{\includesvg[width=0.6\linewidth]{Bowline/Portuguese_Bowline_on-the-bight}}}
		\end{center}
	\end{minipage}
\caption{Portuguese Bowline on-the-bight.}
\label{ris:Portuguese_Bowline_on-the-bight}
\end{figure}

\subsection{Portuguese Bowline with Splayed Loops}

Другой вариант Португальского беседочного с петлями по разные стороны узла.

\begin{figure}[H]\centering
	\subfloat[Завязывание]{\label{ris:Portuguese_Bowline_with_Splayed_Loops_1}
	\tcbox[enhanced jigsaw,colframe=black,opacityframe=0.5,opacityback=0.5,height=5cm]
		{\centering
			\includesvg[width=0.32\linewidth]{Bowline/Portuguese_Bowline_with_Splayed_Loops}}
		}
\hfil
	\subfloat[Результат]{\label{ris:Portuguese_Bowline_with_Splayed_Loops_2}
	\tcbox[enhanced jigsaw,colframe=black,opacityframe=0.5,opacityback=0.5,height=5cm]
		{\centering
			\includesvg[width=0.45\linewidth]{Bowline/Portuguese_Bowline_with_Splayed_Loops_1}}
		}
	\caption{Portuguese Bowline with Splayed Loops.}\label{ris:Portuguese_Bowline_with_Splayed_Loops}
\end{figure}

\subsection{Linesman bowline}

% FIXIT Нужен ли?

Логическое продолжение Fisherman's Bowline (рис.~\ref{ris:Fishermans_Bowline}) с двумя петлями. Три варианта.

\begin{figure}[H]\centering
	\subfloat[Первый вариант]{\label{ris:Linesman_bowline_1}
	\tcbox[enhanced jigsaw,colframe=black,opacityframe=0.5,opacityback=0.5,height=10.5cm]
		{\centering
			\includesvg[width=0.4\linewidth]{Bowline/Linesman_bowline_2}}
		}
\hfil
	\subfloat[Второй вариант]{\label{ris:Linesman_bowline_2}
	\tcbox[enhanced jigsaw,colframe=black,opacityframe=0.5,opacityback=0.5,height=10.5cm]
		{\centering
			\includesvg[width=0.39\linewidth]{Bowline/Linesman_bowline}}
		}
\end{figure}

\begin{figure}[H]\centering
	\subfloat[Третий вариант]{\label{ris:Linesman_bowline_3}
	\tcbox[enhanced jigsaw,colframe=black,opacityframe=0.5,opacityback=0.5]
		{\centering
			\includesvg[width=0.4\linewidth]{Bowline/Linesman_bowline_1}}
		}
	\caption{Linesman bowline.}\label{ris:Linesman_bowline}
\end{figure}

\subsection{Фиксация Булиня}\label{sec:Fiks_bowline}

При пульсирующих нагрузках, особенно на скользких веревках, он может развязаться. Избежать этого можно зафиксировав ходовой конец. Для этого существует несколько способов:

\subsection*{Наложение марки}

Простой и надежный вариант. Недостаток - долго и трудоемко. Как правило, делают это один раз и навсегда. То есть, возможность развязывания такого узла не подразумевается.

\begin{figure}[H]\centering
	\begin{minipage}{1\linewidth}
		\begin{center}
			\tcbox[enhanced jigsaw,colframe=black,opacityframe=0.5,opacityback=0.5]
			{\centering{\includesvg[width=0.6\linewidth]{Bowline/Besedochny_marka}}}
		\end{center}
	\end{minipage}
\caption{Наложение марки.}
\label{ris:Besedochny_marka}
\end{figure}

\subsection*{Узел для утолщения}

На ходовом конце делается простой узел, чтобы конец не мог пролезть внутрь затянутого узла. Для утолщения веревки узлов существует великое множество, можно использовать любой.

\begin{figure}[H]\centering
	\begin{minipage}{1\linewidth}
		\begin{center}
			\tcbox[enhanced jigsaw,colframe=black,opacityframe=0.5,opacityback=0.5]
			{\centering{\includesvg[width=0.6\linewidth]{Bowline/Besedochny_utolsh}}}
		\end{center}
	\end{minipage}
\caption{Узел для утолщения ходового конца.}
\label{ris:Besedochny_utolsh}
\end{figure}

\subsection*{Контрольный узел}

Можно использовать Простой узел, можно какой-либо из Штыков, а можно любую затягивающуюся или незатягивающуюся петлю (тот же Булинь, например).

\begin{figure}[H]\centering
	\subfloat[На ходовом конце]{\label{ris:Besedochny_kontrol_1}
	\tcbox[enhanced jigsaw,colframe=black,opacityframe=0.5,opacityback=0.5]
		{\centering
			\includesvg[width=0.6\linewidth]{Bowline/Besedochny_po_petle}}
		}
\end{figure}

\begin{figure}[H]\centering
	\subfloat[На коренном конце]{\label{ris:Besedochny_kontrol_2}
	\tcbox[enhanced jigsaw,colframe=black,opacityframe=0.5,opacityback=0.5]
		{\centering
			\includesvg[width=0.65\linewidth]{Bowline/Besedochny_na_korennom}}
		}
	\caption{Контрольный узел.}\label{ris:Besedochny_kontrol}
\end{figure}

\subsection*{Ходовой конец внутри узла}

\addtocounter{LoopNoName}{1}

Можно зажать ходовой конец непосредственно внутри узла. Тем самым осуществляется фиксация и вероятность проскальзывания уменьшается.

\begin{figure}[H]\centering
	\subfloat[Ходовой конец перпендикулярен коренному.\\Первый вариант]{\label{ris:Besedochny_hodovoy_inside_1}
	\tcbox[enhanced jigsaw,colframe=black,opacityframe=0.5,opacityback=0.5]
		{\centering
			\includesvg[width=0.55\linewidth]{Bowline/Besedochny-zaprav}}
		}
\end{figure}

\begin{figure}[H]\centering
	\subfloat[Ходовой конец перпендикулярен коренному.\\Второй вариант]{\label{ris:Besedochny_hodovoy_inside_2}
	\tcbox[enhanced jigsaw,colframe=black,opacityframe=0.5,opacityback=0.5]
		{\centering
			\includesvg[width=0.6\linewidth]{Bowline/Besedochny-zaprav-2}}
		}
\end{figure}

\begin{figure}[H]\centering
	\subfloat[Ходовой конец перпендикулярен коренному.\\Третий вариант]{\label{ris:Besedochny_hodovoy_inside_3}
	\tcbox[enhanced jigsaw,colframe=black,opacityframe=0.5,opacityback=0.5]
		{\centering
			\includesvg[width=0.55\linewidth]{Bowline/Besedochny-zaprav-3}}
		}
\end{figure}

\begin{figure}[H]\centering
	\subfloat[Ходовой конец параллелен коренному.\\Первый вариант]{\label{ris:Besedochny_hodovoy_inside_4}
	\tcbox[enhanced jigsaw,colframe=black,opacityframe=0.5,opacityback=0.5]
		{\centering
			\includesvg[width=0.65\linewidth]{Bowline/Besedochny-zaprav-4}}
		}
\end{figure}

\begin{figure}[H]\centering
	\subfloat[Ходовой конец параллелен коренному.\\Второй вариант]{\label{ris:Besedochny_hodovoy_inside_5}
	\tcbox[enhanced jigsaw,colframe=black,opacityframe=0.5,opacityback=0.5]
		{\centering
			\includesvg[width=0.65\linewidth]{Bowline/Besedochny-zaprav-5}}
		}
	\caption{Петля без названия \arabic{LoopNoName}.}\label{ris:Besedochny_hodovoy_inside}
\end{figure}

\subsubsection*{Yosemite Bowline}

После аккуратного затягивания узел выглядит очень гармонично. Чем-то напоминает Восьмерку. Тут ходовой конец выходит вверх параллельно коренному концу.

\begin{figure}[H]\centering
	\subfloat[Классический вариант]{\label{ris:Besedochny-Yosemite_1}
	\tcbox[enhanced jigsaw,colframe=black,opacityframe=0.5,opacityback=0.5]
		{\centering
			\includesvg[width=0.65\linewidth]{Bowline/Besedochny-Yosemite}}
		}
\end{figure}

\begin{figure}[H]\centering
	\subfloat[Второй вариант]{\label{ris:Besedochny-Yosemite_2}
	\tcbox[enhanced jigsaw,colframe=black,opacityframe=0.5,opacityback=0.5]
		{\centering
			\includesvg[width=0.65\linewidth]{Bowline/Besedochny-Yosemite-3}}
		}
\end{figure}

\begin{figure}[H]\centering
	\subfloat[Третий вариант]{\label{ris:Besedochny-Yosemite_3}
	\tcbox[enhanced jigsaw,colframe=black,opacityframe=0.5,opacityback=0.5]
		{\centering
			\includesvg[width=0.7\linewidth]{Bowline/Besedochny-Yosemite-2}}
		}
	\caption{Yosemite.}\label{ris:Besedochny-Yosemite}
\end{figure}

Обратите внимание, третий вариант (рис.~\ref{ris:Besedochny-Yosemite_3}) вяжется из Левостороннего Булиня (рис.~\ref{ris:Besedochny_Left-Hand}).

% TODO Проверить второй и третий вариант (коренной станет ходовым)

\subsubsection*{Fool's Bowline}

\begin{figure}[H]\centering
	\begin{minipage}{1\linewidth}
		\begin{center}
			\tcbox[enhanced jigsaw,colframe=black,opacityframe=0.5,opacityback=0.5]
			{\centering{\includesvg[width=0.6\linewidth]{Bowline/Fools_Bowline}}}
		\end{center}
	\end{minipage}
\caption{Fool's Bowline.}
\label{ris:Fools_Bowline}
\end{figure}

Если затянуть внутреннюю основную петлю, получится один из вариантов фиксации ходового конца (см. Беседочный с заправленным концом). Если ее не затягивать, а просто дополнительно зафиксировать одним из способов - поучится узел с двумя петлями. По своим свойствам (одна глухая, одна скользящая петля) похож на Китайский Беседочный с двумя петлями.

\subsubsection*{Double Bowline}

%TODO Название!!!

После вязки Double Bowline (Brimingham method) (рис.~\ref{ris:Bowline_Shortening}) ходовой конец заправляется в петлю, им же образованную.

\begin{figure}[H]\centering
	\begin{minipage}{1\linewidth}
		\begin{center}
			\tcbox[enhanced jigsaw,colframe=black,opacityframe=0.5,opacityback=0.5]
			{\centering{\includesvg[width=0.6\linewidth]{Bowline/Double_Bowline_1}}}
		\end{center}
	\end{minipage}
\caption{Double Bowline.}
\label{ris:Double_Bowline_}
\end{figure}

\subsection*{Удвоение элементов узла}

\subsubsection*{Альпинистский Беседочный}

Он же Mountaineering Bowline. Удваивается петля на коренном конце.

\begin{figure}[H]\centering
	\begin{minipage}{1\linewidth}
		\begin{center}
			\tcbox[enhanced jigsaw,colframe=black,opacityframe=0.5,opacityback=0.5]
			{\centering{\includesvg[width=0.6\linewidth]{Bowline/Besedochny-Mountaineering}}}
		\end{center}
	\end{minipage}
\caption{Альпинистский Беседочный.}
\label{ris:Besedochny-Mountaineering}
\end{figure}

\subsubsection*{Альпинистский Беседочный 2}

Узел, в котором удваивается только петля на ходовом конце. Судя по его схеме, сильно затягиваться он не должен, к тому же менее склонен  к саморазвязыванию, чем обычный Булинь.

\begin{figure}[H]\centering
	\begin{minipage}{1\linewidth}
		\begin{center}
			\tcbox[enhanced jigsaw,colframe=black,opacityframe=0.5,opacityback=0.5]
			{\centering{\includesvg[width=0.65\linewidth]{Bowline/Besedochny-Mountaineering-2}}}
		\end{center}
	\end{minipage}
\caption{Альпинистский Беседочный 2.}
\label{ris:Besedochny-Mountaineering-2}
\end{figure}

\subsubsection*{Двойной Альпинистский Беседочный}

Он же Mountaineering Double Bowline. Удваиваются петли и на коренном, и на ходовом концах.

\begin{figure}[H]\centering
	\begin{minipage}{1\linewidth}
		\begin{center}
			\tcbox[enhanced jigsaw,colframe=black,opacityframe=0.5,opacityback=0.5]
			{\centering{\includesvg[width=0.65\linewidth]{Bowline/Besedochny-Mountaineering-Double}}}
		\end{center}
	\end{minipage}
\caption{Двойной Альпинистский Беседочный 2.}
\label{ris:Besedochny-Mountaineering-Double}
\end{figure}

\subsubsection*{Водный булинь}

По английски - Water Bowline. Тоже удваивается петля на коренном конце, но немного по другому принципу. Не просто шлагами, а в виде Штыков. В классическом варианте используется Простой штык, но точно так же работать будет и с другими, например, с Cow Hitch.

\begin{figure}[H]\centering
	\subfloat[С использованием Clove Hitch (рис.~\ref{ris:Clove_Hitch})]{\label{ris:Besedochny-Water_1}
	\tcbox[enhanced jigsaw,colframe=black,opacityframe=0.5,opacityback=0.5]
		{\centering
			\includesvg[width=0.65\linewidth]{Bowline/Besedochny-Water}}
		}
\end{figure}

\begin{figure}[H]\centering
	\subfloat[С использованием Cow Hitch (рис.~\ref{ris:Cow_Hitch})]{\label{ris:Besedochny-Water_2}
	\tcbox[enhanced jigsaw,colframe=black,opacityframe=0.5,opacityback=0.5]
		{\centering
			\includesvg[width=0.65\linewidth]{Bowline/Besedochny-Water-2}}
		}
	\caption{Водный Булинь.}\label{ris:Besedochny-Water}
\end{figure}

\subsection*{Совмещение разных узлов}

Можно совместить Fool's Bowline с другим узлом, например, с Удавкой.

\begin{figure}[H]\centering
	\begin{minipage}{1\linewidth}
		\begin{center}
			\tcbox[enhanced jigsaw,colframe=black,opacityframe=0.5,opacityback=0.5]
			{\centering{\includesvg[width=0.65\linewidth]{Bowline/Besedochny-udav}}}
		\end{center}
	\end{minipage}
\caption{Совмещение разных.}
\label{ris:Besedochny-udav}
\end{figure}
