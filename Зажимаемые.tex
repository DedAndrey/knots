\section{Зажимаемые}

\subsection{Олимпийская петля}

Зажимная выбленочная петля. \href{http://www.earlham.edu/~peters/writing/explode.htm}{Автор} называет его зажимным выбленочным узлом петли, собственно по устройству и принципу это половина олимпийского узла (рис.~\ref{ris:Olimpik}. Узел достаточно легко развязывается рывком за свободный конец, не закусывает свободную петлю и не скользит, но очень капризен. Вяжется просто, но требует очень аккуратного завязывания. При ошибке может получиться половина бараньей ноги, только с двумя полуштыками, которая очень легко развязывается сама.

\begin{figure}[H]\centering
	\subfloat[Завязывание]{\label{ris:Olimpik_loop_1}
	\tcbox[enhanced jigsaw,colframe=black,opacityframe=0.5,opacityback=0.5]
		{\centering
			\includesvg[width=0.55\linewidth]{Nonslide/Olimpik_loop}}
		}
\hfil
	\subfloat[Результат]{\label{ris:Olimpik_loop_2}
	\tcbox[enhanced jigsaw,colframe=black,opacityframe=0.5,opacityback=0.5]
		{\centering
			\includesvg[width=0.55\linewidth]{Nonslide/Olimpik_loop_1}}
		}
	\caption{Олимпийская петля.}\label{ris:Olimpik_loop}
\end{figure}

\subsection{Восьмерка с зажиманием петли петлей}

В более практическом понимании это быстроразвязывающаяся восьмерочная петля. Петля не затягивается, но развязывается с трудом. Для легкого развязывания за свободный конец нужно сначала ослабить петли восьмерки.

\begin{figure}[H]\centering
	\begin{minipage}{1\linewidth}
		\begin{center}
			\tcbox[enhanced jigsaw,colframe=black,opacityframe=0.5,opacityback=0.5]
			{\centering{\includesvg[width=0.5\linewidth]{Nonslide/Figure_of_eight_zazhim}}}
		\end{center}
	\end{minipage}
\caption{Восьмерка с зажиманием петли петлей.}
\label{ris:Figure_of_eight_zazhim}
\end{figure}

\subsection{Колокольный узел}

Английское название - Bell Ringer’s Knot. Половина Бараньей ноги (рис.~\ref{ris:Sheepshank_Knot}). Недовязанный Булинь петлей (рис.~\ref{ris:Bowline_petley}).

\begin{figure}[H]\centering
	\subfloat[Обычный]{\label{ris:Bell_Ringers_Knot_1}
	\tcbox[enhanced jigsaw,colframe=black,opacityframe=0.5,opacityback=0.5]
		{\centering
			\includesvg[width=0.45\linewidth]{Nonslide/Bell_Ringers_Knot}}
		}
\hfil
	\subfloat[С дополнительным штыком]{\label{ris:Bell_Ringers_Knot_2}
	\tcbox[enhanced jigsaw,colframe=black,opacityframe=0.5,opacityback=0.5]
		{\centering
			\includesvg[width=0.45\linewidth]{Nonslide/Bell_Ringers_Knot_1}}
		}
\end{figure}

\begin{figure}[H]\centering
	\subfloat[С дополнительным штыком и двойной петлей]{\label{ris:Bell_Ringers_Knot_3}
	\tcbox[enhanced jigsaw,colframe=black,opacityframe=0.5,opacityback=0.5]
		{\centering
			\includesvg[width=0.65\linewidth]{Nonslide/Bell_Ringers_Knot_2}}
		}
	\caption{Колокольный узел.}\label{ris:Bell_Ringers_Knot}
\end{figure}

\subsection{Yardarm Knot}

Колокольный узел с маркой.

\begin{figure}[H]\centering
	\begin{minipage}{1\linewidth}
		\begin{center}
			\tcbox[enhanced jigsaw,colframe=black,opacityframe=0.5,opacityback=0.5]
			{\centering{\includesvg[width=0.4\linewidth]{Nonslide/Yardarm_Knot}}}
		\end{center}
	\end{minipage}
\caption{Yardarm Knot.}
\label{ris:Yardarm_Knot}
\end{figure}

\subsection{Баранья нога}

По английски - Sheepshank (баранья ляжка). Русский капитан дальнего плавания В. В. Бахтин в своем \enquote{Объяснительном морском словаре}, изданном в Санкт-Петербурге в 1894 году, этот узел называет \enquote{Колышкой} (нередко встречается и название \enquote{калышка}).

Название \enquote{Узел Камикадзе} стало популярным после того, как британский путешественник и телеведущий канала Дискавери Беар Гриллс в одной из передач цикла \href{http://www.youtube.com/watch?v=mtgCO17QA6U||Bear}{\enquote{Выжить любой ценой}} продемонстрировал возможности этого узла при спуске на веревке по вертикальному склону.

\begin{figure}[H]\centering
	\begin{minipage}{1\linewidth}
		\begin{center}
			\tcbox[enhanced jigsaw,colframe=black,opacityframe=0.5,opacityback=0.5]
			{\centering{\includesvg[width=0.7\linewidth]{Nonslide/Sheepshank_Knot}}}
		\end{center}
	\end{minipage}
\caption{Баранья нога.}
\label{ris:Sheepshank_Knot}
\end{figure}

Под нагрузкой можно разрезать в трех местах. Если верёвка не натянута, то узел легко развязывается. Узел, представляет из себя два простых полуштыка накинутые на сложенную втрое верёвку с двумя свободными незатягивающимися петлями и одним концом на каждом из простых полуштыков.

\begin{figure}[H]\centering
	\begin{minipage}{1\linewidth}
		\begin{center}
			\tcbox[enhanced jigsaw,colframe=black,opacityframe=0.5,opacityback=0.5]
			{\centering{\includesvg[width=0.7\linewidth]{Nonslide/Sheepshank_Knot_kut}}}
		\end{center}
	\end{minipage}
\caption{Баранья нога. Места разрезов.}
\label{ris:Sheepshank_Knot_kut}
\end{figure}

\subsection{Способы вязания Бараньей ноги}

\begin{figure}[H]\centering
	\subfloat[Обычный. Первый вариант]{\label{ris:Sheepshank_Knot_metod_1}
	\tcbox[enhanced jigsaw,colframe=black,opacityframe=0.5,opacityback=0.5]
		{\centering
			\includesvg[width=0.65\linewidth]{Nonslide/Sheepshank_Knot_metod}}
		}
\hfil
	\subfloat[Обычный. Второй вариант]{\label{ris:Sheepshank_Knot_metod_2}
	\tcbox[enhanced jigsaw,colframe=black,opacityframe=0.5,opacityback=0.5]
		{\centering
			\includesvg[width=0.65\linewidth]{Nonslide/Sheepshank_Knot_metod_1}}
		}
\end{figure}

\begin{figure}[H]\centering
	\subfloat[Быстрый способ]{\label{ris:Sheepshank_Knot_metod_3}
	\tcbox[enhanced jigsaw,colframe=black,opacityframe=0.5,opacityback=0.5]
		{\centering
			\includesvg[width=0.6\linewidth]{Nonslide/Sheepshank_Knot_fast}}
		}
	\caption{Способы вязания Бараньей ноги.}\label{ris:Sheepshank_Knot_metod}
\end{figure}

\subsection{Колышка со сваечными узлами}

Sheepshank with Marlingspike Hitches (рис.~\ref{ris:Marlingspike_Hitch}).

\begin{figure}[H]\centering
	\begin{minipage}{1\linewidth}
		\begin{center}
			\tcbox[enhanced jigsaw,colframe=black,opacityframe=0.5,opacityback=0.5]
			{\centering{\includesvg[width=0.75\linewidth]{Nonslide/Sheepshank_with_Marlingspike_Hitches}}}
		\end{center}
	\end{minipage}
\caption{Колышка со сваечными узлами.}
\label{ris:Sheepshank_with_Marlingspike_Hitches}
\end{figure}

\subsection{Колышка Томаса (скрот)}

Баранья нога на основе Tom Fool’s Knot (рис.~\ref{ris:Tom_Fool}). Другое название - Fireman’s Chair Knot.

\begin{figure}[H]\centering
	\subfloat[Завязывание]{\label{ris:Sheepshank_based_on_the_Tom_Fools_Knot_1}
	\tcbox[enhanced jigsaw,colframe=black,opacityframe=0.5,opacityback=0.5]
		{\centering
			\includesvg[width=0.75\linewidth]{Nonslide/Sheepshank_based_on_the_Tom_Fools_Knot_1}}
		}
\end{figure}

\begin{figure}[H]\centering
	\subfloat[Результат]{\label{ris:Sheepshank_based_on_the_Tom_Fools_Knot_2}
	\tcbox[enhanced jigsaw,colframe=black,opacityframe=0.5,opacityback=0.5]
		{\centering
			\includesvg[width=0.75\linewidth]{Nonslide/Sheepshank_based_on_the_Tom_Fools_Knot}}
		}
	\caption{Колышка Томаса (скрот).}\label{ris:Sheepshank_based_on_the_Tom_Fools_Knot}
\end{figure}

\subsection{Колышка с рифом}

Баранья нога на основе Handcuff Knot (рис.~\ref{ris:Handcuff_Knot}).

\begin{figure}[H]\centering
	\subfloat[Завязывание]{\label{ris:Sheepshank_from_a_Handcuff_Knot_1}
	\tcbox[enhanced jigsaw,colframe=black,opacityframe=0.5,opacityback=0.5]
		{\centering
			\includesvg[width=0.75\linewidth]{Nonslide/Sheepshank_from_a_Handcuff_Knot_1}}
		}
\end{figure}

\begin{figure}[H]\centering
	\subfloat[Результат]{\label{ris:Sheepshank_from_a_Handcuff_Knot_2}
	\tcbox[enhanced jigsaw,colframe=black,opacityframe=0.5,opacityback=0.5]
		{\centering
			\includesvg[width=0.75\linewidth]{Nonslide/Sheepshank_from_a_Handcuff_Knot}}
		}
	\caption{Колышка с рифом.}\label{ris:Sheepshank_from_a_Handcuff_Knot}
\end{figure}

\subsection{Способы фиксации петель Sheepshank Knot}

Четыре способа фиксации петель Бараньей ноги. В вариантах А используется наложение марок. В вариантах Б используется свайка как с маркой так и без.

\begin{figure}[H]\centering
	\subfloat[С марками]{\label{ris:Fix_Sheepshank_Knot_1}
	\tcbox[enhanced jigsaw,colframe=black,opacityframe=0.5,opacityback=0.5]
		{\centering
			\includesvg[width=0.75\linewidth]{Nonslide/Fix_Sheepshank_Knot}}
		}
\end{figure}

\begin{figure}[H]\centering
	\subfloat[Со свайками]{\label{ris:Fix_Sheepshank_Knot_2}
	\tcbox[enhanced jigsaw,colframe=black,opacityframe=0.5,opacityback=0.5]
		{\centering
			\includesvg[width=0.75\linewidth]{Nonslide/Fix_Sheepshank_Knot_1}}
		}
	\caption{Способы фиксации петель Бараньей ноги.}\label{ris:Fix_Sheepshank_Knot}
\end{figure}

\subsection{Sheepshank с Clove Hitches на концах}

\begin{figure}[H]\centering
	\begin{minipage}{1\linewidth}
		\begin{center}
			\tcbox[enhanced jigsaw,colframe=black,opacityframe=0.5,opacityback=0.5]
			{\centering{\includesvg[width=0.75\linewidth]{Nonslide/Sheepshank_with_Clove_Hitches_at_each_end}}}
		\end{center}
	\end{minipage}
\caption{Sheepshank с Clove Hitches на концах.}
\label{ris:Sheepshank_with_Clove_Hitches_at_each_end}
\end{figure}

\subsection{Sheepshank with a Sword Knot}

Sheepshank with a Sword Knot, Navy Sheepshank, Man-o’-War Sheepshank.

\begin{figure}[H]\centering
	\subfloat[Завязывание]{\label{ris:Sheepshank_with_a_Sword_Knot_1}
	\tcbox[enhanced jigsaw,colframe=black,opacityframe=0.5,opacityback=0.5]
		{\centering
			\includesvg[width=0.8\linewidth]{Nonslide/Sheepshank_with_a_Sword_Knot_1}}
		}
\end{figure}

\begin{figure}[H]\centering
	\subfloat[Результат]{\label{ris:Sheepshank_with_a_Sword_Knot_2}
	\tcbox[enhanced jigsaw,colframe=black,opacityframe=0.5,opacityback=0.5]
		{\centering
			\includesvg[width=0.8\linewidth]{Nonslide/Sheepshank_with_a_Sword_Knot}}
		}
	\caption{Sheepshank with a Sword Knot.}\label{ris:Sheepshank_with_a_Sword_Knot}
\end{figure}

\addtocounter{SheepshankNoName}{1}

\subsection{Sheepshank без названия \arabic{SheepshankNoName}}

\begin{figure}[H]\centering
	\subfloat[Завязывание]{\label{ris:Sheepshank_noname_1_1}
	\tcbox[enhanced jigsaw,colframe=black,opacityframe=0.5,opacityback=0.5]
		{\centering
			\includesvg[width=0.8\linewidth]{Nonslide/Sheepshank_noname}}
		}
\end{figure}

\begin{figure}[H]\centering
	\subfloat[Результат]{\label{ris:Sheepshank_noname_1_2}
	\tcbox[enhanced jigsaw,colframe=black,opacityframe=0.5,opacityback=0.5]
		{\centering
			\includesvg[width=0.85\linewidth]{Nonslide/Sheepshank_noname_1}}
		}
	\caption{Sheepshank без названия \arabic{SheepshankNoName}.}\label{ris:SheepshankNoName}
\end{figure}

\subsection{Олимпийский узел}

\enquote{Two Hearts That Beat as One} - Два сердца, бьющиеся как одно. В декоративных целях можно сделать и больше шлагов.

\begin{figure}[H]\centering
	\subfloat[Завязывание]{\label{ris:Olimpik_1}
	\tcbox[enhanced jigsaw,colframe=black,opacityframe=0.5,opacityback=0.5]
		{\centering
			\includesvg[width=0.9\linewidth]{Nonslide/Olimpik}}
		}
\end{figure}

\begin{figure}[H]\centering
	\subfloat[Результат]{\label{ris:Olimpik_2}
	\tcbox[enhanced jigsaw,colframe=black,opacityframe=0.5,opacityback=0.5]
		{\centering
			\includesvg[width=1\linewidth]{Nonslide/Olimpik_1}}
		}
	\caption{Олимпийский узел.}\label{ris:Olimpik}
\end{figure}

\subsection{Кокарда}

Узел в литературе не описан.

\begin{figure}[H]\centering
	\subfloat[Завязывание]{\label{ris:Kokard_1}
	\tcbox[enhanced jigsaw,colframe=black,opacityframe=0.5,opacityback=0.5]
		{\centering
			\includesvg[width=0.9\linewidth]{Nonslide/Kokard}}
		}
\end{figure}

\begin{figure}[H]\centering
	\subfloat[Двухпетельный вариант]{\label{ris:Kokard_2}
	\tcbox[enhanced jigsaw,colframe=black,opacityframe=0.5,opacityback=0.5]
		{\centering
			\includesvg[width=0.9\linewidth]{Nonslide/Kokard_1}}
		}
\end{figure}

\begin{figure}[H]\centering
	\subfloat[Трехпетельный вариант]{\label{ris:Kokard_3}
	\tcbox[enhanced jigsaw,colframe=black,opacityframe=0.5,opacityback=0.5]
		{\centering
			\includesvg[width=1\linewidth]{Nonslide/Kokard_2}}
		}
\end{figure}

\begin{figure}[H]\centering
	\subfloat[Пятипетельный вариант]{\label{ris:Kokard_4}
	\tcbox[enhanced jigsaw,colframe=black,opacityframe=0.5,opacityback=0.5]
		{\centering
			\includesvg[width=1\linewidth]{Nonslide/Kokard_3}}
		}
	\caption{Кокарда.}\label{ris:Kokard}
\end{figure}
