\section{Схватывающие}\label{transformers}

Они же Трансформируемые. Узлы имеют уникальное свойство - окончательное завязывание происходит дистанционно.

\subsection{Рюрик}

%TODO Проверить схватывающая петля?

\begin{figure}[H]\centering
	\begin{minipage}{1\linewidth}
		\begin{center}
			\tcbox[enhanced jigsaw,colframe=black,opacityframe=0.5,opacityback=0.5]
			{\centering{\includesvg[width=0.55\linewidth]{Slide/Rurik}}}
		\end{center}
	\end{minipage}
\caption{Рюрик.}
\label{ris:Rurik}
\end{figure}

\subsection{Перекрестный зажим}

%TODO Проверить схватывающая петля?

\begin{figure}[H]\centering
	\begin{minipage}{1\linewidth}
		\begin{center}
			\tcbox[enhanced jigsaw,colframe=black,opacityframe=0.5,opacityback=0.5]
			{\centering{\includesvg[width=0.55\linewidth]{Slide/Perekrestny_zazhim}}}
		\end{center}
	\end{minipage}
\caption{Перекрестный зажим.}
\label{ris:Perekrestny_zazhim}
\end{figure}

\subsection{Hawser Bowline}

\begin{figure}[H]\centering
	\subfloat[Завязывание]{\label{ris:Hawser_Bowline_1}
	\tcbox[enhanced jigsaw,colframe=black,opacityframe=0.5,opacityback=0.5]
		{\centering
			\includesvg[width=0.55\linewidth]{Bowline/Hawser_Bowline}}
		}
\hfil
	\subfloat[Фиксирование]{\label{ris:Hawser_Bowline_2}
	\tcbox[enhanced jigsaw,colframe=black,opacityframe=0.5,opacityback=0.5]
		{\centering
			\includesvg[width=0.55\linewidth]{Slide/Flagstaff_Knot}}
		}
\end{figure}

Не до конца затянутый узел накидывается на опору, после чего рывком затягиваем.

\begin{figure}[H]\centering
	\subfloat[Результат]{\label{ris:Hawser_Bowline_3}
	\tcbox[enhanced jigsaw,colframe=black,opacityframe=0.5,opacityback=0.5]
		{\centering
			\includesvg[width=0.55\linewidth]{Slide/Flagstaff_Knot_1}}
		}
	\caption{Hawser Bowline.}\label{ris:Hawser_Bowline}
\end{figure}

Другое название - Flagstaff Knot, Шлагштоковый узел. В данном случае, идея в том, что потянув за коренной конец - подымаем флаг, дернув за ходовой - фиксируем узел.

\subsection{Эскимосский Булинь}

Он же Казачий или Казацкий узел, Эскимосская петля, Крабья петля, Затягивающийся огон, Затяжной огон. Английское название - Eskimo Bowline, Crab Noose, Crabber’s Eye Knot или Crossing Running Knot. В качестве трансформируемой петли в литературе не рассматривался, только в качестве незатягивающейся петли.

\begin{figure}[H]\centering
	\subfloat[Завязывание]{\label{ris:Kazak_1}
	\tcbox[enhanced jigsaw,colframe=black,opacityframe=0.5,opacityback=0.5]
		{\centering
			\includesvg[width=0.55\linewidth]{Slide/Crab_noose}}
		}
\hfil
	\subfloat[Результат]{\label{ris:Kazak_2}
	\tcbox[enhanced jigsaw,colframe=black,opacityframe=0.5,opacityback=0.5]
		{\centering
			\includesvg[width=0.55\linewidth]{Slide/Crab_noose_1}}
		}
\end{figure}

Способов вязки у этого узла существует не меньше, чем у Беседочного (см. раздел \ref{sec:Zavyazyvanie_bowline}). В частности, на рисунке показан способ, похожий на вязку Hawser Bowline (рис.~\ref{ris:Hawser_Bowline}). Узлы имеют одинаковую структуру и отличаются всего лишь одной деталью - в Беседочном ходовой конец проходит через петлю коренного снизу-вверх и огибает коренной конец, а в Эскимосском наоборот, сверху-вниз и огибает боковину рабочей петли. Различные варианты усложнения и развития узла также можно позаимствовать у Беседочного (см. раздел \ref{sec:Fiks_bowline}).

\begin{figure}[H]\centering
	\subfloat[Быстроразвязывающийся вариант]{\label{ris:Kazak_3}
	\tcbox[enhanced jigsaw,colframe=black,opacityframe=0.5,opacityback=0.5]
		{\centering
			\includesvg[width=0.55\linewidth]{Slide/Kazak}}
		}
	\caption{Эскимосский Булинь.}\label{ris:Kazak}
\end{figure}

\subsection{Кабестановая петля}

\begin{figure}[H]\centering
	\subfloat[Завязывание]{\label{ris:Capstan_1}
	\tcbox[enhanced jigsaw,colframe=black,opacityframe=0.5,opacityback=0.5]
		{\centering
			\includesvg[width=0.6\linewidth]{Slide/Capstan}}
		}
\end{figure}

\begin{figure}[H]\centering
	\subfloat[Результат]{\label{ris:Capstan_2}
	\tcbox[enhanced jigsaw,colframe=black,opacityframe=0.5,opacityback=0.5]
		{\centering
			\includesvg[width=0.6\linewidth]{Slide/Capstan_1}}
		}
	\caption{Кабестановая петля.}\label{ris:Capstan}
\end{figure}

Он же Capstan Knot.
