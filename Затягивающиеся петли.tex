\section{Затягивающиеся петли}

Такие узлы могут называться скользящими или бегущими.

\subsection{Арбор}

Noose, Noose Knot или Simple Noose. Простой узел. Осевой Узел, если завязывается вокруг опоры. Department-Store knot. Если потянуть за коренной конец - узел развяжется. Быстроразвязывающийся Простой узел (рис.~\ref{ris:Single_Knot}).

\begin{figure}[H]\centering
	\subfloat[Завязывание]{\label{ris:Arbor_1}
	\tcbox[enhanced jigsaw,colframe=black,opacityframe=0.5,opacityback=0.5]
		{\centering
			\includesvg[width=0.55\linewidth]{Slide/Arbor}}
		}
\end{figure}
% \vfill
\begin{figure}[H]\centering
	\subfloat[Результат]{\label{ris:Arbor_2}
	\tcbox[enhanced jigsaw,colframe=black,opacityframe=0.5,opacityback=0.5]
		{\centering
			\includesvg[width=0.55\linewidth]{Slide/Arbor_1}}
		}
	\caption{Арбор.}\label{ris:Arbor}
\end{figure}

\subsection{Похожий на Арбор}

\begin{figure}[H]\centering
	\begin{minipage}{1\linewidth}
		\begin{center}
			\tcbox[enhanced jigsaw,colframe=black,opacityframe=0.5,opacityback=0.5]
			{\centering{\includesvg[width=0.55\linewidth]{Slide/Arbor_2}}}
		\end{center}
	\end{minipage}
\caption{Похожий на Арбор.}
\label{ris:Arbor_seems}
\end{figure}

Узел, очень похожий на Арбор, тоже затягивающийся, но не развязывающийся. Принцип узла Хонда (рис.~\ref{ris:Honda}). Именно поэтому завязать его можно только прямым способом.

\subsection{Аркан}

\begin{figure}[H]\centering
	\begin{minipage}{1\linewidth}
		\begin{center}
			\tcbox[enhanced jigsaw,colframe=black,opacityframe=0.5,opacityback=0.5]
			{\centering{\includesvg[width=0.67\linewidth]{Slide/Arkan}}}
		\end{center}
	\end{minipage}
\caption{Аркан.}
\label{ris:Arkan}
\end{figure}

Другие названия - Хонда, лассо.

\subsection{Швартовая (швартовочная) петля}

\begin{figure}[H]\centering
	\begin{minipage}{1\linewidth}
		\begin{center}
			\tcbox[enhanced jigsaw,colframe=black,opacityframe=0.5,opacityback=0.5]
			{\centering{\includesvg[width=0.6\linewidth]{Slide/Shvartovochnaya}}}
		\end{center}
	\end{minipage}
\caption{Швартовая (швартовочная) петля.}
\label{ris:Shvartovochnaya}
\end{figure}

В случае использования в качестве штыка получается Причальный узел (рис.~\ref{ris:Prichalny}).

\subsection{Двойной Арбор}

Он же Близнецы, Сообщающиеся Сосуды.

\begin{figure}[H]\centering
	\subfloat[Двойной Арбор]{\label{ris:Double_arbor_1}
	\tcbox[enhanced jigsaw,colframe=black,opacityframe=0.5,opacityback=0.5,height=6.5cm]
		{\centering
			\includesvg[width=0.35\linewidth]{Slide/Double_arbor}}
		}
\hfil
	\subfloat[Обратный двойной Арбор]{\label{ris:Double_arbor_2}
	\tcbox[enhanced jigsaw,colframe=black,opacityframe=0.5,opacityback=0.5,height=6.5cm]
		{\centering
			\includesvg[width=0.35\linewidth]{Slide/Double_arbor_revers}}
		}
	\caption{Двойной Арбор.}\label{ris:Double_arbor}
\end{figure}

По своим свойствам похож на Tom Fool’s Knot.

\subsection{Цепочка Арбор}

\begin{figure}[H]\centering
	\begin{minipage}{1\linewidth}
		\begin{center}
			\tcbox[enhanced jigsaw,colframe=black,opacityframe=0.5,opacityback=0.5]
			{\centering{\includesvg[width=0.55\linewidth]{Slide/Arbor_tcepochka}}}
		\end{center}
	\end{minipage}
\caption{Цепочка Арбор.}
\label{ris:Arbor_tcepochka}
\end{figure}

\subsection{Handcuff Knot}

\begin{figure}[H]\centering
	\subfloat[Завязывание]{\label{ris:Handcuff_Knot_1}
	\tcbox[enhanced jigsaw,colframe=black,opacityframe=0.5,opacityback=0.5]
		{\centering
			\includesvg[width=0.45\linewidth]{Slide/Handcuff_Knot}}
		}
\end{figure}
% \vfill
\begin{figure}[H]\centering
	\subfloat[Результат]{\label{ris:Handcuff_Knot_2}
	\tcbox[enhanced jigsaw,colframe=black,opacityframe=0.5,opacityback=0.5]
		{\centering
			\includesvg[width=0.5\linewidth]{Slide/Handcuff_Knot_1}}
		}
	\caption{Handcuff Knot.}\label{ris:Handcuff_Knot}
\end{figure}

Наручники, Кандалы, Кандальный узел, Двойной топовый, Незатягивающееся Стремя.

\subsection{Fireman's Chair Knot}

%TODO См. Баранья нога!!!

\begin{figure}[H]\centering
	\subfloat[Завязывание]{\label{ris:Firemans_Chair_Knot_1}
	\tcbox[enhanced jigsaw,colframe=black,opacityframe=0.5,opacityback=0.5]
		{\centering
			\includesvg[width=0.45\linewidth]{Slide/Firemans_Chair_Knot}}
		}
\end{figure}
% \vfill
\begin{figure}[H]\centering
	\subfloat[Результат]{\label{ris:Firemans_Chair_Knot_2}
	\tcbox[enhanced jigsaw,colframe=black,opacityframe=0.5,opacityback=0.5]
		{\centering
			\includesvg[width=0.5\linewidth]{Slide/Firemans_Chair_Knot_1}}
		}
	\caption{Fireman's Chair Knot.}\label{ris:Firemans_Chair_Knot}
\end{figure}

Handcuff Knot с дополнительным шлагом на каждой петле.

\subsection{Tom Fool or Tom Fool's Knot}

Он же Tom Fool's Knot, Дурацкий, Пьяный, Узел Томаса (глупого Томаса) или Узел Дураков. Фактически, обычный Бантик.

\begin{figure}[H]\centering
	\subfloat[Завязывание\\первый вариант]{\label{ris:Tom_Fool_1}
	\tcbox[enhanced jigsaw,colframe=black,opacityframe=0.5,opacityback=0.5]
		{\centering
			\includesvg[width=0.4\linewidth]{Slide/Tom_Fool_or_Tom_Fools_Knot}}
		}
\end{figure}
% \vfill
\begin{figure}[H]\centering
	\subfloat[Результат\\первый вариант]{\label{ris:Tom_Fool_2}
	\tcbox[enhanced jigsaw,colframe=black,opacityframe=0.5,opacityback=0.5]
		{\centering
			\includesvg[width=0.5\linewidth]{Slide/Tom_Fool_or_Tom_Fools_Knot_2}}
		}
\end{figure}
% \vfill
\begin{figure}[H]\centering
	\subfloat[Завязывание\\второй вариант]{\label{ris:Tom_Fool_3}
	\tcbox[enhanced jigsaw,colframe=black,opacityframe=0.5,opacityback=0.5,height=4cm]
		{\centering
			\includesvg[width=0.3\linewidth]{Slide/Tom_Fool_or_Tom_Fools_Knot_1}}
		}
\hfil
	\subfloat[Результат\\второй вариант]{\label{ris:Tom_Fool_4}
	\tcbox[enhanced jigsaw,colframe=black,opacityframe=0.5,opacityback=0.5,height=4cm]
		{\centering
			\includesvg[width=0.35\linewidth]{Slide/Tom_Fool_or_Tom_Fools_Knot_3}}
		}
	\caption{Tom Fool.}\label{ris:Tom_Fool}
\end{figure}

\subsection{Decorative loop}

\begin{figure}[H]\centering
	\subfloat[Завязывание]{\label{ris:Decorative_loop_1_1}
	\tcbox[enhanced jigsaw,colframe=black,opacityframe=0.5,opacityback=0.5]
		{\centering
			\includesvg[width=0.6\linewidth]{Slide/Decorative_loop_1}}
		}
\end{figure}
% \vfill
\begin{figure}[H]\centering
	\subfloat[Двусторонняя петля]{\label{ris:Decorative_loop_1_2}
	\tcbox[enhanced jigsaw,colframe=black,opacityframe=0.5,opacityback=0.5]
		{\centering
			\includesvg[width=0.6\linewidth]{Slide/Decorative_loop}}
		}
\end{figure}
% \vfill
\begin{figure}[H]\centering
	\subfloat[Односторонняя петля]{\label{ris:Decorative_loop_1_3}
	\tcbox[enhanced jigsaw,colframe=black,opacityframe=0.5,opacityback=0.5]
		{\centering
			\includesvg[width=0.6\linewidth]{Slide/Decorative_loop_4}}
		}
	\caption{Decorative loop.}\label{ris:Decorative_loop}
\end{figure}

\subsection{Decorative loop 2}

Вариация на тему предыдущего узла.

\begin{figure}[H]\centering
	\subfloat[Завязывание]{\label{ris:Decorative_loop_2_1}
	\tcbox[enhanced jigsaw,colframe=black,opacityframe=0.5,opacityback=0.5]
		{\centering
			\includesvg[width=0.6\linewidth]{Slide/Decorative_loop_2}}
		}
\end{figure}
% \vfill
\begin{figure}[H]\centering
	\subfloat[Двусторонняя петля]{\label{ris:Decorative_loop_2_2}
	\tcbox[enhanced jigsaw,colframe=black,opacityframe=0.5,opacityback=0.5]
		{\centering
			\includesvg[width=0.6\linewidth]{Slide/Decorative_loop_3}}
		}
	\caption{Decorative loop 2.}\label{ris:Decorative_loop_2}
\end{figure}

\subsection{Арбор быстроразвязывающийся}

Он же Slip Noose, Halter Hitch, Развязывающийся бегущий простой узел.

\begin{figure}[H]\centering
	\subfloat[Завязывание\\(первый вариант)]{\label{ris:Arbor_fast_1}
	\tcbox[enhanced jigsaw,colframe=black,opacityframe=0.5,opacityback=0.5]
		{\centering
			\includesvg[width=0.6\linewidth]{Slide/Arbor_fast_1}}
		}
\end{figure}
% \vfill
\begin{figure}[H]\centering
	\subfloat[Завязывание\\(второй вариант)]{\label{ris:Arbor_fast_2}
	\tcbox[enhanced jigsaw,colframe=black,opacityframe=0.5,opacityback=0.5]
		{\centering
			\includesvg[width=0.6\linewidth]{Slide/Arbor_fast_1_1}}
		}
\end{figure}
% \vfill
\begin{figure}[H]\centering
	\subfloat[Результат]{\label{ris:Arbor_fast_3}
	\tcbox[enhanced jigsaw,colframe=black,opacityframe=0.5,opacityback=0.5]
		{\centering
			\includesvg[width=0.6\linewidth]{Slide/Arbor_fast}}
		}
\end{figure}
% \vfill
\begin{figure}[H]\centering
	\subfloat[Временная фиксация от случайного развязывания]{\label{ris:Arbor_fast_4}
	\tcbox[enhanced jigsaw,colframe=black,opacityframe=0.5,opacityback=0.5]
		{\centering
			\includesvg[width=0.6\linewidth]{Slide/Arbor_fast_2}}
		}
	\caption{Арбор быстроразвязывающийся.}\label{ris:Arbor_fast}
\end{figure}

\subsection{Эвенкийский узел}

\begin{figure}[H]\centering
	\subfloat[Завязывание]{\label{ris:Siberian_Hitch_1}
	\tcbox[enhanced jigsaw,colframe=black,opacityframe=0.5,opacityback=0.5]
		{\centering
			\includesvg[width=0.6\linewidth]{Slide/Siberian_Hitch_1}}
		}
\end{figure}
% \vfill
\begin{figure}[H]\centering
	\subfloat[Результат]{\label{ris:Siberian_Hitch_2}
	\tcbox[enhanced jigsaw,colframe=black,opacityframe=0.5,opacityback=0.5]
		{\centering
			\includesvg[width=0.6\linewidth]{Slide/Siberian_Hitch}}
		}
	\caption{Эвенкийский узел.}\label{ris:Siberian_Hitch}
\end{figure}

Siberian Hitch. Его удобно завязывать даже в варежках. Отличается от быстроразвязывающего Арбора дополнительным твистом. Предохранить от случайного развязывания можно так же, как Быстроразвязывающийся Арбор (рис.~\ref{ris:Arbor_fast_4}).

\subsection{Скользящая восьмерка}

\begin{figure}[H]\centering
	\begin{minipage}{1\linewidth}
		\begin{center}
			\tcbox[enhanced jigsaw,colframe=black,opacityframe=0.5,opacityback=0.5]
			{\centering{\includesvg[width=0.55\linewidth]{Slide/Sliding_figure_of_eight}}}
		\end{center}
	\end{minipage}
\caption{Скользящая восьмерка.}
\label{ris:Sliding_figure_of_eight}
\end{figure}

По английски - Figure-of-Eight Noose.

\subsection{Скользящая восьмерка 2}

\begin{figure}[H]\centering
	\begin{minipage}{1\linewidth}
		\begin{center}
			\tcbox[enhanced jigsaw,colframe=black,opacityframe=0.5,opacityback=0.5]
			{\centering{\includesvg[width=0.55\linewidth]{Slide/Sliding_figure_of_eight_2}}}
		\end{center}
	\end{minipage}
\caption{Скользящая восьмерка 2.}
\label{ris:Sliding_figure_of_eight_2}
\end{figure}

Другой вариант Скользящей восьмерки.

\subsection{Скользящая восьмерка 3}

\begin{figure}[H]\centering
	\begin{minipage}{1\linewidth}
		\begin{center}
			\tcbox[enhanced jigsaw,colframe=black,opacityframe=0.5,opacityback=0.5]
			{\centering{\includesvg[width=0.55\linewidth]{Slide/Sliding_figure_of_eight_3}}}
		\end{center}
	\end{minipage}
\caption{Скользящая восьмерка 3.}
\label{ris:Sliding_figure_of_eight_3}
\end{figure}

%TODO Сослаться на питонов узел

Третий вариант Скользящей восьмерки (Buntline Extinguisher). Фактически, Питонов узел, завязанный на собственном коренном конце. Если завязать Констриктор, то получится Браконьерский узел (рис.~\ref{ris:Brakonersky}).

\subsection{Браконьерский узел}

\begin{figure}[H]\centering
	\begin{minipage}{1\linewidth}
		\begin{center}
			\tcbox[enhanced jigsaw,colframe=black,opacityframe=0.5,opacityback=0.5]
			{\centering{\includesvg[width=0.85\linewidth]{Slide/Brakonersky}}}
		\end{center}
	\end{minipage}
\caption{Браконьерский узел.}
\label{ris:Brakonersky}
\end{figure}

Он же Мертвая петля или Двойная бегущая удавка, Poacher knot, Scaffold knot. Так же, Констриктор, завязанный на собственном коренном конце.

\subsection{Горская петля}

\begin{figure}[H]\centering
	\begin{minipage}{1\linewidth}
		\begin{center}
			\tcbox[enhanced jigsaw,colframe=black,opacityframe=0.5,opacityback=0.5]
			{\centering{\includesvg[width=0.9\linewidth]{Slide/Gorskaya}}}
		\end{center}
	\end{minipage}
\caption{Горская петля.}
\label{ris:Gorskaya}
\end{figure}

Другое название - Петля с блокадой. Ограничиваемая. Внутренний узел ограничивает минимальный диаметр петли. Таким способом можно ограничивать любую затягивающуюся петлю.

\subsection{Курьерский узел}

\begin{figure}[H]\centering
	\subfloat[Завязывание]{\label{ris:Kurersky_1}
	\tcbox[enhanced jigsaw,colframe=black,opacityframe=0.5,opacityback=0.5,height=7cm]
		{\centering
			\includesvg[width=0.32\linewidth]{Slide/Kurersky_1}}
		}
\hfil
	\subfloat[Результат]{\label{ris:Kurersky_2}
	\tcbox[enhanced jigsaw,colframe=black,opacityframe=0.5,opacityback=0.5,height=7cm]
		{\centering
			\includesvg[width=0.3\linewidth]{Slide/Kurersky}}
		}
	\caption{Курьерский узел.}\label{ris:Kurersky}
\end{figure}

\subsection{Manharness Knot}

\begin{figure}[H]\centering
	\subfloat[Завязывание]{\label{ris:Manharness_Knot_1}
	\tcbox[enhanced jigsaw,colframe=black,opacityframe=0.5,opacityback=0.5]
		{\centering
			\includesvg[width=0.45\linewidth]{Slide/Manharness_Knot_2}}
		}
\hfill
	\subfloat[Завязывание]{\label{ris:Manharness_Knot_2}
	\tcbox[enhanced jigsaw,colframe=black,opacityframe=0.5,opacityback=0.5]
		{\centering
			\includesvg[width=0.4\linewidth]{Slide/Manharness_Knot_1}}
		}
\end{figure}
% \vfill
\begin{figure}[H]\centering
	\subfloat[Результат]{\label{ris:Manharness_Knot_3}
	\tcbox[enhanced jigsaw,colframe=black,opacityframe=0.5,opacityback=0.5]
		{\centering
			\includesvg[width=0.5\linewidth]{Slide/Manharness_Knot}}
		}
	\caption{Manharness Knot.}\label{ris:Manharness_Knot}
\end{figure}

\subsection{Силковый узел}

\begin{figure}[H]\centering
	\subfloat[Завязывание]{\label{ris:Silkovy_1}
	\tcbox[enhanced jigsaw,colframe=black,opacityframe=0.5,opacityback=0.5,height=8.5cm]
		{\centering
			\includesvg[angle=90,width=0.66\linewidth]{Slide/Silkovy}}
		}
\hfil
	\subfloat[Результат]{\label{ris:Silkovy_2}
	\tcbox[enhanced jigsaw,colframe=black,opacityframe=0.5,opacityback=0.5,height=8.5cm]
		{\centering
			\includesvg[angle=90,width=0.65\linewidth]{Slide/Silkovy_1}}
		}
	\caption{Силковый узел.}\label{ris:Silkovy}
\end{figure}

\subsection{Висельный узел}

Он же Hangman’s Knot, Hangman’s Noose, Затягивающаяся удавка, Узел висельника, Удушающий, Эшафотный, Узел Джека Кеча. Обычно делают 9 оборотов – \enquote{если у человека 9 жизней, как у кошки, то каждый оборот пусть заберет по одной}.

%TODO (а если провести первый шлаг внутрь петли или восьмеркой?).

\begin{figure}[H]\centering
	\begin{minipage}{1\linewidth}
		\begin{center}
			\tcbox[enhanced jigsaw,colframe=black,opacityframe=0.5,opacityback=0.5]
			{\centering{\includesvg[width=0.7\linewidth]{Slide/Hangmans_Knot}}}
		\end{center}
	\end{minipage}
\caption{Висельный узел.}
\label{ris:Hangmans_Knot}
\end{figure}

\subsection{Эшафотный узел}

Он же Scaffold Knot from Diderot’s Encyclopedia, Узел Линча. Подобно простому узлу и восьмерке кровавый узел применяется для вязки многих более сложных узлов и петель. Во всех случаях кровавый узел вяжется на основе, в роли которой может выступать свой или чужой коренные концы, цевье крючка и т.п. Пожалуй, простейшими производными кровавого узла являются две затягивающиеся, или скользящие, петли - удушающая (эшафотная) и шпиндельная (рис.~\ref{ris:Shpindel_loop}).

\begin{figure}[H]\centering
	\begin{minipage}{1\linewidth}
		\begin{center}
			\tcbox[enhanced jigsaw,colframe=black,opacityframe=0.5,opacityback=0.5]
			{\centering{\includesvg[width=0.75\linewidth]{Slide/Scaffold_Knot}}}
		\end{center}
	\end{minipage}
\caption{Эшафотный узел.}
\label{ris:Scaffold_Knot}
\end{figure}

Вязка удушающей петли начинается с открытой петли - коренной и ходовой концы параллельны и направлены в одну сторону. Затем ходовым концом делают необходимое число обметывающих шлагов в сторону вершины петли и пропускают конец под шлагами в направлении коренного конца. Шпиндельная петля отличается от предыдущей тем, что вяжется на основе закрытой петли, в которой ходовой конец пересекает коренной и направлен в противоположную по отношению к нему сторону. Шлаги кровавого узла делают в направлении основания закрытой петли - пересечения коренного и ходового концов. Следствием использования к качестве основы двух различных петель, открытой и закрытой, становится то, что удушающая петля образует продолжение коренного конца, а шпиндельная - расположена сбоку от него.

\subsection{Шпиндельная петля}

Петля на основе кровавого узла. В отличие от эшафотного узла начинается с закрытой петли.

\begin{figure}[H]\centering
	\begin{minipage}{1\linewidth}
		\begin{center}
			\tcbox[enhanced jigsaw,colframe=black,opacityframe=0.5,opacityback=0.5]
			{\centering{\includesvg[width=0.6\linewidth]{Slide/Shpindel_loop}}}
		\end{center}
	\end{minipage}
\caption{Шпиндельная петля.}
\label{ris:Shpindel_loop}
\end{figure}

\subsection{Adjustable Loop}

\begin{figure}[H]\centering
	\begin{minipage}{1\linewidth}
		\begin{center}
			\tcbox[enhanced jigsaw,colframe=black,opacityframe=0.5,opacityback=0.5]
			{\centering{\includesvg[width=0.6\linewidth]{Slide/Adjustable_Loop}}}
		\end{center}
	\end{minipage}
\caption{Adjustable Loop.}
\label{ris:Adjustable_Loop}
\end{figure}

Регулируемая петля-захват.

\subsection{Newgate Knot}

\begin{figure}[H]\centering
	\begin{minipage}{1\linewidth}
		\begin{center}
			\tcbox[enhanced jigsaw,colframe=black,opacityframe=0.5,opacityback=0.5]
			{\centering{\includesvg[width=0.65\linewidth]{Slide/Newgate_Knot}}}
		\end{center}
	\end{minipage}
\caption{Newgate Loop.}
\label{ris:Newgate_Knot}
\end{figure}

\subsection{Gallows Knot}

\begin{figure}[H]\centering
	\begin{minipage}{1\linewidth}
		\begin{center}
			\tcbox[enhanced jigsaw,colframe=black,opacityframe=0.5,opacityback=0.5]
			{\centering{\includesvg[width=0.7\linewidth]{Slide/Gallows_Knot}}}
		\end{center}
	\end{minipage}
\caption{Gallows Loop.}
\label{ris:Gallows_Knot}
\end{figure}

Он же Скользящая петля, Arbor Knot. От классического Арбора (рис.~\ref{ris:Arbor}) отличается только количеством шлагов.

\subsection{Ichabod Knot}

\begin{figure}[H]\centering
	\begin{minipage}{1\linewidth}
		\begin{center}
			\tcbox[enhanced jigsaw,colframe=black,opacityframe=0.5,opacityback=0.5]
			{\centering{\includesvg[width=0.7\linewidth]{Slide/Ichabod_Knot}}}
		\end{center}
	\end{minipage}
\caption{Ichabod Loop.}
\label{ris:Ichabod_Knot}
\end{figure}

\subsection{Gibbet knot}

\begin{figure}[H]\centering
	\begin{minipage}{1\linewidth}
		\begin{center}
			\tcbox[enhanced jigsaw,colframe=black,opacityframe=0.5,opacityback=0.5]
			{\centering{\includesvg[width=0.7\linewidth]{Slide/Gibbet_Knot}}}
		\end{center}
	\end{minipage}
\caption{Gibbet Loop.}
\label{ris:Gibbet_Knot}
\end{figure}

\subsection{Decorative Noose}

\begin{figure}[H]\centering
	\begin{minipage}{1\linewidth}
		\begin{center}
			\tcbox[enhanced jigsaw,colframe=black,opacityframe=0.5,opacityback=0.5]
			{\centering{\includesvg[width=0.7\linewidth]{Slide/Decorative_Noose}}}
		\end{center}
	\end{minipage}
\caption{Decorative Noose.}
\label{ris:Decorative_Noose}
\end{figure}

Похож на Matthew Walker Knot. Можно сделать несколько шлагов вокруг "короны" в направлении красной стрелки (снизу вверх), от этого декоративность узла только выиграет.

\subsection{Double Ring}

\begin{figure}[H]\centering
	\subfloat[Завязывание]{\label{ris:Double_Ring_1}
	\tcbox[enhanced jigsaw,colframe=black,opacityframe=0.5,opacityback=0.5,height=7cm]
		{\centering
			\includesvg[width=0.45\linewidth]{Slide/Double_Ring}}
		}
\hfil
	\subfloat[Результат]{\label{ris:Double_Ring_2}
	\tcbox[enhanced jigsaw,colframe=black,opacityframe=0.5,opacityback=0.5,height=7cm]
		{\centering
			\includesvg[width=0.3\linewidth]{Slide/Double_Ring_1}}
		}
	\caption{Double Ring.}\label{ris:Double_Ring}
\end{figure}

Double Ringing Knot or Tag Knot. Скользящая глухая петля. В случае завязывания на опоре получается Double Ring Hitch (рис.~\ref{ris:Double_Strap_Hitch_1}).

\subsection{Регулирующая петля}

Double Prusik (рис.~\ref{ris:Double_Prusik_Knot}), он же Двойная скользящая глухая петля или Регулируемая петля. Представляет из себя Double Ringing Knot (рис.~\ref{ris:Double_Ring}) с дополнительными шлагами вокруг коренных концов. По аналогии с Пруссиком, можно вязать несколько разновидностей, как с симметричными плечами, так и с несимметричными.

\begin{figure}[H]\centering
	\begin{minipage}{1\linewidth}
		\begin{center}
			\tcbox[enhanced jigsaw,colframe=black,opacityframe=0.5,opacityback=0.5]
			{\centering{\includesvg[width=0.7\linewidth]{Slide/Reguliruemaya_loop}}}
		\end{center}
	\end{minipage}
\caption{Регулирующая петля.}
\label{ris:Reguliruemaya_loop}
\end{figure}

\subsection{Тарбука}

\begin{figure}[H]\centering
	\begin{minipage}{1\linewidth}
		\begin{center}
			\tcbox[enhanced jigsaw,colframe=black,opacityframe=0.5,opacityback=0.5]
			{\centering{\includesvg[width=0.7\linewidth]{Slide/Tarbuck_Knot}}}
		\end{center}
	\end{minipage}
\caption{Тарбука.}
\label{ris:Tarbuck_Knot}
\end{figure}

Tarbuck Knot. Очень похож на Schwabich (рис.~\ref{ris:Schwabich}), отличается только последним шлагом.

\subsection{Шпульковый узел}

\begin{figure}[H]\centering
	\begin{minipage}{1\linewidth}
		\begin{center}
			\tcbox[enhanced jigsaw,colframe=black,opacityframe=0.5,opacityback=0.5]
			{\centering{\includesvg[width=0.75\linewidth]{Slide/Shpulkovy}}}
		\end{center}
	\end{minipage}
\caption{Шпульковый узел.}
\label{ris:Shpulkovy}
\end{figure}

\section{Из незатягивающихся}\label{iz_nonslide}

Из любой незатягивающейся петли (или огона) можно сделать затягивающуюся. Например, из Хонды или Bowstring Knot получается Lariat or Lasso Noose. А из Seized Eye (рис.~\ref{ris:Seized_Eye}) получается Running Eye, из него, в свою очередь, получится Running Eye Hitch (рис.~\ref{ris:Running_Eye_Hitch}).

\begin{figure}[H]\centering
	\begin{minipage}{1\linewidth}
		\begin{center}
			\tcbox[enhanced jigsaw,colframe=black,opacityframe=0.5,opacityback=0.5]
			{\centering{\includesvg[width=0.5\linewidth]{Slide/Running_Eye}}}
		\end{center}
	\end{minipage}
\caption{Running Eye.}
\label{ris:Running_Eye}
\end{figure}

\subsection{Бегущий Булинь}

Он же Аркан, Bowline Slip Knot или Running Bowline.

\begin{figure}[H]\centering
	\begin{minipage}{1\linewidth}
		\begin{center}
			\tcbox[enhanced jigsaw,colframe=black,opacityframe=0.5,opacityback=0.5]
			{\centering{\includesvg[width=0.65\linewidth]{Slide/Bowline_Slip_Knot_Linfit_way}}}
		\end{center}
	\end{minipage}
\caption{Бегущий Булинь.}
\label{ris:Bowline_Slip_Knot_Linfit_way}
\end{figure}

\subsection{Double Noose}

На примере Булиня. В одиночную петлю или огон проходит не коренной конец, а просовывается петля. Конкретно этот узел используется в качестве уздечки.

\begin{figure}[H]\centering
	\begin{minipage}{1\linewidth}
		\begin{center}
			\tcbox[enhanced jigsaw,colframe=black,opacityframe=0.5,opacityback=0.5]
			{\centering{\includesvg[width=0.65\linewidth]{Slide/Double_Noose}}}
		\end{center}
	\end{minipage}
\caption{Double Noose.}
\label{ris:Double_Noose}
\end{figure}

\subsection{Double Noose из Spanish Bowline}

\begin{figure}[H]\centering
	\begin{minipage}{1\linewidth}
		\begin{center}
			\tcbox[enhanced jigsaw,colframe=black,opacityframe=0.5,opacityback=0.5]
			{\centering{\includesvg[width=0.55\linewidth]{Slide/Double_Noose_1}}}
		\end{center}
	\end{minipage}
\caption{Double Noose из Spanish Bowline (рис.~\ref{ris:Spanish_bowline}).}
\label{ris:Double_Noose_1}
\end{figure}

\subsection{Double Noose из Harness Loop}

\begin{figure}[H]\centering
	\begin{minipage}{1\linewidth}
		\begin{center}
			\tcbox[enhanced jigsaw,colframe=black,opacityframe=0.5,opacityback=0.5]
			{\centering{\includesvg[width=0.55\linewidth]{Slide/Double_Noose_3}}}
		\end{center}
	\end{minipage}
\caption{Double Noose из Harness Loop (рис.~\ref{ris:Harness_Loop}).}
\label{ris:Double_Noose_3}
\end{figure}

\subsection{Triple Noose из Matthew Walker Knot}

\begin{figure}[H]\centering
	\subfloat[Завязывание]{\label{ris:Triple_Noose_1}
	\tcbox[enhanced jigsaw,colframe=black,opacityframe=0.5,opacityback=0.5,height=6.5cm]
		{\centering
			\includesvg[width=0.45\linewidth]{Slide/Triple_Noose}}
		}
\hfil
	\subfloat[Результат]{\label{ris:Triple_Noose_2}
	\tcbox[enhanced jigsaw,colframe=black,opacityframe=0.5,opacityback=0.5,height=6.5cm]
		{\centering
			\includesvg[width=0.34\linewidth]{Slide/Triple_Noose_1}}
		}
	\caption{Triple Noose из Matthew Walker Knot.}\label{ris:Triple_Noose}
\end{figure}

Получается одна незатягивающаяся петля и две затягивающиеся. При завязывании обратите внимание на левую (на рисунке) петлю. Если ее провести поверх не одной, а обеих поперечных петель, то получится другой узел (рис.~\ref{ris:Triple_Noose_2}).

\subsection{Еще один Triple Noose}

\begin{figure}[H]\centering
	\subfloat[Завязывание\\(первый вариант)]{\label{ris:Triple_Noose_2_1}
	\tcbox[enhanced jigsaw,colframe=black,opacityframe=0.5,opacityback=0.5,height=6.5cm]
		{\centering
			\includesvg[width=0.45\linewidth]{Slide/Triple_Noose_2}}
		}
\hfil
	\subfloat[Завязывание\\(второй вариант)]{\label{ris:Triple_Noose_2_2}
	\tcbox[enhanced jigsaw,colframe=black,opacityframe=0.5,opacityback=0.5,height=6.5cm]
		{\centering
			\includesvg[width=0.3\linewidth]{Slide/Triple_Noose_3}}
		}
\end{figure}
% \vfill
\begin{figure}[H]\centering
	\subfloat[Результат]{\label{ris:Triple_Noose_2_3}
	\tcbox[enhanced jigsaw,colframe=black,opacityframe=0.5,opacityback=0.5]
		{\centering
			\includesvg[width=0.36\linewidth]{Slide/Triple_Noose_4}}
		}
	\caption{Еще один Triple Noose.}\label{ris:Triple_Noose_2}
\end{figure}
