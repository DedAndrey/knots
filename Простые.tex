\section{Простые}

Обычно вяжутся на конце троса для предотвращения продергивания через блок или предотвращения расплетания прядей на концах многопрядных тросов.

\subsection{Простой узел}

\begin{figure}[H]\centering
	\subfloat[Левый]{\label{ris:Single_Knot_left}
	\tcbox[enhanced jigsaw,colframe=black,opacityframe=0.5,opacityback=0.5]
		{\centering
			\includesvg[width=0.33\linewidth]{Utolsh/Single_1}}
		}
\hfil
	\subfloat[Правый]{\label{ris:Single_Knot_right}
	\tcbox[enhanced jigsaw,colframe=black,opacityframe=0.5,opacityback=0.5]
		{\centering
			\includesvg[width=0.33\linewidth]{Utolsh/Single}}
		}
	\caption{Простой узел.}\label{ris:Single_Knot}
\end{figure}

Overhand, Simple, Single, Thumb, Common, Ordinary Knot. Русские названия - Калач, Обыкновенный узел.

Принцип Зеркалирования.

\subsection{Пожарная лестница}

\begin{figure}[H]\centering
	\subfloat[Завязывание]{\label{ris:Fire-Escape_1}
	\tcbox[enhanced jigsaw,colframe=black,opacityframe=0.5,opacityback=0.5]
		{\centering
			\includesvg[width=0.45\linewidth]{Utolsh/Fire-Escape}}
		}
\end{figure}
% \vfill
\begin{figure}[H]\centering
	\subfloat[Результат]{\label{ris:Fire-Escape_2}
	\tcbox[enhanced jigsaw,colframe=black,opacityframe=0.5,opacityback=0.5]
		{\centering
			\includesvg[width=0.7\linewidth]{Utolsh/Fire-Escape_1}}
		}
	\caption{Пожарная лестница.}\label{ris:Fire-Escape}
\end{figure}

Fire-Escape или Philadelphia Knot. Пожарная лестница, Шкентель с мусингами. Вязка этого узла начинается с формирования калышек, заведенных друг за друга. У вас получится некое подобие бухты. Ходовой конец пропустите внутрь нее. Медленно, без рывков, тяните за ходовой конец троса. По мере его вытягивания будут завязываться простые узлы. Их число будет соответствовать числу сделанных калышек, а расстояние между ними - длине их окружности.

\subsection{Бисерный узел}

\begin{figure}[H]\centering
	\begin{minipage}{1\linewidth}
		\begin{center}
			\tcbox[enhanced jigsaw,colframe=black,opacityframe=0.5,opacityback=0.5]
			{\includesvg[width=0.3\linewidth]{Utolsh/Bead_Knot}}
		\end{center}
	\end{minipage}
\caption{Бисерный узел}
	\label{ris:Bead_Knot}
\end{figure}

Bead Knot.

\subsection{True-Lover’s Knot}

\begin{figure}[H]\centering
	\begin{minipage}{1\linewidth}
		\begin{center}
			\tcbox[enhanced jigsaw,colframe=black,opacityframe=0.5,opacityback=0.5]
			{\includesvg[width=0.23\linewidth]{Utolsh/True-Lover_Knot}}
		\end{center}
	\end{minipage}
\caption{True-Lover’s Knot}
	\label{ris:True-Lover_Knot}
\end{figure}

Узел настоящих любовников.

\addtocounter{KnotNoName}{1}

\subsection{Узел без названия \arabic{KnotNoName}}

\begin{figure}[H]\centering
	\begin{minipage}{1\linewidth}
		\begin{center}
			\tcbox[enhanced jigsaw,colframe=black,opacityframe=0.5,opacityback=0.5]
			{\includesvg[width=0.25\linewidth]{Utolsh/KnotNoName_18}}
		\end{center}
	\end{minipage}
\caption{Узел без названия \arabic{KnotNoName}}
	\label{ris:KnotNoName_18}
\end{figure}

\addtocounter{KnotNoName}{1}

\subsection{Узел без названия \arabic{KnotNoName}}

\begin{figure}[H]\centering
	\begin{minipage}{1\linewidth}
		\begin{center}
			\tcbox[enhanced jigsaw,colframe=black,opacityframe=0.5,opacityback=0.5]
			{\includesvg[width=0.28\linewidth]{Utolsh/KnotNoName_3}}
		\end{center}
	\end{minipage}
\caption{Узел без названия \arabic{KnotNoName}}
	\label{ris:KnotNoName_3}
\end{figure}

\subsection{Королевский узел}

\begin{figure}[H]\centering
	\subfloat[Завязывание]{\label{ris:Tweenie_sposob}
	\tcbox[enhanced jigsaw,colframe=black,opacityframe=0.5,opacityback=0.5]
		{\centering
			\includesvg[width=0.33\linewidth]{Utolsh/Tweenie}}
		}
\hfil
	\subfloat[Результат]{\label{ris:Tweenie_rez}
	\tcbox[enhanced jigsaw,colframe=black,opacityframe=0.5,opacityback=0.5]
		{\centering
			\includesvg[width=0.33\linewidth]{Utolsh/Tweenie_1}}
		}
	\caption{Королевский узел.}\label{ris:Tweenie}
\end{figure}

Другое название - Tweenie.

\subsection{Апокрифический узел}

\begin{figure}[H]\centering
	\begin{minipage}{1\linewidth}
		\begin{center}
			\tcbox[enhanced jigsaw,colframe=black,opacityframe=0.5,opacityback=0.5]
			{\centering{\includesvg[width=0.35\linewidth]{Utolsh/Apokrif}}}
		\end{center}
	\end{minipage}
\caption{Апокрифический узел.}
	\label{ris:Apokrif}
\end{figure}

\subsection{Двойной Простой узел}

Double Overhand Knot, Blood Knot. Кровавый узел, Капуцин, Косичка стопорная, Двойной простой, среди альпинистов известен так же как БеК (бескарабинный).

\begin{figure}[H]\centering
	\subfloat[Первый вариант]{\label{ris:Double_Knot_1}
	\tcbox[enhanced jigsaw,colframe=black,opacityframe=0.5,opacityback=0.5]
		{\centering
			\includesvg[width=0.39\linewidth]{Utolsh/Double}}
		}
\hfil
	\subfloat[Первый способ вязки]{\label{ris:Double_Knot_sposob_1}
	\tcbox[enhanced jigsaw,colframe=black,opacityframe=0.5,opacityback=0.5]
		{\centering
			\includesvg[width=0.39\linewidth]{Utolsh/Double_1}}
		}
\end{figure}
% \vfill
\begin{figure}[H]\centering
	\subfloat[Второй вариант]{\label{ris:Double_Knot_2}
	\tcbox[enhanced jigsaw,colframe=black,opacityframe=0.5,opacityback=0.5,height=3cm]
		{\centering
			\includesvg[width=0.3\linewidth]{Utolsh/Double_3}}
		}
\hfil
	\subfloat[Второй способ вязки]{\label{ris:Double_Knot_sposob_2}
	\tcbox[enhanced jigsaw,colframe=black,opacityframe=0.5,opacityback=0.5,height=3cm]
		{\centering
			\includesvg[width=0.3\linewidth]{Utolsh/Double_2}}
		}
	\caption{Двойной Простой узел.}\label{ris:Double_Knot}
\end{figure}

При затягивании узел трансформируется. Особенно хорошо это видно при большом количество шлагов. Если потянуть в разные стороны ходовой и коренной концы, внешняя петля узла начинает свиваться, образуя несколько витков. Причем направление свивания будет противоположным направлению, в котором делались шлаги. Витков, охватывающих шлаги снаружи, получится тем больше, чем больше было сделано дополнительных шлагов. Перед окончательным затягиванием узла надо убедиться, что витки получились ровными, плотно прилегают, но не налезают друг на друга.

\subsection{Тройной Простой узел}

\begin{figure}[H]\centering
	\subfloat[Тройной]{\label{ris:Triple_Knot_1}
	\tcbox[enhanced jigsaw,colframe=black,opacityframe=0.5,opacityback=0.5,height=3cm]
		{\centering
			\includesvg[width=0.34\linewidth]{Utolsh/Triple_1}}
		}
\hfil
	\subfloat[Четверной]{\label{ris:Triple_Knot_2}
	\tcbox[enhanced jigsaw,colframe=black,opacityframe=0.5,opacityback=0.5,height=3cm]
		{\centering
			\includesvg[width=0.39\linewidth]{Utolsh/Triple_2}}
		}
\end{figure}

Threefold Overhand Knot, Triple Overhand Knot. Кровавый узел. Больше трех шлагов (четверной и далее) - Multiple Overhand Knot, другое название French Knot.

Можно завязать несколькими различными способами. Два из них представлены на рисунках \ref{ris:Double_Knot_sposob_1} и \ref{ris:Double_Knot_sposob_2}. Завязывая Простой узел (рис.~\ref{ris:Single_Knot}), делаем несколько дополнительных шлагов ходовым концом, чтобы получить тройной (и более) узел. Иногда отдельно выделяют третий способ вязки, но он является просто развитием первого способа.

\begin{figure}[H]\centering
	\subfloat[Завязывание]{\label{ris:Triple_Knot_sposob_1}
	\tcbox[enhanced jigsaw,colframe=black,opacityframe=0.5,opacityback=0.5,height=2.5cm]
		{\centering
			\includesvg[width=0.36\linewidth]{Utolsh/Triple_3}}
		}
\hfil
	\subfloat[Завязывание]{\label{ris:Triple_Knot_sposob_2}
	\tcbox[enhanced jigsaw,colframe=black,opacityframe=0.5,opacityback=0.5,height=2.5cm]
		{\centering
			\includesvg[width=0.36\linewidth]{Utolsh/Triple_4}}
		}
	\caption{Тройной Простой узел.}\label{ris:Triple_Knot}
\end{figure}

\subsection{Long Three-Ply Knot}

\begin{figure}[H]\centering
	\subfloat[Завязывание]{\label{ris:Long_Three-Ply_Knot_1}
	\tcbox[enhanced jigsaw,colframe=black,opacityframe=0.5,opacityback=0.5,height=3.5cm]
		{\centering
			\includesvg[width=0.33\linewidth]{Utolsh/Long_Three-Ply_Knot}}
		}
\hfil
	\subfloat[Завязывание]{\label{ris:Long_Three-Ply_Knot_2}
	\tcbox[enhanced jigsaw,colframe=black,opacityframe=0.5,opacityback=0.5,height=3.5cm]
		{\centering
			\includesvg[width=0.41\linewidth]{Utolsh/Long_Three-Ply_Knot_1}}
		}
\end{figure}
% \vfill
\begin{figure}[H]\centering
	\subfloat[Результат]{\label{ris:Long_Three-Ply_Knot_3}
	\tcbox[enhanced jigsaw,colframe=black,opacityframe=0.5,opacityback=0.5]
		{\centering
			\includesvg[width=0.5\linewidth]{Utolsh/Long_Three-Ply_Knot_2}}
		}
	\caption{Long Three-Ply Knot.}\label{ris:Long_Three-Ply_Knot}
\end{figure}

На рисунке \ref{ris:Long_Three-Ply_Knot_2} стрелками обозначены направления, в которых начнут скручиваться внешние петли при затягивании узла. Как и в Двойном Простом узле (рис.~\ref{ris:Double_Knot_2}), в процессе затягивания необходимо контролировать правильность и ровность прилегания витков.

\subsection{Восьмерка}

\begin{figure}[H]\centering
	\begin{minipage}{1\linewidth}
		\begin{center}
			\tcbox[enhanced jigsaw,colframe=black,opacityframe=0.5,opacityback=0.5]
			{\centering{\includesvg[width=0.3\linewidth]{Utolsh/Figure-Of-Eight}}}
		\end{center}
	\end{minipage}
\caption{Восьмерка}
	\label{ris:Figure-Of-Eight}
\end{figure}

Figure-Of-Eight Knot, Flemish Knot, Савойский узел.

\subsection{Юферсный узел}

\begin{figure}[H]\centering
	\subfloat[Первая схема]{\label{ris:Long_Three-Ply_Knot_1}
	\tcbox[enhanced jigsaw,colframe=black,opacityframe=0.5,opacityback=0.5,height=3.5cm]
		{\centering
			\includesvg[width=0.33\linewidth]{Utolsh/Ufers}}
		}
\hfil
	\subfloat[Вторая схема]{\label{ris:Long_Three-Ply_Knot_2}
	\tcbox[enhanced jigsaw,colframe=black,opacityframe=0.5,opacityback=0.5,height=3.5cm]
		{\centering
			\includesvg[width=0.33\linewidth]{Utolsh/Ufers_3}}
		}
\end{figure}
% \vfill
\begin{figure}[H]\centering
	\subfloat[Способ вязки из Простого узла (рис.~\ref{ris:Single_Knot})]{\label{ris:Ufers_sposob_1}
	\tcbox[enhanced jigsaw,colframe=black,opacityframe=0.5,opacityback=0.5,height=3.5cm]
		{\centering
			\includesvg[width=0.33\linewidth]{Utolsh/Ufers_1}}
		}
\hfil
	\subfloat[Способ вязки из Восьмерки (рис.~\ref{ris:Figure-Of-Eight})]{\label{ris:Ufers_sposob_2}
	\tcbox[enhanced jigsaw,colframe=black,opacityframe=0.5,opacityback=0.5,height=3.5cm]
		{\centering
			\includesvg[width=0.33\linewidth]{Utolsh/Ufers_2}}
		}
	\caption{Юферсный узел.}\label{ris:Ufers}
\end{figure}

Simple Lanyard Knot. Две схемы одного и того же узла.

\addtocounter{KnotNoName}{1}

\subsection{Узел без названия \arabic{KnotNoName}}

% FIXME Охотничий?

\begin{figure}[H]\centering
	\subfloat[Первая схема]{\label{ris:KnotNoName_5_1}
	\tcbox[enhanced jigsaw,colframe=black,opacityframe=0.5,opacityback=0.5,height=3.5cm]
		{\centering
			\includesvg[width=0.33\linewidth]{Utolsh/KnotNoName_5}}
		}
\hfil
	\subfloat[Вторая схема]{\label{ris:KnotNoName_5_2}
	\tcbox[enhanced jigsaw,colframe=black,opacityframe=0.5,opacityback=0.5,height=3.5cm]
		{\centering
			\includesvg[width=0.33\linewidth]{Utolsh/KnotNoName_5_1}}
		}
	\caption{Узел без названия \arabic{KnotNoName}.}\label{ris:KnotNoName_5}
\end{figure}

Так же как и на рис.~\ref{ris:Ufers} две схемы одного и того же узла.

\addtocounter{KnotNoName}{1}

\subsection{Узел без названия \arabic{KnotNoName}}

\begin{figure}[H]\centering
	\begin{minipage}{1\linewidth}
		\begin{center}
			\tcbox[enhanced jigsaw,colframe=black,opacityframe=0.5,opacityback=0.5]
			{\centering{\includesvg[width=0.4\linewidth]{Utolsh/KnotNoName_6}}}
		\end{center}
	\end{minipage}
\caption{Узел без названия \arabic{KnotNoName}.}
	\label{ris:KnotNoName_6}
\end{figure}

\addtocounter{KnotNoName}{1}

\subsection{Узел без названия \arabic{KnotNoName}}

%FIXME сделать узел для связывания двух тросов (рисунок внутри svg-файла)

\begin{figure}[H]\centering
	\begin{minipage}{1\linewidth}
		\begin{center}
			\tcbox[enhanced jigsaw,colframe=black,opacityframe=0.5,opacityback=0.5]
			{\centering{\includesvg[width=0.5\linewidth]{Utolsh/KnotNoName_7}}}
		\end{center}
	\end{minipage}
\caption{Узел без названия \arabic{KnotNoName}.}
	\label{ris:KnotNoName_7}
\end{figure}

\subsection{Филадельфийская лестница}

\begin{figure}[H]\centering
	\subfloat[Завязывание]{\label{ris:Continuous_Figure-of-Eight_1}
	\tcbox[enhanced jigsaw,colframe=black,opacityframe=0.5,opacityback=0.5,height=2.5cm]
		{\centering
			\includesvg[width=0.33\linewidth]{Utolsh/Continuous_Figure-of-Eight}}
		}
\end{figure}
% \vfill
\begin{figure}[H]\centering
	\subfloat[Результат]{\label{ris:Continuous_Figure-of-Eight_2}
	\tcbox[enhanced jigsaw,colframe=black,opacityframe=0.5,opacityback=0.5,height=2.5cm]
		{\centering
			\includesvg[width=0.5\linewidth]{Utolsh/Continuous_Figure-of-Eight_1}}
		}
	\caption{Филадельфийская лестница.}\label{ris:Continuous_Figure-of-Eight}
\end{figure}

Chain of Figure-Of-Eight Knots, Continuous Figure-of-Eight knot.

\subsection{Intermediate Knot}

% FIXME Девятка?

\begin{figure}[H]\centering
	\begin{minipage}{1\linewidth}
		\begin{center}
			\tcbox[enhanced jigsaw,colframe=black,opacityframe=0.5,opacityback=0.5]
			{\centering{\includesvg[width=0.5\linewidth]{Utolsh/Intermediate}}}
		\end{center}
	\end{minipage}
\caption{Intermediate Knot.}
	\label{ris:Intermediate}
\end{figure}

\subsection{Стивидорный узел}

\begin{figure}[H]\centering
	\begin{minipage}{1\linewidth}
		\begin{center}
			\tcbox[enhanced jigsaw,colframe=black,opacityframe=0.5,opacityback=0.5]
			{\centering{\includesvg[width=0.6\linewidth]{Utolsh/Stevedore}}}
		\end{center}
	\end{minipage}
\caption{Стивидорный узел.}
	\label{ris:Stevedore}
\end{figure}

Stevedore Knot.

\subsection{Многократная Восьмерка}

\begin{figure}[H]\centering
	\subfloat[Первый вариант]{\label{ris:Double_Figure-Of-Eight_1}
	\tcbox[enhanced jigsaw,colframe=black,opacityframe=0.5,opacityback=0.5,height=3.5cm]
		{\centering
			\includesvg[width=0.33\linewidth]{Utolsh/Double_Figure-Of-Eight}}
		}
\hfil
	\subfloat[Второй вариант]{\label{ris:Double_Figure-Of-Eight_2}
	\tcbox[enhanced jigsaw,colframe=black,opacityframe=0.5,opacityback=0.5,height=3.5cm]
		{\centering
			\includesvg[width=0.33\linewidth]{Utolsh/Double_Figure-Of-Eight_1}}
		}
\end{figure}
% \vfill
\begin{figure}[H]\centering
	\subfloat[Третий вариант]{\label{ris:Double_Figure-Of-Eight_3}
	\tcbox[enhanced jigsaw,colframe=black,opacityframe=0.5,opacityback=0.5,height=3.5cm]
		{\centering
			\includesvg[width=0.33\linewidth]{Utolsh/Double_Figure-Of-Eight_2}}
		}
\hfil
	\subfloat[Четвертый вариант]{\label{ris:Double_Figure-Of-Eight_4}
	\tcbox[enhanced jigsaw,colframe=black,opacityframe=0.5,opacityback=0.5,height=3.5cm]
		{\centering
			\includesvg[width=0.33\linewidth]{Utolsh/Double_Figure-Of-Eight_3}}
		}
	\caption{Многократная Восьмерка.}\label{ris:Double_Figure-Of-Eight}
\end{figure}

Double Figure-Of-Eight Knot. Многократная восьмерка, Косичка.

\subsection{Multiple Figure-Of-Eight Knot}

\begin{figure}[H]\centering
	\begin{minipage}{1\linewidth}
		\begin{center}
			\tcbox[enhanced jigsaw,colframe=black,opacityframe=0.5,opacityback=0.5]
			{\centering{\includesvg[width=0.7\linewidth]{Utolsh/Multiple_Figure-Of-Eight}}}
		\end{center}
	\end{minipage}
\caption{Multiple Figure-Of-Eight Knot.}
	\label{ris:Multiple_Figure-Of-Eight}
\end{figure}

Multiple Figure-Of-Eight Knot. Варианты вязки такие же как и у Многократной восьмерки (рис.~\ref{ris:Double_Figure-Of-Eight}).

\subsection{Стопорный узел Голдобина}

\begin{figure}[H]\centering
	\begin{minipage}{1\linewidth}
		\begin{center}
			\tcbox[enhanced jigsaw,colframe=black,opacityframe=0.5,opacityback=0.5]
			{\centering{\includesvg[width=0.35\linewidth]{Utolsh/Goldobin}}}
		\end{center}
	\end{minipage}
\caption{Стопорный узел Голдобина.}
	\label{ris:Goldobin}
\end{figure}

% TODO см. Anchor Bend Variant or Double Oysterman’s Knot???

Две петли и два шлага.\footnote{На этот узел в 1988 году Вениамином Федоровичем Голдобиным получено авторское свидетельство \href{http://patents.su/2-1434012-stopornyjj-uzel-goldobina.html}{№ SU 1434012}} Три петли и три шлага еще больше увеличат размер узла.

\subsection{Развязывающийся простой узел}

\begin{figure}[H]\centering
	\begin{minipage}{1\linewidth}
		\begin{center}
			\tcbox[enhanced jigsaw,colframe=black,opacityframe=0.5,opacityback=0.5]
			{\centering{\includesvg[width=0.5\linewidth]{Utolsh/Slip_Knot}}}
		\end{center}
	\end{minipage}
\caption{Развязывающийся простой узел.}
	\label{ris:Slip_Knot}
\end{figure}

Slip Knot. Фактически, это Арбор или Simple Noose (рис.~\ref{ris:Arbor}). Отличие только в назначении узлов. Арбор - затягивающаяся петля, которая скользит по коренному концу, а Развязывающийся простой узел служит для утолщения троса с возможностью быстрого развязывания. В связи с этим коренной конец у него тот, который в Арборе является ходовым. Кроме того, это еще и Slipped Half Hitch (рис.~\ref{ris:Slipped_Half_Hitch}), который используется для привязывания к опоре. В принципе, таким образом можно использовать любую быстроразвязывающуюся петлю (см. рис.~\ref{ris:Slip_Figure-Of-Eight})

\subsection{Устричный узел}

\begin{figure}[H]\centering
	\subfloat[Завязывание]{\label{ris:Oysterman_1}
	\tcbox[enhanced jigsaw,colframe=black,opacityframe=0.5,opacityback=0.5,height=3.5cm]
		{\centering
			\includesvg[width=0.39\linewidth]{Utolsh/Oysterman}}
		}
\hfil
	\subfloat[Результат]{\label{ris:Oysterman_2}
	\tcbox[enhanced jigsaw,colframe=black,opacityframe=0.5,opacityback=0.5,height=3.5cm]
		{\centering
			\includesvg[width=0.27\linewidth]{Utolsh/Oysterman_1}}
		}
	\caption{Устричный узел.}\label{ris:Oysterman}
\end{figure}

Oysterman’s Stopper Knot, Ashley’s Stopper Knot. Вяжется из Развязывающегося простого узла (рис.~\ref{ris:Arbor}).

\addtocounter{KnotNoName}{1}

\subsection{Узел без названия \arabic{KnotNoName}}

\begin{figure}[H]\centering
	\subfloat[Завязывание]{\label{ris:KnotNoName_4_1}
	\tcbox[enhanced jigsaw,colframe=black,opacityframe=0.5,opacityback=0.5,height=3.5cm]
		{\centering
			\includesvg[width=0.38\linewidth]{Utolsh/KnotNoName_4}}
		}
\hfil
	\subfloat[Результат]{\label{ris:KnotNoName_4_2}
	\tcbox[enhanced jigsaw,colframe=black,opacityframe=0.5,opacityback=0.5,height=3.5cm]
		{\centering
			\includesvg[width=0.38\linewidth]{Utolsh/KnotNoName_4_1}}
		}
	\caption{Узел без названия \arabic{KnotNoName}.}\label{ris:KnotNoName_4}
\end{figure}

Как и Устричный узел (рис.~\ref{ris:Oysterman}), этот узел на основе Развязывающегося простого узла (рис.~\ref{ris:Arbor}).

\subsection{Развязывающаяся восьмерка}

\begin{figure}[H]\centering
	\begin{minipage}{1\linewidth}
		\begin{center}
			\tcbox[enhanced jigsaw,colframe=black,opacityframe=0.5,opacityback=0.5]
			{\centering{\includesvg[width=0.4\linewidth]{Utolsh/Slip_Figure-Of-Eight}}}
		\end{center}
	\end{minipage}
\caption{Развязывающаяся восьмерка.}
	\label{ris:Slip_Figure-Of-Eight}
\end{figure}

Slip Figure-Of-Eight Knot. Используется так же как Развязывающийся простой узел (рис.~\ref{ris:Slip_Knot})

\subsection{Стратим}

\begin{figure}[H]\centering
	\subfloat[Первый вариант]{\label{ris:Stratim_1}
	\tcbox[enhanced jigsaw,colframe=black,opacityframe=0.5,opacityback=0.5]
		{\centering
			\includesvg[width=0.38\linewidth]{Utolsh/Stratim}}
		}
\hfil
	\subfloat[Второй вариант]{\label{ris:Stratim_2}
	\tcbox[enhanced jigsaw,colframe=black,opacityframe=0.5,opacityback=0.5]
		{\centering
			\includesvg[width=0.38\linewidth]{Utolsh/Stratim_1}}
		}
	\caption{Стратим.}\label{ris:Stratim}
\end{figure}
