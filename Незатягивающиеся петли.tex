\section{Незатягивающиеся петли}

Петли имеют фиксированный диаметр.

\subsection{Seized Eye}

\begin{figure}[H]\centering
	\begin{minipage}{1\linewidth}
		\begin{center}
			\tcbox[enhanced jigsaw,colframe=black,opacityframe=0.5,opacityback=0.5]
			{\centering{\includesvg[width=0.45\linewidth]{Nonslide/Seized_Eye}}}
		\end{center}
	\end{minipage}
\caption{Seized Eye.}
\label{ris:Seized_Eye}
\end{figure}

\subsection{Seized Shortening}

Колышка, зафиксированная двумя марками. Английские названия - Monkey’s Tail, Single Throat Seizing, Seized Round Turn, Clinch, Pigtail.

\begin{figure}[H]\centering
	\begin{minipage}{1\linewidth}
		\begin{center}
			\tcbox[enhanced jigsaw,colframe=black,opacityframe=0.5,opacityback=0.5]
			{\centering{\includesvg[width=0.45\linewidth]{Nonslide/Seized_Shortening}}}
		\end{center}
	\end{minipage}
\caption{Seized Shortening.}
\label{ris:Seized_Shortening}
\end{figure}

\subsection{Stopped or Seized Half Hitch}

Самый простой незатягивающийся узел. Фиксируется маркой.

\begin{figure}[H]\centering
	\begin{minipage}{1\linewidth}
		\begin{center}
			\tcbox[enhanced jigsaw,colframe=black,opacityframe=0.5,opacityback=0.5]
			{\centering{\includesvg[width=0.55\linewidth]{Nonslide/Stopped_or_Seized_Half_Hitch}}}
		\end{center}
	\end{minipage}
\caption{Stopped or Seized Half Hitch.}
\label{ris:Stopped_or_Seized_Half_Hitch}
\end{figure}

\subsection{Из затягивающихся}\label{iz_slide}

Как из любой незатягивающейся петли можно сделать затягивающуюся (раздел~\ref{iz_nonslide}), так и практически из любой затягивающейся петли можно сделать незатягивающуюся. Например, из Running Eye (рис.~\ref{ris:Running_Eye}) получается Bowstring Knot (рис.~\ref{ris:Bowstring_Knot}), только с маркой.

\begin{figure}[H]\centering
	\subfloat[Завязывание]{\label{ris:Bowstring_Knot_marka_1}
	\tcbox[enhanced jigsaw,colframe=black,opacityframe=0.5,opacityback=0.5]
		{\centering
			\includesvg[width=0.55\linewidth]{Nonslide/Bowstring_Knot_marka}}
		}
\end{figure}

\begin{figure}[H]\centering
	\subfloat[Результат]{\label{ris:Bowstring_Knot_marka_2}
	\tcbox[enhanced jigsaw,colframe=black,opacityframe=0.5,opacityback=0.5]
		{\centering
			\includesvg[angle=180,width=0.6\linewidth]{Nonslide/Bowstring_Knot_marka_1}}
		}
	\caption{Bowstring Knot с маркой.}\label{ris:Bowstring_Knot_marka}
\end{figure}

\subsection{Дубовая петля}

\begin{figure}[H]\centering
	\subfloat[Завязывание]{\label{ris:Dubovaya_loop_1}
	\tcbox[enhanced jigsaw,colframe=black,opacityframe=0.5,opacityback=0.5]
		{\centering
			\includesvg[width=0.35\linewidth]{Nonslide/Dubovaya_1}}
		}
\end{figure}

\begin{figure}[H]\centering
	\subfloat[Результат]{\label{ris:Dubovaya_loop_2}
	\tcbox[enhanced jigsaw,colframe=black,opacityframe=0.5,opacityback=0.5]
		{\centering
			\includesvg[angle=180,width=0.45\linewidth]{Nonslide/Dubovaya_2}}
		}
	\caption{Дубовая петля.}\label{ris:Dubovaya_loop}
\end{figure}

Loop Knot, Overhand Loop, Проводник, Щедрый или Благородный узел.

\subsection{Дубовая лестница}

\begin{figure}[H]\centering
	\begin{minipage}{1\linewidth}
		\begin{center}
			\tcbox[enhanced jigsaw,colframe=black,opacityframe=0.5,opacityback=0.5]
			{\centering{\includesvg[width=0.8\linewidth]{Nonslide/Dubovaya_trap}}}
		\end{center}
	\end{minipage}
\caption{Дубовая лестница.}
\label{ris:Dubovaya_trap}
\end{figure}

Веревочная лестница из нескольких последовательно завязанных дубовых узлов.

\subsection{Palomar}

\begin{figure}[H]\centering
	\subfloat[Завязывание]{\label{ris:Palomar_1}
	\tcbox[enhanced jigsaw,colframe=black,opacityframe=0.5,opacityback=0.5]
		{\centering
			\includesvg[width=0.55\linewidth]{Nonslide/Palomar}}
		}
\end{figure}

Узел Паломар (Паломарский, Palomar knot, Swiwel knot). Подобно простым петельным узлам, паломарский, или, как его у нас иногда называют, калифорнийский, узел вяжется сложенным вдвое ходовым концом с использованием простого узла. Паломарский узел относится к числу наиболее легких в вязке и используется для привязывания различных концевых элементов оснастки - вертлюжков, застежек, грузил, крючков и искусственных приманок. Несмотря на простоту, паломарский узел очень надежен и совсем незначительно ослабляет леску. Благодаря своим качествам, он хорошо подходит для как для моно-, так и для полифиламентных лесок, плетеных и спаянных. Прежде, чем вязать паломарский узел, надо прикинуть, на какую длину ходовой конец следует сложить пополам, поскольку подвязываемый элемент оснастки придется провести через петлю, получающуюся по ходу вязки узла. Например, особенно длинным должен быть сдвоенный ходовой конец для привязывания воблера с крючками. Сначала сложенный вдвое ходовой конец пропускают через колечко и вяжут простой узел так, чтобы колечко оказалось внутри узла. Затем проводят через петлю подвязываемый элемент оснастки и затягивают узел, стараясь не перекручивать сдвоенные участки лески. Перед затягиванием узел лучше увлажнить.

\begin{figure}[H]\centering
	\subfloat[Результат]{\label{ris:Palomar_2}
	\tcbox[enhanced jigsaw,colframe=black,opacityframe=0.5,opacityback=0.5]
		{\centering
			\includesvg[angle=180,width=0.65\linewidth]{Nonslide/Palomar_1}}
		}
	\caption{Palomar.}\label{ris:Palomar}
\end{figure}

\subsection{Лучниковская петля}

Лучниковый узел, Турецкий узел, Турецкая петля. Английские названия - Adjustable Bowstring Loop from the Orient.

\begin{figure}[H]\centering
	\subfloat[Завязывание]{\label{ris:Luchnikovskaya_1}
	\tcbox[enhanced jigsaw,colframe=black,opacityframe=0.5,opacityback=0.5,height=8cm]
		{\centering
			\includesvg[angle=270,width=0.5\linewidth]{Nonslide/Luchnikovskaya}}
		}
\hfil
	\subfloat[Завязывание]{\label{ris:Luchnikovskaya_2}
	\tcbox[enhanced jigsaw,colframe=black,opacityframe=0.5,opacityback=0.5,height=8cm]
		{\centering
			\includesvg[angle=270,width=0.6\linewidth]{Nonslide/Luchnikovskaya_1}}
		}
\end{figure}

\begin{figure}[H]\centering
	\subfloat[Результат]{\label{ris:Luchnikovskaya_3}
	\tcbox[enhanced jigsaw,colframe=black,opacityframe=0.5,opacityback=0.5]
		{\centering
			\includesvg[width=0.65\linewidth]{Nonslide/Luchnikovskaya_2}}
		}
	\caption{Лучниковская петля.}\label{ris:Luchnikovskaya}
\end{figure}

\subsection{Обвязочная петля}

Вяжется она быстро и состоит из двух переплетенных между собой закрытых петель (полуштыков). Ее можно вязать, начиная с простого узла.

\begin{figure}[H]\centering
	\subfloat[Завязывание]{\label{ris:Obvyazochnaya_1}
	\tcbox[enhanced jigsaw,colframe=black,opacityframe=0.5,opacityback=0.5,height=4.5cm]
		{\centering
			\includesvg[width=0.35\linewidth]{Nonslide/Obvyazochnaya_1}}
		}
\hfil
	\subfloat[Завязывание]{\label{ris:Obvyazochnaya_2}
	\tcbox[enhanced jigsaw,colframe=black,opacityframe=0.5,opacityback=0.5,height=4.5cm]
		{\centering
			\includesvg[width=0.38\linewidth]{Nonslide/Obvyazochnaya_2}}
		}
\end{figure}

\begin{figure}[H]\centering
	\subfloat[Результат]{\label{ris:Obvyazochnaya_3}
	\tcbox[enhanced jigsaw,colframe=black,opacityframe=0.5,opacityback=0.5]
		{\centering
			\includesvg[width=0.8\linewidth]{Nonslide/Obvyazochnaya}}
		}
	\caption{Обвязочная петля.}\label{ris:Obvyazochnaya}
\end{figure}

\subsection{Травяная петля}

\begin{figure}[H]\centering
	\subfloat[Завязывание]{\label{ris:Travyanaya_1}
	\tcbox[enhanced jigsaw,colframe=black,opacityframe=0.5,opacityback=0.5,height=4cm]
		{\centering
			\includesvg[width=0.35\linewidth]{Nonslide/Travyanaya_1}}
		}
\hfil
	\subfloat[Точка различия]{\label{ris:Travyanaya_2}
	\tcbox[enhanced jigsaw,colframe=black,opacityframe=0.5,opacityback=0.5,height=4cm]
		{\centering
			\includesvg[width=0.35\linewidth]{Nonslide/Travyanaya_2}}
		}
\end{figure}

Обратите внимание, у Травяной петли по Скрягину\citep{Skryagin} и Обвязочной петли по Балабанову\citep{Balabanov} разница всего в одном пересечении. В первом случае петля, проходящая в Простой узел, перекручивается вправо, во втором - влево. На рабочие свойства узла это никак не влияет, поэтому различия не принципиальны. Можно вообще обойтись без этих перекрутов.

\begin{figure}[H]\centering
	\subfloat[Результат]{\label{ris:Travyanaya_3}
	\tcbox[enhanced jigsaw,colframe=black,opacityframe=0.5,opacityback=0.5]
		{\centering
			\includesvg[width=0.8\linewidth]{Nonslide/Travyanaya}}
		}
\end{figure}

\begin{figure}[H]\centering
	\subfloat[Травяная петля без перекрута]{\label{ris:Travyanaya_4}
	\tcbox[enhanced jigsaw,colframe=black,opacityframe=0.5,opacityback=0.5]
		{\centering
			\includesvg[width=0.8\linewidth]{Nonslide/Travyanaya_3}}
		}
	\caption{Травяная петля.}\label{ris:Travyanaya}
\end{figure}

\subsection{Петля из Double Overhand Knot}

\begin{figure}[H]\centering
	\subfloat[Завязывание]{\label{ris:From_Double_Overhand_Knot_1}
	\tcbox[enhanced jigsaw,colframe=black,opacityframe=0.5,opacityback=0.5]
		{\centering
			\includesvg[width=0.8\linewidth]{Nonslide/From_Double_Overhand_Knot}}
		}
\end{figure}

\begin{figure}[H]\centering
	\subfloat[Результат]{\label{ris:From_Double_Overhand_Knot_2}
	\tcbox[enhanced jigsaw,colframe=black,opacityframe=0.5,opacityback=0.5]
		{\centering
			\includesvg[width=0.7\linewidth]{Nonslide/From_Double_Overhand_Knot_1}}
		}
	\caption{Петля из Double Overhand Knot.}\label{ris:From_Double_Overhand_Knot}
\end{figure}

\subsection{Плотная петля}

Homer-rhode loop knot, Englishman’s, Fisherman’s, Angler’s Loop or True-Lover (Lover’s) Knot. Рыбацкая (английская) петля или рыбацкий огон, Узел верного любовника (простой английский рыбацкий узел).

\begin{figure}[H]\centering
	\subfloat[Результат]{\label{ris:Plotnaya_1}
	\tcbox[enhanced jigsaw,colframe=black,opacityframe=0.5,opacityback=0.5]
		{\centering
			\includesvg[width=0.8\linewidth]{Nonslide/Plotnaya}}
		}
\end{figure}

Кроме прямого способа, когда последовательно друг за другом вяжутся простые узлы,существуют и другие способы вязки:

\begin{figure}[H]\centering
	\subfloat[Завязывание.\\Первый вариант]{\label{ris:Plotnaya_2}
	\tcbox[enhanced jigsaw,colframe=black,opacityframe=0.5,opacityback=0.5,height=6.5cm]
		{\centering
			\includesvg[width=0.37\linewidth]{Nonslide/Englishman_Knot}}
		}
\hfil
	\subfloat[Завязывание.\\Второй вариант]{\label{ris:Plotnaya_3}
	\tcbox[enhanced jigsaw,colframe=black,opacityframe=0.5,opacityback=0.5,height=6.5cm]
		{\centering
			\includesvg[width=0.37\linewidth]{Nonslide/Englishman_Knot_1}}
		}
\end{figure}

\begin{figure}[H]\centering
	\subfloat[Завязывание.\\Третий вариант]{\label{ris:Plotnaya_4}
	\tcbox[enhanced jigsaw,colframe=black,opacityframe=0.5,opacityback=0.5]
		{\centering
			\includesvg[width=0.6\linewidth]{Nonslide/Englishman_Knot_2}}
		}
\end{figure}

\begin{figure}[H]\centering
	\subfloat[Завязывание.\\Четвертый вариант]{\label{ris:Plotnaya_5}
	\tcbox[enhanced jigsaw,colframe=black,opacityframe=0.5,opacityback=0.5]
		{\centering
			\includesvg[width=0.7\linewidth]{Nonslide/Englishman_Knot_3}}
		}
	\caption{Плотная петля.}\label{ris:Plotnaya}
\end{figure}

\subsection{Плотная петля по принципу Хонды}

\begin{figure}[H]\centering
	\begin{minipage}{1\linewidth}
		\begin{center}
			\tcbox[enhanced jigsaw,colframe=black,opacityframe=0.5,opacityback=0.5]
			{\centering{\includesvg[width=0.8\linewidth]{Nonslide/Plotnaya_2}}}
		\end{center}
	\end{minipage}
\caption{Плотная петля по принципу Хонды.}
\label{ris:Plotnaya_Honda}
\end{figure}

\subsection{Плотная петля с другим узлом на ходовом конце}

\begin{figure}[H]\centering
	\begin{minipage}{1\linewidth}
		\begin{center}
			\tcbox[enhanced jigsaw,colframe=black,opacityframe=0.5,opacityback=0.5]
			{\centering{\includesvg[width=0.8\linewidth]{Nonslide/Plotnaya_3}}}
		\end{center}
	\end{minipage}
\caption{Плотная петля с другим узлом на ходовом конце.}
\label{ris:Plotnaya_drugoi_uzel}
\end{figure}

\subsection{\enquote{Figure-of-Eight} Englishman's Loop}

Плотная петля на основе Восьмерки (рис.~\ref{ris:Figure-Of-Eight}).

\begin{figure}[H]\centering
	\subfloat[Завязывание]{\label{ris:Figure-of-Eight_Englishman_Loop_1}
	\tcbox[enhanced jigsaw,colframe=black,opacityframe=0.5,opacityback=0.5]
		{\centering
			\includesvg[width=0.29\linewidth]{Nonslide/Figure-of-Eight_Englishman_Loop}}
		}
\end{figure}

\begin{figure}[H]\centering
	\subfloat[Результат]{\label{ris:Figure-of-Eight_Englishman_Loop_2}
	\tcbox[enhanced jigsaw,colframe=black,opacityframe=0.5,opacityback=0.5]
		{\centering
			\includesvg[width=0.75\linewidth]{Nonslide/Figure-of-Eight_Englishman_Loop_1}}
		}
	\caption{\enquote{Figure-of-Eight} Englishman's Loop.}\label{ris:Figure-of-Eight_Englishman_Loop}
\end{figure}

\subsection{Хирургическая петля}

Можно по принципу Хонды, можно по принципу затягивающей простой петли.
%TODO Нужно проверить – работает ли?
%TODO Сделать ссылки

\begin{figure}[H]\centering
	\subfloat[На основе Хирургического узла]{\label{ris:Hirurg_loop_1}
	\tcbox[enhanced jigsaw,colframe=black,opacityframe=0.5,opacityback=0.5]
		{\centering
			\includesvg[width=0.7\linewidth]{Nonslide/Noname_loop_0}}
		}
\end{figure}

\begin{figure}[H]\centering
	\subfloat[На основе Бабьего узла]{\label{ris:Hirurg_loop_2}
	\tcbox[enhanced jigsaw,colframe=black,opacityframe=0.5,opacityback=0.5]
		{\centering
			\includesvg[width=0.7\linewidth]{Nonslide/Noname_loop_0_1}}
		}
	\caption{Хирургическая петля.}\label{ris:Hirurg_loop}
\end{figure}

\subsection{True-Lover’s Loop}

\begin{figure}[H]\centering
	\begin{minipage}{1\linewidth}
		\begin{center}
			\tcbox[enhanced jigsaw,colframe=black,opacityframe=0.5,opacityback=0.5]
			{\centering{\includesvg[width=0.41\linewidth]{Nonslide/True-Lover_Loop}}}
		\end{center}
	\end{minipage}
\caption{True-Lover’s Loop.}
\label{ris:True-Lover_Loop}
\end{figure}

\subsection{Петля на основе охотничьего узла}

\begin{figure}[H]\centering
	\begin{minipage}{1\linewidth}
		\begin{center}
			\tcbox[enhanced jigsaw,colframe=black,opacityframe=0.5,opacityback=0.5]
			{\centering{\includesvg[width=0.6\linewidth]{Nonslide/Hanter_loop}}}
		\end{center}
	\end{minipage}
\caption{Петля на основе охотничьего узла.}
\label{ris:Hanter_loop}
\end{figure}

\subsection{Фламандская петля}

Flemish Loop, Figure-of-Eight Loop, Figure-Of-Eight Knot tied in doubled ends, Швейцарский проводник, восьмерка двойная. Главное - не перекручивать, добиваясь сохранения параллельности участков сложенного ходового конца.

\begin{figure}[H]\centering
	\begin{minipage}{1\linewidth}
		\begin{center}
			\tcbox[enhanced jigsaw,colframe=black,opacityframe=0.5,opacityback=0.5]
			{\centering{\includesvg[width=0.8\linewidth]{Nonslide/Flemish_Loop}}}
		\end{center}
	\end{minipage}
\caption{Фламандская петля.}
\label{ris:Flemish_Loop}
\end{figure}

\subsection{Двойная фламандская петля}

\begin{figure}[H]\centering
	\begin{minipage}{1\linewidth}
		\begin{center}
			\tcbox[enhanced jigsaw,colframe=black,opacityframe=0.5,opacityback=0.5]
			{\centering{\includesvg[width=0.9\linewidth]{Nonslide/Double_flemish_Loop}}}
		\end{center}
	\end{minipage}
\caption{Двойная фламандская петля.}
\label{ris:Double_flemish_Loop}
\end{figure}

\subsection{Тройная фламандская петля}

% FIXME(Двойной Стивидорный?).

Если необходимо повысить надежность петли, то перед тем, как пропустить сложенный вдвое ходовой конец через начальную петлю, надо сделать им дополнительный шлаг вокруг коренного конца. Считается, что один дополнительный шлаг способен увеличить прочность петли на разрыв примерно на 1/5-1/4, по сравнению с простой петлей-восьмеркой. Чем тоньше леска, и чем она более \enquote{скользкая}, тем больше надо сделать дополнительных шлагов ходовым концом вокруг коренного. Сколько именно -- подскажет опыт, но вряд ли понадобится больше пяти.

\begin{figure}[H]\centering
	\begin{minipage}{1\linewidth}
		\begin{center}
			\tcbox[enhanced jigsaw,colframe=black,opacityframe=0.5,opacityback=0.5]
			{\centering{\includesvg[width=0.9\linewidth]{Nonslide/Triple_flemish_Loop}}}
		\end{center}
	\end{minipage}
\caption{Тройная фламандская петля.}
\label{ris:Triple_flemish_Loop}
\end{figure}

\subsection{Девятка}

\begin{figure}[H]\centering
	\subfloat[Завязывание]{\label{ris:Nine_1}
	\tcbox[enhanced jigsaw,colframe=black,opacityframe=0.5,opacityback=0.5]
		{\centering
			\includesvg[width=0.6\linewidth]{Nonslide/Nine}}
		}
\end{figure}

\begin{figure}[H]\centering
	\subfloat[Результат]{\label{ris:Nine_2}
	\tcbox[enhanced jigsaw,colframe=black,opacityframe=0.5,opacityback=0.5]
		{\centering
			\includesvg[width=0.8\linewidth]{Nonslide/Nine_1}}
		}
	\caption{Девятка.}\label{ris:Nine}
\end{figure}

\subsection{Жилковая петля}

По английски - Double Overhand Loop. Двойной простой петельный (петлевой) узел. Затягивать его надо постепенно, добиваясь сохранения параллельности участков сложенного ходового конца. Получившуюся петлю иногда называют двойной рыбацкой петлей (последнее название неудачно и даже неверно, поскольку рыбацкая петля вяжется одинарным ходовым концом и совершенно по-иному). Жилковая петля вполне годится для синтетических рыболовных лесок. Если на них вязать концевую петлю сдвоенным ходовым концом, то лучше сделать не 2, а 3-4 шлага в простом узле. Получится тройной или четверной простой петельный узел.

\begin{figure}[H]\centering
	\subfloat[Завязывание]{\label{ris:Double_Overhand_Loop_1}
	\tcbox[enhanced jigsaw,colframe=black,opacityframe=0.5,opacityback=0.5]
		{\centering
			\includesvg[width=0.6\linewidth]{Nonslide/Double_Overhand_Loop}}
		}
\end{figure}

\begin{figure}[H]\centering
	\subfloat[Результат]{\label{ris:Double_Overhand_Loop_2}
	\tcbox[enhanced jigsaw,colframe=black,opacityframe=0.5,opacityback=0.5]
		{\centering
			\includesvg[width=0.8\linewidth]{Nonslide/Double_Overhand_Loop_1}}
		}
	\caption{Жилковая петля.}\label{ris:Double_Overhand_Loop}
\end{figure}

\subsection{Triple Overhand Loop}

\begin{figure}[H]\centering
	\subfloat[Завязывание]{\label{ris:Triple_Overhand_Loop_1}
	\tcbox[enhanced jigsaw,colframe=black,opacityframe=0.5,opacityback=0.5]
		{\centering
			\includesvg[width=0.6\linewidth]{Nonslide/Triple_Overhand_Loop}}
		}
\end{figure}

\begin{figure}[H]\centering
	\subfloat[Завязывание]{\label{ris:Triple_Overhand_Loop_2}
	\tcbox[enhanced jigsaw,colframe=black,opacityframe=0.5,opacityback=0.5]
		{\centering
			\includesvg[width=1\linewidth]{Nonslide/Triple_Overhand_Loop_1}}
		}
\end{figure}

\begin{figure}[H]\centering
	\subfloat[Завязывание]{\label{ris:Triple_Overhand_Loop_3}
	\tcbox[enhanced jigsaw,colframe=black,opacityframe=0.5,opacityback=0.5]
		{\centering
			\includesvg[width=1\linewidth]{Nonslide/Triple_Overhand_Loop_2}}
		}
\end{figure}

\begin{figure}[H]\centering
	\subfloat[Результат]{\label{ris:Triple_Overhand_Loop_4}
	\tcbox[enhanced jigsaw,colframe=black,opacityframe=0.5,opacityback=0.5]
		{\centering
			\includesvg[width=1\linewidth]{Nonslide/Triple_Overhand_Loop_3}}
		}
	\caption{Triple Overhand Loop.}\label{ris:Triple_Overhand_Loop}
\end{figure}

Тройной простой петельный (петлевой) узел.

\addtocounter{LoopNoName}{1}

\subsection{Петля без названия \arabic{LoopNoName}}

По мотивам Triple Overhand Loop (рис.~\ref{ris:Triple_Overhand_Loop_3}).

\begin{figure}[H]\centering
	\begin{minipage}{1\linewidth}
		\begin{center}
			\tcbox[enhanced jigsaw,colframe=black,opacityframe=0.5,opacityback=0.5]
			{\centering{\includesvg[width=1\linewidth]{Nonslide/Triple_Overhand_Loop_2_1}}}
		\end{center}
	\end{minipage}
\caption{Петля без названия \arabic{LoopNoName}.}
\label{ris:Triple_Overhand_Loop_2_1}
\end{figure}

\subsection{Одностороння восьмерка}

Она же Направленная (направляющая) восьмерка или Огон с восьмеркой. Конец веревки лежит внутри первой петли. Это очень важный шаг в процессе завязывания направляющей восьмерки правильно. Если оставить конец веревки вне петли, то в итоге получится узел, которому гн. Эшли дал название Одинарного Беседочного узла (рис.~\ref{ris:Single_Bowline_on_the_bight}).

\begin{figure}[H]\centering
	\subfloat[Завязывание]{\label{ris:One_strand_figure_of_eight_1}
	\tcbox[enhanced jigsaw,colframe=black,opacityframe=0.5,opacityback=0.5]
		{\centering
			\includesvg[width=0.6\linewidth]{Nonslide/One_strand_figure_of_eight}}
		}
\end{figure}

\begin{figure}[H]\centering
	\subfloat[Результат]{\label{ris:One_strand_figure_of_eight_2}
	\tcbox[enhanced jigsaw,colframe=black,opacityframe=0.5,opacityback=0.5]
		{\centering
			\includesvg[width=0.65\linewidth]{Nonslide/One_strand_figure_of_eight_1}}
		}
	\caption{Жилковая петля.}\label{ris:One_strand_figure_of_eight}
\end{figure}

\subsection{Лепесток}

\begin{figure}[H]\centering
	\subfloat[Завязывание]{\label{ris:Lepestok_1}
	\tcbox[enhanced jigsaw,colframe=black,opacityframe=0.5,opacityback=0.5]
		{\centering
			\includesvg[width=0.6\linewidth]{Nonslide/Lepestok}}
		}
\end{figure}

\begin{figure}[H]\centering
	\subfloat[Результат]{\label{ris:Lepestok_2}
	\tcbox[enhanced jigsaw,colframe=black,opacityframe=0.5,opacityback=0.5]
		{\centering
			\includesvg[width=0.7\linewidth]{Nonslide/Lepestok_1}}
		}
\end{figure}

Законченный вид узел приобретает после трансформации, при этом становится похожим на Harness Loop (рис.~\ref{ris:Harness_Loop}).

\begin{figure}[H]\centering
	\subfloat[Результат после трансформации]{\label{ris:Lepestok_2}
	\tcbox[enhanced jigsaw,colframe=black,opacityframe=0.5,opacityback=0.5]
		{\centering
			\includesvg[width=0.5\linewidth]{Nonslide/Lepestok_2}}
		}
	\caption{Лепесток.}\label{ris:Lepestok}
\end{figure}

\subsection{Bight Loop}

\begin{figure}[H]\centering
	\subfloat[Завязывание]{\label{ris:Bight_Loop_1}
	\tcbox[enhanced jigsaw,colframe=black,opacityframe=0.5,opacityback=0.5]
		{\centering
			\includesvg[width=0.55\linewidth]{Nonslide/Bight_Loop}}
		}
\end{figure}

\begin{figure}[H]\centering
	\subfloat[Результат]{\label{ris:Bight_Loop_2}
	\tcbox[enhanced jigsaw,colframe=black,opacityframe=0.5,opacityback=0.5]
		{\centering
			\includesvg[width=0.55\linewidth]{Nonslide/Bight_Loop_1}}
		}
	\caption{Bight Loop.}\label{ris:Bight_Loop}
\end{figure}

\subsection{Loop Knot}

\begin{figure}[H]\centering
	\subfloat[Завязывание]{\label{ris:Loop_Knot_1}
	\tcbox[enhanced jigsaw,colframe=black,opacityframe=0.5,opacityback=0.5]
		{\centering
			\includesvg[width=0.55\linewidth]{Nonslide/Loop_Knot}}
		}
\end{figure}

\begin{figure}[H]\centering
	\subfloat[Результат]{\label{ris:Loop_Knot_2}
	\tcbox[enhanced jigsaw,colframe=black,opacityframe=0.5,opacityback=0.5]
		{\centering
			\includesvg[width=0.6\linewidth]{Nonslide/Loop_Knot_1}}
		}
	\caption{Loop Knot.}\label{ris:Loop_Knot}
\end{figure}

\subsection{Срединный Булинь}

По английски - Single Bowline on the bight.

\begin{figure}[H]\centering
	\subfloat[Завязывание.\\Первый вариант]{\label{ris:Single_Bowline_on_the_bight_1}
	\tcbox[enhanced jigsaw,colframe=black,opacityframe=0.5,opacityback=0.5,height=3cm]
		{\centering
			\includesvg[width=0.34\linewidth]{Nonslide/Single_Bowline_on_the_bight}}
		}
\hfil
	\subfloat[Завязывание.\\Второй вариант]{\label{ris:Single_Bowline_on_the_bight_3}
	\tcbox[enhanced jigsaw,colframe=black,opacityframe=0.5,opacityback=0.5,height=3cm]
		{\centering
			\includesvg[width=0.45\linewidth]{Nonslide/Single_Bowline_on_the_bight_2}}
		}
\end{figure}

\begin{figure}[H]\centering
	\subfloat[Результат.\\Первый вариант]{\label{ris:Single_Bowline_on_the_bight_2}
	\tcbox[enhanced jigsaw,colframe=black,opacityframe=0.5,opacityback=0.5]
		{\centering
			\includesvg[width=0.5\linewidth]{Nonslide/Single_Bowline_on_the_bight_1}}
		}
\end{figure}

\begin{figure}[H]\centering
	\subfloat[Результат.\\Второй вариант]{\label{ris:Single_Bowline_on_the_bight_4}
	\tcbox[enhanced jigsaw,colframe=black,opacityframe=0.5,opacityback=0.5]
		{\centering
			\includesvg[width=0.75\linewidth]{Nonslide/Single_Bowline_on_the_bight_3}}
		}
	\caption{Срединный Булинь.}\label{ris:Single_Bowline_on_the_bight}
\end{figure}

\subsection{Скользящий баттерфляй}

Этимология названия малопонятна. Тоже один из узлов на базе восьмерки.

\begin{figure}[H]\centering
	\subfloat[Первый вариант]{\label{ris:Sliding_butterfly_1}
	\tcbox[enhanced jigsaw,colframe=black,opacityframe=0.5,opacityback=0.5]
		{\centering
			\includesvg[width=0.7\linewidth]{Nonslide/Sliding_butterfly}}
		}
\end{figure}

\begin{figure}[H]\centering
	\subfloat[Второй вариант]{\label{ris:Sliding_butterfly_2}
	\tcbox[enhanced jigsaw,colframe=black,opacityframe=0.5,opacityback=0.5]
		{\centering
			\includesvg[width=0.7\linewidth]{Nonslide/Sliding_butterfly_1}}
		}
	\caption{Скользящий баттерфляй.}\label{ris:Sliding_butterfly}
\end{figure}

\subsection{Односторонняя девятка}

Может быть нагружен только в одном направлении. Внизу два варианта готового узла. Вариант Б получается из варианта А просто в результате скручивания, см. Triple Overhand Loop (рис.~\ref{ris:Triple_Overhand_Loop}) или Стивидорный узел.

\begin{figure}[H]\centering
	\subfloat[Завязывание]{\label{ris:One_strand_figure_of_nine_1}
	\tcbox[enhanced jigsaw,colframe=black,opacityframe=0.5,opacityback=0.5]
		{\centering
			\includesvg[width=0.6\linewidth]{Nonslide/One_strand_figure_of_nine}}
		}
\end{figure}

\begin{figure}[H]\centering
	\subfloat[Завязывание]{\label{ris:One_strand_figure_of_nine_2}
	\tcbox[enhanced jigsaw,colframe=black,opacityframe=0.5,opacityback=0.5]
		{\centering
			\includesvg[width=0.9\linewidth]{Nonslide/One_strand_figure_of_nine_2}}
		}
% \end{figure}
% 
% \begin{figure}[H]\centering
% 	\subfloat[Результат]{\label{ris:One_strand_figure_of_nine_3}
% 	\tcbox[enhanced jigsaw,colframe=black,opacityframe=0.5,opacityback=0.5]
% 		{\centering
% 			\includesvg[width=0.9\linewidth]{Nonslide/One_strand_figure_of_nine_2}}
% 		}
	\caption{Односторонняя девятка.}\label{ris:One_strand_figure_of_nine}
\end{figure}

\subsection{Трехчетвертная восьмерка}

По английски - Three-Quarter Figure-of-Eight Loop.

\begin{figure}[H]\centering
	\subfloat[Завязывание.\\Первый способ вязки]{\label{ris:Three-Quarter_Figure-of-Eight_Loop_1}
	\tcbox[enhanced jigsaw,colframe=black,opacityframe=0.5,opacityback=0.5]
		{\centering
			\includesvg[width=0.45\linewidth]{Nonslide/Three-Quarter_Figure-of-Eight_Loop}}
		}
\end{figure}

\begin{figure}[H]\centering
	\subfloat[Завязывание.\\Продолжение первого способа вязки]{\label{ris:Three-Quarter_Figure-of-Eight_Loop_2}
	\tcbox[enhanced jigsaw,colframe=black,opacityframe=0.5,opacityback=0.5]
		{\centering
			\includesvg[width=0.65\linewidth]{Nonslide/Three-Quarter_Figure-of-Eight_Loop_1}}
		}
\end{figure}

\begin{figure}[H]\centering
	\subfloat[Завязывание.\\Второй способ вязки]{\label{ris:Three-Quarter_Figure-of-Eight_Loop_3}
	\tcbox[enhanced jigsaw,colframe=black,opacityframe=0.5,opacityback=0.5]
		{\centering
			\includesvg[width=0.55\linewidth]{Nonslide/Three-Quarter_Figure-of-Eight_Loop_2}}
		}
\end{figure}

\begin{figure}[H]\centering
	\subfloat[Результат]{\label{ris:Three-Quarter_Figure-of-Eight_Loop_4}
	\tcbox[enhanced jigsaw,colframe=black,opacityframe=0.5,opacityback=0.5]
		{\centering
			\includesvg[width=0.7\linewidth]{Nonslide/Three-Quarter_Figure-of-Eight_Loop_3}}
		}
	\caption{Трехчетвертная восьмерка.}\label{ris:Three-Quarter_Figure-of-Eight_Loop}
\end{figure}

\subsection{Срединная восьмерка-проводник}

\begin{figure}[H]\centering
	\subfloat[Завязывание]{\label{ris:Middle_figure_of_eight_1}
	\tcbox[enhanced jigsaw,colframe=black,opacityframe=0.5,opacityback=0.5]
		{\centering
			\includesvg[width=0.6\linewidth]{Nonslide/Middle_figure_of_eight}}
		}
\end{figure}

\begin{figure}[H]\centering
	\subfloat[Результат]{\label{ris:Middle_figure_of_eight_2}
	\tcbox[enhanced jigsaw,colframe=black,opacityframe=0.5,opacityback=0.5]
		{\centering
			\includesvg[width=0.8\linewidth]{Nonslide/Middle_figure_of_eight_1}}
		}
\end{figure}

Если больше вытянуть правую петлю, то получится узел с двумя петлями. Этот узел используется для соединения грудной и поясной обвязок основной веревкой при отсутствии или невозможности использования карабина.Воспринимает тягу веревка-веревка, левая веревка-правая петля, левая петля-правая веревка и петля-петля.

\begin{figure}[H]\centering
	\subfloat[Срединная восьмерка-проводник с двумя петлями]{\label{ris:Middle_figure_of_eight_3}
	\tcbox[enhanced jigsaw,colframe=black,opacityframe=0.5,opacityback=0.5]
		{\centering
			\includesvg[width=0.8\linewidth]{Nonslide/Middle_figure_of_eight_2}}
		}
	\caption{Срединная восьмерка-проводник.}\label{ris:Middle_figure_of_eight}
\end{figure}

\subsection{Il nodo ad otto direzionale con coda}

В переводе с итальянского - направленная восьмерка с хвостом. Не ломает веревку и не затягивается намертво. Основан на восьмерке. Очень похож на Срединную восьмерку-проводник с двумя петлями (рис.~\ref{ris:Middle_figure_of_eight_3}).

\begin{figure}[H]\centering
	\subfloat[Первый вариант]{\label{ris:Il_nodo_ad_otto_direzionale_con_coda_1}
	\tcbox[enhanced jigsaw,colframe=black,opacityframe=0.5,opacityback=0.5]
		{\centering
			\includesvg[width=0.7\linewidth]{Nonslide/Il_nodo_AD_OTTO_DIREZIONALE_CON_CODA}}
		}
\end{figure}

\begin{figure}[H]\centering
	\subfloat[Второй вариант]{\label{ris:Il_nodo_ad_otto_direzionale_con_coda_2}
	\tcbox[enhanced jigsaw,colframe=black,opacityframe=0.5,opacityback=0.5]
		{\centering
			\includesvg[width=0.85\linewidth]{Nonslide/Il_nodo_AD_OTTO_DIREZIONALE_CON_CODA_1}}
		}
	\caption{Il nodo ad otto direzionale con coda.}\label{ris:Il_nodo_ad_otto_direzionale_con_coda}
\end{figure}

\subsection{Римская восьмерка}

Она же Noeud Romano.

\begin{figure}[H]\centering
	\subfloat[Завязывание]{\label{ris:Rome_figure_of_eight_1}
	\tcbox[enhanced jigsaw,colframe=black,opacityframe=0.5,opacityback=0.5]
		{\centering
			\includesvg[width=0.6\linewidth]{Nonslide/Rome_figure_of_eight}}
		}
\end{figure}

\begin{figure}[H]\centering
	\subfloat[Результат]{\label{ris:Rome_figure_of_eight_2}
	\tcbox[enhanced jigsaw,colframe=black,opacityframe=0.5,opacityback=0.5]
		{\centering
			\includesvg[width=0.65\linewidth]{Nonslide/Rome_figure_of_eight_1}}
		}
	\caption{Римская восьмерка.}\label{ris:Rome_figure_of_eight}
\end{figure}

\subsection{Промежуточная петля}

Она же Span Loop.

\begin{figure}[H]\centering
	\subfloat[Завязывание]{\label{ris:Span_Loop_1}
	\tcbox[enhanced jigsaw,colframe=black,opacityframe=0.5,opacityback=0.5]
		{\centering
			\includesvg[width=0.45\linewidth]{Nonslide/Span_Loop}}
		}
\end{figure}

\begin{figure}[H]\centering
	\subfloat[Результат]{\label{ris:Span_Loop_2}
	\tcbox[enhanced jigsaw,colframe=black,opacityframe=0.5,opacityback=0.5]
		{\centering
			\includesvg[width=0.5\linewidth]{Nonslide/Span_Loop_1}}
		}
	\caption{Промежуточная петля.}\label{ris:Span_Loop}
\end{figure}

\subsection{Harness Loop}

Harness hitch, Artilleryman’s Knot. Бурлацкая петля (упряжковая петля, бурлацкая лямка, пушкарский или артиллерийский узел, мотыль артиллеристский.
%TODO (французский шкотовый)?????

\begin{figure}[H]\centering
	\subfloat[Завязывание]{\label{ris:Harness_Loop_1}
	\tcbox[enhanced jigsaw,colframe=black,opacityframe=0.5,opacityback=0.5]
		{\centering
			\includesvg[width=0.45\linewidth]{Nonslide/Harness_Loop}}
		}
\end{figure}

\begin{figure}[H]\centering
	\subfloat[Результат]{\label{ris:Harness_Loop_2}
	\tcbox[enhanced jigsaw,colframe=black,opacityframe=0.5,opacityback=0.5]
		{\centering
			\includesvg[width=0.5\linewidth]{Nonslide/Harness_Loop_1}}
		}
	\caption{Harness Loop.}\label{ris:Harness_Loop}
\end{figure}

\subsection{Обратная Harness Loop}

Очень похож на Harness Loop, но петли входят друг в друга в обратном порядке.
%TODO Посмотреть что за узел

\begin{figure}[H]\centering
	\subfloat[Завязывание]{\label{ris:Revers_Harness_Loop_1}
	\tcbox[enhanced jigsaw,colframe=black,opacityframe=0.5,opacityback=0.5]
		{\centering
			\includesvg[width=0.45\linewidth]{Nonslide/Revers_Harness_Loop_1}}
		}
\end{figure}

\begin{figure}[H]\centering
	\subfloat[Результат]{\label{ris:Revers_Harness_Loop_2}
	\tcbox[enhanced jigsaw,colframe=black,opacityframe=0.5,opacityback=0.5]
		{\centering
			\includesvg[width=0.45\linewidth]{Nonslide/Revers_Harness_Loop_1_1}}
		}
	\caption{Обратная Harness Loop.}\label{ris:Revers_Harness_Loop}
\end{figure}

\subsection{Loosen Harness Loop}

\begin{figure}[H]\centering
	\subfloat[Отличие от Harness Loop]{\label{ris:Loosen_Harness_Loop_1}
	\tcbox[enhanced jigsaw,colframe=black,opacityframe=0.5,opacityback=0.5,height=6.5cm]
		{\centering
			\includesvg[width=0.49\linewidth]{Nonslide/Loosen_Harness_Loop_1}}
		}
\hfil
	\subfloat[Результат]{\label{ris:Loosen_Harness_Loop_2}
	\tcbox[enhanced jigsaw,colframe=black,opacityframe=0.5,opacityback=0.5,height=6.5cm]
		{\centering
			\includesvg[width=0.29\linewidth]{Nonslide/Loosen_Harness_Loop}}
		}
	\caption{Loosen Harness Loop.}\label{ris:Loosen_Harness_Loop}
\end{figure}

Перевернутая Harness Loop.

\subsection{Двойная Harness Loop}

Английское название - Double Harness Loop.

\begin{figure}[H]\centering
	\subfloat[Завязывание]{\label{ris:Double_Harness_Loop_1}
	\tcbox[enhanced jigsaw,colframe=black,opacityframe=0.5,opacityback=0.5]
		{\centering
			\includesvg[width=0.6\linewidth]{Nonslide/Double_Harness_Loop}}
		}
\end{figure}

\begin{figure}[H]\centering
	\subfloat[Результат]{\label{ris:Double_Harness_Loop_2}
	\tcbox[enhanced jigsaw,colframe=black,opacityframe=0.5,opacityback=0.5]
		{\centering
			\includesvg[width=0.45\linewidth]{Nonslide/Double_Harness_Loop_1}}
		}
	\caption{Двойная Harness Loop.}\label{ris:Double_Harness_Loop}
\end{figure}

\subsection{Двойной проводник}

\begin{figure}[H]\centering
	\subfloat[Завязывание]{\label{ris:Double_provodnik_1}
	\tcbox[enhanced jigsaw,colframe=black,opacityframe=0.5,opacityback=0.5]
		{\centering
			\includesvg[width=0.5\linewidth]{Nonslide/Double_provodnik}}
		}
\end{figure}

\begin{figure}[H]\centering
	\subfloat[Завязывание]{\label{ris:Double_provodnik_2}
	\tcbox[enhanced jigsaw,colframe=black,opacityframe=0.5,opacityback=0.5]
		{\centering
			\includesvg[width=0.6\linewidth]{Nonslide/Double_provodnik_1}}
		}
\end{figure}

\begin{figure}[H]\centering
	\subfloat[Результат]{\label{ris:Double_provodnik_3}
	\tcbox[enhanced jigsaw,colframe=black,opacityframe=0.5,opacityback=0.5]
		{\centering
			\includesvg[width=0.6\linewidth]{Nonslide/Double_provodnik_2}}
		}
	\caption{Двойной проводник.}\label{ris:Double_provodnik}
\end{figure}

\subsection{Бабочка}

%TODO Двуглавый Мотыль???

Узел \enquote{Бабочка} - Срединный, демпфирующий узел, работает как компенсатор и только в одном направлении!

Крик души в интернете:
- Совершенно устал “биться” с дремучей безграмотностью, когда узел “австрийский проводник” называют и “бабочкой” в том числе! Благодаря халатности отдельных деятелей, пишущих книжки про узлы, в начале 90-х эта ошибка плотно засела в бумаге, а теперь и в глобальной сети... Приводимое в качестве примера название “Alpine butterfly” (альпийская бабочка) - неологизм, появилось гораздо позже устоявшегося у нас названия - “австрийский проводник”, одним словом (butterfly) на Западе не используется, в русской транскрипции нормальными альпинистами не употребляется. Господа! Это два совершенно различных узла, хотя и внешне, при беглом осмотре, сразу найти различия весьма нелегко, уж слишком походят они друг на друга! Но, разные.. и по функционалу, в том числе.
%TODO Уточнить в Интернете

\begin{figure}[H]\centering
	\subfloat[Завязывание]{\label{ris:Butterfly_1}
	\tcbox[enhanced jigsaw,colframe=black,opacityframe=0.5,opacityback=0.5,height=5cm]
		{\centering
			\includesvg[width=0.44\linewidth]{Nonslide/Butterfly_1}}
		}
\hfil
	\subfloat[Результат]{\label{ris:Butterfly_2}
	\tcbox[enhanced jigsaw,colframe=black,opacityframe=0.5,opacityback=0.5,height=5cm]
		{\centering
			\includesvg[width=0.35\linewidth]{Nonslide/Butterfly_2}}
		}
	\caption{Бабочка.}\label{ris:Butterfly}
\end{figure}

\subsection{Австрийский проводник}

\begin{figure}[H]\centering
	\subfloat[Результат]{\label{ris:Alpine_butterfly_loop_1}
	\tcbox[enhanced jigsaw,colframe=black,opacityframe=0.5,opacityback=0.5]
		{\centering
			\includesvg[width=0.45\linewidth]{Nonslide/Alpine_butterfly_loop}}
		}
\end{figure}

Одиночный баттерфляй, Ездовая петля, узел третьего, пчелка, бурлацкая петля, бабочка, альпийская бабочка, артиллерийская петля, срединный проводник, бергшафт, бергштоф, альпийский мотыль. Английские названия - alpine butterfly loop, Lineman’s Loop, Linesman Loop, Farmer’s Loop. Срединный узел, работает как опорная петля во всех направлениях.

\begin{figure}[H]\centering
	\subfloat[Завязывание. 1 Вариант]{\label{ris:Alpine_butterfly_loop_2}
	\tcbox[enhanced jigsaw,colframe=black,opacityframe=0.5,opacityback=0.5,height=5.5cm]
		{\centering
			\includesvg[width=0.28\linewidth]{Nonslide/Alpine_butterfly_loop_2}}
		}
\hfil
	\subfloat[Завязывание. 2 Вариант]{\label{ris:Alpine_butterfly_loop_3}
	\tcbox[enhanced jigsaw,colframe=black,opacityframe=0.5,opacityback=0.5,height=5.5cm]
		{\centering
			\includesvg[width=0.51\linewidth]{Nonslide/Alpine_butterfly_loop_1_2}}
		}
\end{figure}

Австрийский проводник можно использовать не только как петлю, но и для крепления к опоре.

\begin{figure}[H]\centering
	\subfloat[Привязывание к опоре\\ Первый способ]{\label{ris:Alpine_butterfly_loop_4}
	\tcbox[enhanced jigsaw,colframe=black,opacityframe=0.5,opacityback=0.5,height=4.5cm]
		{\centering
			\includesvg[width=0.35\linewidth]{Nonslide/Alpine_butterfly_loop_3}}
		}
\hfil
	\subfloat[Привязывание к опоре\\ Первый способ]{\label{ris:Alpine_butterfly_loop_5}
	\tcbox[enhanced jigsaw,colframe=black,opacityframe=0.5,opacityback=0.5,height=4.5cm]
		{\centering
			\includesvg[width=0.35\linewidth]{Nonslide/Alpine_butterfly_loop_3_1}}
		}
\end{figure}

\begin{figure}[H]\centering
	\subfloat[Результат привязывания к опоре первым способом]{\label{ris:Alpine_butterfly_loop_6}
	\tcbox[enhanced jigsaw,colframe=black,opacityframe=0.5,opacityback=0.5]
		{\centering
			\includesvg[width=0.4\linewidth]{Nonslide/Alpine_butterfly_loop_3_2}}
		}
\end{figure}

\begin{figure}[H]\centering
	\subfloat[Привязывание к опоре\\ Второй способ]{\label{ris:Alpine_butterfly_loop_7}
	\tcbox[enhanced jigsaw,colframe=black,opacityframe=0.5,opacityback=0.5,height=4cm]
		{\centering
			\includesvg[width=0.35\linewidth]{Nonslide/Alpine_butterfly_loop_4}}
		}
\hfil
	\subfloat[Привязывание к опоре\\ Второй способ]{\label{ris:Alpine_butterfly_loop_8}
	\tcbox[enhanced jigsaw,colframe=black,opacityframe=0.5,opacityback=0.5,height=4cm]
		{\centering
			\includesvg[width=0.35\linewidth]{Nonslide/Alpine_butterfly_loop_4_1}}
		}
\end{figure}

\begin{figure}[H]\centering
	\subfloat[Привязывание к опоре\\ Второй способ]{\label{ris:Alpine_butterfly_loop_9}
	\tcbox[enhanced jigsaw,colframe=black,opacityframe=0.5,opacityback=0.5,height=4.5cm]
		{\centering
			\includesvg[width=0.35\linewidth]{Nonslide/Alpine_butterfly_loop_4_2}}
		}
\hfil
	\subfloat[Результат привязывания к опоре вторым способом]{\label{ris:Alpine_butterfly_loop_10}
	\tcbox[enhanced jigsaw,colframe=black,opacityframe=0.5,opacityback=0.5,height=4.5cm]
		{\centering
			\includesvg[width=0.4\linewidth]{Nonslide/Alpine_butterfly_loop_4_3}}
		}
	\caption{Австрийский проводник.}\label{ris:Alpine_butterfly_loop}
\end{figure}

\subsection{Одиночный баттерфляй}

Почти то же самое, что и Австрийский проводник (рис.~\ref{ris:Alpine_butterfly_loop}, но первоначальное закручивание осуществляется не против, а по часовой стрелке и петля поднимается не спереди, а сзади узла.

\begin{figure}[H]\centering
	\subfloat[Завязывание]{\label{ris:Single_butterfly_1}
	\tcbox[enhanced jigsaw,colframe=black,opacityframe=0.5,opacityback=0.5,height=5cm]
		{\centering
			\includesvg[width=0.37\linewidth]{Nonslide/Single_butterfly_2}}
		}
\hfil
	\subfloat[Завязывание]{\label{ris:Single_butterfly_2}
	\tcbox[enhanced jigsaw,colframe=black,opacityframe=0.5,opacityback=0.5,height=5cm]
		{\centering
			\includesvg[width=0.42\linewidth]{Nonslide/Single_butterfly}}
		}
\end{figure}

\begin{figure}[H]\centering
	\subfloat[Результат]{\label{ris:Single_butterfly_3}
	\tcbox[enhanced jigsaw,colframe=black,opacityframe=0.5,opacityback=0.5,height=5cm]
		{\centering
			\includesvg[width=0.35\linewidth]{Nonslide/Single_butterfly_1}}
		}
	\caption{Одиночный баттерфляй.}\label{ris:Single_butterfly}
\end{figure}

\subsection{Баттерфляй двойной}

Если получившуюся петлю наполовину вернуть обратно в узел и охватить ею получившиеся половинки - получится узел с двумя петлями. Аналогичным способом можно завязать любую срединную петлю.

\begin{figure}[H]\centering
	\subfloat[Завязывание]{\label{ris:Double_butterfly_1}
	\tcbox[enhanced jigsaw,colframe=black,opacityframe=0.5,opacityback=0.5,height=5.5cm]
		{\centering
			\includesvg[width=0.41\linewidth]{Nonslide/Double_butterfly_2}}
		}
\hfil
	\subfloat[Результат]{\label{ris:Double_butterfly_2}
	\tcbox[enhanced jigsaw,colframe=black,opacityframe=0.5,opacityback=0.5,height=5.5cm]
		{\centering
			\includesvg[width=0.38\linewidth]{Nonslide/Double_butterfly_3}}
		}
	\caption{Двойной баттерфляй.}\label{ris:Double_butterfly}
\end{figure}

% TODO Узел \enquote{заячьи ушки} (название по принятой в туристской практике транскрипции), на самом деле является известным узлом \enquote{двойной проводник}. При всей его надежности, он может быть крайне опасным. Дело в том, что завязывая узел, невозможно добиться полной одинаковости выходящих из узла двух петель. Одна из них, даже незаметно для глаза, будет короче другой. Это означает, что при чрезмерной нагрузке в первую очередь лопнет наиболее короткая петля. Кажущаяся надежность (в точке закрепления лежит двойной репшнур!) не соответствует истине, рвется одинарная прядь репшнура, а вторая может лопнуть вслед за первой.

\subsection{Петлевой узел}

%TODO Из Кровавого?

\begin{figure}[H]\centering
	\begin{minipage}{1\linewidth}
		\begin{center}
			\tcbox[enhanced jigsaw,colframe=black,opacityframe=0.5,opacityback=0.5]
			{\centering{\includesvg[width=0.5\linewidth]{Nonslide/Petlevoy}}}
		\end{center}
	\end{minipage}
\caption{Петлевой узел.}
\label{ris:Petlevoy}
\end{figure}

Английское название - Blood Loop Dropper Knot.

\subsection{Петля из Left Half Knot 1}

\begin{figure}[H]\centering
	\subfloat[Завязывание]{\label{ris:made_from_Left_Half_Knot_1_1}
	\tcbox[enhanced jigsaw,colframe=black,opacityframe=0.5,opacityback=0.5,height=6.5cm]
		{\centering
			\includesvg[width=0.42\linewidth]{Nonslide/made_from_Left_Half_Knot}}
		}
\hfil
	\subfloat[Завязывание]{\label{ris:made_from_Left_Half_Knot_1_2}
	\tcbox[enhanced jigsaw,colframe=black,opacityframe=0.5,opacityback=0.5,height=6.5cm]
		{\centering
			\includesvg[width=0.37\linewidth]{Nonslide/made_from_Left_Half_Knot_1}}
		}
\end{figure}

\begin{figure}[H]\centering
	\subfloat[Результат]{\label{ris:made_from_Left_Half_Knot_1_3}
	\tcbox[enhanced jigsaw,colframe=black,opacityframe=0.5,opacityback=0.5]
		{\centering
			\includesvg[width=0.3\linewidth]{Nonslide/made_from_Left_Half_Knot_2}}
		}
	\caption{Петля из Left Half Knot 1.}\label{ris:made_from_Left_Half_Knot_1}
\end{figure}

\subsection{Петля из Left Half Knot 2}

\begin{figure}[H]\centering
	\subfloat[Завязывание]{\label{ris:made_from_Left_Half_Knot_2_1}
	\tcbox[enhanced jigsaw,colframe=black,opacityframe=0.5,opacityback=0.5,height=7cm]
		{\centering
			\includesvg[width=0.29\linewidth]{Nonslide/made_from_Left_Half_Knot_3}}
		}
\hfil
	\subfloat[Результат]{\label{ris:made_from_Left_Half_Knot_2_2}
	\tcbox[enhanced jigsaw,colframe=black,opacityframe=0.5,opacityback=0.5,height=7cm]
		{\centering
			\includesvg[width=0.3\linewidth]{Nonslide/made_from_Left_Half_Knot_3_1}}
		}
	\caption{Петля из Left Half Knot 2.}\label{ris:made_from_Left_Half_Knot_2}
\end{figure}

\subsection{Two-Strand Diamond Knot}

Носит чисто декоративный характер. См. также Two-Strand Matthew Walker Knot, Two-Strand Chinese Crown Knot (рис.~\ref{ris:Decorative_Chinese_Loop}), Two-Strand Knife Lanyard Knot и Two-Strand Sinnet Knot.
%TODO Может, развернуть верхнюю петлю?

\begin{figure}[H]\centering
	\subfloat[Завязывание]{\label{ris:Two_Strand_Diamond_Knot_1}
	\tcbox[enhanced jigsaw,colframe=black,opacityframe=0.5,opacityback=0.5,height=6cm]
		{\centering
			\includesvg[width=0.35\linewidth]{Nonslide/Two-Strand_Diamond_Knot}}
		}
\hfil
	\subfloat[Результат]{\label{ris:Two_Strand_Diamond_Knot_2}
	\tcbox[enhanced jigsaw,colframe=black,opacityframe=0.5,opacityback=0.5,height=6cm]
		{\centering
			\includesvg[width=0.35\linewidth]{Nonslide/Two-Strand_Diamond_Knot_1}}
		}
	\caption{Two-Strand Diamond Knot.}\label{ris:Two_Strand_Diamond_Knot}
\end{figure}

\subsection{Китайская петля}

Китайский коронный узел. Английские названия - Decorative Chinese Loop, Chinese Crown Knot.

\begin{figure}[H]\centering
	\subfloat[Завязывание]{\label{ris:Decorative_Chinese_Loop_1}
	\tcbox[enhanced jigsaw,colframe=black,opacityframe=0.5,opacityback=0.5,height=5.5cm]
		{\centering
			\includesvg[width=0.3\linewidth]{Nonslide/Decorative_Chinese_Loop_1_1}}
		}
\hfil
	\subfloat[Завязывание]{\label{ris:Decorative_Chinese_Loop_2}
	\tcbox[enhanced jigsaw,colframe=black,opacityframe=0.5,opacityback=0.5,height=5.5cm]
		{\centering
			\includesvg[width=0.23\linewidth]{Nonslide/Decorative_Chinese_Loop_1_2}}
		}
\end{figure}

\begin{figure}[H]\centering
	\subfloat[Завязывание]{\label{ris:Decorative_Chinese_Loop_3}
	\tcbox[enhanced jigsaw,colframe=black,opacityframe=0.5,opacityback=0.5,height=5cm]
		{\centering
			\includesvg[width=0.3\linewidth]{Nonslide/Decorative_Chinese_Loop_1_3}}
		}
\hfil
	\subfloat[Завязывание]{\label{ris:Decorative_Chinese_Loop_4}
	\tcbox[enhanced jigsaw,colframe=black,opacityframe=0.5,opacityback=0.5,height=5cm]
		{\centering
			\includesvg[width=0.31\linewidth]{Nonslide/Decorative_Chinese_Loop_1_4}}
		}
\end{figure}

\begin{figure}[H]\centering
	\subfloat[Результат]{\label{ris:Decorative_Chinese_Loop_5}
	\tcbox[enhanced jigsaw,colframe=black,opacityframe=0.5,opacityback=0.5]
		{\centering
			\includesvg[width=0.65\linewidth]{Nonslide/Decorative_Chinese_Loop}}
		}
	\caption{Китайская петля.}\label{ris:Decorative_Chinese_Loop}
\end{figure}

\subsection{Скользящая китайская корона}

Скользящий китайский коронный узел.

\begin{figure}[H]\centering
	\subfloat[Завязывание]{\label{ris:Sliding_chinese_crown_1}
	\tcbox[enhanced jigsaw,colframe=black,opacityframe=0.5,opacityback=0.5]
		{\centering
			\includesvg[width=0.55\linewidth]{Nonslide/Sliding_chinese_crown}}
		}
\end{figure}

\begin{figure}[H]\centering
	\subfloat[Завязывание]{\label{ris:Sliding_chinese_crown_2}
	\tcbox[enhanced jigsaw,colframe=black,opacityframe=0.5,opacityback=0.5]
		{\centering
			\includesvg[width=0.6\linewidth]{Nonslide/Sliding_chinese_crown_1}}
		}
\end{figure}

\begin{figure}[H]\centering
	\subfloat[Результат]{\label{ris:Sliding_chinese_crown_3}
	\tcbox[enhanced jigsaw,colframe=black,opacityframe=0.5,opacityback=0.5]
		{\centering
			\includesvg[width=0.65\linewidth]{Nonslide/Sliding_chinese_crown_2}}
		}
	\caption{Скользящая китайская корона.}\label{ris:Sliding_chinese_crown}
\end{figure}

\subsection{Pendant Hitch}

Г-н Эшли пишет, что узел применяется для изготовления различных декоративных подвесок. Очень похож на The True Lovers or Fisherman's Knot.

\begin{figure}[H]\centering
	\begin{minipage}{1\linewidth}
		\begin{center}
			\tcbox[enhanced jigsaw,colframe=black,opacityframe=0.5,opacityback=0.5]
			{\centering{\includesvg[width=0.45\linewidth]{Nonslide/Pendant_Hitch}}}
		\end{center}
	\end{minipage}
\caption{Pendant Hitch.}
\label{ris:Pendant_Hitch}
\end{figure}

\subsection{Перекрут Бимини}

\begin{figure}[H]\centering
	\subfloat[Завязывание]{\label{ris:Bimini_Twist_1}
	\tcbox[enhanced jigsaw,colframe=black,opacityframe=0.5,opacityback=0.5]
		{\centering
			\includesvg[width=0.9\linewidth]{Nonslide/Bimini_Twist}}
		}
\end{figure}

\begin{figure}[H]\centering
	\subfloat[Завязывание]{\label{ris:Bimini_Twist_2}
	\tcbox[enhanced jigsaw,colframe=black,opacityframe=0.5,opacityback=0.5]
		{\centering
			\includesvg[width=0.95\linewidth]{Nonslide/Bimini_Twist_1}}
		}
\end{figure}

\begin{figure}[H]\centering
	\subfloat[Результат]{\label{ris:Bimini_Twist_3}
	\tcbox[enhanced jigsaw,colframe=black,opacityframe=0.5,opacityback=0.5]
		{\centering
			\includesvg[width=1\linewidth]{Nonslide/Bimini_Twist_2}}
		}
	\caption{Перекрут Бимини.}\label{ris:Bimini_Twist}
\end{figure}

Английское название - Bimini Twist.

\subsection{Австралийская коса}

%TODO Посмотреть в Косичках

Основан на Косичке из Простого узла (рис.~\ref{ris:Single_braid}).
%FIXME или на основа косички из Восьмерки?

Завязывание Австралийской косички (Австралийского жгута):

\begin{enumerate}
 \item Сделайте петлю, оставляя конец веревки длинным.
 \item Завяжите косичкой петлю и длинный конец, плотно затягивая (длина косы будет зависеть от нагрузки на леске).
 \item Завершите косичку с помощью открытой петли на конце веревки.
 \item Протяните исходную петлю через открытую петлю.
 \item Смочите узел водой, затяните его, аккуратно потянув за конец веревки.
 \item Подрежьте оставшийся длинный конец.
\end{enumerate}

\begin{table}[H]
\begin{center}
\begin{tabular}{ c|c }
\toprule %%% верхняя линейка
Нагрузка на леску/трос (кг) & Длина жгута/косички (мм) \\
\midrule %%% тонкий разделитель
До 2,7 & \textbf{26} \\
3,6 - 5,4 & \textbf{50} \\
7,2 - 9 & \textbf{80} \\
13,5 & \textbf{150} \\
22,5 и более & \textbf{200} \\
\bottomrule %%% нижняя линейка
\end{tabular}
\end{center}
\caption{Рекомендации от \href{http://www.leadertec.com}{Leadertec.com}.}
\label{tabl:Nagruzka_1}
\end{table}

Таблица показывает рекомендованную длину плетения для разных лесок/тросов. Сопротивление разрыву: Утверждается, что австралийская коса (или жгут), как и скрутка Бимини, сопротивляется разрыву на 100\%. Надо заметить, что этот результат скорее всего был получен в лаборатории, где условия идеальны - сухо, тепло, и к тросу не приложена ударная нагрузка. Внезапные толчки вырабатывают тепло в процессе трения и скорее всего со временем появятся повреждения, когда будет приложена большая рывковая нагрузка.

AUSTRIALIAN PLAIT OR BRAID. Некоторые рыболовы даже считают этот узел основой для вязания BIMINI TWIST. Однако его вяжут несколько дольше. Хотя оба этих узла отличаются стопроцентной прочностью, те, кто использует австралийскую косичку, считают, что он дополнительно амортизирует напряжение лески. Завязанный узел образует петлю (подобно BIMINI) и может быть использован везде, где применяется BIMINI TWIST.

\begin{enumerate}
 \item Основная леска должна быть натянутой до окончания завязывания узла. Оставьте рабочий конец лески длиной около 45-60 см. Размер петли зависит от потребностей. Завязывание начните, поддерживая нижнюю часть петли (или обе части лески - в случае более длинной петли) безымянным пальцем и мизинцем левой руки. В этот момент конец лески необходимо придерживать большим и указательным пальцами левой руки. Заметьте, что рабочий конец лески находится позади основной. Придерживайте конец лески и ее главную часть большим и указательным пальцем в точке пересечения.
 \item Скрестите основную леску с ее рабочим концом и перекиньте этот конец через петлю. Продолжайте придерживать то место, где лески пересекаются.
 \item Вновь перекиньте конец лески вокруг ее главной части и просуньте в петлю.
 \item Проверьте ваши успехи. Держите петлю (или обе части лески) в одной руке и конец лески в другой. Мы сделали два витка - конец лески закручен дважды вокруг ее главной части.
 \item Потяните конец лески назад, по направлению к главной ее части, чтобы затянуть витки в шаге 3. Используйте правую руку, чтобы удержать это соединение.
 \item Продолжая поддерживать соединение основной лески и ее конца, перекиньте его вокруг верхней части петли и вниз через нее.
 \item Потяните концы лески назад к началу косички и затяните ее. Затем перекиньте главную часть лески вокруг другой части петли и потяните за основную леску, чтобы стянуть ее с косичкой. Повторите весь процесс от 15 до 20 раз на каждой стороне, меняя порядок переплетения то с одной, то с другой стороны петли. Это - просто следующее друг за другом переплетение, где каждую заплетенную косичку нужно затянуть, прежде чем начать вязать новую.
 \item Завязав нужное число косичек, необходимо закончить узел так же, как в случае Bimini. Придерживайте две стороны петли на расстоянии около 5 см от косички, затем перекиньте ее через созданную таким образом малую петлю. Это действие необходимо повторить, по крайней мере, 4 раза, а затем точно затянуть весь узел. Не забудьте окончательно затянуть узел с помощью плоскогубцев.
 \item Отрежьте конец лески, чтобы готовый узел выглядел так, как на рисунке.
\end{enumerate}

\begin{figure}[H]\centering
	\subfloat[Завязывание]{\label{ris:Australian_1}
	\tcbox[enhanced jigsaw,colframe=black,opacityframe=0.5,opacityback=0.5]
		{\centering
			\includesvg[width=1\linewidth]{Nonslide/Australian}}
		}
\end{figure}

\begin{figure}[H]\centering
	\subfloat[Завязывание]{\label{ris:Australian_2}
	\tcbox[enhanced jigsaw,colframe=black,opacityframe=0.5,opacityback=0.5]
		{\centering
			\includesvg[width=1\linewidth]{Nonslide/Australian_1}}
		}
\end{figure}

\begin{figure}[H]\centering
	\subfloat[Завязывание]{\label{ris:Australian_3}
	\tcbox[enhanced jigsaw,colframe=black,opacityframe=0.5,opacityback=0.5]
		{\centering
			\includesvg[width=1\linewidth]{Nonslide/Australian_2}}
		}
\end{figure}

\begin{figure}[H]\centering
	\subfloat[Результат]{\label{ris:Australian_4}
	\tcbox[enhanced jigsaw,colframe=black,opacityframe=0.5,opacityback=0.5]
		{\centering
			\includesvg[width=1\linewidth]{Nonslide/Australian_3}}
		}
	\caption{Австралийская косичка.}\label{ris:Australian}
\end{figure}

\subsection{Обезьянья цепочка}

Английские названия - Monkey Chain, Monkey Chain Lanyard Knot или Chain Shortening. Начинать вязку обычно начинают со скользящей петли, но, в принципе, это не обязательно. Можно начать с любой петли, как затягивающейся, так и незатягивающейся.

\begin{figure}[H]\centering
	\subfloat[Завязывание.\\Первый вариант]{\label{ris:Monkey_Chain_1}
	\tcbox[enhanced jigsaw,colframe=black,opacityframe=0.5,opacityback=0.5]
		{\centering
			\includesvg[width=0.4\linewidth]{Nonslide/Monkey_Chain_or_Chain_Shortening}}
		}
\end{figure}

В первом варианте ходовой конец входит в петлю снаружи.

\begin{figure}[H]\centering
	\subfloat[Результат.\\Первый вариант]{\label{ris:Monkey_Chain_2}
	\tcbox[enhanced jigsaw,colframe=black,opacityframe=0.5,opacityback=0.5]
		{\centering
			\includesvg[width=0.8\linewidth]{Nonslide/Monkey_Chain_or_Chain_Shortening_1}}
		}
\end{figure}

\begin{figure}[H]\centering
	\subfloat[Завязывание.\\Второй вариант]{\label{ris:Monkey_Chain_3}
	\tcbox[enhanced jigsaw,colframe=black,opacityframe=0.5,opacityback=0.5]
		{\centering
			\includesvg[width=0.4\linewidth]{Nonslide/Monkey_Chain_or_Chain_Shortening_2}}
		}
\end{figure}

Во втором варианте ходовой конец входит в петлю изнутри.

\begin{figure}[H]\centering
	\subfloat[Результат.\\Второй вариант]{\label{ris:Monkey_Chain_4}
	\tcbox[enhanced jigsaw,colframe=black,opacityframe=0.5,opacityback=0.5]
		{\centering
			\includesvg[width=1\linewidth]{Nonslide/Monkey_Chain_or_Chain_Shortening_2_1}}
		}
	\caption{Обезьянья цепочка.}\label{ris:Monkey_Chain}
\end{figure}

\subsection{Double Monkey Chain Lanyard Knot}

Он же Trumpet Cord.

\begin{figure}[H]\centering
	\subfloat[Завязывание]{\label{ris:Trumpet_Cord_1}
	\tcbox[enhanced jigsaw,colframe=black,opacityframe=0.5,opacityback=0.5]
		{\centering
			\includesvg[width=0.5\linewidth]{Nonslide/Trumpet_Cord}}
		}
\end{figure}

\begin{figure}[H]\centering
	\subfloat[Завязывание]{\label{ris:Trumpet_Cord_2}
	\tcbox[enhanced jigsaw,colframe=black,opacityframe=0.5,opacityback=0.5]
		{\centering
			\includesvg[width=0.7\linewidth]{Nonslide/Trumpet_Cord_1}}
		}
\end{figure}

\begin{figure}[H]\centering
	\subfloat[Результат]{\label{ris:Trumpet_Cord_3}
	\tcbox[enhanced jigsaw,colframe=black,opacityframe=0.5,opacityback=0.5]
		{\centering
			\includesvg[width=1\linewidth]{Nonslide/Trumpet_Cord_2}}
		}
	\caption{Double Monkey Chain Lanyard Knot.}\label{ris:Trumpet_Cord}
\end{figure}

\subsection{Петля из Констриктора}

\begin{figure}[H]\centering
	\subfloat[Завязывание]{\label{ris:made_from_Constrictor_Knot_1}
	\tcbox[enhanced jigsaw,colframe=black,opacityframe=0.5,opacityback=0.5,height=6.5cm]
		{\centering
			\includesvg[width=0.33\linewidth]{Nonslide/made_from_Constrictor_Knot}}
		}
\hfil
	\subfloat[Результат]{\label{ris:made_from_Constrictor_Knot_2}
	\tcbox[enhanced jigsaw,colframe=black,opacityframe=0.5,opacityback=0.5,height=6.5cm]
		{\centering
			\includesvg[width=0.46\linewidth]{Nonslide/made_from_Constrictor_Knot_1}}
		}
	\caption{Петля из Констриктора.}\label{ris:made_from_Constrictor_Knot}
\end{figure}

После завязывания внешне похожа на Дубовую петлю.

\subsection{Knot Shortening}

Укорачивающий узел.

\begin{figure}[H]\centering
	\subfloat[Завязывание]{\label{ris:Knot_Shortening_1}
	\tcbox[enhanced jigsaw,colframe=black,opacityframe=0.5,opacityback=0.5]
		{\centering
			\includesvg[width=0.6\linewidth]{Nonslide/Knot_Shortening}}
		}
\end{figure}

\begin{figure}[H]\centering
	\subfloat[Завязывание]{\label{ris:Knot_Shortening_2}
	\tcbox[enhanced jigsaw,colframe=black,opacityframe=0.5,opacityback=0.5]
		{\centering
			\includesvg[width=0.7\linewidth]{Nonslide/Knot_Shortening_1}}
		}
\end{figure}

\begin{figure}[H]\centering
	\subfloat[Результат]{\label{ris:Knot_Shortening_3}
	\tcbox[enhanced jigsaw,colframe=black,opacityframe=0.5,opacityback=0.5]
		{\centering
			\includesvg[width=0.75\linewidth]{Nonslide/Knot_Shortening_2}}
		}
	\caption{Knot Shortening.}\label{ris:Knot_Shortening}
\end{figure}

\subsection{Затягивающаяся Баранья нога}

Затягивающийся Sheepshank Knot (рис.~\ref{ris:Sheepshank_Knot}), который фиксируется узлами. Обратите внимание, на обоих концах получаются Булини.

\begin{figure}[H]\centering
	\begin{minipage}{1\linewidth}
		\begin{center}
			\tcbox[enhanced jigsaw,colframe=black,opacityframe=0.5,opacityback=0.5]
			{\centering{\includesvg[width=0.95\linewidth]{Nonslide/Sheepshank_Knot_slide}}}
		\end{center}
	\end{minipage}
\caption{Затягивающаяся Баранья нога.}
\label{ris:Sheepshank_Knot_slide}
\end{figure}

\subsection{Топовый узел}

Masthead, Jury Mast Knot или Pitcher Knot. Есть несколько вариантов, отличающихся расположением петель.

\begin{figure}[H]\centering
	\subfloat[Завязывание.\\Первый вариант]{\label{ris:Masthead_1}
	\tcbox[enhanced jigsaw,colframe=black,opacityframe=0.5,opacityback=0.5]
		{\centering
			\includesvg[width=0.7\linewidth]{Nonslide/Masthead}}
		}
\end{figure}

\begin{figure}[H]\centering
	\subfloat[Результат.\\Первый вариант]{\label{ris:Masthead_2}
	\tcbox[enhanced jigsaw,colframe=black,opacityframe=0.5,opacityback=0.5]
		{\centering
			\includesvg[width=0.8\linewidth]{Nonslide/Masthead_1}}
		}
\end{figure}

\begin{figure}[H]\centering
	\subfloat[Завязывание.\\Второй вариант]{\label{ris:Masthead_3}
	\tcbox[enhanced jigsaw,colframe=black,opacityframe=0.5,opacityback=0.5]
		{\centering
			\includesvg[width=0.7\linewidth]{Nonslide/Masthead_2}}
		}
\end{figure}

\begin{figure}[H]\centering
	\subfloat[Результат.\\Второй вариант]{\label{ris:Masthead_4}
	\tcbox[enhanced jigsaw,colframe=black,opacityframe=0.5,opacityback=0.5]
		{\centering
			\includesvg[width=0.85\linewidth]{Nonslide/Masthead_2_1}}
		}
\end{figure}

\begin{figure}[H]\centering
	\subfloat[Завязывание.\\Третий вариант]{\label{ris:Masthead_5}
	\tcbox[enhanced jigsaw,colframe=black,opacityframe=0.5,opacityback=0.5]
		{\centering
			\includesvg[width=0.7\linewidth]{Nonslide/Masthead_3}}
		}
\end{figure}

\begin{figure}[H]\centering
	\subfloat[Результат.\\Третий вариант]{\label{ris:Masthead_5}
	\tcbox[enhanced jigsaw,colframe=black,opacityframe=0.5,opacityback=0.5]
		{\centering
			\includesvg[width=0.85\linewidth]{Nonslide/Masthead_3_1}}
		}
	\caption{Топовый узел.}\label{ris:Masthead}
\end{figure}

\subsection{Французский топовый}

Английское название - French Masthead Knot.

\begin{figure}[H]\centering
	\subfloat[Завязывание]{\label{ris:French_Masthead_Knot_1}
	\tcbox[enhanced jigsaw,colframe=black,opacityframe=0.5,opacityback=0.5]
		{\centering
			\includesvg[width=0.5\linewidth]{Nonslide/French_Masthead_Knot}}
		}
\end{figure}

\begin{figure}[H]\centering
	\subfloat[Результат]{\label{ris:French_Masthead_Knot_2}
	\tcbox[enhanced jigsaw,colframe=black,opacityframe=0.5,opacityback=0.5]
		{\centering
			\includesvg[width=0.5\linewidth]{Nonslide/French_Masthead_Knot_1}}
% 			\includesvg[width=0.75\linewidth]{Nonslide/French_Masthead_Knot}}
		}
	\caption{Французский топовый.}\label{ris:French_Masthead_Knot}
\end{figure}
