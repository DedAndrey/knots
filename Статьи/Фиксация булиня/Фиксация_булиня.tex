\documentclass{artikel1}
%~~~~~~~~~~Математика~~~~~~~~~~
\usepackage{amsmath,amsthm,amssymb}
% \usepackage[warn]{mathtext} % русские буквы в формулах, с предупреждением

%~~~~~~~~~~Текст~~~~~~~~~~
\usepackage{verse}
\usepackage[utf8]{inputenc} % кодовая страница документа
\usepackage[T2A]{fontenc} % внутренняя кодировка TeX
\usepackage[english, russian]{babel} % локализация и переносы
\usepackage[babel]{csquotes} % вставка кавычек
\usepackage{indentfirst} % русский стиль: отступ первого абзаца раздела
\usepackage{anyfontsize} % решает проблему с размером шрифтов
% \usepackage{soul} % Разряженный текст \so{} и подчеркивание \ul{}
% \usepackage{soulutf8} % Поддержка UTF8 в soul
% \usepackage{ulem} % Зачеркивание

%~~~~~~~~~~Графика~~~~~~~~~~
\usepackage{graphicx} % Работа с графикой \includegraphics{}
	\graphicspath{Bowline/} % Папки с иллюстрациями
\usepackage{svg} % Работа с SVG
	\svgsetup{inkscapearea=page,inkscapepath=svgdir}
\usepackage[most]{tcolorbox} % Рамки вокруг рисунков
	\tcbset{boxrule=0.4pt,drop lifted shadow=black,valign=center}
\usepackage{float} % плавающие объекты
\usepackage{wrapfig} % Обтекание картинок текстом
\usepackage[labelformat=simple]{subfig}
% 	\renewcommand{\thesubfigure}{\relax} % без меток на подписях
	\renewcommand{\thesubfigure}{\asbuk{subfigure}.} % метка subfigure: "(а)" вместо дефолтного "а"
\usepackage[justification=centering]{caption} % Центрирует многострочные подписи к рисункам
	\DeclareCaptionLabelFormat{cont}{#1~#2~(\asbuk{ContinuedFloat})}
\usepackage{color,colortbl,xcolor,transparent} % работа с цветом
	\definecolor{BlueGreen}{RGB}{49,152,255} % цвета ссылок
	\definecolor{Violet}{RGB}{120,80,120}
\usepackage{titlepic} % Логотип на титульной странице

%~~~~~~~~~~Разное~~~~~~~~~~
\usepackage{ifthen} % Включаем логику \ifthenelse{условие проверки}{если НЕ истина}{если истина}
\usepackage{epigraph} % Вставка эпиграфа
\usepackage{qrcode} % Поддержка QR-code
\usepackage[section,above,below]{placeins} % указывает \FloatBarrier перед каждым \section автоматически
\usepackage{afterpage} % вставить барьер сразу после начала новой страницы
\usepackage{flafter} % помещает флоат ПОСЛЕ первой ссылки на него
\usepackage[
	type={CC},
	modifier={by-nc-sa},
	version={4.0},
	]{doclicense} % Лицензия

%~~~~~~~~~~Таблицы~~~~~~~~~~
\usepackage{array} % работа с таблицами
\usepackage{booktabs} % возможность набирать красивые таблицы

%~~~~~~~~~~Библиография~~~~~~~~~~
\usepackage[round,sort,numbers]{natbib}
\bibliographystyle{unsrt} % Стиль библиографии
	\makeatletter
	\renewcommand{\@biblabel}[1]{#1.} % Заменяем библиографию с квадратных скобок на точку
	\makeatother

%~~~~~~~~~~Ссылки~~~~~~~~~~
\usepackage[
	unicode=true,
	colorlinks=true,
	urlcolor=BlueGreen,
	linkcolor=Violet,
	citecolor=Violet]{hyperref}

\pdfsuppresswarningpagegroup=1

\begin{document}                                     % начало документа

\title{Фиксация Булиня}
\date{}
\author{А.В. Дедиков}

\maketitle

Булинь (другое название - Беседочный узел) по праву называют ``Король узлов''! Он сочетает в себе простоту и надежность, никогда намертво не затягивается, вяжется очень легко. Его можно завязать даже одной рукой в полной темноте.

Этот узел известен уже несколько тысяч лет, им пользовались еще древнеегипетские и финикийские моряки, а англичане его завязывание описывали в стихах:

~\
\begin{verse}%[\versewidth]
Lay the bight to make a hole \\
Then under the back and around the pole \\
Over the top and thru the eye \\
Cinch it tight and let it lie.
\end{verse}
~\

Но при всех достоинствах есть у Булиня и недостаток. Иногда при пульсирующих нагрузках, особенно на скользких веревках, он может развязаться. Чтобы застраховаться от этого нужно как-то зафиксировать ходовой конец. Для этого существует несколько способов:

\section{Наложение марки}

\begin{figure}[H]\centering
	\begin{minipage}{1\linewidth}
		\begin{center}
			\tcbox[enhanced jigsaw,colframe=black,opacityframe=0.5,opacityback=0.5]
			{\centering{\includesvg[width=0.32\linewidth]{Bowline/Besedochny_marka}}}
		\end{center}
	\end{minipage}
\caption{Наложение марки.}
\label{ris:Besedochny_marka}
\end{figure}

Простой и надежный вариант. То есть, возможность развязывания такого узла не подразумевается. Недостаток - долго и трудоемко. Как правило, делают это один раз и навсегда. В связи с этим смысл использования именно Беседочного узла остается только при оперативной необходимости увеличения размера имеющейся незатягивающейся петли. В этом случае узел фактически вяжется коренным концом, который нужно пропустить через фиксированную петлю ходового. Если он достаточно длинный, то сделать это будет проблематично, а если он где-нибудь уже закреплен, то и невозможно.

\begin{figure}[H]\centering
	\begin{minipage}{1\linewidth}
		\begin{center}
			\tcbox[enhanced jigsaw,colframe=black,opacityframe=0.5,opacityback=0.5]
			{\centering{\includesvg[width=0.32\linewidth]{Bowline/Besedochny_marka_1}}}
		\end{center}
	\end{minipage}
\caption{Завязывание коренным концом.}
\label{ris:Besedochny_marka_sposob}
\end{figure}

\section{Узел для утолщения}

На ходовом конце делается Простой узел, чтобы конец не мог пролезть внутрь затянутого узла. Узлов для утолщения троса существует великое множество, можно использовать любой.

\begin{figure}[H]\centering
	\begin{minipage}{1\linewidth}
		\begin{center}
			\tcbox[enhanced jigsaw,colframe=black,opacityframe=0.5,opacityback=0.5]
			{\centering{\includesvg[width=0.32\linewidth]{Bowline/Besedochny_utolsh}}}
		\end{center}
	\end{minipage}
\caption{Узел для утолщения ходового конца.}
\label{ris:Besedochny_utolsh}
\end{figure}

\section{Контрольный узел}

Можно использовать Простой узел, можно какой-либо из Штыков, а можно любую затягивающуюся или незатягивающуюся петлю (тот же Булинь, например).

\begin{figure}[H]\centering
	\subfloat[На ходовом конце.]{\label{ris:Besedochny_kontrol_1}
	\tcbox[enhanced jigsaw,colframe=black,opacityframe=0.5,opacityback=0.5,height=2.5cm]
		{\centering
			\includesvg[width=0.33\linewidth]{Bowline/Besedochny_po_petle}}
		}
\hfill
	\subfloat[На коренном конце.]{\label{ris:Besedochny_kontrol_2}
	\tcbox[enhanced jigsaw,colframe=black,opacityframe=0.5,opacityback=0.5,height=2.5cm]
		{\centering
			\includesvg[width=0.33\linewidth]{Bowline/Besedochny_na_korennom}}
		}
	\caption{Контрольный узел.}\label{ris:Besedochny_kontrol}
\end{figure}

\section{Удвоение элементов узла}

Для повышения надежности можно применить метод удвоения одного или сразу нескольких элементов узла.

\subsection{Альпинистский Беседочный}

Английское название - Mountaineering Bowline. Удваивается петля коренного конца.

\begin{figure}[H]\centering
	\begin{minipage}{1\linewidth}
		\begin{center}
			\tcbox[enhanced jigsaw,colframe=black,opacityframe=0.5,opacityback=0.5]
			{\centering{\includesvg[width=0.32\linewidth]{Bowline/Besedochny-Mountaineering}}}
		\end{center}
	\end{minipage}
\caption{Альпинистский Беседочный.}
\label{ris:Besedochny-Mountaineering}
\end{figure}

\subsection{Альпинистский Беседочный 2}

Узел, в котором удваивается петля ходового конца.

\begin{figure}[H]\centering
	\begin{minipage}{1\linewidth}
		\begin{center}
			\tcbox[enhanced jigsaw,colframe=black,opacityframe=0.5,opacityback=0.5]
			{\centering{\includesvg[width=0.32\linewidth]{Bowline/Besedochny-Mountaineering-2}}}
		\end{center}
	\end{minipage}
\caption{Альпинистский Беседочный 2.}
\label{ris:Besedochny-Mountaineering-2}
\end{figure}

\subsection{Двойной Альпинистский Беседочный}

Он же Mountaineering Double Bowline. Удваиваются петли и коренного, и ходового конца.

\begin{figure}[H]\centering
	\begin{minipage}{1\linewidth}
		\begin{center}
			\tcbox[enhanced jigsaw,colframe=black,opacityframe=0.5,opacityback=0.5]
			{\centering{\includesvg[width=0.32\linewidth]{Bowline/Besedochny-Mountaineering-Double}}}
		\end{center}
	\end{minipage}
\caption{Двойной Альпинистский Беседочный.}
\label{ris:Besedochny-Mountaineering-Double}
\end{figure}

\subsection{Водный булинь}

По английски - Water Bowline. Тоже удваивается петля на коренном конце, но немного по другому принципу. Не просто шлагами, а в виде Штыков. Классический вариант с использованием Простого штыка, но точно так же работать будет и с другими, например, с Cow Hitch.

\begin{figure}[H]\centering
	\subfloat[С использованием Clove Hitch.]{\label{ris:Besedochny-Water_1}
	\tcbox[enhanced jigsaw,colframe=black,opacityframe=0.5,opacityback=0.5,height=2.5cm]
		{\centering
			\includesvg[width=0.33\linewidth]{Bowline/Besedochny-Water}}
		}
\hfill
	\subfloat[С использованием Cow Hitch.]{\label{ris:Besedochny-Water_2}
	\tcbox[enhanced jigsaw,colframe=black,opacityframe=0.5,opacityback=0.5,height=2.5cm]
		{\centering
			\includesvg[width=0.33\linewidth]{Bowline/Besedochny-Water-2}}
		}
	\caption{Водный Булинь.}\label{ris:Besedochny-Water}
\end{figure}

\section{Фиксирующийся Укорачивающий Булинь}

Завязывая Булинь сложенным вдвое ходовым концом получим узел, который называется Укорачивающий Булинь или Bowline Shortening. После его вязки ходовой конец можно заправить в им же образованную петлю и затянуть.

\begin{figure}[H]\centering
	\subfloat[Первый вариант.]{\label{ris:Double_Bowline_Brimingham_1}
	\tcbox[enhanced jigsaw,colframe=black,opacityframe=0.5,opacityback=0.5,height=2.5cm]
		{\centering
			\includesvg[width=0.33\linewidth]{Bowline/Double_Bowline_1}}
		}
\hfill
	\subfloat[Второй вариант.]{\label{ris:Double_Bowline_Brimingham_2}
	\tcbox[enhanced jigsaw,colframe=black,opacityframe=0.5,opacityback=0.5,height=2.5cm]
		{\centering
			\includesvg[width=0.33\linewidth]{Bowline/Double_Bowline_2}}
		}
	\caption{Фиксирующийся Укорачивающий Булинь.}\label{ris:Double_Bowline_Brimingham}
\end{figure}

\section{Ходовой конец внутри узла}

Можно зажать ходовой конец непосредственно внутри узла. Тем самым осуществляется фиксация и вероятность проскальзывания сводится к нулю.

\subsection{Yosemite Bowline}

У этого способа даже есть собственное название - Bowline with a Yosemite backup. Русское название - Полуторный булинь. Ходовой конец выходит вверх параллельно коренному концу. После аккуратного затягивания узел выглядит очень красиво и гармонично.

\begin{figure}[H]\centering
	\subfloat[Первый вариант.]{\label{ris:Besedochny-Yosemite_1}
	\tcbox[enhanced jigsaw,colframe=black,opacityframe=0.5,opacityback=0.5,height=2.5cm]
		{\centering
			\includesvg[width=0.33\linewidth]{Bowline/Besedochny-Yosemite}}
		}
\hfill
	\subfloat[Второй вариант.]{\label{ris:Besedochny-Yosemite_2}
	\tcbox[enhanced jigsaw,colframe=black,opacityframe=0.5,opacityback=0.5,height=2.5cm]
		{\centering
			\includesvg[width=0.33\linewidth]{Bowline/Besedochny-Yosemite-3}}
		}
	\caption{Yosemite Bowline.}\label{ris:Besedochny-Yosemite}
\end{figure}

\subsection{Ходовой конец параллелен коренному}

Другие варианты, где ходовой конец параллелен коренному. Собственных названий не имеют.

\begin{figure}[H]\centering
	\subfloat[Первый вариант.]{\label{ris:Besedochny_hodovoy_inside_6}
	\tcbox[enhanced jigsaw,colframe=black,opacityframe=0.5,opacityback=0.5,height=2.5cm]
		{\centering
			\includesvg[width=0.33\linewidth]{Bowline/Besedochny-Yosemite-2}}
		}
\hfill
	\subfloat[Второй вариант.]{\label{ris:Besedochny_hodovoy_inside_4}
	\tcbox[enhanced jigsaw,colframe=black,opacityframe=0.5,opacityback=0.5,height=2.5cm]
		{\centering
			\includesvg[width=0.33\linewidth]{Bowline/Besedochny-zaprav-4}}
		}
\hfill
	\subfloat[Третий вариант.]{\label{ris:Besedochny_hodovoy_inside_5}
	\tcbox[enhanced jigsaw,colframe=black,opacityframe=0.5,opacityback=0.5,height=2.5cm]
		{\centering
			\includesvg[width=0.33\linewidth]{Bowline/Besedochny-zaprav-5}}
		}
	\caption{Ходовой конец параллелен коренному.}\label{ris:Besedochny_hodovoy_inside_paral}
\end{figure}

Обратите внимание, вариант на рис.~(\ref{ris:Besedochny_hodovoy_inside_6}) вяжется из Левостороннего Булиня.

% TODO Проверить первый и второй варианты (коренной станет ходовым)

\subsection{Ходовой конец перпендикулярен коренному}

У следующих вариантов ходовой конец перпендикулярен коренному. Собственных названий эти узлы так же не имеют.

\begin{figure}[H]\centering
	\subfloat[Первый вариант.]{\label{ris:Besedochny_hodovoy_inside_1}
	\tcbox[enhanced jigsaw,colframe=black,opacityframe=0.5,opacityback=0.5,height=2.5cm]
		{\centering
			\includesvg[width=0.33\linewidth]{Bowline/Besedochny-zaprav}}
		}
\hfill
	\subfloat[Второй вариант.]{\label{ris:Besedochny_hodovoy_inside_2}
	\tcbox[enhanced jigsaw,colframe=black,opacityframe=0.5,opacityback=0.5,height=2.5cm]
		{\centering
			\includesvg[width=0.33\linewidth]{Bowline/Besedochny-zaprav-3}}
		}
\hfill
	\subfloat[Третий вариант.]{\label{ris:Besedochny_hodovoy_inside_3}
	\tcbox[enhanced jigsaw,colframe=black,opacityframe=0.5,opacityback=0.5,height=2.5cm]
		{\centering
			\includesvg[width=0.33\linewidth]{Bowline/Besedochny-zaprav-2}}
		}
	\caption{Ходовой конец перпендикулярен коренному.}\label{ris:Besedochny_hodovoy_inside_perp}
\end{figure}

\section{Fool's Bowline}

\begin{figure}[H]\centering
	\begin{minipage}{1\linewidth}
		\begin{center}
			\tcbox[enhanced jigsaw,colframe=black,opacityframe=0.5,opacityback=0.5]
			{\centering{\includesvg[width=0.32\linewidth]{Bowline/Fools_Bowline}}}
		\end{center}
	\end{minipage}
\caption{Fool's Bowline.}
\label{ris:Fools_Bowline}
\end{figure}

Затянув ходовой конец, получим еще один вариант его фиксации, где он перпендикулярен коренному как на рис.~\ref{ris:Besedochny_hodovoy_inside_perp}. Если его не затягивать, а просто дополнительно зафиксировать каким-либо способом (например, полуштыком на коренном конце) - получим узел с двумя незатягивающимися петлями, величину одной из которых можно оперативно регулировать.

% TODO По своим свойствам (одна глухая, одна скользящая петля) похож на Китайский Беседочный с двумя петлями.

\section{Совмещение разных узлов}

% TODO Добавить ссылки на узлы.

Для фиксации Беседочного узла его можно совместить с другими узлами.

\subsection{Совмещение с Удавкой}

Fool's Bowline (рис.~\ref{ris:Fools_Bowline}) совмещается с Удавкой.

\begin{figure}[H]\centering
	\begin{minipage}{1\linewidth}
		\begin{center}
			\tcbox[enhanced jigsaw,colframe=black,opacityframe=0.5,opacityback=0.5]
			{\centering{\includesvg[width=0.35\linewidth]{Bowline/Besedochny-udav}}}
		\end{center}
	\end{minipage}
\caption{Совмещение с Удавкой.}
\label{ris:Besedochny-udav}
\end{figure}

\subsection{Совмещение с Cow Hitch}

Если Булинь предполагается использовать для привязывания к чему-либо, то очевидно, что существует два способа. Можно обвязать его непосредственно вокруг опоры, а можно, как и любую другую незатягивающуюся петлю, привязать его Cow Hitch или другим аналогичным узлом.

\begin{figure}[H]\centering
	\begin{minipage}{1\linewidth}
		\begin{center}
			\tcbox[enhanced jigsaw,colframe=black,opacityframe=0.5,opacityback=0.5]
			{\centering{\includesvg[width=0.35\linewidth]{Bowline/Besedochny_Cow_Hitch}}}
		\end{center}
	\end{minipage}
	\caption{Совмещение с Cow Hitch.}
	\label{ris:Besedochny_Cow_Hitch}
\end{figure}

В этом случае дополнительная фиксация самого Булиня практически не требуется, так как стороны его петли не испытывают разнонаправленных пульсирующих нагрузок. Но и эту систему узлов желающие могут усовершенствовать.

\begin{figure}[H]\centering
	\subfloat[Снаружи. Первый вариант.]{\label{ris:Besedochny_Cow_Hitch_fix_1}
	\tcbox[enhanced jigsaw,colframe=black,opacityframe=0.5,opacityback=0.5,height=2.5cm]
		{\centering
			\includesvg[width=0.33\linewidth]{Bowline/Besedochny_Cow_Hitch_1}}
		}
\hfill
	\subfloat[Снаружи. Второй вариант.]{\label{ris:Besedochny_Cow_Hitch_fix_2}
	\tcbox[enhanced jigsaw,colframe=black,opacityframe=0.5,opacityback=0.5,height=2.5cm]
		{\centering
			\includesvg[width=0.33\linewidth]{Bowline/Besedochny_Cow_Hitch_2}}
		}
\hfill
	\subfloat[Изнутри. Первый вариант.]{\label{ris:Besedochny_Cow_Hitch_fix_4}
	\tcbox[enhanced jigsaw,colframe=black,opacityframe=0.5,opacityback=0.5,height=2.5cm]
		{\centering
			\includesvg[width=0.33\linewidth]{Bowline/Besedochny_Cow_Hitch_4}}
		}
\hfill
	\subfloat[Изнутри. Второй вариант.]{\label{ris:Besedochny_Cow_Hitch_fix_3}
	\tcbox[enhanced jigsaw,colframe=black,opacityframe=0.5,opacityback=0.5,height=2.5cm]
		{\centering
			\includesvg[width=0.33\linewidth]{Bowline/Besedochny_Cow_Hitch_3}}
		}
	\caption{Фиксация с Cow Hitch.}\label{ris:Besedochny_Cow_Hitch_fix}
\end{figure}

\section*{Заключение}

\begin{wrapfigure}[5]{R}{0.35\linewidth}
	\vspace{-7ex}
	\qrcode[height=1.2in]{mailto:dedikovav+book@googlemail.com}
\end{wrapfigure}

Свои предложения и замечания присылайте мне на e-mail. В электронной версии книги можно кликнуть по QR-коду, читатели же бумажной версии могут его просто просканировать с помощью мобильного телефона.

\vfill

% \begin{wrapfigure}[5]{C}{0.3\textwidth}
% \vspace{-3ex}
% 	\qrcode[height=1.2in]{https://pda.litres.ru/andrey-dedikov/mayak/}
% \end{wrapfigure}

% Если Вам понравилась эта книга, ее можно купить на сайте
% \href{https://pda.litres.ru/andrey-dedikov/mayak/}{Литерс}, тем самым поддержав
% автора. Для этого достаточно отсканировать QR-код с помощью мобильного
% телефона.
% 
% \vfill

\begin{center}
	Это произведение доступно по лицензии \doclicenseNameRef \\ \doclicenseImage[imagewidth=5em]
\end{center}

\end{document}
