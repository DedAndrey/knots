\documentclass{artikel1}
%~~~~~~~~~~Математика~~~~~~~~~~
\usepackage{amsmath,amsthm,amssymb}
% \usepackage[warn]{mathtext} % русские буквы в формулах, с предупреждением

%~~~~~~~~~~Текст~~~~~~~~~~
\usepackage{verse}
\usepackage[utf8]{inputenc} % кодовая страница документа
\usepackage[T2A]{fontenc} % внутренняя кодировка TeX
\usepackage[english, russian]{babel} % локализация и переносы
\usepackage[babel]{csquotes} % вставка кавычек
\usepackage{indentfirst} % русский стиль: отступ первого абзаца раздела
\usepackage{anyfontsize} % решает проблему с размером шрифтов
% \usepackage{soul} % Разряженный текст \so{} и подчеркивание \ul{}
% \usepackage{soulutf8} % Поддержка UTF8 в soul
% \usepackage{ulem} % Зачеркивание

%~~~~~~~~~~Графика~~~~~~~~~~
\usepackage{graphicx} % Работа с графикой \includegraphics{}
	\graphicspath{{Nonslide/}{Slide/}} % Папки с иллюстрациями
\usepackage{svg} % Работа с SVG
	\svgsetup{inkscapearea=page,inkscapepath=svgdir}
\usepackage[most]{tcolorbox} % Рамки вокруг рисунков
	\tcbset{boxrule=0.4pt,drop lifted shadow=black,valign=center}
\usepackage{float} % плавающие объекты
\usepackage{wrapfig} % Обтекание картинок текстом
\usepackage[labelformat=simple]{subfig}
% 	\renewcommand{\thesubfigure}{\relax} % без меток на подписях
	\renewcommand{\thesubfigure}{\asbuk{subfigure}.} % метка subfigure: "(а)" вместо дефолтного "а"
\usepackage[justification=centering]{caption} % Центрирует многострочные подписи к рисункам
	\DeclareCaptionLabelFormat{cont}{#1~#2~(\asbuk{ContinuedFloat})}
\usepackage{color,colortbl,xcolor,transparent} % работа с цветом
	\definecolor{BlueGreen}{RGB}{49,152,255} % цвета ссылок
	\definecolor{Violet}{RGB}{120,80,120}
\usepackage{titlepic} % Логотип на титульной странице

%~~~~~~~~~~Разное~~~~~~~~~~
\usepackage{ifthen} % Включаем логику \ifthenelse{условие проверки}{если НЕ истина}{если истина}
\usepackage{epigraph} % Вставка эпиграфа
\usepackage{qrcode} % Поддержка QR-code
\usepackage[section,above,below]{placeins} % указывает \FloatBarrier перед каждым \section автоматически
\usepackage{afterpage} % вставить барьер сразу после начала новой страницы
\usepackage{flafter} % помещает флоат ПОСЛЕ первой ссылки на него
\usepackage[
	type={CC},
	modifier={by-nc-sa},
	version={4.0},
	]{doclicense} % Лицензия

%~~~~~~~~~~Таблицы~~~~~~~~~~
\usepackage{array} % работа с таблицами
\usepackage{booktabs} % возможность набирать красивые таблицы

%~~~~~~~~~~Библиография~~~~~~~~~~
\usepackage[round,sort,numbers]{natbib}
\bibliographystyle{unsrt} % Стиль библиографии
	\makeatletter
	\renewcommand{\@biblabel}[1]{#1.} % Заменяем библиографию с квадратных скобок на точку
	\makeatother

%~~~~~~~~~~Ссылки~~~~~~~~~~
\usepackage[
	unicode=true,
	colorlinks=true,
	urlcolor=BlueGreen,
	linkcolor=Violet,
	citecolor=Violet]{hyperref}

\pdfsuppresswarningpagegroup=1

\begin{document}                                     % начало документа

\title{Эскимосский, Якутский, Калмыцкий, Чабанский и Скифовский узлы}
\date{}
\author{А.В. Дедиков}

\maketitle

\section{Эскимосский Булинь}

Он же Казачий или Казацкий узел, Эскимосская петля, Крабья петля, Затягивающийся огон, Затяжной огон. Английские названия - Eskimo Bowline, Crab Noose, Crabber’s Eye Knot или Crossing Running Knot. Является трансформируемой петлей. То есть затягивающаяся петля путем более сильного затягивания (обычно рывком) превращается в незатягивающуюся. В качестве трансформируемой петли в литературе нигде не рассматривался, только в качестве незатягивающейся петли.

Способов вязки у этого узла существует не меньше, чем у Беседочного. Узлы имеют одинаковую структуру и отличаются всего лишь одной деталью - в Беседочном ходовой конец проходит через петлю коренного снизу-вверх и огибает коренной конец, а в Эскимосском наоборот, сверху-вниз и огибает боковину рабочей петли. Различные варианты усложнения и развития узла также можно позаимствовать у Беседочного.

%TODO про поворот петли коренного узла по часовой стрелке - ничего не получится

\begin{figure}[H]\centering
	\subfloat[Завязывание]{\label{ris:Kazak_1}
	\tcbox[enhanced jigsaw,colframe=black,opacityframe=0.5,opacityback=0.5,height=2cm]
		{\centering
			\includesvg[width=0.33\linewidth]{Slide/Crab_noose}}
		}
\hfill
	\subfloat[Результат]{\label{ris:Kazak_2}
	\tcbox[enhanced jigsaw,colframe=black,opacityframe=0.5,opacityback=0.5,height=2cm]
		{\centering
			\includesvg[width=0.33\linewidth]{Slide/Crab_noose_1}}
		}
\hfill
	\subfloat[Быстроразвязывающийся вариант]{\label{ris:Kazak_3}
	\tcbox[enhanced jigsaw,colframe=black,opacityframe=0.5,opacityback=0.5]
		{\centering
			\includesvg[width=0.33\linewidth]{Slide/Kazak}}
		}
	\caption{Эскимосский Булинь.}\label{ris:Kazak}
\end{figure}

\section{Калмыцкий узел}

Очень похож на Эскимосский булинь в быстроразвязывающемся варианте (рис.~\ref{ris:Kazak_3}). Однако, петля коренного конца повернута не против часовой стрелки, а в обратную сторону, в связи с чем ходовой конец ею зажимается несколько по-другому.

\begin{figure}[H]\centering
	\begin{minipage}{1\linewidth}
		\begin{center}
			\tcbox[enhanced jigsaw,colframe=black,opacityframe=0.5,opacityback=0.5]
			{\centering{\includesvg[width=0.4\linewidth]{Nonslide/Kalmyk}}}
		\end{center}
	\end{minipage}
\caption{Калмыцкий узел.}
\label{ris:Kalmyk}
\end{figure}

\section{Чабанский узел}

Развитие быстроразвязывающегося Эскимосского булиня и Калмыцкого узла. Ходовой конец страхует узел, временно блокируя от возможности случайного саморазвязывания. Отличие только в направлении прохождения ходового конца через фиксируемую петлю. Разница не принципиальна.

\begin{figure}[H]\centering
	\subfloat[Эскимосский.\\Первый вариант.]{\label{ris:Chaban_1}
	\tcbox[enhanced jigsaw,colframe=black,opacityframe=0.5,opacityback=0.5]
		{\centering
			\includesvg[width=0.33\linewidth]{Nonslide/Chaban}}
		}
\hfill
	\subfloat[Калмыцкий.\\Первый вариант.]{\label{ris:Chaban_2}
	\tcbox[enhanced jigsaw,colframe=black,opacityframe=0.5,opacityback=0.5]
		{\centering
			\includesvg[width=0.33\linewidth]{Nonslide/Chaban_2}}
		}
\hfill
	\subfloat[Эскимосский.\\Второй вариант.]{\label{ris:Chaban_3}
	\tcbox[enhanced jigsaw,colframe=black,opacityframe=0.5,opacityback=0.5]
		{\centering
			\includesvg[width=0.33\linewidth]{Nonslide/Chaban_4}}
		}
\hfill
	\subfloat[Калмыцкий.\\Второй вариант.]{\label{ris:Chaban_4}
	\tcbox[enhanced jigsaw,colframe=black,opacityframe=0.5,opacityback=0.5]
		{\centering
			\includesvg[width=0.33\linewidth]{Nonslide/Chaban_5}}
		}
\hfill
	\subfloat[Эскимосский.\\Третий вариант.]{\label{ris:Chaban_5}
	\tcbox[enhanced jigsaw,colframe=black,opacityframe=0.5,opacityback=0.5]
		{\centering
			\includesvg[width=0.33\linewidth]{Nonslide/Chaban_1}}
		}
\hfill
	\subfloat[Калмыцкий.\\Третий вариант.]{\label{ris:Chaban_6}
	\tcbox[enhanced jigsaw,colframe=black,opacityframe=0.5,opacityback=0.5]
		{\centering
			\includesvg[width=0.33\linewidth]{Nonslide/Chaban_3}}
		}
	\caption{Чабанский узел.}\label{ris:Chaban}
\end{figure}

\section{Якутский узел}

По-якутски этот узел называется \enquote{Туомтуу баайыы}. Другое русское название - Распускной узел. Интереснейшее исследование есть \href{http://ilin-yakutsk.narod.ru/2002-4/savinov.htm}{в журнале \enquote{Илин} за 2002г}. Там, в частности, описаны способы вязки, а так же отличия Калмыцкого и Якутского узлов.

Вновь \enquote{открыт} \href{http://www.muzel.ru/article/za/skifovsky.htm}{Юрием Елизаровым}, где он называет его Скифовский узел.

\begin{figure}[H]\centering
	\subfloat[Результат]{\label{ris:Skif_1}
	\tcbox[enhanced jigsaw,colframe=black,opacityframe=0.5,opacityback=0.5]
		{\centering
			\includesvg[width=0.33\linewidth]{Nonslide/Skif}}
		}
\vfill
	\subfloat[Точка отличия от\\Калмыцкого узла]{\label{ris:Skif_2}
	\tcbox[enhanced jigsaw,colframe=black,opacityframe=0.5,opacityback=0.5]
		{\centering
			\includesvg[width=0.33\linewidth]{Nonslide/Skif_0}}
		}
\hfill
	\subfloat[Точка отличия от\\Беседочного узла]{\label{ris:Skif_3}
	\tcbox[enhanced jigsaw,colframe=black,opacityframe=0.5,opacityback=0.5]
		{\centering
			\includesvg[width=0.33\linewidth]{Nonslide/Skif_00}}
		}
	\caption{Якутский узел.}\label{ris:Skif}
\end{figure}

На рисунке выше красными кружками отмечены единственные пересечения, которые отличают Якутский (Скифовский) от Калмыцкого и Беседочного узлов.

\section{Простой Скифовский узел}

Тоже самое, но без возврата ходового конца обратно для быстрого развязывания.

\begin{figure}[H]\centering
	\begin{minipage}{1\linewidth}
		\begin{center}
			\tcbox[enhanced jigsaw,colframe=black,opacityframe=0.5,opacityback=0.5]
			{\centering{\includesvg[width=0.4\linewidth]{Nonslide/Skif_2}}}
		\end{center}
	\end{minipage}
\caption{Простой Скифовский узел.}
\label{ris:Skif_simpl}
\end{figure}

\section*{Заключение}

\begin{wrapfigure}[5]{R}{0.35\linewidth}
	\vspace{-7ex}
	\qrcode[height=1.2in]{mailto:dedikovav+book@googlemail.com}
\end{wrapfigure}

Свои вопросы, предложения и замечания присылайте мне на e-mail. В электронной версии статьи достаточно кликнуть по QR-коду, в бумажной версии можно его просканировать с помощью мобильного телефона.

\vfill

\begin{center}
	Это произведение доступно по лицензии \doclicenseNameRef \\ \doclicenseImage[imagewidth=5em]
\end{center}

\end{document}
