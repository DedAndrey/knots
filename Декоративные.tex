\section{Декоративные}

\addtocounter{KnotNoName}{1}

\subsection{Узел без названия \arabic{KnotNoName}}

\begin{figure}[H]\centering
	\subfloat[Завязывание]{\label{ris:KnotNoName_9_1}
	\tcbox[enhanced jigsaw,colframe=black,opacityframe=0.5,opacityback=0.5,height=5cm]
		{\centering
			\includesvg[width=0.31\linewidth]{Utolsh/KnotNoName_9_1}}
		}
\hfil
	\subfloat[Результат]{\label{ris:KnotNoName_9_2}
	\tcbox[enhanced jigsaw,colframe=black,opacityframe=0.5,opacityback=0.5,height=5cm]
		{\centering
			\includesvg[width=0.35\linewidth]{Utolsh/KnotNoName_9}}
		}
	\caption{Узел без названия \arabic{KnotNoName}.}\label{ris:KnotNoName_9}
\end{figure}

\subsection{Трехпетельный узел}

\begin{figure}[H]\centering
	\subfloat[Завязывание]{\label{ris:TriplLoop_1}
	\tcbox[enhanced jigsaw,colframe=black,opacityframe=0.5,opacityback=0.5,height=4cm]
		{\centering
			\includesvg[width=0.38\linewidth]{Utolsh/TriplLoop}}
		}
\hfil
	\subfloat[Завязывание]{\label{ris:TriplLoop_2}
	\tcbox[enhanced jigsaw,colframe=black,opacityframe=0.5,opacityback=0.5,height=4cm]
		{\centering
			\includesvg[width=0.26\linewidth]{Utolsh/TriplLoop_2}}
		}
\end{figure}
% \vfill
\begin{figure}[H]\centering
	\subfloat[Результат]{\label{ris:TriplLoop_3}
	\tcbox[enhanced jigsaw,colframe=black,opacityframe=0.5,opacityback=0.5]
		{\centering
			\includesvg[width=0.33\linewidth]{Utolsh/TriplLoop_1}}
		}
	\caption{Трехпетельный узел.}\label{ris:TriplLoop}
\end{figure}

Three-Ply Knot. Результат похож на Устричный узел (рис.~\ref{ris:Oysterman}).

\subsection{Кордовый узел}

\begin{figure}[H]\centering
	\begin{minipage}{1\linewidth}
		\begin{center}
			\tcbox[enhanced jigsaw,colframe=black,opacityframe=0.5,opacityback=0.5]
			{\centering{\includesvg[width=0.35\linewidth]{Utolsh/Kord}}}
		\end{center}
	\end{minipage}
\caption{Кордовый узел.}
	\label{ris:Kord}
\end{figure}

One-Cord Knot.

\subsection{Японский венценосный узел}

\begin{figure}[H]\centering
	\begin{minipage}{1\linewidth}
		\begin{center}
			\tcbox[enhanced jigsaw,colframe=black,opacityframe=0.5,opacityback=0.5]
			{\centering{\includesvg[width=0.35\linewidth]{Utolsh/Japan}}}
		\end{center}
	\end{minipage}
\caption{Японский венценосный узел.}
	\label{ris:Japan}
\end{figure}

Другое название - Узел успеха.

\subsection{Узел Анны Богемской}

\begin{figure}[H]\centering
	\begin{minipage}{1\linewidth}
		\begin{center}
			\tcbox[enhanced jigsaw,colframe=black,opacityframe=0.5,opacityback=0.5]
			{\centering{\includesvg[width=0.35\linewidth]{Utolsh/Bogemia}}}
		\end{center}
	\end{minipage}
\caption{Узел Анны Богемской.}
	\label{ris:Bogemia}
\end{figure}

\subsection{Стопорный узел Шамова}

% TODO Наверняка он есть еще где-нибудь (в декоративных)!

\begin{figure}[H]\centering
	\subfloat[Завязывание]{\label{ris:Shamov_1}
	\tcbox[enhanced jigsaw,colframe=black,opacityframe=0.5,opacityback=0.5,height=4.5cm]
		{\centering
			\includesvg[width=0.33\linewidth]{Utolsh/Shamov}}
		}
\hfil
	\subfloat[Результат]{\label{ris:Shamov_2}
	\tcbox[enhanced jigsaw,colframe=black,opacityframe=0.5,opacityback=0.5,height=4.5cm]
		{\centering
			\includesvg[width=0.33\linewidth]{Utolsh/Shamov_1}}
		}
	\caption{Стопорный узел Шамова.}\label{ris:Shamov}
\end{figure}

Узел описан в книге А.П. Шамова \enquote{Способы и схемы вязки узлов и их применение в туристической практике}\citep{Shamov}.

\addtocounter{KnotNoName}{1}

\subsection{Узел без названия \arabic{KnotNoName}}

\begin{figure}[H]\centering
	\begin{minipage}{1\linewidth}
		\begin{center}
			\tcbox[enhanced jigsaw,colframe=black,opacityframe=0.5,opacityback=0.5]
			{\centering{\includesvg[width=0.35\linewidth]{Utolsh/KnotNoName_10}}}
		\end{center}
	\end{minipage}
\caption{Узел без названия \arabic{KnotNoName}.}
	\label{ris:KnotNoName_10}
\end{figure}

При затягивании получится петля из Плоского узла, похожая на Carrick Loop (рис.~\ref{ris:Carrick_Loop}).

\subsection{Четырехпетельный узел}

\begin{figure}[H]\centering
	\subfloat[Завязывание]{\label{ris:FourLoop_1_1}
	\tcbox[enhanced jigsaw,colframe=black,opacityframe=0.5,opacityback=0.5,height=4.5cm]
		{\centering
			\includesvg[width=0.34\linewidth]{Utolsh/FourLoop_2}}
		}
\hfil
	\subfloat[Результат]{\label{ris:FourLoop_1_2}
	\tcbox[enhanced jigsaw,colframe=black,opacityframe=0.5,opacityback=0.5,height=4.5cm]
		{\centering
			\includesvg[width=0.32\linewidth]{Utolsh/FourLoop_3}}
		}
	\caption{Четырехпетельный узел.}\label{ris:FourLoop}
\end{figure}

Four-Ply Knot.

\subsection{Четырехпетельный узел 2}

\begin{figure}[H]\centering
	\subfloat[Завязывание]{\label{ris:FourLoop_2_1}
	\tcbox[enhanced jigsaw,colframe=black,opacityframe=0.5,opacityback=0.5,height=4.5cm]
		{\centering
			\includesvg[width=0.4\linewidth]{Utolsh/FourLoop}}
		}
\hfil
	\subfloat[Результат]{\label{ris:FourLoop_2_2}
	\tcbox[enhanced jigsaw,colframe=black,opacityframe=0.5,opacityback=0.5,height=4.5cm]
		{\centering
			\includesvg[width=0.32\linewidth]{Utolsh/FourLoop_1}}
		}
	\caption{Четырехпетельный узел 2.}\label{ris:FourLoop_2}
\end{figure}

Double Three-Ply Knot.

\subsection{Наполеон}

\begin{figure}[H]\centering
	\begin{minipage}{1\linewidth}
		\begin{center}
			\tcbox[enhanced jigsaw,colframe=black,opacityframe=0.5,opacityback=0.5]
			{\centering{\includesvg[width=0.35\linewidth]{Utolsh/Napoleon}}}
		\end{center}
	\end{minipage}
\caption{Наполеон.}
\label{ris:Napoleon}
\end{figure}

\subsection{Китайский ткацкий узел}

\begin{figure}[H]\centering
	\begin{minipage}{1\linewidth}
		\begin{center}
			\tcbox[enhanced jigsaw,colframe=black,opacityframe=0.5,opacityback=0.5]
			{\centering{\includesvg[width=0.4\linewidth]{Utolsh/Chinese_tkach}}}
		\end{center}
	\end{minipage}
\caption{Китайский ткацкий узел.}
\label{ris:Chinese_tkach}
\end{figure}

% TODO нарисовать способ вязки

Five-Strand French Sinnet.

\addtocounter{KnotNoName}{1}

\subsection{Узел без названия \arabic{KnotNoName}}

\begin{figure}[H]\centering
	\begin{minipage}{1\linewidth}
		\begin{center}
			\tcbox[enhanced jigsaw,colframe=black,opacityframe=0.5,opacityback=0.5]
			{\centering{\includesvg[width=0.35\linewidth]{Utolsh/KnotNoName_1}}}
		\end{center}
	\end{minipage}
\caption{Узел без названия \arabic{KnotNoName}.}
\label{ris:KnotNoName_1}
\end{figure}

\addtocounter{KnotNoName}{1}

\subsection{Узел без названия \arabic{KnotNoName}}

\begin{figure}[H]\centering
	\begin{minipage}{1\linewidth}
		\begin{center}
			\tcbox[enhanced jigsaw,colframe=black,opacityframe=0.5,opacityback=0.5]
			{\centering{\includesvg[width=0.35\linewidth]{Utolsh/KnotNoName_8}}}
		\end{center}
	\end{minipage}
\caption{Узел без названия \arabic{KnotNoName}.}
\label{ris:KnotNoName_8}
\end{figure}

\addtocounter{KnotNoName}{1}

\subsection{Узел без названия \arabic{KnotNoName}}

\begin{figure}[H]\centering
	\begin{minipage}{1\linewidth}
		\begin{center}
			\tcbox[enhanced jigsaw,colframe=black,opacityframe=0.5,opacityback=0.5]
			{\centering{\includesvg[width=0.3\linewidth]{Utolsh/KnotNoName_2}}}
		\end{center}
	\end{minipage}
\caption{Узел без названия \arabic{KnotNoName}.}
\label{ris:KnotNoName_2}
\end{figure}

\addtocounter{KnotNoName}{1}

\subsection{Узел без названия \arabic{KnotNoName}}

\begin{figure}[H]\centering
	\begin{minipage}{1\linewidth}
		\begin{center}
			\tcbox[enhanced jigsaw,colframe=black,opacityframe=0.5,opacityback=0.5]
			{\centering{\includesvg[width=0.5\linewidth]{Utolsh/KnotNoName_16}}}
		\end{center}
	\end{minipage}
\caption{Узел без названия \arabic{KnotNoName}.}
\label{ris:KnotNoName_16}
\end{figure}

\subsection{Terminal Knot}

\begin{figure}[H]\centering
	\begin{minipage}{1\linewidth}
		\begin{center}
			\tcbox[enhanced jigsaw,colframe=black,opacityframe=0.5,opacityback=0.5]
			{\centering{\includesvg[width=0.35\linewidth]{Utolsh/Decorative_Terminal}}}
		\end{center}
	\end{minipage}
\caption{Terminal Knot.}
\label{ris:Decorative_Terminal}
\end{figure}

\subsection{Decorative Terminal Knot}

\begin{figure}[H]\centering
	\begin{minipage}{1\linewidth}
		\begin{center}
			\tcbox[enhanced jigsaw,colframe=black,opacityframe=0.5,opacityback=0.5]
			{\centering{\includesvg[width=0.35\linewidth]{Utolsh/Decorative_Terminal_1}}}
		\end{center}
	\end{minipage}
\caption{Terminal Knot.}
\label{ris:Decorative_Terminal_1}
\end{figure}

\subsection{Quatrefoil}

\begin{figure}[H]\centering
	\subfloat[Завязывание]{\label{ris:Quatrefoil_1}
	\tcbox[enhanced jigsaw,colframe=black,opacityframe=0.5,opacityback=0.5,height=5.5cm]
		{\centering
			\includesvg[width=0.3\linewidth]{Utolsh/Quatrefoil}}
		}
\hfil
	\subfloat[Результат]{\label{ris:Quatrefoil_2}
	\tcbox[enhanced jigsaw,colframe=black,opacityframe=0.5,opacityback=0.5,height=5.5cm]
		{\centering
			\includesvg[width=0.3\linewidth]{Utolsh/Quatrefoil_1}}
		}
	\caption{Quatrefoil.}\label{ris:Quatrefoil}
\end{figure}

\subsection{Cinquefoil}

\begin{figure}[H]\centering
	\begin{minipage}{1\linewidth}
		\begin{center}
			\tcbox[enhanced jigsaw,colframe=black,opacityframe=0.5,opacityback=0.5]
			{\centering{\includesvg[width=0.5\linewidth]{Utolsh/Cinquefoil}}}
		\end{center}
	\end{minipage}
\caption{Cinquefoil.}
\label{ris:Cinquefoil}
\end{figure}

\subsection{Двойной Устричный узел}

\begin{figure}[H]\centering
	\begin{minipage}{1\linewidth}
		\begin{center}
			\tcbox[enhanced jigsaw,colframe=black,opacityframe=0.5,opacityback=0.5]
			{\centering{\includesvg[width=0.5\linewidth]{Utolsh/Double_Oysterman}}}
		\end{center}
	\end{minipage}
\caption{Двойной Устричный узел.}
\label{ris:Double_Oysterman}
\end{figure}

Double Oysterman’s Knot.

\subsection{Настоящий Двойной Устричный узел}

\begin{figure}[H]\centering
	\begin{minipage}{1\linewidth}
		\begin{center}
			\tcbox[enhanced jigsaw,colframe=black,opacityframe=0.5,opacityback=0.5]
			{\centering{\includesvg[width=0.5\linewidth]{Utolsh/Truly_Double_Oysterman}}}
		\end{center}
	\end{minipage}
\caption{Настоящий Двойной Устричный узел.}
\label{ris:Truly_Double_Oysterman}
\end{figure}

Truly Double Oysterman’s Knot.

\subsection{Двойной Трехпетельный узел}

\begin{figure}[H]\centering
	\begin{minipage}{1\linewidth}
		\begin{center}
			\tcbox[enhanced jigsaw,colframe=black,opacityframe=0.5,opacityback=0.5]
			{\centering{\includesvg[width=0.4\linewidth]{Utolsh/Double_Three-Ply_Knot}}}
		\end{center}
	\end{minipage}
\caption{Двойной Трехпетельный узел.}
\label{ris:Double_Three-Ply_Knot}
\end{figure}

Double Three-Ply Knot.
