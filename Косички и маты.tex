\section{Косички и маты}

\subsection{На основе Простого узла}

\begin{figure}[H]\centering
	\subfloat[Завязывание]{\label{ris:Single_braid_1}
	\tcbox[enhanced jigsaw,colframe=black,opacityframe=0.5,opacityback=0.5,height=2.5cm]
		{\centering
			\includesvg[width=0.33\linewidth]{Utolsh/Single_braid}}
		}
\hfil
	\subfloat[Завязывание]{\label{ris:Single_braid_2}
	\tcbox[enhanced jigsaw,colframe=black,opacityframe=0.5,opacityback=0.5,height=2.5cm]
		{\centering
			\includesvg[width=0.33\linewidth]{Utolsh/Single_braid_1}}
		}
\end{figure}
% \vfill
\begin{figure}[H]\centering
	\subfloat[Результат]{\label{ris:Single_braid_3}
	\tcbox[enhanced jigsaw,colframe=black,opacityframe=0.5,opacityback=0.5]
		{\centering
			\includesvg[width=0.33\linewidth]{Utolsh/Single_braid_2}}
		}
	\caption{Косичка на основе Простого узла (рис.~\ref{ris:Single_Knot}).}\label{ris:Single_braid}
\end{figure}

\subsection{На основе Восьмерки}

\begin{figure}[H]\centering
	\subfloat[Завязывание]{\label{ris:Figure-Of-Eight_braid_1}
	\tcbox[enhanced jigsaw,colframe=black,opacityframe=0.5,opacityback=0.5,height=2.5cm]
		{\centering
			\includesvg[width=0.33\linewidth]{Utolsh/Figure-Of-Eight_braid}}
		}
\hfil
	\subfloat[Результат]{\label{ris:Figure-Of-Eight_braid_2}
	\tcbox[enhanced jigsaw,colframe=black,opacityframe=0.5,opacityback=0.5,height=2.5cm]
		{\centering
			\includesvg[width=0.33\linewidth]{Utolsh/Figure-Of-Eight_braid_1}}
		}
	\caption{Косичка на основе Восьмерки (рис.~\ref{ris:Figure-Of-Eight}).}\label{ris:Figure-Of-Eight_braid}
\end{figure}

\subsection{Двойная косичка из Простого узла}

\begin{figure}[H]\centering
	\subfloat[Завязывание]{\label{ris:Double_Single_braid_1}
	\tcbox[enhanced jigsaw,colframe=black,opacityframe=0.5,opacityback=0.5]
		{\centering
			\includesvg[width=0.35\linewidth]{Utolsh/Double_Single_braid}}
		}
\end{figure}
% \vfill
\begin{figure}[H]\centering
	\subfloat[Результат]{\label{ris:Double_Single_braid_2}
	\tcbox[enhanced jigsaw,colframe=black,opacityframe=0.5,opacityback=0.5]
		{\centering
			\includesvg[width=0.6\linewidth]{Utolsh/Double_Single_braid_1}}
		}
	\caption{Двойная косичка из Простого узла.}\label{ris:Double_Single_braid}
\end{figure}

Double Twist Braid Knot based on an Overhand Knot.

\subsection{Двойная косичка из Восьмерки}

\begin{figure}[H]\centering
	\subfloat[Завязывание]{\label{ris:Double_Figure-Of-Eight_braid_1}
	\tcbox[enhanced jigsaw,colframe=black,opacityframe=0.5,opacityback=0.5]
		{\centering
			\includesvg[width=0.45\linewidth]{Utolsh/Double_Figure-Of-Eight_braid}}
		}
\end{figure}
% \vfill
\begin{figure}[H]\centering
	\subfloat[Результат]{\label{ris:Double_Figure-Of-Eight_braid_2}
	\tcbox[enhanced jigsaw,colframe=black,opacityframe=0.5,opacityback=0.5]
		{\centering
			\includesvg[width=0.65\linewidth]{Utolsh/Double_Figure-Of-Eight_braid_1}}
		}
	\caption{Двойная косичка из Восьмерки.}\label{ris:Double_Figure-Of-Eight_braid}
\end{figure}

\subsection{Five-Lead Flat-Sinnet Terminal Knot}

\begin{figure}[H]\centering
	\subfloat[Завязывание]{\label{ris:Five-Lead_Flat-Sinnet_Terminal_Knot_1_1}
	\tcbox[enhanced jigsaw,colframe=black,opacityframe=0.5,opacityback=0.5,height=2.5cm]
		{\centering
			\includesvg[width=0.3\linewidth]{Utolsh/Five-Lead_Flat-Sinnet_Terminal_Knot}}
		}
\hfil
	\subfloat[Завязывание]{\label{ris:Five-Lead_Flat-Sinnet_Terminal_Knot_1_2}
	\tcbox[enhanced jigsaw,colframe=black,opacityframe=0.5,opacityback=0.5,height=2.5cm]
		{\centering
			\includesvg[width=0.45\linewidth]{Utolsh/Five-Lead_Flat-Sinnet_Terminal_Knot_1}}
		}
\end{figure}
% \vfill
\begin{figure}[H]\centering
	\subfloat[Завязывание]{\label{ris:Five-Lead_Flat-Sinnet_Terminal_Knot_1_3}
	\tcbox[enhanced jigsaw,colframe=black,opacityframe=0.5,opacityback=0.5]
		{\centering
			\includesvg[width=0.5\linewidth]{Utolsh/Five-Lead_Flat-Sinnet_Terminal_Knot_1_1}}
		}
\end{figure}
% \vfill
\begin{figure}[H]\centering
	\subfloat[Завязывание]{\label{ris:Five-Lead_Flat-Sinnet_Terminal_Knot_1_4}
	\tcbox[enhanced jigsaw,colframe=black,opacityframe=0.5,opacityback=0.5]
		{\centering
			\includesvg[width=0.55\linewidth]{Utolsh/Five-Lead_Flat-Sinnet_Terminal_Knot_1_2}}
		}
\end{figure}
% \vfill
\begin{figure}[H]\centering
	\subfloat[Завязывание]{\label{ris:Five-Lead_Flat-Sinnet_Terminal_Knot_1_5}
	\tcbox[enhanced jigsaw,colframe=black,opacityframe=0.5,opacityback=0.5]
		{\centering
			\includesvg[width=0.6\linewidth]{Utolsh/Five-Lead_Flat-Sinnet_Terminal_Knot_1_3}}
		}
\end{figure}
% \vfill
\begin{figure}[H]\centering
	\subfloat[Результат]{\label{ris:Five-Lead_Flat-Sinnet_Terminal_Knot_1_6}
	\tcbox[enhanced jigsaw,colframe=black,opacityframe=0.5,opacityback=0.5]
		{\centering
			\includesvg[width=0.65\linewidth]{Utolsh/Five-Lead_Flat-Sinnet_Terminal_Knot_1_4}}
		}
	\caption{Five-Lead Flat-Sinnet Terminal Knot.}\label{ris:Five-Lead_Flat-Sinnet_Terminal_Knot_1}
\end{figure}

\subsection{Five-Lead Flat-Sinnet Terminal Knot 2}

\begin{figure}[H]\centering
	\subfloat[Завязывание]{\label{ris:Five-Lead_Flat-Sinnet_Terminal_Knot_2_1}
	\tcbox[enhanced jigsaw,colframe=black,opacityframe=0.5,opacityback=0.5]
		{\centering
			\includesvg[width=0.3\linewidth]{Utolsh/Five-Lead_Flat-Sinnet_Terminal_Knot_2}}
		}
\end{figure}
% \vfill
\begin{figure}[H]\centering
	\subfloat[Завязывание]{\label{ris:Five-Lead_Flat-Sinnet_Terminal_Knot_2_2}
	\tcbox[enhanced jigsaw,colframe=black,opacityframe=0.5,opacityback=0.5]
		{\centering
			\includesvg[width=0.45\linewidth]{Utolsh/Five-Lead_Flat-Sinnet_Terminal_Knot_2_1}}
		}
\end{figure}
% \vfill
\begin{figure}[H]\centering
	\subfloat[Завязывание]{\label{ris:Five-Lead_Flat-Sinnet_Terminal_Knot_2_3}
	\tcbox[enhanced jigsaw,colframe=black,opacityframe=0.5,opacityback=0.5]
		{\centering
			\includesvg[width=0.5\linewidth]{Utolsh/Five-Lead_Flat-Sinnet_Terminal_Knot_2_2}}
		}
\end{figure}
% \vfill
\begin{figure}[H]\centering
	\subfloat[Завязывание]{\label{ris:Five-Lead_Flat-Sinnet_Terminal_Knot_2_4}
	\tcbox[enhanced jigsaw,colframe=black,opacityframe=0.5,opacityback=0.5]
		{\centering
			\includesvg[width=0.55\linewidth]{Utolsh/Five-Lead_Flat-Sinnet_Terminal_Knot_2_3}}
		}
\end{figure}
% \vfill
\begin{figure}[H]\centering
	\subfloat[Результат]{\label{ris:Five-Lead_Flat-Sinnet_Terminal_Knot_2_5}
	\tcbox[enhanced jigsaw,colframe=black,opacityframe=0.5,opacityback=0.5]
		{\centering
			\includesvg[width=0.65\linewidth]{Utolsh/Five-Lead_Flat-Sinnet_Terminal_Knot_2_4}}
		}
	\caption{Five-Lead Flat-Sinnet Terminal Knot 2.}\label{ris:Five-Lead_Flat-Sinnet_Terminal_Knot_2}
\end{figure}

\subsection{Flat Lanyard Knot на базе Five-Strand French Sinnet}

\begin{figure}[H]\centering
	\subfloat[Завязывание]{\label{ris:Flat_Lanyard_Five-Strand_French_Sinnet_1}
	\tcbox[enhanced jigsaw,colframe=black,opacityframe=0.5,opacityback=0.5,height=3cm]
		{\centering
			\includesvg[width=0.29\linewidth]{Utolsh/Flat_Lanyard_Five-Strand_French_Sinnet_1}}
		}
\hfil
	\subfloat[Завязывание]{\label{ris:Flat_Lanyard_Five-Strand_French_Sinnet_2}
	\tcbox[enhanced jigsaw,colframe=black,opacityframe=0.5,opacityback=0.5,height=3cm]
		{\centering
			\includesvg[width=0.32\linewidth]{Utolsh/Flat_Lanyard_Five-Strand_French_Sinnet_2}}
		}
\end{figure}
% \vfill
\begin{figure}[H]\centering
	\subfloat[Завязывание]{\label{ris:Flat_Lanyard_Five-Strand_French_Sinnet_3}
	\tcbox[enhanced jigsaw,colframe=black,opacityframe=0.5,opacityback=0.5]
		{\centering
			\includesvg[width=0.4\linewidth]{Utolsh/Flat_Lanyard_Five-Strand_French_Sinnet_3}}
		}
\end{figure}
% \vfill
\begin{figure}[H]\centering
	\subfloat[Завязывание]{\label{ris:Flat_Lanyard_Five-Strand_French_Sinnet_4}
	\tcbox[enhanced jigsaw,colframe=black,opacityframe=0.5,opacityback=0.5]
		{\centering
			\includesvg[width=0.45\linewidth]{Utolsh/Flat_Lanyard_Five-Strand_French_Sinnet_4}}
		}
\end{figure}
% \vfill
\begin{figure}[H]\centering
	\subfloat[Результат]{\label{ris:Flat_Lanyard_Five-Strand_French_Sinnet_5}
	\tcbox[enhanced jigsaw,colframe=black,opacityframe=0.5,opacityback=0.5]
		{\centering
			\includesvg[width=0.5\linewidth]{Utolsh/Flat_Lanyard_Five-Strand_French_Sinnet_5}}
		}
	\caption{Flat Lanyard Knot на базе Китайского ткацкого узла (рис.~\ref{ris:Chinese_tkach}).}\label{ris:Five-Lead_Flat-Flat_Lanyard_Five-Strand_French_Sinnet}
\end{figure}

\subsection{Two-Plane Knot}

\begin{figure}[H]\centering
	\subfloat[Первый вариант]{\label{ris:Two-Plane_Knot_1_1}
	\tcbox[enhanced jigsaw,colframe=black,opacityframe=0.5,opacityback=0.5]
		{\centering
			\includesvg[width=0.35\linewidth]{Utolsh/Two-Plane_Knot}}
		}
\hfil
	\subfloat[Второй вариант]{\label{ris:Two-Plane_Knot_1_2}
	\tcbox[enhanced jigsaw,colframe=black,opacityframe=0.5,opacityback=0.5]
		{\centering
			\includesvg[width=0.35\linewidth]{Utolsh/Two-Plane_Knot_1}}
		}
	\caption{Two-Plane Knot.}\label{ris:Two-Plane_Knot}
\end{figure}

\subsection{Two-Plane Knot 2}

\begin{figure}[H]\centering
	\subfloat[Первый вариант]{\label{ris:Two-Plane_Knot_2_1}
	\tcbox[enhanced jigsaw,colframe=black,opacityframe=0.5,opacityback=0.5]
		{\centering
			\includesvg[width=0.35\linewidth]{Utolsh/Two-Plane_Knot_3}}
		}
\hfil
	\subfloat[Второй вариант]{\label{ris:Two-Plane_Knot_2_2}
	\tcbox[enhanced jigsaw,colframe=black,opacityframe=0.5,opacityback=0.5]
		{\centering
			\includesvg[width=0.35\linewidth]{Utolsh/Two-Plane_Knot_2}}
		}
	\caption{Two-Plane Knot.}\label{ris:Two-Plane_Knot_2}
\end{figure}

\subsection{Rectangular Knot}

\begin{figure}[H]\centering
	\begin{minipage}{1\linewidth}
		\begin{center}
			\tcbox[enhanced jigsaw,colframe=black,opacityframe=0.5,opacityback=0.5]
			{\centering{\includesvg[width=0.5\linewidth]{Utolsh/Rectangular_Knot}}}
		\end{center}
	\end{minipage}
\caption{Rectangular Knot.}
\label{ris:Rectangular_Knot}
\end{figure}

Прямоугольный узел.

\subsection{Larger Rectangular Knot}

\begin{figure}[H]\centering
	\begin{minipage}{1\linewidth}
		\begin{center}
			\tcbox[enhanced jigsaw,colframe=black,opacityframe=0.5,opacityback=0.5]
			{\centering{\includesvg[width=0.7\linewidth]{Utolsh/larger_Rectangular_Knot}}}
		\end{center}
	\end{minipage}
\caption{Larger Rectangular Knot.}
\label{ris:larger_Rectangular_Knot}
\end{figure}

\subsection{Rectangular Two-Plane Lanyard Knot}

\begin{figure}[H]\centering
	\begin{minipage}{1\linewidth}
		\begin{center}
			\tcbox[enhanced jigsaw,colframe=black,opacityframe=0.5,opacityback=0.5]
			{\centering{\includesvg[width=0.65\linewidth]{Utolsh/Rectangular_Two-Plane}}}
		\end{center}
	\end{minipage}
\caption{Rectangular Two-Plane Lanyard Knot.}
\label{ris:Rectangular_Two-Plane}
\end{figure}

\subsection{Pentagon in Two Planes}

\begin{figure}[H]\centering
	\begin{minipage}{1\linewidth}
		\begin{center}
			\tcbox[enhanced jigsaw,colframe=black,opacityframe=0.5,opacityback=0.5]
			{\centering{\includesvg[width=0.55\linewidth]{Utolsh/Pentagon_Two_Planes}}}
		\end{center}
	\end{minipage}
\caption{Pentagon in Two Planes.}
\label{ris:Pentagon_Two_Planes}
\end{figure}

\addtocounter{KnotNoName}{1}

\subsection{Узел без названия \arabic{KnotNoName}}

\begin{figure}[H]\centering
	\subfloat[Первый вариант]{\label{ris:KnotNoName_14_1}
	\tcbox[enhanced jigsaw,colframe=black,opacityframe=0.5,opacityback=0.5]
		{\centering
			\includesvg[width=0.5\linewidth]{Utolsh/KnotNoName_14}}
		}
\end{figure}
% \vfill
\begin{figure}[H]\centering
	\subfloat[Второй вариант]{\label{ris:KnotNoName_14_2}
	\tcbox[enhanced jigsaw,colframe=black,opacityframe=0.5,opacityback=0.5]
		{\centering
			\includesvg[width=0.5\linewidth]{Utolsh/KnotNoName_14_1}}
		}
	\caption{Узел без названия \arabic{KnotNoName}.}\label{ris:KnotNoName_14}
\end{figure}

\subsection{Symmetrical Chinese Button}

\begin{figure}[H]\centering
	\subfloat[Первый вариант]{\label{ris:Symmetrical_Chinese_Button_1}
	\tcbox[enhanced jigsaw,colframe=black,opacityframe=0.5,opacityback=0.5,height=5cm]
		{\centering
			\includesvg[width=0.32\linewidth]{Utolsh/Eight-Part_Button_1}}
		}
\hfil
	\subfloat[Второй вариант]{\label{ris:Symmetrical_Chinese_Button_2}
	\tcbox[enhanced jigsaw,colframe=black,opacityframe=0.5,opacityback=0.5,height=5cm]
		{\centering
			\includesvg[width=0.34\linewidth]{Utolsh/Eight-Part_Button}}
		}
	\caption{Two-Plane Knot.}\label{ris:Symmetrical_Chinese_Button}
\end{figure}

Eight-Part Button, Pajama Knot.

Можно завязать Chinese Button Two-Ply (doubled), Three-Ply (triple).

\subsection{Турецкий узел}

\begin{figure}[H]\centering
	\subfloat[Первый вариант]{\label{ris:Turk-Head_001_1}
	\tcbox[enhanced jigsaw,colframe=black,opacityframe=0.5,opacityback=0.5,height=5cm]
		{\centering
			\includesvg[width=0.34\linewidth]{Utolsh/Turk-Head_001}}
		}
\hfil
	\subfloat[Второй вариант]{\label{ris:Turk-Head_001_2}
	\tcbox[enhanced jigsaw,colframe=black,opacityframe=0.5,opacityback=0.5,height=5cm]
		{\centering
			\includesvg[width=0.32\linewidth]{Utolsh/Turk-Head_002}}
		}
	\caption{Турецкий узел.}\label{ris:Turk-Head_001}
\end{figure}

Основа для Турецкой оплетки – Головы турка.

\subsection{Double Chinese Button}

\begin{figure}[H]\centering
	\subfloat[Первый вариант]{\label{ris:Double_Chinese_Button_1}
	\tcbox[enhanced jigsaw,colframe=black,opacityframe=0.5,opacityback=0.5,height=5cm]
		{\centering
			\includesvg[width=0.34\linewidth]{Utolsh/Eight-Part_Button_2}}
		}
\hfil
	\subfloat[Второй вариант]{\label{ris:Double_Chinese_Button_2}
	\tcbox[enhanced jigsaw,colframe=black,opacityframe=0.5,opacityback=0.5,height=5cm]
		{\centering
			\includesvg[width=0.32\linewidth]{Utolsh/Eight-Part_Button_3}}
		}
	\caption{Double Chinese Button.}\label{ris:Double_Chinese_Button}
\end{figure}

Double Eight-Part Button, Chinese Button Two-Ply (doubled). Аналогичным образом можно завязать Three-Ply (triple).

\subsection{Two-Ply Eight-Part Button}

\begin{figure}[H]\centering
	\begin{minipage}{1\linewidth}
		\begin{center}
			\tcbox[enhanced jigsaw,colframe=black,opacityframe=0.5,opacityback=0.5]
			{\centering{\includesvg[width=0.45\linewidth]{Utolsh/Two-Ply_Eight-Part_Button}}}
		\end{center}
	\end{minipage}
\caption{Two-Ply Eight-Part Button.}
\label{ris:Two-Ply_Eight-Part_Button}
\end{figure}

Second Chinese Button.

\subsection{Chinese Button 2}

\begin{figure}[H]\centering
	\begin{minipage}{1\linewidth}
		\begin{center}
			\tcbox[enhanced jigsaw,colframe=black,opacityframe=0.5,opacityback=0.5]
			{\centering{\includesvg[width=0.4\linewidth]{Utolsh/Chinese_Button_2}}}
		\end{center}
	\end{minipage}
\caption{Chinese Button 2.}
\label{ris:Chinese_Button_2}
\end{figure}

Больше первого (рис.~\ref{ris:Symmetrical_Chinese_Button}), но не симметричен.

\addtocounter{KnotNoName}{1}

\subsection{Узел без названия \arabic{KnotNoName}}

\begin{figure}[H]\centering
	\begin{minipage}{1\linewidth}
		\begin{center}
			\tcbox[enhanced jigsaw,colframe=black,opacityframe=0.5,opacityback=0.5]
			{\centering{\includesvg[width=0.4\linewidth]{Utolsh/KnotNoName_11}}}
		\end{center}
	\end{minipage}
\caption{Узел без названия \arabic{KnotNoName}.}
\label{ris:KnotNoName_11}
\end{figure}

\addtocounter{KnotNoName}{1}

\subsection{Узел без названия \arabic{KnotNoName}}

\begin{figure}[H]\centering
	\begin{minipage}{1\linewidth}
		\begin{center}
			\tcbox[enhanced jigsaw,colframe=black,opacityframe=0.5,opacityback=0.5]
			{\centering{\includesvg[width=0.5\linewidth]{Utolsh/KnotNoName_12}}}
		\end{center}
	\end{minipage}
\caption{Узел без названия \arabic{KnotNoName}.}
\label{ris:KnotNoName_12}
\end{figure}

\addtocounter{KnotNoName}{1}

\subsection{Узел без названия \arabic{KnotNoName}}

\begin{figure}[H]\centering
	\begin{minipage}{1\linewidth}
		\begin{center}
			\tcbox[enhanced jigsaw,colframe=black,opacityframe=0.5,opacityback=0.5]
			{\centering{\includesvg[width=0.5\linewidth]{Utolsh/KnotNoName_13}}}
		\end{center}
	\end{minipage}
\caption{Узел без названия \arabic{KnotNoName}.}
\label{ris:KnotNoName_13}
\end{figure}

\addtocounter{KnotNoName}{1}

\subsection{Узел без названия \arabic{KnotNoName}}

\begin{figure}[H]\centering
	\begin{minipage}{1\linewidth}
		\begin{center}
			\tcbox[enhanced jigsaw,colframe=black,opacityframe=0.5,opacityback=0.5]
			{\centering{\includesvg[width=0.6\linewidth]{Utolsh/KnotNoName_17}}}
		\end{center}
	\end{minipage}
\caption{Узел без названия \arabic{KnotNoName}.}
\label{ris:KnotNoName_17}
\end{figure}

\subsection{Six-Bight rim}

\begin{figure}[H]\centering
	\begin{minipage}{1\linewidth}
		\begin{center}
			\tcbox[enhanced jigsaw,colframe=black,opacityframe=0.5,opacityback=0.5]
			{\centering{\includesvg[width=0.5\linewidth]{Utolsh/Six-Bight_rim}}}
		\end{center}
	\end{minipage}
\caption{Six-Bight rim.}
\label{ris:Six-Bight_rim}
\end{figure}

Six-Bight rim and a Three–Bight center is not a Turk’s-Head.

\subsection{Flat Oblong Toggle}

\begin{figure}[H]\centering
	\begin{minipage}{1\linewidth}
		\begin{center}
			\tcbox[enhanced jigsaw,colframe=black,opacityframe=0.5,opacityback=0.5]
			{\centering{\includesvg[width=0.45\linewidth]{Utolsh/Flat_Oblong_Toggle}}}
		\end{center}
	\end{minipage}
\caption{Flat Oblong Toggle.}
\label{ris:Flat_Oblong_Toggle}
\end{figure}

\subsection{Oblong Knot}

\begin{figure}[H]\centering
	\begin{minipage}{1\linewidth}
		\begin{center}
			\tcbox[enhanced jigsaw,colframe=black,opacityframe=0.5,opacityback=0.5]
			{\centering{\includesvg[width=0.6\linewidth]{Utolsh/Oblong_Knot}}}
		\end{center}
	\end{minipage}
\caption{Oblong Knot.}
\label{ris:Oblong_Knot}
\end{figure}

\subsection{Square Knot}

\begin{figure}[H]\centering
	\begin{minipage}{1\linewidth}
		\begin{center}
			\tcbox[enhanced jigsaw,colframe=black,opacityframe=0.5,opacityback=0.5]
			{\centering{\includesvg[width=0.49\linewidth]{Utolsh/Square_Knot}}}
		\end{center}
	\end{minipage}
\caption{Square Knot.}
\label{ris:Square_Knot}
\end{figure}

\subsection{Square Knot with snug bights around the edges}

\begin{figure}[H]\centering
	\begin{minipage}{1\linewidth}
		\begin{center}
			\tcbox[enhanced jigsaw,colframe=black,opacityframe=0.5,opacityback=0.5]
			{\centering{\includesvg[width=0.55\linewidth]{Utolsh/Square_Knot_with_snug}}}
		\end{center}
	\end{minipage}
\caption{Square Knot with snug bights around the edges.}
\label{ris:Square_Knot_with_snug}
\end{figure}

\subsection{Elliptical Knot}

\begin{figure}[H]\centering
	\begin{minipage}{1\linewidth}
		\begin{center}
			\tcbox[enhanced jigsaw,colframe=black,opacityframe=0.5,opacityback=0.5]
			{\centering{\includesvg[width=0.5\linewidth]{Utolsh/Elliptical_Knot}}}
		\end{center}
	\end{minipage}
\caption{Elliptical Knot.}
\label{ris:Elliptical_Knot}
\end{figure}

\subsection{Elliptical Flat-Topped Knot}

\begin{figure}[H]\centering
	\begin{minipage}{1\linewidth}
		\begin{center}
			\tcbox[enhanced jigsaw,colframe=black,opacityframe=0.5,opacityback=0.5]
			{\centering{\includesvg[width=0.5\linewidth]{Utolsh/Elliptical_Flat-Topped_Knot}}}
		\end{center}
	\end{minipage}
\caption{Elliptical Flat-Topped Knot.}
\label{ris:Elliptical_Flat-Topped_Knot}
\end{figure}

\subsection{Triangular Knot}

\begin{figure}[H]\centering
	\begin{minipage}{1\linewidth}
		\begin{center}
			\tcbox[enhanced jigsaw,colframe=black,opacityframe=0.5,opacityback=0.5]
			{\centering{\includesvg[width=0.5\linewidth]{Utolsh/Triangular_Knot}}}
		\end{center}
	\end{minipage}
\caption{Triangular Knot.}
\label{ris:Triangular_Knot}
\end{figure}

\subsection{Triangular Knot with snug bights around the edges}

\begin{figure}[H]\centering
	\begin{minipage}{1\linewidth}
		\begin{center}
			\tcbox[enhanced jigsaw,colframe=black,opacityframe=0.5,opacityback=0.5]
			{\centering{\includesvg[width=0.55\linewidth]{Utolsh/Triangular_Knot_with_snug}}}
		\end{center}
	\end{minipage}
\caption{Triangular Knot with snug bights around the edges.}
\label{ris:Triangular_Knot_with_snug}
\end{figure}

\subsection{Flattened Cylindroid Toggle}

\begin{figure}[H]\centering
	\begin{minipage}{1\linewidth}
		\begin{center}
			\tcbox[enhanced jigsaw,colframe=black,opacityframe=0.5,opacityback=0.5]
			{\centering{\includesvg[width=0.8\linewidth]{Utolsh/Flattened_Cylindroid_Toggle}}}
		\end{center}
	\end{minipage}
\caption{Flattened Cylindroid Toggle.}
\label{ris:Flattened_Cylindroid_Toggle}
\end{figure}

\subsection{Upright Cylindroid Toggle}

\begin{figure}[H]\centering
	\begin{minipage}{1\linewidth}
		\begin{center}
			\tcbox[enhanced jigsaw,colframe=black,opacityframe=0.5,opacityback=0.5]
			{\centering{\includesvg[width=0.8\linewidth]{Utolsh/Upright_Cylindroid_Toggle}}}
		\end{center}
	\end{minipage}
\caption{Upright Cylindroid Toggle.}
\label{ris:Upright_Cylindroid_Toggle}
\end{figure}

\subsection{Circular Flat-Topped Knot}

\begin{figure}[H]\centering
	\begin{minipage}{1\linewidth}
		\begin{center}
			\tcbox[enhanced jigsaw,colframe=black,opacityframe=0.5,opacityback=0.5]
			{\centering{\includesvg[width=0.5\linewidth]{Utolsh/Circular_Flat-Topped_Knot}}}
		\end{center}
	\end{minipage}
\caption{Circular Flat-Topped Knot.}
\label{ris:Circular_Flat-Topped_Knot}
\end{figure}

\subsection{Узел на основе Chinese Butterfly knot}

%TODO Найти этот Chinese Butterfly knot

\begin{figure}[H]\centering
	\begin{minipage}{1\linewidth}
		\begin{center}
			\tcbox[enhanced jigsaw,colframe=black,opacityframe=0.5,opacityback=0.5]
			{\centering{\includesvg[width=0.5\linewidth]{Utolsh/Chinese_Butterfly_knot_based}}}
		\end{center}
	\end{minipage}
\caption{Узел на основе Chinese Butterfly knot.}
\label{ris:Chinese_Butterfly_knot_based}
\end{figure}

\subsection{Butterfly knot in triangular form}

\begin{figure}[H]\centering
	\begin{minipage}{1\linewidth}
		\begin{center}
			\tcbox[enhanced jigsaw,colframe=black,opacityframe=0.5,opacityback=0.5]
			{\centering{\includesvg[width=0.5\linewidth]{Utolsh/Triangular_Butterfly_knot}}}
		\end{center}
	\end{minipage}
\caption{Butterfly knot in triangular form.}
\label{ris:Triangular_Butterfly_knot}
\end{figure}

\subsection{Button Knot (handsomest)}

\begin{figure}[H]\centering
	\begin{minipage}{1\linewidth}
		\begin{center}
			\tcbox[enhanced jigsaw,colframe=black,opacityframe=0.5,opacityback=0.5]
			{\centering{\includesvg[width=0.5\linewidth]{Utolsh/Button_Knot}}}
		\end{center}
	\end{minipage}
\caption{Button Knot (handsomest).}
\label{ris:Button_Knot}
\end{figure}

\subsection{Knot with diagonal texture}

\begin{figure}[H]\centering
	\begin{minipage}{1\linewidth}
		\begin{center}
			\tcbox[enhanced jigsaw,colframe=black,opacityframe=0.5,opacityback=0.5]
			{\centering{\includesvg[width=0.75\linewidth]{Utolsh/Knot_with_diagonal_texture}}}
		\end{center}
	\end{minipage}
\caption{Knot with diagonal texture.}
\label{ris:Knot_with_diagonal_texture}
\end{figure}

Узел с диагонвльной структурой.

\addtocounter{KnotNoName}{1}

\subsection{Узел без названия \arabic{KnotNoName}}

\begin{figure}[H]\centering
	\begin{minipage}{1\linewidth}
		\begin{center}
			\tcbox[enhanced jigsaw,colframe=black,opacityframe=0.5,opacityback=0.5]
			{\centering{\includesvg[width=0.75\linewidth]{Utolsh/KnotNoName_15}}}
		\end{center}
	\end{minipage}
\caption{Узел без названия \arabic{KnotNoName}.}
\label{ris:KnotNoName_15}
\end{figure}

В зависимости от затягивания получается два разных варианта – вертикально-горизонтальный или диагональный.

\subsection{Circle in Two Planes}

\begin{figure}[H]\centering
	\begin{minipage}{1\linewidth}
		\begin{center}
			\tcbox[enhanced jigsaw,colframe=black,opacityframe=0.5,opacityback=0.5]
			{\centering{\includesvg[width=0.8\linewidth]{Utolsh/Circle_Two_Planes}}}
		\end{center}
	\end{minipage}
\caption{Circle in Two Planes.}
\label{ris:Circle_Two_Planes}
\end{figure}
